% Copyright (c) 2019 Bochen Tan
% Public domain.
%本模板的宗旨是尽量绿色,不需要附加安装任何东西。
%按照教务部下发的WORD说明文档格式,下简称“说明”
%没有封面和评阅表,这两部分请直接在Cover&ReviewTable.doc中写再输出pdf拼到一起
%doc小改动:封面校徽和文字替换为了高清版本,“题目:”和中文题目对齐,中英文题目分在了表的两行
%doc小改动:插入了两个白页,使得连续打印的时候封面和表格都在奇数页
%正文部分改动:在每一页下方中央加了页码,因为说明中页眉不分奇偶页,所以页码就都在中央吧
%不含自动的参考文献,因为说明中参考文献格式不典型,请手动输入或自行写程序
%在Windows或Linux下渲染出字体更接近说明,Mac OS上字体不太一样
%有警告\headheight is too small,fancyhdr的上距离有点小,似乎问题不大

\documentclass[UTF8,openany,AutoFakeBold,AutoFakeSlant,cs4size]{ctexbook}
%openany 使一章可以从偶数页开始,因为说明中每一章并没有只能从奇数页开始,虽然这是常理
%AutoFakeBold 和 AutoFakeSlant 因为 CJK 里没有真正的加粗和倾斜,如果额外字体则效果更好
%cs4size 因为要求主题是小四号字

\usepackage[a4paper,left=3.18cm,right=3.18cm,top=2.54cm,bottom=2.54cm]{geometry}
%office中正常页边距



\usepackage{amsmath}
\usepackage{amsthm}
\usepackage{bm}
\usepackage{amsfonts}
\usepackage{enumerate}
\usepackage{fancyhdr}
\usepackage{color}
\usepackage{proof}
\usepackage{float}
\usepackage{algorithm}
\usepackage{algorithmic}

\renewcommand{\algorithmicrequire}{\textbf{输入:}}

\renewcommand{\algorithmicensure}{\textbf{输出:}}

\usepackage{cite}
\newcommand{\upcite}[1]{\textsuperscript{\cite{#1}}} %引用在右上角

\usepackage{hyperref}
\usepackage{gbt7714}

\usepackage{multirow,booktabs,makecell}
\usepackage{graphicx}
\usepackage[font=small,labelsep=space]{caption} %五号,宋体/Time new roman
\renewcommand{\thetable}{\arabic{table}} %表格和图片编号不分章节,直接1,2,3 ...
\renewcommand{\thefigure}{\arabic{figure}}
\renewcommand{\theequation}{\arabic{chapter}.\arabic{equation}} %公式标签 章.公式(均为阿拉伯数字)



\usepackage{tocloft} %自定义目录,说明中没有明确规定,和WORD自动生成目录格式一致

%“全文目录”四个字的格式
\renewcommand\cftbeforetoctitleskip{0pt}
\renewcommand\cftaftertoctitleskip{0pt}
\renewcommand\cfttoctitlefont{\bfseries\heiti\zihao{2}}

\renewcommand\cftchapfont{\heiti\normalsize} %黑体小四
\renewcommand\cftchapdotsep{\cftdotsep} %有点连到页码,点间距不确定,待改
\renewcommand\cftchappagefont{\songti\normalsize} %宋体小四页码
\renewcommand\cftbeforechapskip{0pt}

%1. 第一级 五号宋体,缩进两个字符,页码一致
\renewcommand\cftsecfont{\songti\small}
\renewcommand\cftsecpagefont{\songti\small}
\renewcommand\cftsecaftersnum{.} %一级目录号后加点
\renewcommand\cftsecindent{2em}
\renewcommand\cftbeforesecskip{0pt}

%1.1 第二级 五号宋体,缩进四个字符,页码一致
\renewcommand\cftsubsecfont{\songti\small}
\renewcommand\cftsubsecpagefont{\songti\small}
\renewcommand\cftsubsecindent{4em}
\renewcommand\cftbeforesubsecskip{0pt}

%1.1.1 第二级 五号宋体,缩进四个字符,页码一致
\renewcommand\cftsubsubsecfont{\songti\small}
\renewcommand\cftsubsubsecpagefont{\songti\small}
\renewcommand\cftsubsubsecindent{4em}
\renewcommand\cftbeforesubsubsecskip{0pt}



\usepackage{titlesec}%自定义章节标题
\CTEXsetup[format={\bfseries\center\heiti\zihao{2}},beforeskip=0pt]{chapter}
%第一章  绪论(二号、黑体) beforeskip为上方垂直距离看起来还比说明偏大,待改

\setcounter{tocdepth}{3}
\setcounter{secnumdepth}{3}
%使目录中有三级标题,即subsubsection

\renewcommand\thesection{\arabic{section}} % 使得不显示章名,只显示节名
\titleformat{\section}
{\raggedright\zihao{3}\bfseries\songti}
{\thesection.\quad}
{0pt}
{}%1. 第一级(三号、宋体/Time new roman、加粗)

\titleformat{\subsection}
{\raggedright\bfseries\zihao{4}\songti}
{\thesubsection\quad}
{0pt}
{}%1.1 第二级(四号,宋体/Time new roman,加粗)

\titleformat{\subsubsection}
{\raggedright\bfseries\zihao{-4}\songti}
{\thesubsubsection\quad}
{0pt}
{}%1.1.1 第三级(小四,宋体/Time new roman,加粗)




% 封面依赖的宏包
\input{CoverHead}
% 评阅表依赖的宏包
\input{ReviewTableHead}


\title{}
\author{}
\date{}
\begin{document}

% 封面中需要修改的内容直接在此处更改即可
\newcommand{\chineseTitle}{一种利用Redex实现重组糖的轻量级方法}
\newcommand{\englishTitle}{A Lightweight Resugaring Method using PLT Redex}
\newcommand{\name}{杨子毅}
\newcommand{\studentID}{1600011063}
\newcommand{\school}{信息科学技术学院}
\newcommand{\major}{软件工程}
\newcommand{\advisor}{胡振江}
% 插入封面
\input{cover}
\clearpage


% 插入导师评阅表
%!TEX root=./Template.tex
\thispagestyle{empty}
\newgeometry{left=2cm, right=2cm, top=2.64cm, bottom=2.54cm}
\renewcommand\arraystretch{1.2}

\begin{center}
{\songti\zihao{3}{北京大学本科毕业论文导师评阅表}}
\end{center}

\begin{table}[H]
	\centering
    \begin{tabular}{|rrrrrc|}
    \hline
    \multicolumn{1}{|p{4em}|}{学生姓名} & \multicolumn{1}{p{3em}|}{杨子毅} & \multicolumn{1}{p{5em}|}{学生学号} & \multicolumn{1}{p{6.5em}|}{1600011063} & \multicolumn{1}{p{6.565em}|}{论文成绩} &  \multicolumn{1}{r|}{}\\
    \hline
    \multicolumn{1}{|p{4em}|}{学院(系)} & \multicolumn{3}{c|}{信息科学技术学院} & \multicolumn{1}{p{6.565em}|}{学生所在专业} &  
    \multicolumn{1}{c|}{软件工程}\\
    \hline
    \multicolumn{1}{|r|}{\multirow{2}[2]{*}{导师姓名}} & \multicolumn{1}{c|}{\multirow{2}[2]{*}{\centering{胡振江}}} & \multicolumn{1}{p{5em}|}{导师单位/} & \multicolumn{1}{c|}{\multirow{2}[2]{*}{\centering{软件工程研究所}}} & \multicolumn{1}{p{6.565em}|}{\multirow{2}[2]{*}{导师职称}} & \multirow{2}[2]{*}{\centering{教授}} \\
    \multicolumn{1}{|r|}{} & \multicolumn{1}{r|}{} & \multicolumn{1}{p{5em}|}{所在研究所} & \multicolumn{1}{r|}{} & \multicolumn{1}{r|}{} &  \\
    \hline
    \multicolumn{2}{|p{9em}|}{\centering{论文题目}} & \multicolumn{4}{c|}{\centering{一种利用Redex实现重组糖的轻量级方法}} \\
    \multicolumn{2}{|p{9em}|}{\centering{(中、英文)}} & \multicolumn{4}{c|}{A Lightweight Resugaring Method using PLT Redex} \\
    \hline
    \multicolumn{6}{|c|}{\centering{导师评语}} \\
    \multicolumn{6}{|p{35.88em}|}{\kaiti{(包含对论文的性质、难度、分量、综合训练等是否符合培养目标的目的等评价)}} \\
    %语法糖作为设计和实现领域特定语言的一个简单而有效的方法正受到越来越多的关注,但是它有一个缺陷,即语法糖表达式被展开成内部语言表达式后,其执行状态变得难以理解。为了解决这个问题,人们提出了重组糖的概念,将嵌入在内部语言的语法糖表达式执行序列尽可能地反映到带有语法糖的表面语言上,从而得到在语法糖层面的执行序列。杨子毅同学在本科毕业设计的研究中,提出了一个新的重组糖方法,并在PLT Redex的基础上实现了一套重组糖工具。新的重组糖方法不仅比原来的方法简单直接,而且能解决原来的方法难以解决的递归糖、高阶糖的重组糖问题,对卫生宏的处理也更简单和自然。 杨子毅同学的本科毕业设计论文写作规范,逻辑性强,是一篇优秀的本科毕业论文。
    \multicolumn{6}{|l|}{\qquad 语法糖作为设计和实现领域特定语言的一个简单而有效的方法正受到越来越多的关注,} \\
    \multicolumn{6}{|l|}{但是它有一个缺陷,即语法糖表达式被展开成内部语言表达式后,其执行状态变得难以理解} \\
    \multicolumn{6}{|l|}{。为了解决这个问题,人们提出了重组糖的概念,将嵌入在内部语言的语法糖表达式执行序} \\
    \multicolumn{6}{|l|}{列尽可能地反映到带有语法糖的表面语言上,从而得到在语法糖层面的执行序列。杨子毅同} \\
    \multicolumn{6}{|l|}{学在本科毕业设计的研究中,提出了一个新的重组糖方法,并在PLT Redex的基础实现了以} \\
    \multicolumn{6}{|l|}{重组糖工具。新的重组糖方法不仅比原来的方法简单直接,而且能解决原来的方法难以解决} \\
    \multicolumn{6}{|l|}{的递归糖、高阶糖的重组糖问题,对卫生宏的处理也更简单和自然。} \\
    \multicolumn{6}{|l|}{\qquad 杨子毅同学的本科毕业设计论文写作规范,逻辑性强,是一篇优秀的本科毕业论文。} \\
    \multicolumn{6}{|r|}{} \\
    \multicolumn{6}{|r|}{} \\
    \multicolumn{6}{|r|}{} \\
    \multicolumn{6}{|r|}{} \\
    \multicolumn{6}{|r|}{} \\
    \multicolumn{6}{|r|}{} \\
    \multicolumn{6}{|r|}{} \\
    \multicolumn{6}{|p{35.88em}|}{                                                                             \hfill 导师签名:\qquad\qquad\qquad\qquad\qquad\qquad\qquad\qquad } \\
    \multicolumn{6}{|r|}{} \\
    \multicolumn{6}{|p{35.88em}|}{\hfill 年 \qquad\quad 月 \qquad\quad 日 \qquad\qquad\qquad} \\
    \multicolumn{6}{|r|}{} \\
    \hline
    \end{tabular}
\end{table}

\renewcommand\arraystretch{1}
\restoregeometry
\clearpage

\zihao{-4}\songti\linespread{1.5}\selectfont
\linespread{1.5}\selectfont
\chapter*{版权声明}
\setcounter{page}{0}
% 本页不计页码
\thispagestyle{empty}
% 本页无页眉和页脚
任何收存和保管本论文各种版本的单位和个人,未经本论文作者同意,不得将本论文转借他人,亦不得随意复制、抄录、拍照或以任何方式传播。否则,引起有碍作者著作权之问题,将可能承担法律责任。
\clearpage

%版权声明后空白一页,使得摘要从奇数页开始。
%\quad
%\setcounter{page}{0}
% 本页不计页码
%\thispagestyle{empty}
% 本页无页眉和页脚
%\clearpage



\pagestyle{fancy}
\normalsize
\linespread{1.5}\selectfont
%小四号,宋体/Time new roman,1.5倍行距



\chapter*{摘要}

\phantomsection
\addcontentsline{toc}{chapter}{摘要} %手动加入目录
随着计算机科学的普及和程序设计语言的发展,程序设计语言、特别是领域特定语言的应用越来越日常化。语法糖作为实现领域特定语言的一项重要技术在近年来发展火热,相关的研究以及为支持DSL设计特性的的程序设计语言(例如Racket、Scala)都有显著进展显著。

重组糖是一项关于语法糖的研究---将嵌入在内部语言的语法糖表达式执行序列在表面语言(语法糖)上提取出来,以得到在语法糖层面的执行序列(详见\ref{mark:resugaring})。其目的是为了解决语法糖解糖的单向性,让语法糖表达式的执行过程能表现在语法糖结构上。但现存方法很难处理递归糖、高阶糖等语法糖特性,且对于卫生宏处理很繁琐。

我们基于PLT Redex设计了一个轻量级重组糖算法,简单实现了一套工具并在一些例子上应用进行测试。结果显示我们的轻量级重组糖算法相较于现有重组糖算法可以多处理一些语法糖特性,也更容易处理卫生宏等特性。除此之外,我们的方法在表示语法糖的方式上更加灵活。

%背景 解决什么问题 得到什么结果

\bigskip
\noindent{\bfseries\songti 关键词: 领域特定语言、语法糖、解释器、重写系统 }


\fancypagestyle{plain} %因为latex默认每章第一页是plain所以需要重置一下plain和说明统一
{
	\fancyhf{} %清空

	\fancyhead[RE,RO]{摘要}
	%偶数页右页眉,奇数页右页眉均为“摘要”,及章名\leftmark

	\fancyhead[LE,LO]{北京大学本科生毕业论文}
	%偶数页左页眉,奇数页左页眉均为“北京大学本科生毕业论文”

	\fancyfoot[CO,CE]{~\thepage~}
	%偶数页和奇数页中页脚为页码,从对称考虑,因为每页在说明中都是一样的,不分奇偶

	\renewcommand{\headrulewidth}{0.7pt} %页眉线宽度,可调,不太清楚说明中是多少,待改

	\renewcommand{\footrulewidth}{0pt} %页脚线宽度为0,既没有
}

%默认的风格是fancy,设置于下,用于每章非第一页
\fancyhf{}
\fancyhead[RE,RO]{摘要}
\fancyhead[LE,LO]{北京大学本科生毕业论文}
\fancyfoot[CO,CE]{~\thepage~}
\renewcommand{\headrulewidth}{0.7pt}
\renewcommand{\footrulewidth}{0pt}
\clearpage






\small
\linespread{1.5}\selectfont
%5号,Time new roman,1.5倍行距

\chapter*{\bfseries Abstract}
\phantomsection
\addcontentsline{toc}{chapter}{\bfseries Abstract} %Abstract加粗
With the popularization of computer science and the development of programming languages, the application of programming languages, especially domain-specific languages, is becoming more and more routine. Syntactic sugar, as an important technique for implementing a domain-specific language, has developed fiercely in recent years, and related research and programming languages designing features for DSL(such as Racket, Scala) are making significant progress.

Resugaring is a research on syntactic sugar--lifting the evaluation sequence of syntactic sugar expression embedded in the core language on the surface language (syntactic sugar) to get the sequence at the level of syntactic sugar. (see detail\ref{mark:resugaring}) The purpose of resugaring is to solve the unidirectionality of desugaring the syntactic sugar expression, so that the evaluation sequence of syntactic sugar expression can be expressed in the structure of syntactic sugar. However, the existing methods are difficult to deal with the features of syntactic sugar such as recursive sugar and high-order sugar, and are very cumbersome for hygienic macro processing.

We designed a lightweight resugaring algorithm based on PLT Redex, simply implemented a set of tools and tested on some examples. The results show that our lightweight resugaring algorithm can handle more syntactic sugar features than existing resugaring algorithms, also handle features like hygienic macro more simply. In addition, our method is more flexible in the way of representing syntactic sugar.


\bigskip
\noindent

{\bfseries Key Words: Domain-specific language, Syntactic sugar, Interpreter, Rewriting system }



\fancypagestyle{plain}
{
	\fancyhf{}
	\fancyhead[RE,RO]{Abstract}
	\fancyhead[LE,LO]{北京大学本科生毕业论文}
	\fancyfoot[CO,CE]{~\thepage~}
	\renewcommand{\headrulewidth}{0.7pt}
	\renewcommand{\footrulewidth}{0pt}
}
\fancyhf{}
\fancyhead[RE,RO]{Abstract}
\fancyhead[LE,LO]{北京大学本科生毕业论文}
\fancyfoot[CO,CE]{~\thepage~}
\renewcommand{\headrulewidth}{0.7pt}
\renewcommand{\footrulewidth}{0pt}
\clearpage





\fancypagestyle{plain}
{
	\fancyhf{}
	\fancyhead[RE,RO]{全文目录}
	\fancyhead[LE,LO]{北京大学本科生毕业论文}
	\fancyfoot[CO,CE]{~\thepage~}
	\renewcommand{\headrulewidth}{0.7pt}
	\renewcommand{\footrulewidth}{0pt}
}
\fancyhf{}
\fancyhead[RE,RO]{全文目录}
\fancyhead[LE,LO]{北京大学本科生毕业论文}
\fancyfoot[CO,CE]{~\thepage~}
\renewcommand{\headrulewidth}{0.7pt}
\renewcommand{\footrulewidth}{0pt}
\renewcommand{\contentsname}{\centerline{全文目录}}
\phantomsection
\tableofcontents
\addcontentsline{toc}{chapter}{全文目录}
\clearpage





\normalsize
\linespread{1.5}\selectfont
%正文,小四号,中文宋体,英文Time new roman,1.5倍行距
\fancypagestyle{plain}
{
	\fancyhf{}
	\fancyhead[RE,RO]{\leftmark}
	\fancyhead[LE,LO]{北京大学本科生毕业论文}
	\fancyfoot[CO,CE]{~\thepage~}
	\renewcommand{\headrulewidth}{0.7pt}
	\renewcommand{\footrulewidth}{0pt}
}
\fancyhf{}
\fancyhead[RE,RO]{\leftmark}
\fancyhead[LE,LO]{北京大学本科生毕业论文}
\fancyfoot[CO,CE]{~\thepage~}
\renewcommand{\headrulewidth}{0.7pt}
\renewcommand{\footrulewidth}{0pt}



%正文,五号,中文宋体,英文Time new roman,1倍行距
\section{Introduction}

What is the research background and and what motivate you to do this research?

What is the research issue and how the issue has been addressed so far?

What is the remained research problem and how challenge it is?

What is your key idea (insight) of your solution to be discussed in this paper?

What are the three main technical contributions oof this paper?

The rest of the paper is organized as follows. ...
\pagestyle{fancy}
\normalsize
\linespread{1.5}\selectfont
\chapter{背景知识}
\addtocontents{los}{\protect\addvspace{10pt}}

\section{重组糖形式化定义}
对于给定求值规则的内部语言CoreLang,和在CoreLang基础上用语法糖构造的表面语言SurfLang;对于任意SurfLang的表达式,得到其在SurfLang上的求值序列,且该求值序列满足三个性质:

1.	仿真性:求值序列需要和在CoreLang上的求值顺序相同,即存在CoreLang上的求值序列中的部分中间过程与该序列中的元素对应。该性质是重组糖有意义的前提。

2.	抽象性:求值序列中只存在SurfLang中存在的术语,没有引入CoreLang中的术语。该性质是重组糖研究的目的。

3.	覆盖性:在求值序列中没有跳过一些中间过程。该性质不是正确性的必要条件,却是在应用中极其重要的;加上前两条性质满足的正确性,构成了重组糖的全部重要性质。

例子:
\begin{equation}
and(or(\#f,\#t),and(\#t,\#f))
\end{equation}

语法糖规则(Surflang):
\begin{equation}
\parbox[t]{\textwidth}{%
	\begin{center}  
	and(e1,e2)==if(e1,e2,\#f)\\
	or(e1,e2)==if(e1,\#t,e2)
	\end{center}  
}%  
\end{equation}


其中if的规则是在coreLang上规定的,其具体规约规则如下:
\begin{equation}
\parbox[t]{\textwidth}{%
	\begin{center}  
	if(\#t,e1,e2)==e1\\
	if(\#f,e1,e2)==e2
	\end{center}  
}%  
\end{equation}

则我们期望得到的重组糖序列是如下的序列:

\begin{equation}
\framebox[20em][l]{%  
	and(or(\#f,\#t),and(\#t,\#f))-->\parbox[t]{\textwidth}{%
		\begin{flushleft}  
			and(\#t,and(\#t,\#f))\\
			and(\#t,\#f)\\
			\#f
		\end{flushleft}  
	}%  
}  
\end{equation}
全文将围绕这个例子展开初步的讲解。

\section{完全β规约、归约语义和PLT Redex}
与β规约的概念不同,完全β规约是一种基于β规约的求值顺序规则。对于一个嵌套的lambda表达式,每个表达式都可能进行β规约,而常规的call-by-name和call-by-value都是对规约顺序进行了约定,而完全β规约就是一种不定序的求值规则,每个可β规约的位置都有可能进行规约,因此得到的规约路径不是一条,而是一个图,且这个图的起点和终点只有一个。如图所示的例子就是一个完全β规约的求值图

\begin{figure}[h]
	\centering
	\includegraphics[width=12cm]{images/chapter2/fullbeta.png}
	\caption{完全β规约}
\end{figure}

可以看出,在完全beta规则中,对任何位置的可beta规约的lambda表达式,都可以进行规约。因此,与call-by-name和call-by-value不同的是,这种求值规则是不定序的。


规约语义:我们需要在求值规则中约定每一个表达式的规约规则,。。。PLT Redex\upcite{SEwPR}是基于此语义的语义工程工具。todo

本文工作的最初思想就是基于完全beta规约。我们在对规约语义不限制上下文环境的情况下,其规约路径也将成为类似完全β规约的图。还是基于上文的例子and(or(\#f,\#t),and(\#t,\#f)),我们可以得到如下的完全规约图\ref{fig:full_reduction}。

\begin{figure}[h]
	\centering
	\includegraphics[width=12cm]{images/chapter2/fullreduction.png}
	\caption{完全规约}
	\label{fig:full_reduction}
\end{figure}

在这个图中,我们可以看出,用红色标出的子序列是将语法糖直接展开后进行规约的求值序列,而这其中有许多中间表达式是可以重组成语法糖的,用蓝色标出。我们可以发现,在这个中间表达式中可重组的部分就是(2.3)中的重组糖序列。而又可以在图中找到绿色标出的序列---可以惊喜而又自然的发现这一条规约规约路径包含了我们想要的重组糖序列。自然是因为我们对语法糖的规约做了类似完全β规约的处理,导致每个子表达式都有可能首先被规约处理,因此重组糖序列一定在我们的完全图中。

\pagestyle{fancy}
\normalsize
\linespread{1.5}\selectfont
\chapter{算法定义及正确性证明}
\addtocontents{los}{\protect\addvspace{10pt}}

\section{对语言的规定}
{\bfseries 首先},我们需要将整个语言限定在基于树形表达式的结构化语言。

树形表达式:此处我们使用类似Lisp的S表达式的递归树,基础定义如下。

\framebox[20em][l]{%  
	Exp::=\parbox[t]{\textwidth}{%
		\begin{flushleft}  
			(Headid~Exp*)\\  
			|Value\\
			|Variable
		\end{flushleft}  
	}%  
}  

结构化:对于每个表达式中的子表达式,其规约规则只和子表达式本身有关。此限制约束了Corelang的作用域,限制语言子表达式不能有对外的副作用。我们将在第五章详细讨论副作用的一些具体解决办法

{\bfseries 此外}:我们对CoreLang和SurfLang进行一些简单的限制。

对每个子表达式都是CoreLang表达式的表达式Exp,最多只能有一条规约路径(通过求值顺序约束)。这一点约束并不过分,为了保证每个程序只能有一条执行路径。

对SurfLang,任意一个语法糖只能有一个CoreLang的表达式与之对应。这也是很自然的要求,因为同一个语法糖不应该有二义性。对于求值顺序,SurfLang上的子表达式约定类似完全规约的规则,任何子表达式都可以首先进行规约。

在PLT Redex中,我们将CoreLang和SurfLang视为同一个语言。则当我们定义了一个语言内部各种规约规则后,对于任意Exp都有其对应的一条或多条规约规则。根据对CoreLang的约定,有多条规约规则的表达式必然存在SurfLang的表达式。

为了区分CoreLang的语言和SurfLang的语言,我们将表达式的文法定义为如下

\framebox[30em][l]{%  
	\parbox[t]{\textwidth}{
		\begin{flushleft}
			Exp::=\parbox[t]{\textwidth}{
				\begin{flushleft}  
					Coreexp\\
					|Surfexp\\
					|Commonexp\\
					|OtherSurfexp\\
					|OtherCommonexp
				\end{flushleft}  
			}\\
			Coreexp ::= (CoreHead Exp*)\\
			Surfexp ::= (SurfHead (Surfexp|Commonexp)*)\\
			Commonexp::=\parbox[t]{\textwidth}{
				\begin{flushleft}
					(CommonHead (Surfexp|Commonexp)*)\\
					|value\\
					|variable
				\end{flushleft}	
			}\\
			OtherSurfexp ::= (SurfHead Exp* Coreexp Exp*)\\
			OtherCommonexp ::= (CommonHead Exp* Coreexp Exp*)
		\end{flushleft}
	
}
	
}  

在这里,我们将CoreLang的表达式一部分提取处理作为公共表达式,是因为在重组糖序列中必定有一些表达式是属于CoreLang的,但需要在序列中输出(比如说数字,布尔表达式,以及一些可能的基础运算)。在这种情况下,对于Commonexp来说,满足CoreLang的约束,但是也可以作为重组糖的中间序列输出

可以看出,在我们的重组糖方法中,可以输出的表达式是Surfexp和Commonexp,即不存在任何子表达式中存在Coreexp。

{\bfseries 小结:}todo
\newpage

\section{算法描述}
本节讨论建立在符合约定的语言基础上。

{\bfseries 核心算法f}\footnote{核心思想:对于每个表达式Exp,我们将对它所有规约规则中选择一条符合resugaring的仿真性规则的规约,且尽可能不破坏任何语法糖。}定义如下:(输入为一个任意Exp,输出为应用应该执行的规约规则后的表达式)

\begin{flushleft}
\fbox{
	\parbox{\textwidth}{
	对Exp尝试所有规约规则,得到多个可能的表达式ListofExp'=\{$Exp'_{1}$,$Exp'_{2}$,$\ldots$\}
	
	\begin{flushleft}
		\large{\bfseries{
				1.	如果Exp是Coreexp或Commonexp或OtherCommonexp,则其规约规则	
			}	
		}	
	\end{flushleft}
	\begin{itemize}
		\item 或是将表达式规约到另一个表达式,此时只有一条规则,应用后输出Exp';
		\item 或是其规约不满足导致内部子表达式需要规约,此时因为CoreLang的定序性,只会有一个子表达式被规约(且此表达式为Surfexp),此时对该子表达式Subexp递归调用核心算法f得到Subexp’,则在ListofExp'中找到将此Exp中子表达式Subexp规约为Subexp’的表达式就是我们需要的表达式;
		\item 或是已经无法被规约(ListofExp'为空),此时返回的表达式为空。
	\end{itemize}

	\begin{flushleft}
		\large{\bfseries{
				2.	如果Exp是Surfexp或OtherSurfexp	
			}	
		}
	\end{flushleft}
	\begin{itemize}
		\item 如果内部子表达式无可规约的,则必然会展开该语法糖,此时输出表达式为Exp解糖后的表达式;
		\item 如果存在可规约的子表达式对于每个子表达式,如果可规约,则根据我们的设定,存在一条关于此子表达式的规约规则。因此每个子表达式都可能被规约的前提下,我们需要对Surfexp或OtherSurfexp的语法糖进行展开为Exp’(此展开只有一种规约规则对应),之后对Exp’调用核心算法f,检测内部哪个子表达式Subexp在f调用过程中被规约,则此子表达式Subexp需要在Exp处首先被规约,对应对该子表达式被规约的表达式。
	\end{itemize}
	}
}
\end{flushleft}
\pagebreak

{\bfseries 整体算法lightweight-resugaring}定义如下

算法Lightweight-resugaring:给定Surfexp的表达式$Exp$,输出其重组糖序列\\

\fbox{
\parbox{\textwidth}{
	$Lightweight-resugaring$($Exp$)
	
	\qquad while($tmpexp$==f($Exp$)))
	
	\qquad \qquad if($tmpexp$ is empty):
	
	\qquad \qquad \qquad return;
	
	\qquad \qquad else if($tmpexp$ is surfexp or commonexp):
	
	\qquad \qquad \qquad output $tmpexp$;
	
	\qquad \qquad else:
	
	\qquad \qquad \qquad $Lightweight-resugaring$($tmpexp$)
}
}

\section{正确性证明}
\subsection{仿真性}

\subsection{抽象性}
抽象性的正确性是显然的,因为我们在每次输出都判断了输出的$Exp$是否是$Surfexp$或$Commonexp$。

\subsection{覆盖性}




%!TEX root = ./main.tex
\section{Implementation and Case Studies}
\label{sec5}

\subsection{Implementation}

Our basic resugaring approach is implemented using PLT Redex\cite{SEwPR}, which is an semantic engineering tool based on reduction semantics\cite{reduction}. The framework of the implementation is as Figure \ref{fig:frame}.

\begin{figure}[thb]
	\centering
	\includegraphics[width=8cm]{images/frame.png}
	\caption{framework of implementation}
	\label{fig:frame}
\end{figure}

In the language model, desugaring rules are written as reduction rules of \m{SurfExp}. And context rules of \m{SurfExp} have no restrict (every subexpressions is reducible as a hole). Then for each resugaring step, we should choose the exact reduction which satisfies the reduction of mixed language's rule (see in Section \ref{mark:miexedreduction}).

\label{mark:blackbox}
And one may notice the traditional resugaring approach does not need the whole evaluation rules of core language, a black-box stepper is needed instead. We have proved that our approach can also work by just given a black-box stepper, with a tricky extension.
We use $\redc{}{}$ to denote a reduction step of core language's expression in the black-box stepper, and $\rede{}{}$ to denote a step in the extension evaluator for the mixed language. We may use $\redm{}{}$ to denote one-step reduction in our mixed language, defined in the next section.
\infrule[CoreRed]
{ \forall~i.~e_i\in \m{CoreExp}\\
\redc{(\m{CoreHead}~e_1~\ldots~e_n)}{e'}}
{\rede{(\m{CoreHead}~e_1~\ldots~e_n)}{e'}}
\infrule[CoreExt1]
{ \forall~i.~subst_i= (e_i \in \m{SurfExp}~?~\m{tmpexp}~:~e_i),~where~\m{tmpexp}~is~any~reduciable~\m{CoreExp}\\
\redc{(\m{CoreHead}~subst_1~\ldots~subst_i~\ldots~subst_n)}{(\m{CoreHead}~subst_1~\ldots~subst_i'~\ldots~subst_n)}}
{\rede{(\m{CoreHead}~e_1~\ldots~e_i~\ldots~e_n)}{(\m{CoreHead}~e_1~\ldots~e_i'~\ldots~e_n)}\\where~\redm{e_i}{e_i'}~if~e_i~\in~\m{SurfExp},~else~\redc{e_i}{e_i'}}
\infrule[CoreExt2]
{ \forall~i.~subst_i= (e_i \in \m{SurfExp}~?~\m{tmpexp}~:~e_i),~where~\m{tmpexp}~is~any~reduciable~\m{CoreExp}\\
\redc{(\m{CoreHead}~subst_1~\ldots~subst_n)}{e'}~\note{// not reduced in subexpressions}}
{\rede{(\m{CoreHead}~e_1~\ldots~e_n)}{e'[e_1/subst_1~\ldots~e_n/subst_n]}}
For expression \Code{(CoreHead $e_1$ ... $e_n$)}, replacing all subexpression not in core language with any reducible core language's term \m{tmpexp}. Then getting a result after inputting the new expression \Code{e'} to the original blackbox stepper. If reduction appears at a subexpression at $e_i$ or what the $e_i$ replaced by, then the stepper with the extension should return \Code{(CoreHead $e_1$ ... $e_i'$ ... $e_n$)}, where $e_i'$ is $e_i$ after the mixed language's one-step reduction ($\redm{}{}$) or after core language's reduction ($\redc{}{}$) (the rule \m{CoreExt1}, an example in Figure \ref{fig:e1}). Otherwise, stepper should return \Code{e'}, with all the replaced subexpressions replacing back. (the rule \m{CoreExt2}, an example in Figure \ref{fig:e2}) The extension will not violate properties of original core language's evaluator. It is obvious that the evaluator with the extension will reduce at the subexpression as it needs in core language, if the reduction appears in a subexpression. One may notice that the stepper with extension behaves the same as mixing the evaluation rules of core language and surface language. The extension is just to make it works when the evaluator of core language is a blackbox stepper. That's why the extension is tricky.
\begin{center}
\begin{figure}[thb]
\centering
\Code{(if (and e1 e2) true false)}\\ $\Downarrow_{replace}$\\ \Code{(if tmpe1 true false)}\\ $\Downarrow_{blackbox}$\\ \Code{(if tmpe1' true false)}\\ $\Downarrow_{desugar}$\\ \Code{(if (if e1 e2 false) true false)}
\caption{\m{CoreExt1}'s example}
\label{fig:e1}
\end{figure}

\begin{figure}[thb]
\centering
\Code{(if (if true ture false) (and ...) (or ...))}\\ $\Downarrow_{replace}$ \\\Code{ (if (if true ture false) tmpe2 tmpe3)}\\ $\Downarrow_{blackbox}$\\  \Code{(if true tmpe2 tmpe3)}\\ $\Downarrow_{replaceback}$\\ \Code{(if true (and ...) (or ...))}
\caption{\m{CoreExt2}'s example}
\label{fig:e2}
\end{figure}


 % \Code{(if true (and ...) (or ...))} $\Rightarrow_{replace}$ \Code{(if true tmpe2 tmpe3)} $\Rightarrow_{blackbox}$ \\ \Code{tmpe2} $\Rightarrow_{replaceback}$ \Code{(and ...)}

\end{center}



% Instead of implementing a blackbox stepper of core language, we just used the core language's reduction semantics, because its behavior is same as the stepper with extension for mixed language. We have proved or discussed the correctness with the assumption that the core language's evaluator is a blackbox stepper. 

\label{mark:optimize}
Note that in \m{SurfRed1} rule and \m{CoreExt1} rule, there is a recursive call on $\redm{}{}$. We can optimize the resugaring algorithm by recursively resugaring. For example, \Code{(Sugar1 (Sugar2 $e_{21}$ $e_{22}$ ...) $e_{11}$ $e_{12}$ ...)} as the input, and find the first subexpression should be reduced. We can firstly get the resugaring sequences of \Code{(Sugar2 $e_{21}$ $e_{22}$)}
\begin{Codes}
    (Sugar2 e_2_1 e_2_2 ...)
\OneStep{ exp1}
\DeStep{  exp...} \note{// may be 0 or more steps}
\OneStep{ expn}
\end{Codes}
Then a resugaring subsequence is got as
\begin{Codes}
    (Sugar1 (Sugar2 e_2_1 e_2_2 ...) e_1_1 e_1_2 ...)
\OneStep{ (Sugar1 exp1 e_1_1 e_1_2 ...)}
\DeStep{  (Sugar1 exp... e_1_1 e_1_2 ...)} \note{// may be 0 or more steps}
\OneStep{ (Sugar1 expn e_1_1 e_1_2 ...)}
\end{Codes}
Thus, we will not need to try to expand the outermost sugar for each inner step (recursively resugaring for inner expression).

As for the automatic derivation of evaluation rules, it is just like what we describe. Just be careful during a merge on IFAs. 

\subsection{Case Studies}

We test some applications on the tool as case studies. Note that we set call-by-value lambda calculus as terms in \m{CommonExp}, because we need to output some intermediate sequences including lambda expressions in some examples. It's easy if we want to skip them.

\subsubsection{simple sugar}
\label{mark:simple}

We construct some simple syntactic sugars and try it on our tool. Some sugar is inspired by the first work of resugaring\cite{resugaring}. The result shows that our approach can handle all sugar features of their first work.

We take a SKI combinator syntactic sugar as an example. We will show why our approach is efficient.
\[
\begin{array}{l}
\drule{\m{S}}{\Code{(lambdaN (x1 x2 x3) (x1 x2 (x1 x3)))}}\\
\drule{\m{K}}{\Code{(lambdaN (x1 x2) x1)}}\\
\drule{\m{I}}{\Code{(lambdaN (x) x)}}
\end{array}
\]




Although SKI combinator calculus is a reduced version of lambda calculus, we can construct combinators' sugar based on call-by-need lambda calculus in our CoreLang. For sugar expression \Code{(S (K (S I)) K xx yy)}, we get the following resugaring sequences.
\begin{Codes}
    (S (K (S I)) K xx yy)
\OneStep (((K (S I)) xx (K xx)) yy)
\OneStep (((S I) (K xx)) yy)
\OneStep (I yy ((K xx) yy))
\OneStep (yy ((K xx) yy))
\OneStep (yy xx)
\end{Codes}


For the traditional approach, the sugar expression should firstly desugar to
\begin{Codes}
((lambdaN
   (x1 x2 x3)
   (x1 x3 (x2 x3)))
  ((lambdaN (x1 x2) x1)
   ((lambdaN
     (x1 x2 x3)
     (x1 x3 (x2 x3)))
    (lambdaN (x) x)))
  (lambdaN (x1 x2) x1)
  xx yy)
\end{Codes}

Then in our CoreLang, the execution of expanded expression will contain 33 steps. For each step, there will be many attempts to match and substitute the syntactic sugars to resugar the expression. It will omit more steps for a larger expression.

We have described an example of \m{and} sugar and \m{or} sugar in overview. But what if the \m{or} sugar written like follows?
\[\drule{\Code{(Or $e_1$ $e_2$)}}{\Code{(let (x $e_1$) (if x x $e_2$))}}\]
Of course, we got the same evaluation rules as the example in overview.
\[
\begin{array}{c}
\infer {(\mbox{or}~e_1~e_2) \rightarrow (\mbox{or}~e_1'~e_2)} {e_1~ \rightarrow~e_1'}
\qquad
(\mbox{or}~\#t~e2) \rightarrow \#t
\quad
(\mbox{or}~\#f~e2) \rightarrow e_2 \\
\end{array}
\]

Then for expressions headed with \m{or}, we won't need the one-step try to figure out whether desugaring or processing on a subexpression, which makes our approach more concise. Overall, the unidirectional resugaring algorithm makes our approach efficient, because no attempts for resugaring the expression are needed.
\subsubsection{hygienic sugar}
\label{mark:hygienic}


The second work\cite{hygienic} of traditional resugaring approach mainly processes hygienic sugar compared to first work. It use a DAG to represent the expression. However, hygiene is not hard to handle by our lazy desugaring strategy. Our algorithm can easily process hygienic sugar without special data structure.


A typical hygienic problem is as the following example.
\[
\drule{\Code{(Hygienicadd $e_1$ $e_2$)}}{\Code{(let (x $e_1$) (+ x $e_2$))}}
\]
% \begin{Codes}
% 	(Hygienicadd e1 e2) \DeStep{ (let ((x e1)) (+ x e2))}
% \end{Codes}

For traditional resugaring approach, if we want to get sequences of \Code{(let ((x 2)) (Hygienicadd 1 x))}, it will firstly desugar to \Code{(let ((x 2)) (let ((x 1)) (+ x x)))}, which is awful because the two $x$ in \Code{(+ x x)} should be bind to different value. So traditional hygienic resugaring approach use abstract syntax DAG to distinct different \m{x} in the desugared expression. But for our approach based on lazy desugaring, the \m{hygienicadd} sugar does not have to desugar until necessary, so, getting following sequences based on a  rewriting system which renaming the variables during the rewriting.

\begin{Codes}
    (let ((x 2)) (Hygienicadd 1 x)
\OneStep{ (Hygienicadd 1 2)}
\OneStep{ (+ 1 2)}
\OneStep{ 3}
\end{Codes}

The lazy desugaring is also convenient for hygienic resugaring for non-hygienic rewriting. For example, \Code{(let ((x 1)) (+ x (let ((x 2)) (+ x 1))))} may be reduced to \Code{(+ 1 (let ((1 2)) (+ 1 1)))} by a simple core language whose \Code{let} expression does not handle cases like that. But by writing a simple sugar Let,
\[\drule{\Code{(Let~$e_1$~$e_2$~$e_3$)}}{\Code{(let~(($e_1$~$e_2$))~$e_3$)}}\]
and some simple modifies in the reduction of mixed language, we will get the following sequences in our system.
\begin{Codes}
    (Let x 1 (+ x (Let x 2 (+ x 1))))
\OneStep{ (Let x 1 (+ x (+ 2 1)))}
\OneStep{ (Let x 1 (+ x 3))}
\OneStep{ (+ 1 3)}
\OneStep{ 4}
\end{Codes}

In practical application, we think hygiene can be easily processed by rewriting system, so we just use a rewriting system which can rename variable automatically. 

And for the derivation method, there is no rewriting system at all, but the hygiene is handled more concisely. we build a hygienic sugar \m{Hygienicor} based the \m{or} sugar.
\[
\begin{array}{l}
\drule{\Code{(Or $e_1$ $e_2$)}}{\Code{(let (x $e_1$) (if x x $e_2$))}}\\
\drule{\Code{(Hygienicor $e_1$ $e_2$)}}{\Code{(let (x $e_1$) (or $e_2$ x))}}
\end{array}
\]
Though no need to write the sugar like that, something wrong may happen without hygienic rewriting system (\Code{(if x x x)} appears). But by the method introduced in \todo{..}, we can easily get the following rules, which will behave as it should be in resugaring.
\[
\begin{array}{c}
\infer {(\mbox{Hygienicor}~e_1~e_2) \rightarrow (\mbox{Hygienicor}~e_1'~e_2)} {e_1~ \rightarrow~e_1'}
\qquad
\infer {(\mbox{Hygienicor}~v_1~e_2) \rightarrow (\mbox{Hygienicor}~v_1~e_2')} {e_2~ \rightarrow~e_2'}
\\
(\mbox{Hygienicor}~v_1~\#t) \rightarrow \#t
\quad
(\mbox{Hygienicor}~v_1~\#f) \rightarrow v_1\\
\end{array}
\]

Overall, our result shows lazy desugaring is really a good way to handle hygienic sugar in any systems.

\subsubsection{recursive sugar}
\label{sec:recursiveSugar}

Recursive sugar is a kind of syntactic sugars where call itself or each other during the expanding. For example,
\[
\begin{array}{l}
\drule{(\m{Odd}~$e$)~}{\Code{(if (> $e$ 0) (Even (- $e$ 1)) \false)}}\\
\drule{(\m{Even}~$e$)}{\Code{(if (> $e$ 0) (Odd (- $e$ 1)) \true)}}
\end{array}
\]
are common recursive sugars. The traditional resugaring approach can't process syntactic sugar written as this (non pattern-based) easily, because boundary conditions are in the sugar itself.

Take $(Odd~2)$ as an example. The previous work will firstly desugar the expression using the rewriting system. Then the rewriting system will never terminate as following shows.
\begin{Codes}
   (Odd 2)
\DeStep{ (if (> 2 0) (Even (- 2 1) \#f))}
\DeStep{ (if (> (- 2 1) 0) (Odd (- (- 2 1) 1) \#t))}
\DeStep{ (if (> (- (- 2 1) 1) 0) (Even (- (- (- 2 1) 1) 1) \#f))}
\DeStep{ ...}
\end{Codes}


Then the advantage of our approach is embodied. Our lightweight approach doesn't require a whole expanding of sugar expression, which gives the framework chances to judge boundary conditions in sugars themselves, and showing more intermediate sequences. We get the resugaring sequences of the former example using our tool.
\begin{Codes}
    (Odd 2)
\OneStep{ (Even (- 2 1))}
\OneStep{ (Even 1)}
\OneStep{ (Odd (- 1 1))}
\OneStep{ (Odd 0)}
\OneStep{ \#f}
\end{Codes}


We also construct some higher-order syntactic sugars and test them. The higher-order feature is important for constructing practical syntactic sugars. And many higher-order sugars should be constructed by recursive definition. The first sugar is \m{filter}, implemented by pattern matching term rewriting.
\[\Code{(filter $e$ (list $v_1$ $v_2$ ...))}\]
\[
\drule{}
{\Code{(if ($e$ $v_1$) (cons $v_1$ (filter $e$ (list $v_2$ ...)))\ (filter $e$ (list $v_2$ ...)))}}
\]
\[
\drule{\Code{(filter $e$ (list))}}{\Code{(list)}}
\]
and getting the following result.

\begin{Codes}
    (filter (lambda (x) (and (> x 1) (< x 4))) (list 1 2 3 4))
\OneStep{ (filter (lambda (x) (and (> x 1) (< x 4))) (list 2 3 4))}
\OneStep{ (cons 2 (filter (lambda (x) (and (> x 1) (< x 4))) (list 3 4)))}
\OneStep{ (cons 2 (cons 3 (filter (lambda (x) (and (> x 1) (< x 4))) (list 4))))}
\OneStep{ (cons 2 (cons 3 (filter (lambda (x) (and (> x 1) (< x 4))) (list))))}
\OneStep{ (cons 2 (cons 3 (list)))}
\OneStep{ (cons 2 (list 3))}
\OneStep{ (list 2 3)}
\end{Codes}

Here, although the sugar can be processed by traditional resugaring approach, it will be redundant. The reason is that, a filter for a list of length $n$ will match to find possible resugaring $n*(n-1)/2$ times. Thus, lazy desugaring is really important to reduce the resugaring complexity of recursive sugar.

Moreover, just like the \emph{Odd and Even} sugar above, there are some simple rewriting systems which do not allow pattern-based rewriting. Or there are some sugars which need to be expressed by the terms in core language as rewriting conditions. Take the example of another higher-order sugar \m{map} as an example.
\[
\begin{array}{l}
\drule{\Code{(map $e_1$ e2)}}{}\\
\Code{(let ((x e2)) (if (empty x) (list) (cons (e1 (first x)) (map e1 (rest x)))))}
\end{array}
\]

Get following resugaring sequences.
\begin{Codes}
    (map (lambda (x) (+ x 1)) (cons 1 (list 2)))
\OneStep{ (map (lambda (x) (+ x 1)) (list 1 2))}
\OneStep{ (cons 2 (map (lambda (x) (+ 1 x)) (list 2)))}
\OneStep{ (cons 2 (cons 3 (map (lambda (x) (+ 1 x)) (list))))}
\OneStep{ (cons 2 (cons 3 (list)))}
\OneStep{ (cons 2 (list 3))}
\OneStep{ (list 2 3)}
\end{Codes}

Note that the \m{let} term is to limit the subexpression only appears once in RHS. In this example, we can find that the list \Code{(cons 1 (list 2))}, though equal to \Code{(list 1 2)}, is represented by core language's term. So it will be difficult to handle such inline boundary conditions by rewriting system. But our approach is easy to handle cases like this. So our resugaring approach by lazy desugaring is powerful.

\pagestyle{fancy}
\normalsize
\linespread{1.5}\selectfont
\chapter{实现细节与评估}
\addtocontents{los}{\protect\addvspace{10pt}}

%!TEX root = ./main.tex
\label{sec7}\section{Conclusion}

Summarize the paper, explaining what you have shown, what results you have achieved, and what future work is.


\clearpage
\phantomsection
\addcontentsline{toc}{chapter}{参考文献}
\small
\bibliographystyle{gbt7714-numerical}

\bibliography{ref}


\linespread{1}\selectfont
\normalsize
%小四号,中文宋体,英文Time new roman,1倍行距
\chapter*{本科期间的主要工作和成果}

\noindent 本科期间参加的主要科研项目

\noindent 本研基金
\begin{enumerate}
	\item 校长基金. 校长基金(理). 熊英飞. 1年
\end{enumerate}

\noindent 各种科研项目
\begin{enumerate}
	\item TV Backscatter. 课程项目
	\item Static Analysis using Inductive Logic Programming. 本科生科研
	\item DryadSynth Solver. 暑期科研
\end{enumerate}



\addcontentsline{toc}{chapter}{本科期间的主要工作和成果}
\fancypagestyle{plain}
{
	\fancyhf{}
	\fancyhead[RE,RO]{本科期间的主要工作和成果}
	\fancyhead[LE,LO]{北京大学本科生毕业论文}
	\fancyfoot[CO,CE]{~\thepage~}
	\renewcommand{\headrulewidth}{0.7pt}
	\renewcommand{\footrulewidth}{0pt}
}
\fancyhf{}
\fancyhead[RE,RO]{本科期间的主要工作和成果}
\fancyhead[LE,LO]{北京大学本科生毕业论文}
\fancyfoot[CO,CE]{~\thepage~}
\renewcommand{\headrulewidth}{0.7pt}
\renewcommand{\footrulewidth}{0pt}
\clearpage





\linespread{1.5}\selectfont
\normalsize
%正文,小四号,中文宋体,英文Time new roman,1.5倍行距
\chapter*{致谢}
感谢胡振江老师对我毕设工作期间的指导。前期和胡老师的对DSL项目的多次探讨激发了本文的初步想法,并在胡老师的指引下对工作不断完善。胡老师对我研究方法、思考方式、写作等方面的指导都受益匪浅。尽管有时无心学习进度缓慢,胡老师依然对有限的进展予以肯定,让我不好意思继续拖延,最终能够让标志着本科结束的工作有一个令自己满意的结果。

感谢熊英飞老师对我本科生科研期间及之后的指导。熊老师是我正式步入科研的引路人,且对我很多学习习惯的培养、和我对工作生活态度的探讨对我帮助很大。并且毕设是熊老师推荐我和胡老师做的。

感谢17级本科生关智超同学。关智超同学在今年3月加入胡老师和我的项目讨论,并在之后对我论文写作方面提出宝贵建议。特别是本文中“Resugaring”一词的中文翻译---重组糖是关智超同学提出的。

感谢我的舍友们。尽管由于疫情原因,本文工作及写作大部分时间不是在宿舍进行的,但他们三个人努力学习,创造宿舍良好的学习环境,让不爱学习的我羞愧难当,没有完全不学习;并且宿舍的卫生条件、生活气氛等等都十分不错,让我在本科三年半的时间里生活愉快。

感谢女朋友周宇航同学。周宇航同学在我整个本科期间督促我学习、陪我玩、帮助我、支持我。毕设工作的代码和论文经常是在夜间进行,她和她的猫经常熬夜陪我。

感谢父母和其他家人。他们一直在支持我的各种选择,并且在物质上和精神上帮助我。父母对我要求极低,导致我有一个自认为不错的性格,以至于快乐的生活。

感谢本科期间遇见的所有老师和同学及其他人士、动植物和其他事物。这些人和事构成了我本科期间的无数的碎片化回忆。


\addcontentsline{toc}{chapter}{致谢}
\fancypagestyle{plain}
{
	\fancyhf{}
	\fancyhead[RE,RO]{致谢}
	\fancyhead[LE,LO]{北京大学本科生毕业论文}
	\fancyfoot[CO,CE]{~\thepage~}
	\renewcommand{\headrulewidth}{0.7pt}
	\renewcommand{\footrulewidth}{0pt}
}
\fancyhf{}
\fancyhead[RE,RO]{致谢}
\fancyhead[LE,LO]{北京大学本科生毕业论文}
\fancyfoot[CO,CE]{~\thepage~}
\renewcommand{\headrulewidth}{0.7pt}
\renewcommand{\footrulewidth}{0pt}





\end{document}
