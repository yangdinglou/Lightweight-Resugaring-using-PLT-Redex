\pagestyle{fancy}
\normalsize
\linespread{1.5}\selectfont
\label{mark:chapter6}\chapter{总结与展望}
\addtocontents{los}{\protect\addvspace{10pt}}
\section{结论}
我们的重组糖方法相对于现有方法,是一种更轻量级的算法---用单步尝试加上不完全展开的基础思想,解决了现有工作不能处理递归糖、高阶糖的问题,并且让处理语法糖尤其是卫生宏的流程简单而自然。我们基于PLT Redex工具,实现了我们的轻量级重组糖框架,并在其基础上进行对多种特性的语法糖进行测试,并对特殊的应用进行尝试,结果显示我们的重组糖方法确实处理了更多的语法糖特性,且工具使用简单。我们在开头的{\bfseries 本文主要贡献}\ref{mark:contribution}小节总结了本文,此处不再赘述。



\section{未来可能的工作}

\subsection{副作用语法糖}

正如前文\ref{mark:side}所提到的,目前该工作不能很好的处理有副作用的语法糖,尽管本身对带有副作用的语法糖进行重组糖没有什么意义,我们依然希望将副作用在语法糖中的确切难点弄清楚并试图解决,因为这对于下面的DSL解释器工作有相当的影响。

\subsection{DSL解释器}
我们的工作初衷是为了解决如下的问题:

\begin{quote}
	给定内部(通用语言)的解释器(或求值规则),加上外部语言(领域特定语言)的映射关系,用算法自动推导出外部语言(不依赖内部语言)的解释器(或求值规则)。
\end{quote}

在对问题初步思考后,结合学习阅读resugaring系列工作,提出了本文主要工作的想法。而我们的最终目标依然没有变。

广义上讲,本文实现的工具是一个解释器---它将带有语法糖的表达式,在表面语言上一步一步的解释执行。但最终这个解释执行依然是包含算法对内部语言的一些尝试,没有脱离对内部语言的依赖。然而本工作核心算法的部分思想对此目标有借鉴意义,今后的工作可能会朝这方向发展,将该方法抽象到符号层面。

\subsection{多步程序综合}

程序综合研究在今年来发展火热,而目前大多数基于例子的程序综合都是给定一个输入和对应的输出,得到想要的表达式。针对多步执行序列为例子的程序综合目前还没有被深入研究,而该研究是具有意义的。我们的工作和DSL的解释器工作一定程度上就是一个对序列进行综合的问题;如果能够将DSL解释器研究清楚,可能对于多步程序综合的研究具有借鉴意义。

\clearpage
