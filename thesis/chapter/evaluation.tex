\pagestyle{fancy}
\normalsize
\linespread{1.5}\selectfont
\chapter{实现细节与评估}
\addtocontents{los}{\protect\addvspace{10pt}}

\section{实现细节}
在PLT Redex中,对一个语言的定义主要需要三个内容。
\begin{itemize}
	\item 文法
	\item 上下文环境
	\item 规约规则
\end{itemize}

我们在实现中,对所有表达式按如下文法进行子类划分。
\begin{verbatim}
exp ::= coreexp | surfexp | commonexp
\end{verbatim}

与(第三章第1节)的定义有所差别的是,这里的surfexp包括了第三章第1节的Surfexp和OtherSurfexp,commonexp包括了第三章第一节的Commonexp和OtherCommonexp。

对于coreexp,自然是包含了所有内部语言的术语。因为我们的定义是将Exp的形式限制在$(Headid\;Subexp_{1}\ldots)$,通常情况下可以根据Headid判断是否是coreexp。

例:
\begin{verbatim}
coreexp ::= (if exp exp exp)
		|   (let ((x exp) ...) exp)
		|   (first exp)
		|   (rest exp)
		|   ...
\end{verbatim}

对于surfexp,就是$Headid$是表示DSL的标识的表达式。

例:
\begin{verbatim}
surfexp ::= (and exp exp)
		|   (or exp exp)
		|   (map exp exp)
		|   (filter exp exp)
		|   ...
\end{verbatim}

对于commonexp,是为了让我们的重组糖序列中有一些包含在内部语言、但是可以输出的中间过程。例如对于高阶糖

$(Map\;f\;lst)-->(cons\;(f\;(first\;lst))\;(Map\;f\;(rest\;lst)))$

\begin{flushleft}
	我们需要一些中间序列用cons表示,来输出有用的中间过程。我们主要需要的commonexp有所有值、所有基础运算(包括算数运算和列表运算)。
\end{flushleft}

对于上下文规则,对coreLang为了限制每个表达式只有一条规则可以规约,需要仔细限制位置。
而对于surfLang,由于我们不知道哪个子表达式需要先规约,要把每一个位置都设为可规约的洞。
\begin{verbatim}
(E  (v ... E e ...)
	(let ((x v) ... (x E) (x e) ...) e)
	(if E e e)
	(and E e)
	(and e E)
	(or E e)
	(or e E)
	...
	hole)
\end{verbatim}

规约规则没有特殊说明。具体可参见代码。\footnote{\url{http://wikipedia.org}}
\section{评估}

\subsection{普通糖}

\subsection{卫生宏}

\subsection{递归糖}

\subsection{高阶糖}

\section{一些其他讨论}

\subsection{副作用}