\pagestyle{fancy}
\normalsize
\linespread{1.5}\selectfont
\label{mark:chapter4}\chapter{应用及讨论}
\addtocontents{los}{\protect\addvspace{10pt}}



\section{应用}
我们就不同类型(特性)的语法糖分别进行尝试。

\subsection{普通糖}

我们首先就简单的

$(and\;e1\;e2)→(if\;e1\;e2\;\#f)$

$(or\;e1\;e2)→(if\;e1\;\#t\;e2)$
糖进行测试。

因为我们的CoreLang上规定的求值顺序是,对$(if \;e1\;e2\;e3)$的$e1$求到值后就规约,因此我们应该看到的是对$and$和$or$糖,先将第一个值求到值,如何对糖进行破坏。

\begin{center}
	\framebox[35em][l]{
		\parbox[t]{\textwidth}{
			\begin{center}
				(and (and \#t \#f) (or \#f \#t))\\
				↓\\
				(and \#f (or \#f \#t))\\
				↓\\
				\#f
			\end{center}
			
		}
	}
\end{center}

\begin{center}
	\framebox[35em][l]{
		\parbox[t]{\textwidth}{
			\begin{center}
				(and (or \#f \#t) (or \#f \#f))\\
				↓\\
				(and \#t (or \#f \#f))\\
				↓\\
				(or \#f \#f)\\
				↓\\
				\#f
			\end{center}
			
		}
	}
\end{center}


我们可以看到如上的两个例子皆按照我们预想的样子输出了重组糖序列。\\[6pt]

另外尽管我们的重组糖序列不考虑有副作用的语言,但我们依然测试了一下考虑副作用时的糖的正确性。

$(Myor\;e1\;e2)→(let\;((tmp\;e1))\;(if\;tmp\;tmp\;e2))$
其中没有用到任何其他糖,因此我们可以期待它的效果和上面的or糖一样。

\begin{center}
	\framebox[35em][l]{
		\parbox[t]{\textwidth}{
			\begin{center}
				(Myor (Myor \#f \#f) (and \#t \#t))\\
				↓\\
				(Myor \#f (and \#t \#t))\\
				↓\\
				(and \#t \#t)\\
				↓\\
				\#t
			\end{center}
			
		}
	}
\end{center}
结果显示我们的方法确实正确得到了该重组糖的序列。

\subsection{复合糖}

尽管和普通糖没有本质区别,但本节测试的糖在解糖后依然存在语法糖结构。我们构造如下Sg糖为例

$(Sg\;e1\;e2\;e3)→(and\;(or\;e1\;e2)\;(not\;e3))$

\begin{center}
	\framebox[35em][l]{
		\parbox[t]{\textwidth}{
			\begin{center}
				(Sg (and \#t \#f) (not \#f) \#f)\\
				↓\\
				(Sg \#f (not \#f) \#f)\\
				↓\\
				(and (or \#f (not \#f)) (not \#f))\\
				↓\\
				(and (not \#f) (not \#f))\\
				↓\\
				(and \#t (not \#f))\\
				↓\\
				(not \#f)\\
				↓\\
				\#t
			\end{center}
			
		}
	}
\end{center}


通过简单的展开验证我们发现,对于该语法糖表达式我们的方法确实做到了在语法糖需要展开的时候展开。

\label{mark:hygienic}\subsection{卫生宏}

卫生宏定义:展开宏时变量作用域不被改变的宏。

{\bfseries 例:}

对语法糖$(Let\;x\;v\;exp)$→$(Apply\;(\lambda\;(x)\;exp)\;v)$

表达式$(Let\;x\;1\;(Let\;x\;2\;(+\;x\;1)))$中的(+ x 1)将用值为2的x,因为他在内部Let的作用域内。

实现卫生宏的方法有很多,但对于现有的重组糖方法,卫生宏的重组糖\upcite{hygienic}并不是那么容易。而在我们的方法中,实现卫生宏的重组糖很简单。

以表达式$(Let\;x\;1\;(Let\;x\;2\;(+\;x\;1)))$为例。我们会先将它单步展开到
$(Apply\;(\lambda\;(x)\;(Let\;x\;2\;(+\;x\;1)))\;1)$,接着用Apply的规则进行规约,发现将$(Let\;x\;2\;(+\;x\;1)))$中的x置换为1后,表达式并不符合规则,因此这不是一条有效的规约,不能对外层Let糖先展开。因此输出的第一个重组糖中间序列是$(Let\;x\;1\;(+\;2\;1))$

而在实现过程中,我们发现借助PLT Redex的$\#:refers-to$命令,我们可以更简单的实现更漂亮的卫生宏重组糖。

测试结果如下。
\begin{center}
	\framebox[35em][l]{
		\parbox[t]{\textwidth}{
			\begin{center}
				(Let x (+ 1 2) (+ x (Let x (+ 1 4) (+ x 1))))\\
				↓\\
				(Let x 3 (+ x (Let x (+ 1 4) (+ x 1))))\\
				↓\\
				((λ (x) (+ x (Let x (+ 1 4) (+ x 1)))) 3)\\
				↓\\
				(+ 3 (Let x (+ 1 4) (+ x 1)))\\
				↓\\
				(+ 3 (Let x 5 (+ x 1)))\\
				↓\\
				(+ 3 ((λ (x) (+ x 1)) 5))\\
				↓\\
				(+ 3 (+ 5 1))\\
				↓\\
				(+ 3 6)\\
				↓\\
				9
			\end{center}
			
		}
	}
\end{center}

\subsection{递归糖}

递归糖是对于现有重组糖方法很难处理的一种语法糖,指的是多个语法糖相互调用。在本节我们主要用如下例子说明。

$(Odd\;e)$→$(if\;(>\;e\;0)\;(Even (-\;e\;1)\;\#f))$

$(Even\;e)$→$(if\;(>\;e\;0)\;(Odd (-\;e\;1)\;\#t))$

看似不复杂的两条语法糖规则,对于现有方法来说并不是一件容易的事情。具体原因我们将在下一高阶糖和下一节中仔细探讨。测试结果如下:

\begin{center}
	\framebox[35em][l]{
		\parbox[t]{\textwidth}{
			\begin{center}
				(Odd 4)\\
				↓\\
				(Even (- 4 1))\\
				↓\\
				(Even 3)\\
				↓\\
				(Odd (- 3 1))\\
				↓\\
				(Odd 2)\\
				↓\\
				(Even (- 2 1))\\
				↓\\
				(Even 1)\\
				↓\\
				(Odd (- 1 1))\\
				↓\\
				(Odd 0)\\
				↓\\
				\#f
			\end{center}
			
		}
	}
\end{center}

\subsection{高阶糖}

高阶糖本质上和递归糖没有区别,但在实际应用中,对高阶糖的支持是很有意义的。

我们分别用两种不同的语法糖形式实现了$map$和$filter$两个语法糖。
\begin{flushleft}
	$(map\;e\;(list\;v_1\ldots))$→
	
	$(if\;(empty\;(list\;v_1\ldots))\;(list)\;(cons\;(e\;(first\;(list\;v_1\ldots)))\;(map\;e\;(rest\;(list\;v_1\ldots)))))$
\end{flushleft}

\begin{flushleft}
	$(filter\;e\;(list\;v_1\;v_2\ldots))$→
	
	$(if\;(e\;v_1)\;(cons\;v_1\;(filter\;e\;(list\;v_2\ldots)))\;(filter\;e\;(list\;v_2\ldots)))$
	
	$(filter\;e\;(list))$→$(list)$
\end{flushleft}

可以看出,对map糖我们将边界条件用解糖后的if表达式进行了限制;而对filter糖,我们用两个糖的参数区分了不同的条件用来表示边界条件。在保证同名不同参数的糖没有二义性的前提下,我们允许一糖多参,让书写语法糖更加简单。

测试结果如下
\begin{center}
	\framebox[35em][l]{
		\parbox[t]{\textwidth}{
			\begin{center}
				(map (λ (x) (+ 1 x)) (list 1 2 3))\\
				↓\\
				(cons 2 (map (λ (x) (+ 1 x)) (list 2 3)))\\
				↓\\
				(cons 2 (cons 3 (map (λ (x) (+ 1 x)) (list 3))))\\
				↓\\
				(cons 2 (cons 3 (cons 4 (map (λ (x) (+ 1 x)) (list)))))\\
				↓\\
				(cons 2 (cons 3 (cons 4 (list))))\\
				↓\\
				(cons 2 (cons 3 (list 4)))\\
				↓\\
				(cons 2 (list 3 4))\\
				↓\\
				(list 2 3 4)
			\end{center}
			
		}
	}
\end{center}

\begin{center}
	\framebox[35em][l]{
		\parbox[t]{\textwidth}{
			\begin{center}
				(filter (λ (x) (and (> x 3) (< x 6))) (list 1 2 3 4 5 6 7))\\
				↓\\
				(filter (λ (x) (and (> x 3) (< x 6))) (list 2 3 4 5 6 7))\\
				↓\\
				(filter (λ (x) (and (> x 3) (< x 6))) (list 3 4 5 6 7))\\
				↓\\
				(filter (λ (x) (and (> x 3) (< x 6))) (list 4 5 6 7))\\
				↓\\
				(cons 4 (filter (λ (x) (and (> x 3) (< x 6))) (list 5 6 7)))\\
				↓\\
				(cons 4 (cons 5 (filter (λ (x) (and (> x 3) (< x 6))) (list 6 7))))\\
				↓\\
				(cons 4 (cons 5 (filter (λ (x) (and (> x 3) (< x 6))) (list 7))))\\
				↓\\
				(cons 4 (cons 5 (filter (λ (x) (and (> x 3) (< x 6))) (list))))\\
				↓\\
				(cons 4 (cons 5 (list)))\\
				↓\\
				(cons 4 (list 5))\\
				↓\\
				(list 4 5)
			\end{center}
			
		}
	}
\end{center}


我们接下来讨论为什么我们的方法能处理递归语法糖。以

$(map\;(\lambda\;(x)\;(+\;1\;x))\;(list\;1\;2\;3))$为例。如果将该糖完全展开,将变成如下表达式
\begin{flushleft}
	$(if\;(empty\;(list\;1\;2\;3))$
	
	$\qquad(list)$
	
	$\qquad(cons\;((\lambda\;(x)\;(+\;1\;x))\;(first\;(list\;1\;2\;3)))$
	
	$\qquad\qquad(if\;(empty\;(rest\;(list\;1\;2\;3)))$
	
	$\qquad\qquad\qquad(list)$
	
	$\qquad\qquad\qquad(cons\;((\lambda\;(x)\;(+\;1\;x))\;(first\;(rest\;(list\;1\;2\;3))))$ $\ldots$
\end{flushleft}

我们可以看出,由于糖的分支条件写在了语法糖内部,展开过程中不会判断if分支,因此语法糖第二个子表达式$(list~1~2~3)$会在递归中不断变为
\begin{flushleft}
	$(rest~(list~1~2~3))$~$(rest~(rest~(list~1~2~3)))$~$\ldots$
\end{flushleft}
而不会停止。而我们的重组糖方法则不需要将语法糖完全展开,因此不会发生这种事情。实际上,将语法糖展开条件写在语法糖内部是一个很自然的想法,而现有的语法糖基本上并不支持这么做,而更多的需要手动书写边界条件,这对于作为实现领域特定语言的工具的语法糖来说,是不太友好的。

\subsection{其他例子}

\subsubsection{惰性求值}

我们的工具设计之初以call-by-value的λ演算作为内部语言的基础。而实际应用时,惰性求值,也就是call-by-need也是比较常见的。我们试图构造一个惰性求值λ演算的语法糖。

\begin{flushleft}
	$(shell~(\lambda_{N}~x~\ldots~e))$→$(\lambda~(x~\ldots)~(shell~e))$
	
	$(shell~(e_{1}~e_{2}))$→$((shell~e_{1})~(\lambda~(x_{new})~(shell~e_{2})))$
	
	$(shell~(\lambda~(x~\ldots)~(shell~e)))$→$e$
\end{flushleft}

并进行一个样例测试。

\begin{center}
	\framebox[35em][l]{
		\parbox[t]{\textwidth}{
			\begin{center}
				(shell (((λN x y x) xx) ((λN x x) yy)))\\
				↓\\
				((shell ((λN x y x) xx)) (λ ($x_{new}$) (shell ((λN x x) yy))))\\
				↓\\
				(((shell (λN x y x)) (λ ($x_{new1}$) (shell xx))) (λ ($x_{new}$) (shell ((λN x x) yy))))\\
				↓\\
				(((λ (x y) (shell x)) (λ ($x_{new1}$) (shell xx))) (λ ($x_{new}$) (shell ((λN x x) yy))))\\
				↓\\
				((λ (y) (shell (λ ($x_{new1}$) (shell xx)))) (λ ($x_{new}$) (shell ((λN x x) yy))))\\
				↓\\
				(shell (λ ($x_{new1}$) (shell xx)))\\
				↓\\
				xx
			\end{center}
		}
	}
\end{center}

结果发现,我们构造的shell糖和λN糖虽然能够在我们的框架下执行得到求值序列,但并没有进行重组糖。经过简单思考,我们发现原因在于惰性求值本身就是不会对子表达式优先进行计算,因此根据核心算法的$Rule2.2.2$,语法糖展开后不会重组,因此得到的序列虽然满足重组糖的性质,但看起来没有重组。而用shell糖来构造这种call-by-need到call-by-value的转化是一种自认为不错的想法,因此也记录在这里。

\subsubsection{SKI组合子}

SKI组合子演算\upcite{SKI}是无类型λ演算的一种简化版,其运用$S$~$K$~$I$三个组合子的组合使用,可以表达无类型λ演算的所有行为,是一个极小的图灵完备语言。\upcite{mathlogic}我们在我们的工具上尝试构造这三个组合子,并尝试输出SKI组合子表达式的重组糖序列。语法糖定义如下\footnote{此处实现时,为了方便,我们将$S$写做$(S~comb)$(满足Racket的pair形式即可)}:

\begin{flushleft}
	$S$ → $(\lambda~(x_{1}~x_{2}~x_{3})~(x_{1}~x_{2}~(x_{1}~x_{3})))$
	
	$K$ → $(\lambda~(x_{1}~x_{2})~x_{1})$
	
	$I$ → $(\lambda~(x)~x)$
\end{flushleft}

并构造一个将两个参数对置的SKI演算。

\begin{center}
	\framebox[35em][l]{
		\parbox[t]{\textwidth}{
			\begin{center}
				{\linespread{1}\selectfont
				(S (K (S I)) K xx yy)\\
				↓\\
				(S (K (S (λ (x) x))) K xx yy)\\
				↓\\
				(S (K ((λ ($x_{1}$ $x_{2}$ $x_{3}$) ($x_{1}$ $x_{3}$ ($x_{2}$ $x_{3}$))) (λ (x) x))) K xx yy)\\
				↓\\
				(S (K (λ ($x_{2}$ $x_{3}$) ((λ (x) x) $x_{3}$ ($x_{2}$ $x_{3}$)))) K xx yy)\\
				↓\\
				(S ((λ ($x_{1}$ $x_{2}$) $x_{1}$) (λ ($x_{2}$ $x_{3}$) ((λ (x) x) $x_{3}$ ($x_{2}$ $x_{3}$)))) K xx yy)\\
				↓\\
				(S (λ ($x_{2}$) (λ ($x_{2}$ $x_{3}$) ((λ (x) x) $x_{3}$ ($x_{2}$ $x_{3}$)))) K xx yy)\\
				↓\\
				(S (λ ($x_{2}$) (λ ($x_{2}$ $x_{3}$) ((λ (x) x) $x_{3}$ ($x_{2}$ $x_{3}$)))) (λ ($x_{1}$ $x_{2}$) $x_{1}$) xx yy)\\
				↓\\
				((λ ($x_{1}$ $x_{2}$ $x_{3}$) ($x_{1}$ $x_{3}$ ($x_{2}$ $x_{3}$))) (λ ($x_{2}$) (λ ($x_{2}$ $x_{3}$) ((λ (x) x) $x_{3}$ ($x_{2}$ $x_{3}$)))) (λ ($x_{1}$ $x_{2}$) $x_{1}$) xx yy)\\
				↓\\
				((λ ($x_{2}$ $x_{3}$) ((λ ($x_{2}$) (λ ($x_{2}$ $x_{3}$) ((λ (x) x) $x_{3}$ ($x_{2}$ $x_{3}$)))) $x_{3}$ ($x_{2}$ $x_{3}$))) (λ ($x_{1}$ $x_{2}$) $x_{1}$) xx yy)\\
				↓\\
				((λ ($x_{3}$) ((λ ($x_{2}$) (λ ($x_{2}$ $x_{3}$) ((λ (x) x) $x_{3}$ ($x_{2}$ $x_{3}$)))) $x_{3}$ ((λ ($x_{1}$ $x_{2}$) $x_{1}$) $x_{3}$))) xx yy)\\
				↓\\
				(((λ ($x_{2}$) (λ ($x_{2}$ $x_{3}$) ((λ (x) x) $x_{3}$ ($x_{2}$ $x_{3}$)))) xx ((λ ($x_{1}$ $x_{2}$) $x_{1}$) xx)) yy)\\
				↓\\
				(((λ ($x_{2}$) (λ ($x_{2}$ $x_{3}$) ((λ (x) x) $x_{3}$ ($x_{2}$ $x_{3}$)))) xx (λ ($x_{2}$) xx)) yy)\\
				↓\\
				(((λ ($x_{2}$ $x_{3}$) ((λ (x) x) $x_{3}$ ($x_{2}$ $x_{3}$))) (λ ($x_{2}$) xx)) yy)\\
				↓\\
				((λ ($x_{3}$) ((λ (x) x) $x_{3}$ ((λ ($x_{2}$) xx) $x_{3}$))) yy)\\
				↓\\
				((λ (x) x) yy ((λ ($x_{2}$) xx) yy))\\
				↓\\
				((λ (x) x) yy xx)\\
				↓\\
				(yy xx)}
			\end{center}
			
		}
	}
\end{center}

而实际上,对SKI组合子,构建在惰性求值的λ演算是一种更合适的语法糖,于是我们构造了惰性求值的$\lambda_{N}$作为coreLang的一部分,并重新测试该表达式。

\begin{center}
	\framebox[35em][l]{
		\parbox[t]{\textwidth}{
			\begin{center}
				(S (K (S I)) K xx yy)\\
				↓\\
				(((K (S I)) xx (K xx)) yy)\\
				↓\\
				(((S I) (K xx)) yy)\\
				↓\\
				(I yy ((K xx) yy))\\
				↓\\
				(yy ((K xx) yy))\\
				↓\\
				(yy xx)
				
			\end{center}
			
		}
	}
\end{center}

结果显示我们的重组糖工具漂亮的显示了该SKI表达式的序列。

\section{与现有方法对比}
正如前文多次提到的,我们的方法与现有重组糖方法最大的不同是,不会完全将语法糖展开。我们的轻量级重组糖方法先比于现有工作有如下优点。
\begin{itemize}
	\item 轻量级。不将语法糖完全展开,意味着不需要对语法糖表达式展开后的内部语言表达式的多步执行过程进行一次又一次的匹配和替代。因为对于一个相对比较大的程序来说,这样的匹配开销并不小,因此我们的做法---只在需要(不得不)展开语法糖时再将语法糖破坏的方法,理论上很大程度节省了重组糖的时间开销。
	\item 对卫生宏友好。重组糖系列工作第二篇用一个新的数据结构---抽象语法有向无环图,来处理卫生宏的重组糖问题;而正如上一小节所介绍\ref{mark:hygienic},我们的系统处理卫生宏和普通宏的区别只在于借用PLT Redex的重命名;并且即使不用这个功能,算法依然可以有效运行。\footnote{需要对Rule2.2进行微小改动,代码中也有运行实例}这条优点本身也属于一种“轻量级”的表现---算法本身简单以至于拓展性强,而现有工作则对算法进行重新设计。
	\item 更多语法糖特性(尤其递归语法糖)。正如前一节所介绍的,我们的方法处理递归糖和高阶糖的能力,是现有方法很难达到的。原因很简单:递归糖需要处理终止条件,而本身语法糖处理终止条件就不是一件容易的事情。而得益于规约语义的强大能力,加上算法本身的精心设计,让处理递归这件事成为可能,进而让高阶语法糖成为可能。我们可以发现,高阶函数作为函数式语言的一个非常重要的特性,在近年来被多种其他编程范式的语言所支持。因此支持递归的语法糖是本工作十分重要的优势。
	\item 更简单的书写语法糖。相比于resugaring系列工作\upcite{resugaring},我们的方法得益于规约语义的友好形式,对语法糖的书写并不必须是模式匹配为基础的。只要可以保证对同一个语法糖没有出现多种解糖方式,我们可以对语法糖的各种条件进行分类处理。比如说实现一个斐波那契数列语法糖,我们可以设计如下解糖规则。(对应写在规约规则中)
	
	$(f\;0)$→$1$
	
	$(f\;1)$→$1$
	
	$(f\;x)$→$(+\;(f\;(-\;x\;1))\;(f\;(-\;x\;2)))$\qquad side-condition (x>1)
	
	这种语法糖写法相对于简单的pattern based语法糖,更加友好。
\end{itemize}

在前文有提到,我们需要限制语法糖和内部语言的结构化\ref{mark:struct}。该限制最大的影响就是语言不能有副作用,我们将在下一节对该问题进行讨论。
\section{一些其他讨论}



\label{mark:side}\subsection{副作用}

表达式不能用对外副作用一定程度我们我们方法的便利性。在一定解决范围内,我可以基于let(变量绑定)解决部分需求。

不能有副作用的主要原因,是因为:一旦语法糖的子表达式带有副作用,而当表达式并没有执行到带副作用语句前是可以重组糖的,执行后便不能重组。而对于嵌套表达式来说,判断是否执行过有副作用的语句是一件很困难的事情。可能的解决方案是设计一个检测副作用的算法来增强我们的系统。

如下语法糖

$(Sg~exp_{1}~exp_{2})→(begin~(if~exp_{1}~(set!~x~0)~(set!~x~exp_{2}))~x)$

如果$exp_{1}$不包含副作用,那么显然可以不展开Sg糖,先对$exp_{1}$进行规约,再根据$exp_{1}$的值决定之后的重组糖序列。

但是一旦$exp_{1}$表达式包含了副作用,我们就不得不将Sg糖展开。此时该糖就不会有重组糖序列,只会得到最后x的值。

另一方面,我们可以发现的是,对于有副作用的语法糖来说,重组糖的是没有很大作用的。因为当一个语法糖表达式内部副作用语句被执行,该糖就不能重组了。

另一种可能的解决方法,是将副作用封装到内部语言的两个函数中,比如表达式$exp_{1}$中将x改为x',y改为y',则$f(exp_{1})$为$(list~(cons~x~x')~(cons~y~y'))$,如果别的表达式需要得到$exp_{1}$的副作用,就将$f(exp_{1})$保存到一个let绑定中,供别的表达式使用。

\subsection{其他相关工作}
