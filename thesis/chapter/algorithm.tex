\pagestyle{fancy}
\normalsize
\linespread{1.5}\selectfont
\chapter{算法定义及正确性证明}
\addtocontents{los}{\protect\addvspace{10pt}}

\section{对语言的规定}
对于给定求值规则的内部语言CoreLang,和在CoreLang基础上用语法糖构造的SurfLang;对于任意SurfLang的表达式,得到其在SurfLang上的求值序列且该求值序列满足三个性质:

首先,我们需要将语言限定在基于树形表达式的结构化语言。

树形表达式:此处我们使用类似Lisp的S表达式的递归树,基础定义如下。

Exp::= (Headid Exp*)

|  Value

|  Variable

结构化:对于每个表达式中的子表达式,其规约规则只和子表达式本身有关。

此外:我们对CoreLang和SurfLang进行一些简单的限制。

对CoreLang,任意一个Exp(每个子表达式都不包含surfexp)最多只能有一条规约路径。这一点约束并不过分,为了保证每个程序只能有一条执行路径。

对SurfLang,任意一个语法糖只能有一个CoreLang的表达式与之对应。

在PLT Redex中,我们将CoreLang和SurfLang视为同一个语言。则当我们定义了一个语言内部各种规约规则后,对于任意Exp都有其对应的一条或多条规约规则。根据对CoreLang的约定,有多条规约规则的表达式必然存在SurfLang的表达式。

为了区分CoreLang的语言和SurfLang的语言,我们将表达式的文法定义为如下

Exp::= Coreexp

|  Surfexp

|  Commonexp

|  Otherexp

Coreexp::= (CoreHead Exp*)

Surfexp::= (SurfHead (Surfexp|Commonexp)*)

Commonexp ::= (CommonHead (Surfexp|Commonexp)*)

|   value | variable

OtherSurfexp ::= (SurfHead Exp* Coreexp Exp*)

OtherCommonexp ::= (CommonHead Exp* Coreexp Exp*)

可以看出,在我们的重组糖方法中,可以输出的表达式是Surfexp和Commonexp,即不存在任何子表达式中存在Coreexp。


\section{算法描述}

核心算法f定义如下:

对于每个表达式Exp,我们将对它所有规约规则中选择一条符合resugaring的仿真性规则的规约,且尽可能不破坏任何语法糖。

1.	如果Exp是Coreexp或Commonexp或OtherCommonexp,则其规约规则或是将表达式规约到另一个表达式,此时只有一条规则;

2.	或是其规约不满足导致内部子表达式需要规约,此时因为CoreLang的定序性,只会有一个子表达式被规约(且此表达式为Surfexp),此时对该子表达式Subexp递归调用核心算法f得到Subexp’,则将此Exp中子表达式Subexp规约为Subexp’的规则就是我们需要的规则。

3.	如果Exp是Surfexp或OtherSurfexp,如果内部子表达式无可规约的,则必然会展开该语法糖;

4.	如果存在可规约的子表达式对于每个子表达式,如果可规约,则根据我们的设定,存在一条关于此子表达式的规约规则。因此每个子表达式都可能被规约的前提下,我们需要对Surfexp或OtherSurfexp的语法糖进行展开为Exp’(此展开只有一种规约规则对应),之后对Exp’调用核心算法f,检测内部哪个子表达式在f调用过程中被规约,则此子表达式需要在Exp处首先被规约,对应对该子表达式进行规约的规约规则即为所需要的路径。
