\pagestyle{fancy}
\normalsize
\linespread{1.5}\selectfont
\chapter{背景知识}
\addtocontents{los}{\protect\addvspace{10pt}}

\section{重组糖形式化定义}
对于给定求值规则的内部语言CoreLang,和在CoreLang基础上用语法糖构造的SurfLang;对于任意SurfLang的表达式,得到其在SurfLang上的求值序列且该求值序列满足三个性质:

1.	仿真性:求值序列需要和在CoreLang上的求值顺序相同,即存在CoreLang上的求值序列中的部分中间过程与该序列中的元素对应。该性质是重组糖有意义的前提。

2.	抽象性:求值序列中只存在SurfLang中存在的术语,没有引入CoreLang中的术语。该性质是重组糖研究的目的。

3.	覆盖性:在求值规则中没有跳过一些中间过程。该性质不是正确性的必要条件,却是在应用中极其重要的;加上前两条性质满足的正确性,构成了重组糖的全部重要性质。

\section{完全β规约、归约语义和PLT Redex}
与β规约的概念不同,完全β规约是一种基于β规约的求值顺序规则。对于一个嵌套的lambda表达式,每个表达式都可能进行β规约,而常规的call-by-name和call-by-value都是对规约顺序进行了约定,而完全β规约就是一种不定序的求值规则,每个可β规约的位置都有可能进行规约,因此得到的规约路径不是一条,而是一个图,且这个图的起点和终点只有一个。如图所示的例子就是一个完全β规约的求值图

(图)

可以看出,在完全beta规则中,对任何位置的可beta规约的lambda表达式,都可以进行规约。因此,与call-by-name和call-by-value不同的是,这种求值规则是不定序的。

本文工作的最初思想就是基于完全beta规约。

规约语义:我们需要在求值规则中约定每一个表达式的规约规则,。。。PLT Redex是基于此语义的语义工程工具。
