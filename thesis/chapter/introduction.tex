\pagestyle{fancy}
\normalsize
\linespread{1.5}\selectfont
\chapter{绪论}
\addtocontents{los}{\protect\addvspace{10pt}}

\section{研究背景}
领域特定语言\cite{dsl}的研究及应用日益广泛,其中不乏正则表达式\cite{regexp}、SQL\cite{sql}、XML\cite{xml}等应用场合广泛的领域特定语言。而随着计算机科学技术的发展,DSL的应用场景步入了日常生活中,如IFTTT应用、IOS快捷指令。而在一般场合中,语言设计者是计算机科学家,语言使用者却是领域内的专家。因此,当领域特定语言的使用出现一些问题时,领域内的专家(使用者)需要得到领域特定的信息来进行处理。

语法糖是近些年一种流行的领域特定语言实现方法,其方法源于Peter Landin对λ演算应用计算的替换\cite{landin}。函数式语言方面,经过Lisp的宏系统(Macro),Scheme语言的发展,再到Racket\cite{racket}语言的扩展\cite{frommacro},如今形成了比较完善的宏系统来用语法糖构造DSL。而像如Scala\cite{scala}这种结合了面向对象和函数式语言的程序设计语言也对语法糖的构造很重视。语法糖相关的研究\cite{sugarj}也十分火热。

语法糖构造DSL的最主要优点就是简单高效,语言设计者只需要写一个简单的映射,就可以构造DSL。然而,语法糖的一项缺陷,导致其应用领域大多局限于计算机科学内部。我们先来看下面这个例子。

	我们拟构造一个自动化饭店的DSL(如图\ref{fig:restaurant})为例。
	
	\begin{figure}[H]
		\centering
		\includegraphics[width=12cm]{images/chapter1/restaurant.png}
		\caption{自动化饭店DSL}
		\label{fig:restaurant}
	\end{figure}
其中箭头表示语法糖展开的形式。
在这个DSL内,执行

\begin{flushleft}
	打卤面(皮皮虾卤(大,~ 咸,~ 胡萝卜),~ 煮面(毛细),~ 2)
\end{flushleft}

因为其过程中会有很多中间过程,我们期望得到的执行过程是这样的\ref{fig:expect}。
\begin{figure}[H]
	\begin{center}
		\framebox[35em][l]{
			\parbox[t]{\textwidth}{
				\begin{center}
					打卤面(皮皮虾卤(大,~ 咸,~ 胡萝卜),~ 煮面(毛细),~ 2)\\
					↓\\
					打卤面(皮皮虾卤(大,~ 咸,~ 胡萝卜丁),~ 煮面(毛细),~ 2)\\
					↓\\
					打卤面(大份咸皮皮虾卤(胡萝卜丁),~ 煮面(毛细),~ 2)\\
					↓\\
					打卤面(大份咸皮皮虾卤(胡萝卜丁),~ 毛细抻面,~ 2)\\
					↓\\
					两份皮皮虾打卤面(大份、咸、胡萝卜丁、毛细)
				\end{center}
				
			}
		}
	\end{center}
	\caption{期待的执行序列}
	\label{fig:expect}
\end{figure}


而实际上,这个DSL的执行序列是这样的\ref{fig:fact}。(先将语法糖展开成难懂的内部语言)
\begin{figure}[H]
	\begin{center}
		\framebox[35em][l]{
			\parbox[t]{\textwidth}{
				\begin{center}
					打卤面(皮皮虾卤(大,~ 咸,~ 胡萝卜),~ 煮面(毛细),~ 2)\\
					↓\\
					(begin~for$\ldots$~let($\ldots$)~$\ldots$)\\
					↓\\
					$\ldots$\\
					↓\\
					两份皮皮虾打卤面(大份、咸、胡萝卜丁、毛细)
				\end{center}
				
			}
		}
	\end{center}
	\caption{实际执行序列}
	\label{fig:fact}
\end{figure}


将这个例子抽象到简单的例子
	$and(or(\#f,~\#t),~and(\#t,~\#f))$

(其中and、or是语法糖,展开成if的表达式)有如下执行序列\ref{fig:desugar}

\begin{figure}[ht]
	\framebox[35em][l]{
		\parbox[t]{\textwidth}{
			\begin{center}  
				and(or(\#f,~\#t),~and(\#t,~\#f))\\
				↓\\
				if(if(\#f, \#t, \#t), \#t, if(\#f, \#t, \#f))\\
				↓\\
				if(\#t , if(\#f, \#t, \#f), \#f)\\
				↓\\
				if(\#f, \#t, \#f)\\
				↓\\
				\#f
			\end{center}  
		}
	}  
\caption{语法糖表达式执行示例}
\label{fig:desugar}
\end{figure}



我们可以看到,在执行过程中,语法糖表达式被展开成通用语言表达式,在通用语言中继续执行,得到最终结果。但实际应用到特定领域时,我们很明显不希望得到这样复杂的执行过程,特别是对于对计算机内部语言不熟悉的领域专家来说,这种执行过程是没有意义的。


\section{语法糖的困境}
\label{mark:onedirect}我们将上述问题总结为语法糖解糖的单向性,对上面的例子\ref{fig:desugar}的第三行

\begin{flushleft}
	$\mbox{if}(\#t , \mbox{if}(\#f, \#t, \#f), \#f)$
\end{flushleft}
进行观察,我们可以看出,其存在等价的领域特定语言表示$\mbox{and}(\#t, \mbox{and}(\#t, \#f))$,这正是初始表达式的第一个子表达式规约后的结果。如果语法糖的解糖有一个逆过程(重组糖),我们就可以找到语法糖表达式其对应的在语法糖层面的执行序列。对上面的and、or糖例子,我们希望得到的重组糖序列如下图\ref{fig:resugar}。

\begin{figure}[ht]
	\framebox[35em][l]{
		\parbox[t]{\textwidth}{
			\begin{center}  
				and(or(\#f,~\#t),~and(\#t,~\#f))\\
				↓\\
				and(\#t,~and(\#t,~\#f))\\
				↓\\
				and(\#t,~\#f)\\
				↓\\
				\#f
			\end{center}  
		}
	}  
	\caption{语法糖表达式重组糖序列实例}
	\label{fig:resugar}
\end{figure}

我们将在第二章对重组糖问题进行形式化定义。在后文中,将领域特定语言视为外部语言,通用语言视为内部语言。

\section{现有工作及存在的问题}
我们的方法主要借鉴和对比Resugaring系列\cite{resugaring}\cite{hygienic}\cite{resugaringscpoe}\cite{resugaringtype}的一些工作,其中前两篇和本文的工作紧密相关,我们将在这里简单讲述一下这两篇工作的方法及缺陷,并在相关工作部分\ref{mark:relative}进一步讨论。

第一篇工作\upcite{resugaring}提出了重组糖问题的概念,并介绍了一个解决重组糖的方法,其基本思想是将内部语言的求值序列每一步加上标签,进行搜索(匹配和替代),试图得到其对应的在外部语言的表示。这也是重组糖名字的由来---将解糖的语法糖表达式重新组成到语法糖。

其算法希望具有如下三个性质(详见\ref{mark:three}):

仿真性/抽象性/覆盖性(没有被证明)
\\[12pt]

第二篇工作\upcite{hygienic}在第一篇的基础上,新增了三个优点:

\begin{itemize}
	\item 解决卫生宏的重组糖
	\item 拓展语法糖规则
	\item 覆盖性得到形式化证明
\end{itemize}

然而该工作仍然存在一些问题:
\begin{itemize}
	\item 语法糖规则依然不够丰富
	\item 算法定义繁琐,通用性差
\end{itemize}

\section{基本思路}

本文基于语义工程工具PLT Redex\cite{SEwPR},设计了一个全新的语法糖——重组糖框架。其想法源于完全β规约的完备性,如下图\ref{fig:base}。

\begin{figure}[ht]
	\centering
	\includegraphics[width=12cm]{images/chapter1/example.png}
	\caption{基本思路}
	\label{fig:base}
\end{figure}

该例子中,对一个结构化(类似S表达式,定义在第二章)的语言,定义其基础语义和规约规则。其中and和or是语法糖,语法糖展开内部语言的if表达式。在没有限制其上下文规则(求值顺序)的前提下,生成了其所以规约路径的流程图。我们可以看出,在该图中,既包含了将语法糖展开的规约,也包含了我们期待的$\mbox{and}(\#t,~\mbox{and}(\#t,~\#f))$等等这些中间求值路径。因此我们希望使用基于规约语义\cite{reduction}的PLT Redex,实现一个轻量级的重组糖,在这个完全图中提取出我们需要的重组糖序列。

在实现过程中,我们发现生成中国完全图并不是必须的,而我们可以在从最初的表达式开始{\bfseries 每次进行单步规约,在一条或多条规约规则中选择我们需要的那条规约规则 }(是本工作的核心算法),且该规则的多次执行保证重组糖的三个基本重要性质。则在这个核心算法迭代执行过程中,会留下一个对应的求值序列,在其中提取出符合输出规则的中间序列,则此序列即为重组糖的输出。


\label{mark:contribution}\section{本文主要贡献}
\begin{flushleft}
	1.	我们针对现有重组糖的方法法进行改进,得到新的轻量级重组糖方法。
\end{flushleft}

\begin{itemize}
	\item 我们的方法不需要将所有语法糖都展开就可以得到重组糖序列,而现有方法需要展开后在内部语言执行,并基于match和substitute对可重组的语法糖进行搜索。
	
\end{itemize}

\begin{flushleft}
	2.	我们基于轻量级重组糖算法,用PLT Redex实现了一套工具。
\end{flushleft}
\begin{flushleft}
	3.	基于我们实现的工具,测试得到我们的方法相对于现有重组糖方法,支持更多语法糖特性。
\end{flushleft}

\begin{itemize}
	\item 对递归糖,我们的方法可以很简单的处理。而现有方法只能用letrec处理一下递归绑定。
	\item 对高阶糖,我们的方法也可以很容易处理。而现有方法不能处理。
	\item 对卫生宏,我们的方法处理卫生宏很简单而自然,而现有工作处理卫生宏需要引入新的数据结构以及很多其他处理。
\end{itemize}

\section{全文结构}

我们将在第二章讲述本文工作的一些背景知识及思考路线;第三章讲述工作的算法定义及正确性证明;第四章对利用PLT Redex的轻量级重组糖工具进行实现上的简单介绍;第五章讲述一些轻量级重组糖的应用,对一些语法糖例子进行讨论评估;第六章总结我们的工作,并对一些未来可能的方向进行简单探讨与展望。
