%!TEX root = ./main.tex

\section{Resugaring by Lazy Desugaring}
\label{sec:algo}

In this section, we present our new approach to resugaring.  We will show that any expression in the mixed language can be evaluated in such a smart way that producing all necessary resugaring sequences. We will discuss the properties of our approach to show the correctness and the practicality.




\subsection{Resugaring Algorithm}

Our resugaring algorithm works on the mixed language, based on the evaluation rules of the core language and the desugaring rules for defining the surface language. The process of  getting the resugaring sequence contains two separate parts.

\begin{enumerate}
\item Calculating the context rules of syntactic sugars.
\item Filtering \m{DisplayableExp} during the execution of the mixed language.
\end{enumerate}



\subsubsection{Context Rule Derivation}
\begin{algorithm}
	\caption{\m{calcontext}}
	\label{alg:f}     % 给算法一个标签,以便其它地方引用该算法
	\begin{algorithmic}[1]       % 数字 "1" 表示为算法显示行号的时候,每几行显示一个行号,如:"1" 表示每行都显示行号,"2" 表示每两行显示一个行号,也是为了方便其它地方的引用
		\REQUIRE ~~\\      % 算法的输入参数说明部分
		\Code{currentLHS = (SurfHead~$t_1$~$t_2$~$\ldots$~$t_n$)}\hfill \\
		\note{//where $t_i$ is $e$(expression) or $v$(value).}\\
		\Code{currentContext = (Head~$\ldots~e_1'~e_2'~\ldots~e_m'$)} \\
		\note{//where $e_i'$ can be at any depth of sub-expressions.}\\
		\Code{currentIncal = $\{\ldots\}$} \note{//list of contexts in calculation.}
		\ENSURE ~~\\     % 算法的输出说明
		\Code{ListofRule, tmpLHS}
		\STATE     \Code{Let ListofRule = \{\}, tmpLHS = currentLHS,\hfill\\ \qquad InCal = append(currentIncal, SurfHead)}
		\IF {$\not \exists~\text{contexts~rules~of}~\m{Head}$}
			\IF {$\exists$ \m{Head} in \m{InCal}}
				\RETURN error
			\ELSE
				\STATE \Code{ListofRule = append(ListofRule,}
				\STATE \qquad\Code{calcontext(\m{Head}.LHS,\m{Head}.RHS,InCal))}
			\ENDIF
		\ENDIF
		\STATE \Code{Let OrderList = $\{e_i',~e_j',~\ldots\}$} \note{//RHS's computational order got by context rules}
		\FOR {\Code{subExp} in \m{OrderList}}
			\IF {$\exists i, s.t. e_i=\Code{subExp}$}
			\STATE \note{//$e_i$ means the i-th of $t_i$, and $t_i$ is not a value}
				\STATE \Code{ListofRule= append(ListofRule,~tmpLHS$[[\bigcdot]/e_i]$)}
				\STATE \Code{tmpLHS = tmpLHS$[v_i/e_i]$}
			\ELSE
				\STATE \Code{Let recRule, recLHS = calcontext(tmpLHS,~subExp,~Incal)}
				\STATE \Code{tmpLHS = recLHS}
				\STATE \Code{ListofRule = append(ListofRule, recRule)}
				\STATE {\bfseries break}\note{//means the RHS has to be destroyed.}
			\ENDIF
		\ENDFOR
		\RETURN \Code{ListofRule, tmpLHS}

	\end{algorithmic}
\end{algorithm}
Given a sugar \m{SurfHead} defined by
\[
\drule{(\m{SurfHead}~e_1~e_2~\ldots~e_n)}{(\m{Head}~\ldots~e_1'~e_2'~\ldots~e_m')}
\]
the context rule derivation is to infer which and in which order $e_i$ should be computed before this desugaring rule is applied. This information can be derived by analyzing computational order of each sub-expression in RHS of the desugaring rule. The reason we can derive the context rules of syntactic sugar is that, for any sugar's RHS, a reduction will either reduce on the $e_i$ of its $LHS$, or reduction on other component of RHS to destroy the RHS's form (which means the sugar has to be expanded). So we can trace the order before the destruction. (We will give the formal definition of RHS's destruction in next section.) The algorithm for this derivation, \texttt{calcontext}, is described as Algorithm \ref{alg:f}. Running
\[
 \Code{calcontext(\m{SurfHead}.LHS,~\m{SurfHead}.RHS, \{\})}
\]
will yield the context rules for \m{SurfHead}.

Before explaining the algorithm in detail, we use a simple example to illustrate the idea.
Consider a sugar \m{Sg0} defined by the following desugaring rule:
\[
\drule{\m{(Sg0}~e_1~e_2~e_3~e_4\m{)}}{\m{(+}~e_1~\m{(if}~e_2~e_3~e_4\m{))}}
\]
where we assume that $+$ computes its first argument and then its second argument before performing reduction, i.e., the context rules for $+$ are  $(+~[\bigcdot]~e_2)$ and $(+~v_1~[\bigcdot])$. Now from $\m{(+}~e_1~\m{(if}~e_2~e_3~e_4\m{))}$, we can infer that $e_1$ should be computed first from the context rule of $+$, and then $e_2$ is computed from the context rule of the inner $\m{if}$, and finally it stops because no context rule is applicable for any remaining sub-expression. Therefore, we get the following context rules for \m{Sg0} as
$
(\m{Sg0}~[\bigcdot]~e_2~e_3~e_4),~(\m{Sg0}~v_1~[\bigcdot]~e_3~e_4).
$

Return to the algorithm. The most important thing is when the algorithm should be stopped. If \m{Head} is a \m{CoreHead}, for each context rule of the \m{Head} in order, we should just recursively make context rules for each hole, until a whole sub-expression is iterated. For instance of \m{Sg1}:
\[
\drule{\m{(Sg1}~e_1~e_2~e_3~e_4\m{)}}{\m{(+}~e_1~\m{(+}~e_2~\m{(+}~e_3~e_4\m{)))}}
\]
the sugar does not have to be expanded until all four sub-expressions are reduced to value, as demonstrated in the following execution trace.

\begin{footnotesize}
\[
\begin{array}{l}
\mbox{$OrderList$ = $\{e_1,~(+~e_2~(+~e_3~e_4))\}$} \\
\quad \mbox{$\Rightarrow$ \Code{(Sg1 $[\bigcdot]~e_2~e_3~e_4$)}}\\
\mbox{$OrderList$ = $\{e_2,~(+~e_3~e_4)\}$}\\
\quad \mbox{$\Rightarrow$ \Code{(Sg1 $v_1~[\bigcdot]~e_3~e_4$)}}\\
\mbox{$OrderList$ = $\{e_3,~e_4\}$} \\
\quad \mbox{$\Rightarrow$ \Code{(Sg1 $v_1~v_2~[\bigcdot]~e_4$),(Sg1 $v_1~v_2~v_3~[\bigcdot]$)}}
\end{array}
\]
\end{footnotesize}
But for instance of \m{Sg2}:
\[
\drule{(\m{Sg2}~e_1~e_2~e_3~e_4)}{\m{(+~(+~(+}~e_1~e_2\m{)}~e_3\m{)}~e_4\m{)}}
\]
once the sub-expressions $e_1$ and $e_2$ are reduced to value, the sugar has to be expanded, because if being desugared to the core language, the sub-expression \Code{(+ $v_1~v_2$)} will be reduced, then the sugar form of RHS is destroyed. 
\begin{footnotesize}
	

	\[
\begin{array}{l}
\mbox{$OrderList$ = $\{(+~(+~e_1~e_2)~e_3),~e_4\}$}\\
\quad \Rightarrow \mbox{nothing}\\
\mbox{$OrderList$ = $\{(+~e_1~e_2),~e_3\}$}\\
\quad \Rightarrow \mbox{nothing}\\
\mbox{$OrderList$ = $\{e_1,~e_2\}$}\\
\quad \Rightarrow \mbox{\Code{(Sg2 $[\bigcdot]~e_2~e_3~e_4$), (Sg2 $v_1~[\bigcdot]~e_3~e_4$)})}
\end{array}
\]
\end{footnotesize}


If the \m{Head} is a \m{SurfHead} with its context rules calculated, then we regard it as \m{CoreHead}. If it has no context rule, we will try calculating its context rules first. However, if an infinite recursive process happens, it means that the original recursive sugars are of ill-form, such as the following:
\[
\begin{array}{l}
\drule{(\m{Odd}~e)}{(\m{Even}~(-~e~1))}\\
\drule{(\m{Even}~e)}{(\m{Odd}~(-~e~1))}
\end{array}
\]

After calculating all context rules of a surface language, we can mix the context rules with core language's. And we can modify the desugaring rules of sugar expression then generate the mixed language.

%\todo{Add explanantion of the above rule.}

\subsubsection{Filtering and Main Algorithm}

As the second part of the whole process, our resugaring algorithm can be defined based on evaluation rules of the mixed language. Let $\redm{}{}$ be one-step reduction in the mixed language.

\[
\begin{array}{llll}
\m{resugar} (e) &=& \key{if}~\m{isNormal}(e)~\key{then}~return\\
              & & \key{else}~\\
							& & \qquad \key{let}~\redm{e}{e'}~\key{in}\\
							& & \qquad \key{if}~e' \in~\m{DisplayableExp} \\
							& & \qquad \key{then}~\m{output}(e'),~\m{resugar}(e')\\
							& & \qquad \key{else}~\m{resugar}(e')
\end{array}
\]
During the resugaring, we just apply the reduction ($\redm{}{}$) on the input program step by step until no reduction can be applied (\m{isNormal}, \m{value} in our setting), while outputting the intermediate expressions that belong to \m{DisplayableExp}.

\subsection{Resugaring Examples}
We will present some resugaring examples in details by some sugars defined in Section \ref{mark:suflang}, to demonstrate the expressive power of our approach.

\subsubsection{Simple Sugars}
\label{mark:simple}

We construct several simple sugars by our approach. Take the SKI combinator as an example, for the program \Code{(S (K (S I)) K xx yy)}, we get the resugaring sequences as follows.
\[
	\begin{array}{l}
		\quad\;\;\Code{ (S (K (S I)) K xx yy)}\\
	\OneStep{ \Code{ (((K (S I)) xx (K xx)) yy)}}\\
	\OneStep{ \Code{ (((S I) (K xx)) yy)}}\\
	\OneStep{ \Code{ (I yy ((K xx) yy))}}\\
	\OneStep{ \Code{ (yy ((K xx) yy))}}\\
	\OneStep{ \Code{ (yy xx)}}\\
	\end{array}
\]
	% \begin{flushleft}
	% {\scriptsize
	% % \begin{Codes}
	% 	\quad\;\;\;\Code{ (S (K (S I)) K xx yy)}\\
	% \OneStep{ \Code{ (((K (S I)) xx (K xx)) yy)}}\\
	% \OneStep{ \Code{ (((S I) (K xx)) yy)}}\\
	% \OneStep{ \Code{ (I yy ((K xx) yy))}}\\
	% \OneStep{ \Code{ (yy ((K xx) yy))}}\\
	% \OneStep{ \Code{ (yy xx)}}\\
	% % \end{Codes}
	% }
	% \end{flushleft}

	


Here the sugars contain no sub-expression (so no context rules), then the sugar should just desugar to the core expression by the context rules in Fig. \ref{fig:core}.
We obtain the resugaring sequences as expected without any reverse desugaring.

Another interesting point we get from this example is, we can use the call-by-need lambda calculus to force an expansion of sugar. Considering the sugar \m{S} in the example above, we may write it by another form with \Code{(S $e_1$ $e_2$ $e_3$)} as its LHS, then the context rules can be calculated, and the sugar may not be expanded. By writing a sugar with call-by-need lambda calculus, such as the following one,
\[
\drule{\Code{ForceAnd}}{\Code{($\lambda_N$ ($\m{x}_1$ $\m{x}_2$) (if $\m{x}_1$ $\m{x}_2$ \false))}}
\]
given any program with \Code{(ForceAnd $e_1$ $e_2$)} as its inner expression, when the reduction should happen at the sub-expression \Code{(ForceAnd $e_1$ $e_2$)}, we will get the evaluation sequences as follows,
\[
	\begin{array}{l}
		\quad\;\;\Code{ ($\ldots$ (ForceAnd $e_1$ $e_2$) $\ldots$)}\\
	\OneStep{ \Code{ ($\ldots$ (($\lambda_N$ ($\m{x}_1$ $\m{x}_2$) (if $\m{x}_1$ $\m{x}_2$ \false)) $e_1$ $e_2$) $\ldots$)}}\\
	\OneStep{ \Code{ ($\ldots$ (if $e_1$ $e_2$ \false) $\ldots$)}}\\
	\OneStep{ \Code{ $\ldots$}}\\
	\end{array}
\]
\subsubsection{Sugars with Let-Binding}
\label{mark:hygienic}

As an important property for sugar or macro system, we used to think it is necessary to achieve hygiene by combining the approach with an existing hygienic desugaring system. But during the experiment, we find it naturally solve the hygienic problem with the original desugaring system in our language setting. 

A typical hygienic problem can be introduced by the sugar \m{HygienicAdd} (see Section \ref{mark:suflang}). For the program \Code{(let (x 2) (HygienicAdd 1 x))}, the existing approach uses an abstract syntax DAG to distinct different x in the desugared expression \Code{(let (x 2) (let x 1 (+ x x)))}. But for our approach based on lazy desugaring, the \m{HygienicAdd} sugar does not have to expand until necessary, thus, getting resugaring sequences as follows based on a non-hygienic transformer system.
\[
	\begin{array}{l}
		\quad\;\;\Code{ (let x 2 (HygienicAdd 1 x))}\\
		\OneStep{ \Code{ (HygienicAdd 1 2)}}\\
		\OneStep{ \Code{ (+ 1 2)}}\\
		\OneStep{ \Code{ 3}}
	\end{array}
\]

 To discuss more cases of hygienic problem, we should find when the problem happens. In our approach, the sugar can contain some bindings, written by the core language's \m{let}. The hygienic problem only happens when binders of an expanded sugar conflict with other binders. We classify them into following two cases. Any hygienic problems are composite by the two cases.

The first one is that, a sugar expression exists in binding's evaluation context. For the sugar \m{Or1} with following rule,
\[\drule{\Code{(Or1~$e_1$~$e_2$)}}{\Code{(let (t $e_1$) (if t t $e_2$))}}\]
The program \Code{(let (t \#t) (Or1 \#f t))} is of the case. But because of the context rule of \m{let}, the sugar \m{Or1} will not be expanded before the \m{t} is substituted. So the program reduces to \Code{(Or1 \#f \#t)} first, so avoiding the hygienic problem. Because the bound variables in sugar expressions are only introduced by let-binding, all of them can "delay" the expansion of the syntactic sugar.

The second one is that, a sugar expression which introduced binding by the sugar expansion contains bindings in its sub-expression. For the sugar \m{Subst} with following rule,
\[
\drule{\Code{(Subst $e_1$ $e_2$ $e_3$)}}{\Code{(let ($e_2$ $e_3$) $e_1$)}}
\]
The program \Code{(Subst (+ f (let (f 1) f)) f 5)} is of the case. The sugar introduces a local-binding on the variable \m{f}, with its sub-expression contains multiple \m{f}. By calculating the context rules of sugar \m{Subst}, the sugar has to be expanded after the $e_3$ being a value. After desugaring to \Code{(let (f 5) (+ f (let (f 1) (+ f 1))))},  no hygienic problem will take place because of the capture-avoiding substitution in the core language.

Because of the definition of desugaring in our approach, we cannot achieve hygiene by proving the $\alpha-equivalence$.
Here what we want to show is that, even without complex things in macro systems, scope specification and so on, the lazy desugaring itself will solve the common hygienic problem with carefully-designed core language. And of course the lazy desugaring will also work together with a hygienic desugaring system (e.g., by specific the binding scope \cite{10.5555/1792878.1792884}).

\todo{check}

\subsubsection{Recursive Sugars}
\label{sec:recursiveSugar}

It is common to write recursive sugar because of its usefulness. However, the existing syntactic sugar has a disadvantage when meeting a special case---though the sugar's definition expresses the correct meaning, the sugar can be ill-formed (for traditional sugar system), such as the \m{Odd/Even} sugars in Section \ref{mark:suflang}. Take \Code{(Odd 2)} as an example. The previous work will first desugar the program using the rules. Then the desugaring will never terminate as the following shows.
\[
	\begin{array}{l}
		\quad\;\ \Code{(Odd 2)}\\
		\DeStep{\Code{ (let (x 2) (if (> x 0) (Even (- x 1)) \false))}}\\
		\DeStep{\Code{ (let (x 2) (if (> x 0)}}\\
		\qquad\qquad\qquad\qquad\quad \Code{(let (x1 (- x 1)) (if (> x1 0) (Odd (- x1 1)) \#t))}\\
		\qquad\qquad\qquad\qquad\quad \Code{\false))}\\
		\DeStep{\Code{ $\ldots$}}
	\end{array}
\]



Then an advantage of our approach is embodied. Our approach does not require a fully desugaring of the sugar expression, which gives the framework chances to judge boundary conditions in sugars themselves and shows more intermediate sequences. We get the resugaring sequences of the former example as follows,
\[
	\begin{array}{l}
		\quad\;\;\Code{ (Odd 2)}\\
		\OneStep{ \Code{ (Even (- 2 1))}}\\
		\OneStep{ \Code{ (Even 1)}}\\
		\OneStep{ \Code{ (Odd (- 1 1))}}\\
		\OneStep{ \Code{ (Odd 0)}}\\
		\OneStep{ \Code{ \#f}}
	\end{array}
\]

We construct some higher-order syntactic sugars and test them. The higher-order feature is important for constructing practical syntactic sugars. And many higher-order sugars should be constructed by recursive definition. Note that we set call-by-value lambda calculus as terms in \m{CommonExp}, because we want to output some intermediate expressions including it in the following examples. It is easy if we want to skip them. The first sugar is \m{Filter} in Section \ref{mark:suflang}, implemented by pattern matching. We obtain a resugaring sequences as follows,
\[
	\begin{array}{l}
		\quad\;\;\Code{ (Filter ($\lambda$ (x) (and (> x 1) (< x 4))) (list 1 2 3 4))}\\
			\OneStep{ \Code{ (Filter ($\lambda$ (x) (and (> x 1) (< x 4))) (list 2 3 4))}}\\
			\OneStep{ \Code{ (cons 2 (Filter ($\lambda$ (x) (and (> x 1) (< x 4))) (list 3 4)))}}\\
			\OneStep{ \Code{ (cons 2 (cons 3 (Filter ($\lambda$ (x) (and (> x 1) (< x 4))) (list 4))))}}\\
			\OneStep{ \Code{ (cons 2 (cons 3 (Filter ($\lambda$ (x) (and (> x 1) (< x 4))) (list))))}}\\
			\OneStep{ \Code{ (cons 2 (cons 3 (list)))}}\\
			\OneStep{ \Code{ (cons 2 (list 3))}}\\
			\OneStep{ \Code{ (list 2 3)}}
	\end{array}
\]
Although the resugaring can be processed by the existing resugaring approach, it will be redundant. The reason is that a \m{Filter} program for a list of length $n$ will match to find possible reverse desugaring $n*(n-1)/2$ times.

Moreover, just like the sugar \m{Map}, it can be convenient if the desugaring allows inline boundary conditions. We obtain the resugaring sequences for a \m{Map} sugar as follows,
\[
	\begin{array}{l}
		\quad\;\;\Code{ (Map ($\lambda$ (x) (+ x 1)) (cons 1 (list 2)))}\\
			\OneStep{ \Code{ (Map ($\lambda$ (x) (+ x 1)) (list 1 2))}}\\
			\OneStep{ \Code{ (cons 2 (Map ($\lambda$ (x) (+ 1 x)) (list 2)))}}\\
			\OneStep{ \Code{ (cons 2 (cons 3 (Map ($\lambda$ (x) (+ 1 x)) (list))))}}\\
			\OneStep{ \Code{ (cons 2 (cons 3 (list)))}}\\
			\OneStep{ \Code{ (cons 2 (list 3))}}\\
			\OneStep{ \Code{ (list 2 3)}}\\
	\end{array}
\]

In this example, we can find that the list \Code{(cons 1 (list 2))}, though equal to \Code{(list 1 2)}, is represented by core language's expression. So it will be difficult to be handled by existing desugaring systems. (The case can be specific by some setting, such as local-expansion\cite{10.1017/S0956796812000093} in Racket language's macro.) But it is easy to be processed by our approach without specifying the expansion.

To sum up, we tried some basic syntactic sugar features to see if our approach could handle complex sugar. The result shows that our approach can easily handle which existing resugaring approach can deal with, together with some other features which existing one cannot support.

\subsection{Correctness}
\label{mark:correct}


We define the following properties of our desugaring algorithm for demonstrating the correctness. Before stepping in, we intuitively explain why the algorithm is correct. For any program headed with \m{SurfHead}, we can expand the outermost sugar by the desugaring rule of \m{SurfHead}. The desugared expression contains some sub-expressions. Some of them are the original program's sub-expressions, some origin from the desugaring rule, while others consist of the previous two parts. If the desugared expression should be reduced or recursively reduced at one of the original sub-expressions, the expression after that reduction can be resugared by reverse expansion of the desugaring rule. So these sub-expressions become the holes for context rules. Otherwise, the recursive reduction will not save the sugar's RHS structure. Such a program in the mixed language have to expand the outermost sugar because no more resugaring is available for it. We will give the properties to describe the correctness by certain definitions, and give proof sketch of them.



\subsubsection{Some Definition}
First, we define a function $\mathtt{D}$ to recursively desugar all sugars in a program using desugaring rules.

\begin{Def}[fully desugaring]
\[
\begin{array}{lll}
	\mathtt{D}(\Code{value}) = \Code{value}\\
	\mathtt{D}(\Code{CoreHead}~e_1~e_2~ ...) = (\Code{CoreHead}~(\mathtt{D}(e_1))~(\mathtt{D}(e_2))~...)\\
	\mathtt{D}(\Code{SurfHead}~e_1~e_2~ ...) = \mathtt{D}(e[e_i/x_i])\\
\quad \mbox{\bf where}~\drule{(\Code{SurfHead}~x_1~x_2~ ...)}{e}

\end{array}
\]

\end{Def}


An expression \m{E} can be fully desugared if \Code{$\mathtt{D}(\m{E})$} terminates. We use $\m{C}[\m{E}]$ to denote filling in the hole of the evaluation context \m{C} with \m{E}.  The fully desugaring of the evaluation context is also the same form, by following desugaring rules of evaluation context.

\begin{Def}[Desugaring rule of evaluation context]
	For syntactic sugar $S$
	\[
	\drule{(\m{SurfHead}~e_1~e_2~\ldots~e_n)}{(\m{Head}~\ldots~e_1'~e_2'~\ldots~e_m')}
	\]
	and evaluation context \m{C} = $\m{S}.LHS[[\bigcdot]/e_i]$, where $[\bigcdot]$ is at $e_i$'s location, then
	\[
	\drule{\m{C}}{\m{S}.RHS[[\bigcdot]/e_i]}
	\]

\end{Def}
As the evaluation rules of the mixed language defined, there are only two kinds of reductions---(1) desugaring on an expression headed with \m{SurfHead}; (2) reduction on an expression headed with \m{CoreHead}. And because we need the mixed language's reduction corresponds to the execution of the fully desugared program, for any expression \m{E} in the mixed language, if it reduces by expanding a sugar, then the reduction will occur in  the expression after the sugar expanded in $\mathtt{D}(\m{E})$; otherwise (reduced by \m{CoreHead}), the reduction will be also by the same \m{CoreHead} in $\mathtt{D}(\m{E})$.
So for convenience, we define destruction of a sugar's RHS when the second case happens in $\mathtt{D}(\m{E})$.

\begin{Def}[Destruction of a sugar's RHS]
	For a program \m{E} = $(\m{SurfHead}~e_1~e_2~\ldots~e_n)$,
	and $\mathtt{D}(\m{E})$ = $(\m{Head}~\ldots~e_1'~e_2'~\ldots~e_m')$ headed with \m{SurfHead},
	$\redc{\mathtt{D}(\m{E})}{\m{E'}}$.
	The $\redc{}{}$ is a destruction of \m{E}'s outermost sugar if the reduction is not at a recursive $\mathtt{D}(e_i)$.
\end{Def}
For example, \m{E} is \Code{(And (And \true~\true) \false)}, then $\mathtt{D}(\m{E})$ is \Code{(if (if \true~\true~\false) \false~\false)}. $\redc{\mathtt{D}(\m{E})}{\m{E'}}$ will reduce at the \Code{(if \true~\true~\false)} which is $\mathtt{D}(e_1)$ of \m{E}, so it is not a destruction of \m{And}'s RHS.
\subsubsection{Properties}

To describe the following properties in a word---no matter the expression \m{E} reduced by $\redm{}{}$ is headed by \m{SurfHead} or \m{CoreHead}, the reduction is correct because it showed or was to show the reduction of $\mathtt{D}(P)$.

\begin{property} \label{thm1}
	For an expression \m{E}=$\m{C}[\m{S}]$ of the mixed language which can be fully desugared, where \m{E'}=$\mathtt{D}(\m{E})$=$\m{C'}[\m{H}]$, and \m{S}=\Code{(SurfHead $e_1$ ... $e_n$)} in the expression \m{E} together with \m{C'}=$\mathtt{D}(\m{C})$ (then of course \m{H}=$\mathtt{D}(\m{S})$); if $\redm{\m{C}[\m{S}]}{\m{C}[\m{S'}]}$ and $\drule{\m{S}}{\m{S'}}$, then $\redc{\m{C'}[\m{H}]}{\m{C'}[\m{H'}]}$ together with destroying the sugar's RHS form of \m{S} by $\redc{\m{H}}{\m{H'}}$. An example in Fig. \ref{example:ppt1}.
\end{property}
\example{\footnotesize
\begin{tabular}{|l | l | l |}\hline
    \m{E}(above)/\m{E'}(below) & \m{C}(above)/\m{C'}(below) & \m{S}(above)/\m{H}(below)\\ \hline
    \Code{(And (And \#t \#f) \#f)} & \Code{(And $[\bigcdot]$ \#f)} & \Code{(And \#t \#f)}  \\ \hline
    \Code{(if (if \#t \#f \#f) \#f \#f)} & \Code{(if $[\bigcdot]$ \#f \#f)} & \Code{(if \#t \#f \#f)}   \\ \hline
  \end{tabular}
\begin{flushleft}
	$\redm{\Code{(And (And \#t \#f) \#f)}}{\Code{(And (if \#t \#f \#f) \#f)}}$, reduced by \m{And}.\\
	$\redc{\Code{(if (if \#t \#f \#f) \#f \#f)}}{\Code{(if \#f \#f \#f)}}$, reduced on the expression expanded from \m{And} sugar.\\
	So \m{S'}=\Code{(if \#t \#f \#f)}, \m{H'}=\Code{\#f}; thus the $\redc{}{}$ destroyed the sugar form of \Code{(And \#t \#f)}.

\end{flushleft}

}{Example of property \ref{thm1}}{example:ppt1}



\begin{property} \label{thm2}
	For an expression \m{E}=$\m{C}[\m{HH}]$ of the mixed language which can be fully desugared, where \m{E'}=$\mathtt{D}(\m{E})$=$\m{C'}[\m{H}]$, and \m{HH}=\Code{(CoreHead $e_1$ ... $e_n$)} in the expression \m{E} together with \m{C'}=$\mathtt{D}(\m{C})$ (then of course \m{H}=$\mathtt{D}(\m{HH})$); if $\redm{\m{C}[\m{HH}]}{\m{C}[\m{HH'}]}$ reduced by the \m{CoreHead}'s reduction rule on \m{HH}, then for $\redm{\m{C'}[\m{H}]}{\m{C'}[\m{H'}]}$, it also reduced by the \m{CoreHead}'s reduction rule on \m{H}. An example in Fig. \ref{example:ppt2}.
\end{property}

\example{\footnotesize
\begin{tabular}{|l | l | l |}\hline
    \m{E}(above)/\m{E'}(below) & \m{C}(above)/\m{C'}(below) & \m{H}(above)/\m{HH}(below)\\ \hline
    \Code{(if (if \#t (And \#t \#f) \#t) \#f \#f)} & \Code{(if $[\bigcdot]$ \#f \#f)} & \Code{(if \#t (And \#t \#f) \#t)}  \\ \hline
    \Code{(if (if \#t (if \#t \#f \#f) \#t) \#f \#f)} & \Code{(if $[\bigcdot]$ \#f \#f)} & \Code{(if \#t (if \#t \#f \#t) \#f)}   \\ \hline
  \end{tabular}
\begin{flushleft}
	$\redm{\Code{(if (if \#t (And \#t \#f) \#t) \#f \#f)}}{\Code{(if (And \#t \#f) \#f \#f)}}$, reduced on \m{if}.\\
	$\redc{\Code{(if (if \#t (if \#t \#f \#f) \#t) \#f \#f)}}{\Code{((if (if \#t \#f \#f) \#f \#f)}}$, on the same \m{if}.\\
	So \m{H'}=\Code{(And \#t \#f)},  \m{HH'}=\Code{(if \#t \#f \#f)}; thus \m{H} and \m{HH} are both reduced by \m{if}'s reduction rule.

\end{flushleft}

}{Example of property \ref{thm2}}{example:ppt2}

The properties restrict how the lazy desugaring of our mixed language should be. We give the proof sketch as follows.

\begin{lemma}
	For a syntactic sugar $S$, with the desugaring rule
	\[
	\drule{(\m{SurfHead}~e_1~e_2~\ldots~e_n)}{(\m{Head}~\ldots~e_1'~e_2'~\ldots~e_m')}
	\]
	If the algorithm \m{calcontext} gets the context rules as follows.
	\[
		\begin{array}{l}
			S.LHS[[\bigcdot]/e_i]\\
			S.LHS[v_i/e_i, [\bigcdot]/e_j]\\
			\ldots\\
			S.LHS[v_i/e_i, v_j/e_j, \ldots, [\bigcdot]/e_p]\\
			S.LHS[v_i/e_i, v_j/e_j, \ldots, v_p/e_p, [\bigcdot]/e_q]\\
			\ldots\\
			S.LHS[v_i/e_i, v_j/e_j, \ldots, v_p/e_p, v_q/e_q, \ldots, [\bigcdot]/e_x]\\
		\end{array}
	\]

	



Then for any expression \m{E} headed with \m{SurfHead}, if the sub-expressions $\{e_i, e_j, \ldots, e_p\}$ are all values, then $\redc{\mathtt{D}(\m{E})}{\m{E'}}$ will reduce on $\mathtt{D}(e_q)$.
\end{lemma}
\begin{proof}[Proof Sketch]
In the algorithm \m{calcontext}, the computational order of $S.RHS$ is iterated. So the context rules of the sugar correspond to the computational order.
So if the expression \m{E} will reduce on $e_q$, the desugared $\mathtt{D}(\m{E})$ will also reduce on the same part.
\end{proof}

\begin{lemma}
For a syntactic sugar $S$, with the desugaring rule
\[
\drule{(\m{SurfHead}~e_1~e_2~\ldots~e_n)}{(\m{Head}~\ldots~e_1'~e_2'~\ldots~e_m')}
\]
If the algorithm \m{calcontext} gets the context rules as follows.
	\[
		\begin{array}{l}
			S.LHS[[\bigcdot]/e_i]\\
			S.LHS[v_i/e_i, [\bigcdot]/e_j]\\
			\ldots\\
			S.LHS[v_i/e_i, v_j/e_j, \ldots, [\bigcdot]/e_x]\\
		\end{array}
	\]

Then for any expression \m{E} headed with \m{SurfHead}, if the sub-expressions $\{e_i, e_j, \ldots, e_x\}$ are all values, $\redc{\mathtt{D}(\m{E})}{\m{E'}}$ will destroy the $S.RHS$'s form.
\end{lemma}
\begin{proof}[Proof Sketch]
In the algorithm \m{calcontext}, the iteration recursively runs on the first inner sub-expression which is not $e_i$ or values, then it should be a reduciable expression. Because the computational order is on it, the inner sub-expression can be reduced (which means the $RHS$'s form is destroyed) if its context rules are iterated. So whenever the recursive call on \m{calcontext} is returned, the whole \m{calcontext} should break.

If $\redc{\mathtt{D}(\m{E})}{\m{E'}}$ is not a destruction of $S.RHS$'s form, the reduction will be on $e_i$, which is conflicted with our algorithm \m{calcontext}.
\end{proof}


\begin{proof} [Proof Sketch of Property 3.1]
If $\redm{\m{C}[\m{S}]}{\m{C}[\m{S'}]}$ and $\drule{\m{S}}{\m{S'}}$, according to lemma 1 and context rules of core language's expression, the expression \m{E} reduces recursively on the correct sub-expression, so $\redc{\m{C'}[\m{H}]}{\m{C'}[\m{H'}]}$.
According to lemma 2, the reduction will destroy the $RHS$'s form of the sugar $S$.

\end{proof}

\begin{proof} [Proof Sketch of Property 3.2]
If $\redm{\m{C}[\m{HH}]}{\m{C}[\m{HH'}]}$ reduced by the \m{CoreHead}'s reduction rule on \m{HH}, according to lemma 1 and context rules of core language's expression, the expression \m{E} reduces recursively on the correct sub-expression, so for $\redm{\m{C'}[\m{H}]}{\m{C'}[\m{H'}]}$, it will reduce on the same sub-expression.

Because the sub-expression \m{HH} is reduced by its \Code{CoreHead}, so matter how its inner expressions desugared, according to the context rules, the fully desugared \m{H} will also reduced by its \m{CoreHead}.

\end{proof}

\subsection{Output by DisplayableExp}
\label{mark:correctness}


% The existing resugaring approaches \cite{resugaring,hygienic} proposed the following three properties to define the correctness.

% \begin{quote}
% % enhanced formatting
% \begin{itemize}
%     \item Emulation:
% Each term in the generated surface evaluation sequence desugars into the core term which it is meant to represent.
%     \item Abstraction:
% Code introduced by desugaring is never revealed in the surface evaluation sequence, and code originating from the original input program is never hidden by resugaring.
%     \item Coverage: Resugaring is attempted on every core step, and as few core steps are skipped as possible.
% \end{itemize}
% \end{quote}
% Here we will show what are the similarities and differences between theirs and our properties.
\todo{(1) introduce ours (2) compare (3) trade-off}


\emph{Abstraction and Coverage}: Our reduction in the mixed language has some similarities to theirs. But since our framework has no execution for the fully desugared program and no reverse desugaring, there are some differences in details. For the coverage, our approach can guarantee that every step in the core languages' sequences can be reflected in the sequence before filtering, which is similar to the original one. But the abstraction and coverage are both about which kind of intermediate expression should be output. Especially, the abstraction property does not need to be strict if willing to have better coverage for some cases (as mentioned in Section 3.4 in \cite{resugaring}).

So in our approach, we provide a more flexible handling for the output thanks to the mixture of languages. Overall, our approach restricts the output by the \m{Head} of an expression and its sub-expressions. It is quite natural since the motivation of the resugaring is to show useful intermediate sequences, we think it will be better than restricting the output by judging whether the intermediate expressions contain some components desugared from the original program's components. Take the following sugar definitions as an example.
\[
\drule{\Code{(Nor~x~y)}}{\Code{(And~(not~x)~(not~y))}}
\]
\[
\drule{\Code{(And~x~y)}}{\Code{(if~x~y~\false)}}
\]
Then for a logic domain, what should be a resugaring sequence of the program \Code{(not (And (Nor \false~\true) \true))} ?

In our opinion, if the outer \m{not}, \m{And} can be displayed, so they should be after desugared.
The existing approach will produce the sequences as follows.
\begin{footnotesize}
\begin{Codes}
	\qquad(not (And (Nor \false \true) \true))
\OneStep{ (not (And \false \true))}
\OneStep{ (not \false)}
\OneStep{ \true}
\end{Codes}
\end{footnotesize}
while ours will produce the following sequences.
\begin{footnotesize}
\begin{Codes}
	\qquad(not (And (Nor \false \true) \true))
\OneStep{ (not (And (And (not \false) (not \true)) \true))}
\OneStep{ (not (And (And \true (not \true)) \true))}
\OneStep{ (not (And (not \true) \true))}
\OneStep{ (not (And \false \true))}
\OneStep{ (not \false)}
\OneStep{ \true}
\end{Codes}
\end{footnotesize}

Also, if we want to display the core language's expression only when it is originated from the input program, we can just make a mirror for it as a \m{CommonHead}. For example, when we want to show resugaring sequences of \Code{(And (if (And \#t \#f) ...) ...)}
without showing the \m{if} expression expanded from \m{And}, we only need to set \m{If} as \m{CommonHead} together with its evaluation rules same as \m{if}. Then inputting \Code{(And (If (And \#t \#f) ...) ...)} to the main algorithm will get what we need. Thus, our approach is able to get the resugaring sequence as the existing one.

In summary, our approach chooses a slightly different way for the \emph{abstraction} for better \emph{coverage} in the real application.
