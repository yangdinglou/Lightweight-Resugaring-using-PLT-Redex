%!TEX root = ./main.tex
\section{The Mixed Language for Resugaring}
\label{sec:language}

Different from the existing approach that clearly separates the surface language (defined by syntactic sugars over the core language) from the core languages, we intentionally combine them as one mixed language, allowing free use of the language constructs in both languages. In this section, we will introduce our mixed language, which is the basic of our approach.

Note that this section discusses the second step in Section \ref{sec2}, while
assuming that the first step of calculating the context rules for the surface language has been done. The first step will be discussed later in Section \ref{sec:algo}.

%Note that the mixture is unfinished until the process in next section, because of the lack of context rules for expressions headed with \m{SurfHead}.
%We assume that the evaluation of the core language is  compositional (as the definition in \cite{hygienic}), that is, for evaluation contexts $C_1$ and $C_2$, $C_1[C_2]$ is also an evaluation context.

\begin{figure}[t]
	\begin{flushleft}
	{\footnotesize
	\[
	\begin{array}{lllll}
	\m{CoreExp} &::=& x  & \note{variable}\\
	&~|~& v  & \note{value}\\
	&~|~& (\m{CoreHead}~\m{CoreExp}_1~\ldots~\m{CoreExp}_n) & \note{constructor}\\
	\\
	\m{SurfExp} &::=& x  & \note{variable}\\
	&~|~& v  & \note{value}\\
	&~|~& (\m{SurfHead}~\m{SurfExp}_1~\ldots~\m{SurfExp}_n) & \note{sugar constructor}\\
	\end{array}
	\]
	}
	\end{flushleft}


		\caption{Core and Surface Expressions}
		\label{fig:expression}
	\end{figure}

\subsection{Core Language}

There are many possible forms of the core language. For simplicity, we assume that the form of the expressions in our core language is defined as in Fig. \ref{fig:expression}. An expression can be a variable, a value, or a (language) constructor expression. Here, $\m{CoreHead}$ stands for a language constructor such as $\m{if}$ and $\m{let}$. Also, the sub-expressions can be nested for supporting expressions like \Code{(let (x e) $\ldots$)}. To be concrete, in this paper we will use the core language defined in Fig.  \ref{fig:core} to demonstrate our approach. It is a usual functional language and its semantics is defined by the context rules and the reduction rules. Here the [e/x] denotes capture-avoiding substitution.

\begin{figure*}[t]
\begin{subfigure}{0.95\linewidth}
\[
{\footnotesize
		\begin{array}{lcl}
		\m{CoreExp} &::=& \Code{(CoreExp~CoreExp~...)} ~~\note{// application}\\
		&|& \m{(if~CoreExp~CoreExp~CoreExp)} ~~\note{// condition}\\
		&|& \m{(let~(x~CoreExp)~CoreExp)} ~~\note{// binding}\\
		&|& \m{(listop~CoreExp)} ~~\note{// first, rest, empty?}\\
		&|& \m{(cons~CoreExp~CoreExp)} ~~\note{// data structure of list}\\
		&|& \m{(arithop~CoreExp~CoreExp)} ~~\note{// +, -, *, /, >, <, =}\\
		&|& \m{x} ~~\note{// variable}\\
		&|& \m{value}\\
		\m{value} &::=& \m{($\lambda$~(x~...)~CoreExp)} ~~\note{// call-by-value}\\
		&|& \m{c} ~~\note{// boolean, number and list}
		\end{array}
}
\]
\caption{Syntax}
\end{subfigure}
\\
\medskip
\begin{subfigure}{0.9\linewidth}
\[
{\footnotesize
		\begin{array}{lcl}
		\Code{C} &::=& \Code{(value~...~C~CoreExp~...)}\\
		&|& \Code{(if~C~CoreExp~CoreExp)}\\
		&|& \Code{(let~(x~C)~CoreExp)}\\
		&|& \Code{(listop~C)}\\
		&|& \Code{(cons~C~CoreExp)}\\
		&|& \Code{(cons~value~C)}\\
		&|& \Code{(arithop~C~CoreExp)}\\
		&|& \Code{(arithop~value~C)}\\
		&|& [\bigcdot]
		%\\[1.5em]
		\end{array}
}
\]
\caption{Context Rules}
\end{subfigure}
\\
\medskip
\begin{subfigure}{0.9\linewidth}
	\[
{\footnotesize
		\begin{array}{lcl}
		\Code{(($\lambda$~($\m{x}_1$~$\m{x}_2$~...)~CoreExp)~$\m{value}_1$~$\m{value}_2$~...)} &\redc{}{}& \Code{(($\lambda$~($\m{x}_2$~...)~CoreExp[$\m{value}_1$/$\m{x}_1$])~$\m{value}_2$~...)}\\

		\Code{(if~$\m{\true}$~$\m{CoreExp}_1$~$\m{CoreExp}_2$)}&\redc{}{}& \Code{$\m{CoreExp}_1$}\\
		\Code{(if~$\m{\false}$~$\m{CoreExp}_1$~$\m{CoreExp}_2$)}&\redc{}{}& \Code{$\m{CoreExp}_2$}\\
		\Code{(let~(x~$\m{value}$)~CoreExp)}&\redc{}{}&\Code{CoreExp[\m{value}/x]}\\
		\ldots&&
		\end{array}
}
\]
	\caption{Part of Reduction Rules}
\end{subfigure}


\caption{A Core Language}
\label{fig:core}
\end{figure*}

The evaluator of our core language is driven by evaluation rules (context rules and reduction rules), with two natural assumptions. First, the evaluation is deterministic, in the sense that any expression in the core language will be reduced by a unique reduction sequence (defined by the context rules). Second, the context rules have no side conditions, which excludes the following rules (here $f_1$, $f_2$ are two auxiliary functions).

{\footnotesize
\[
\begin{array}{lll}
\m{C}& ::= & (\m{notif}~[\bigcdot]~e_2~e_3)\\
&|& (\m{notif}~v_1~[\bigcdot]~e_3), \qquad \m{when}~(f_1~v_1)\\
&|& (\m{notif}~v_1~e_2~[\bigcdot]), \qquad \m{when}~(f_2~v_1)
\end{array}
\]}


\subsection{Surface Language}
\label{mark:suflang}

A surface language is defined by a set of syntactic sugars, together with some elements in the core language, such as values and variables. The expression of a surface language has a similar form with that of the core as shown in Fig.  \ref{fig:expression}. To separate surface language's \m{Head} from that of the core language, we capitalize the first letter of surface language's \m{Head}.


A syntactic sugar \m{SurfHead} is defined by a desugaring rule in the following form
\[
\drule{(\m{SurfHead}~e_1~e_2~\ldots~e_n)}{\m{exp}} %\todo{\text{font size in drule}}
\]
where its left-hand side (LHS) is a pattern and its right-hand side (RHS) is an expression of the mixed language (described in the next section).
%The LHS may be nested, so we can write sugars like \Code{(SurfHead ($e_1$ ($e_2$ $e_3$)) $\ldots$ $e_n$)}.
For instance, we may define a syntactic sugar \m{And} by
\[
\drule{(\m{And}~e_1~e_2)}{(\m{if}~e_1~e_2~\m{\false})}.
\]

In the following, we give some interesting cases for defining sugars, showing our extension and demonstrating its power in defining surface language.

\subsubsection{Sugars with no Arguments}
\label{mark:SKI}
A sugar definition does not need to have pattern variables at LHS, because we can use the $\lambda$ abstraction in the core language. For example, we can define the sugar of SKI combinator by following rules. ($\lambda_N$ here is the call-by-need lambda calculus.)
\[
\begin{array}{l}
\drule{\m{S}}{\Code{($\lambda_N$ ($\m{x}_1$ $\m{x}_2$ $\m{x}_3$) ($\m{x}_1$ $\m{x}_2$ ($\m{x}_1$ $\m{x}_3$)))}}\\
\drule{\m{K}}{\Code{($\lambda_N$ ($\m{x}_1$ $\m{x}_2$) $\m{x}_1$)}}\\
\drule{\m{I}}{\Code{($\lambda_N$ (x) x)}}
\end{array}
\]

\subsubsection{Sugars with Linear Use of Pattern Variables}
We assume that any pattern variable (e.g., $e_1$) in LHS appears only once in RHS for the syntactic sugar.
If we need to use a pattern variable multiple times in RHS, a \m{let} binding may be used.\footnote{Of course, we can add the support for multiple appearance for a pattern variable by some special judgements. But we just make it simple for presentation.} To see the problem of multiple uses of a variable, suppose we define a sugar by
$
\drule{(\m{Twice}~e_1)}{(+~e_1~e_1)}.
$
If we execute \Code{(Twice (+ 1 1))}, it will first be desugared to \Code{(+ (+ 1 1) (+ 1 1))}, then reduced to \Code{(+ 2 (+ 1 1))} by one step. The sub-expression \Code{(+ 1 1)} has been reduced but should not be resugared to the surface, because the other \Code{(+ 1 1)} has not been reduced yet.
In this paper, we just use a \m{let} binding to resolve this problem, and define it as
\[
\drule{(\m{Twice}~e_1)}{\Code{(let (x $e_1$) (+ x x))} }.
\]

\subsubsection{Sugars with let-binding}
\label{mark:let}
The use of \m{let} binding is useful, but it introduces the hygienic problem for desugaring. For example, considering we have the following syntactic sugar
\[
\drule{\Code{(HygienicAdd $e_1$ $e_2$)}}{\Code{(let (x $e_1$) (+ x $e_2$))}}
\]
for the program \Code{(let (x 2) (HygienicAdd 1 x))}, it would be desugared by a simple expansion to the unexpected expression \Code{(let (x 2) (let x 1 (+ x x)))}. We will show how we solve this problem later.

\subsubsection{Recursive and Higher Order Sugars}
\label{mark:recsugar}
In the desugaring rule, we relax the ordinary restriction that RHS must be a $\m{CoreExp}$. This makes it possible to define recursive sugars as follows:

\[\begin{array}{l}
	\drule{\Code{(Filter $e$ (list $v_1$ $v_2$ ...))}}{}\\
	\quad
	\Code{ (let (f $e$) (if (f $v_1$)}\\
	\qquad\qquad\qquad\qquad\ \Code{ (cons $v_1$ (Filter f (list $v_2$ ...)))}\\
	\qquad\qquad\qquad\qquad\ \Code{ (Filter f (list $v_2$ ...))))}\\
\\
	\drule{\Code{(Filter $e$ (list))}}{\Code{(list)}}
	\end{array}
\]

Moreover, we allow more involved recursive syntactic sugars, which is ill-formed from the traditional view of desugaring. For example, the following sugar is not allowed by simple desugaring, because the expansion of sugar will not stop. Fortunately, it is allowed in our context.
\[
\begin{array}{l}
\drule{\Code{(Odd e)}}{\Code{(let (x~e)~(if~(>~x~0)~(Even~(-~x~1))~\m{\false}))}}\\
\drule{\Code{(Even e)}}{\Code{(let (x~e)~(if~(>~x~0)~(Odd~(-~x~1))~\m{\true}))}}
\end{array}
\]
%However, the "semantics" of such syntactic sugar is sound, so we have a chance to handle this case by our lazy desugaring. This can be practical, though we should have written the \m{Odd, Even} sugars by a pattern matching like \m{Filter}.
%Sometimes we can only use an inline condition like \Code{(empty? x)} to judge the branch as the following \m{Map} sugar.

As another example, we can define the \m{Map} sugar as follows.
\[
\begin{array}{l}
\drule{\Code{(Map $e_1$ $e_2$)}}{}\\
\quad\Code{(let (f $e_1$)}\\
\qquad\Code{(let (x $e_2$)}\\
\qquad\quad

\Code{(if (empty? x)}\\
\qquad\qquad\quad\Code{(list)}\\
\qquad\qquad\quad\Code{(cons (f (first x)) (Map f (rest x))))))}
\end{array}
\]
In this case, our approach can handle something which is hard to deal with by traditional desugaring.

\medskip
We will return to the examples in this section to show how resugaring can be done for the  sugars that are defined here.



\subsection{Mixed Language}
\begin{figure}[t]
\begin{centering}
{\small
\[
			\begin{array}{lcl}
			\m{Exp} &::=& \m{DisplayableExp}\\
			&|& \m{MixedExp}\\
			\\
			\m{DisplayableExp} &::=& \m{(SurfHead~DisplayableExp~$\ldots$)}\\
			&|& \m{(CommonHead~DisplayableExp~$\ldots$)}\\
			&|& v\\
			&|& x\\
			\\
			\m{MixedExp} &::=& \m{(SurfHead~MixedExp~$\ldots$)}\\
			&|& \m{(CoreHead~MixedExp~$\ldots$)}\\
			&|& v\\
			&|& x\\
			\end{array}
			\]
}

\end{centering}
\caption{Syntax of the Mixed Language}
\label{fig:mix}
\end{figure}

A mixed language for resugaring combines the surface language and the core language, with the mixture of syntax, context rules, and reduction rules. The mixture can be done automatically, and the generation of the mixed language can be flexible (we will discuss in Section \ref{mark:correctness}). As discussed at the beginning of this section, since the context rules of the sugars are calculated by the algorithm later, we assume here that we have their context rules.

\subsubsection{Mixing Syntax}
\label{mark:displayable}
The mixed syntax is described in Fig.  \ref{fig:mix}.
%
The differences between expressions in our core language and those in our surface language are identified by their \m{Head}. But there may be some expressions in the core language which are also used in the surface language for convenience, or to say, we need some core language's expressions to help us get better resugaring sequences. So we take \m{CommonHead} as a subset of the \m{CoreHead}, which can be displayed in resugaring sequences (just as the \m{CommonExp} in our Section \ref{sec2}). Then if any sub-expression in an expression contains no \m{CoreHead} except for \m{CommonHead}, we should let them displayed during the evaluation process (named \m{DisplayableExp}). Otherwise, the expression should not be displayed. We just use a \m{MixedExp} expression to represent the expressions that are unnecessarily displayed for concision, and discuss more on \m{DisplayableExp}  in Section \ref{mark:correctness}.





As an example, for the core language in Fig.  \ref{fig:core},
we may assume \m{arithop}, \m{$\lambda$} (call-by-value lambda calculus), \m{cons} as \m{CommonHead}, \m{if}, \m{let}, \m{listop} as \m{CoreHead} but out of \m{CommonHead}. This will allow \m{arithop}, \m{$\lambda$} and \m{cons} to appear in the resugaring sequences, and thus display more useful intermediate steps during resugaring.

\begin{figure}[t]
	\centering
	\begin{subfigure}{0.3\linewidth}{\footnotesize
		\begin{flushleft}
			\[
			\begin{array}{lcl}
			\m{C} &:=& (\m{CoreHead1}~$\ldots$~C~$\ldots$)\\
			&|& (\m{CoreHead2}~$\ldots$~C~$\ldots$)\\
			&|& $\ldots$\\
			&|&[\bigcdot]\\[4em]
			\end{array}
			\]
			\end{flushleft}
		\caption{Core's Context Rules}
		\label{fig:corecontext}
	}
	\end{subfigure}
	\begin{subfigure}{0.3\linewidth}{\footnotesize
		\begin{flushleft}
			\[
			\begin{array}{lcl}
			\m{C} &:=& (\m{SurfHead1}~$\ldots$~C~$\ldots$)\\
			&|& (\m{SurfHead2}~$\ldots$~C~$\ldots$)\\
			&|& $\ldots$\\
			&|&[\bigcdot]\\[4em]
			\end{array}
			\]
			\end{flushleft}
		\caption{Surface's Context Rules}
		\label{fig:surfcontext}
	}
	\end{subfigure}
	\begin{subfigure}{0.3\linewidth}{\footnotesize
		\begin{flushleft}
			\[
			\begin{array}{lcl}
			\m{C} &:=& (\m{CoreHead1}~$\ldots$~C~$\ldots$)\\
			&|& (\m{CoreHead2}~$\ldots$~C~$\ldots$)\\
			&|& $\ldots$\\
			&|& (\m{SurfHead1}~$\ldots$~C~$\ldots$)\\
			&|& (\m{SurfHead2}~$\ldots$~C~$\ldots$)\\
			&|& $\ldots$\\
			&|&[\bigcdot]\\
			\end{array}
			\]
			\end{flushleft}
		\caption{Mixed's Context Rules}
		\label{fig:mixedcontext}
	}
	\end{subfigure}

\caption{Mixture of Context Rules}
\label{fig:mixofcontext}
\end{figure}

\subsubsection{Mixing Context and Reduction Rules}

The context rules specify the computational order for an expression based on the \m{Head}, so that they can be simply mixed as described in Fig. \ref{fig:mixofcontext}. And if we want to limit the unique reduction of an expression, we should modify the desugaring rules to be correct for the reduction. For example, if we have the context rules as in Fig. \ref{fig:mixcontext}, then for the expression
\[
	\Code{(if (And (And \true~\true) \false) \false~\true)}
\]
it will reduce at the sub-expression \Code{(And (And \true~\true) \false)}. But the outer \m{And} sugar should not be expanded according to the context rules of \m{And}, we should change
\[
	\drule{(\m{And}~e_1~e_2)}{(\m{if}~e_1~e_2~\m{\false})}
\]
to
\[
	\drule{(\m{And}~v_1~e_2)}{(\m{if}~v_1~e_2~\m{\false})}
\]
so that the expression above will reduce by proper order as follows, in the mixed language.
\[
	\begin{array}{l}
		\qquad\Code{  (if (And (And \true~\true) \false) \false~\true)}\\
		\redm{}\Code{(if (And (if \true~\true~\false) \false) \false~\true)}\\
		\redm{}\Code{$\ldots$}
	\end{array}
\]

After the mixture of reduction rules in the core language ($\redc{}{}$) and the  desugaring rules for the syntactic sugars in the surface language ($\drule{}{}$), any expression in the mixed language can be reduced step by step. For example, in the mixed language, some expressions with \m{CoreHead} may contain sub-expressions with \m{SurfHead}. We process these expressions by the context rules of the core language, so that the reduction rules of the core language and the desugaring rules of surface language can be mixed as a whole
 (the $\redm{}{}$ in Fig. \ref{fig:mixexample}). For example, suppose we have the context rule of \m{if} expression\footnote{Another presentation of \Code{(if $[\bigcdot]$ e e)}. Use this here for convenience.}
\[
\infer{(\m{if}~e_1~e_2~e_3) \rightarrow (\m{if}~e_1'~e_2~e_3)}{e_1 \rightarrow e_1'}
\]
then if $e_1$ is headed with \m{CoreHead}, the reduction will use the \m{CoreHead}'s reduction rules recursively. So does it if $e_1$ headed with \m{SurfHead}. Finally, the $e_1$ is reduced to a value, then we can use the reduction rule of \m{if}.
