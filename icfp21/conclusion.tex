%!TEX root = ./main.tex
\section{Conclusion}
\label{sec7}

%Summarize the paper, explaining what you have shown, what results you have achieved, and what future work is.

In this paper, we propose a novel resugaring approach using lazy desugaring. We design the approach based on a  core language, with a simple desugaring system. Our algorithm then can output the evaluation sequence in the surface syntax, given some syntactic sugars together with an input program. In our approach, the most important insight is delaying the expansion of syntactic sugars by calculating context rules (Section \ref{sec:language} and \ref{sec:algo}), which decide whether the mixed language should reduce the sub-expression by core's reduction rules or expand the sugar. We show that the system can handle a variety of syntactic sugars and can achieve better efficiency (Section \ref{mark:resugaringexample} and \ref{sec:implementation}). Moreover, the approach is flexible to make some extensions (Section \ref{sec:ext}).

We find the extensions may work if the core language and the syntactic sugar have some properties such like compositional, clear semantics, and unique computational order. So one possible future work of this is to extend the core language and the desugaring system with other components of language design like type system, analyzer, and optimizer. Also, we find it is possible to derive stand-alone evaluation rules for the surface language by means similar to how we calculate context rules, making it more convenient to develop domain-specific languages. This functionality can be added to future systems.
