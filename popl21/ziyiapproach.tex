%!TEX root = ./main.tex

\section{Resugaring by Lazy Desugaring}
\label{sec3}

In this section, we present our new approach to resugaring. Different from the traditional approach that clearly separates the surface from the core languages, we intentionally combine them as one mixed language, allowing free use of the language constructs in both languages. We will show that any expression in the mixed language can be evaluated in such a smart way that a sequence of all expressions that are necessary to be resugared by the traditional approach can be correctly produced.

\subsection{Mixed Language for Resugaring}

\begin{figure}[t]
{\footnotesize
\[
	\begin{array}{lllll}
	 &\m{CoreExp} &::=& x  & \note{variable}\\
	       &&~|~& c  & \note{constant}\\
				 &&~|~& (\m{CoreHead}~\m{CoreExp}_1~\ldots~\m{CoreExp}_n) & \note{constructor}\\
	\\
	 &\m{SurfExp} &::=& x  & \note{variable}\\
	       &&~|~& c  & \note{constant}\\
				  % &&~|~& (\m{CoreHead}~\m{SurfExp}_1~\ldots~\m{SurfExp}_n) & \note{selected core constructor}\\
					 &&~|~& (\m{SurfHead}~\m{SurfExp}_1~\ldots~\m{SurfExp}_n) & \note{sugar expression}\\
	\end{array}
	\]
}

	\caption{Core and Surface Expressions}
	\label{fig:expression}
\end{figure}

As a preparation for our resugaring algorithm, we define a mixed language that combines a core language with a surface language (defined by syntactic sugars over the core language). An expression in this language is reduced step by step by the evaluation rules for the core language and the desugaring rules for the syntactic sugars in the surface language.

\subsubsection{Core Language}

For our core language,  its evaluator is driven by evaluation rules (context rules and reduction rules), with two natural assumptions. First, the evaluation is deterministic, in the sense that any expression in the core language will be reduced by a unique reduction sequence. Second, evaluation of a sub-expression has no side-effect on other parts of the expression.
(We discuss how to trickily extend our approach with a black-box stepper as the evaluator in Appendix A.1)

An expression form of the core language is defined in Figure \ref{fig:expression}. It is a variable, a constant, or a (language) constructor expression. Here, $\m{CoreHead}$ stands for a language constructor such as $\m{if}$ and $\m{let}$. To be concrete, we will use a simplified core language defined in Figure \ref{fig:core} to demonstrate our approach.

\begin{figure}[thb]
\begin{centering}
	\[
	{\footnotesize
			\begin{array}{lcl}
			\m{CoreExp} &::=& \Code{(CoreExp~CoreExp~...)} ~~\note{// apply}\\
			&|& \m{(lambda~(x~...)~CoreExp)} ~~\note{// call-by-value}\\
			&|& \m{(lambdaN~(x~...)~CoreExp)} ~~\note{// call-by-need}\\
			&|& \m{(if~CoreExp~CoreExp~CoreExp)} ~~\note{// condition}\\
			&|& \m{(let~x~CoreExp~CoreExp)} ~~\note{// binding}\\
			&|& \m{(listop~CoreExp)} ~~\note{// first, rest, empty?}\\
			&|& \m{(cons~CoreExp~CoreExp)} ~~\note{// data structure of list}\\
			&|& \m{(arithop~CoreExp~CoreExp)} ~~\note{// +, -, *, /, >, <, =}\\
			&|& \m{x} ~~\note{// variable}\\
			&|& \m{c} ~~\note{// boolean, number and list}
			\end{array}
	}
	\]
\end{centering}
\caption{A Core Language Example}
\label{fig:core}
\end{figure}


%For simplicity, we use the prefix notation. For instance, we write $\m{if-then-else}~e_1~e_2~e_3$, which would be more readable if we write $\m{if}~e_1~\m{then}~e_2~\m{else}~e_3$. In this paper, we may write both if it is clear from the context.

\subsubsection{Surface Language}

Our surface language is defined by a set of syntactic sugars, together with some basic elements in the core language, such as constant, variable. The surface language has expressions as given in Figure \ref{fig:expression}.

A syntactic sugar is defined by a desugaring rule in the following form:
\[
\drule{(\m{SurfHead}~e_1~e_2~\ldots~e_n)}{\m{exp}}
\]
where its LHS is a simple pattern (unnested) and its RHS is an expression of the surface language or the core language, and any pattern variable (e.g., $e_1$) in LHS only appears once in RHS. For instance, we may define syntactic sugar \m{And} by
\[
\drule{(\m{And}~e_1~e_2)}{(\m{if}~e_1~e_2~\false)}.
\]
Note that if the pattern is nested, we can introduce a new syntactic sugar to flatten it. And if we need to use a pattern variable multiple times in RHS, a \m{let} binding may be used (a normal way in syntactic sugar). We take the following sugar as an example
\[
\drule{(\m{Twice}~e_1)}{(+~e_1~e_1)}.
\]
If we execute \Code{(Twice (+ 1 1))}, it will firstly be desugared to \Code{(+ (+ 1 1) (+ 1 1))}, then reduced to \Code{(+ 2 (+ 1 1))} by one step. The subexpression \Code{(+ 1 1)} has been reduced but should not be resugared to the surface, because the other \Code{(+ 1 1)} has not been reduced yet.
%This would let many intermediate steps omitted, if there are many reduction steps between reduction of them.
So we just use a \m{let} binding to resolve this problem. The RHS should be \Code{(let x $e_1$ (+ x x))} in this case.

Note that in the desugaring rule, we do not restrict the RHS to be a $\m{CoreExp}$. We can use $\m{SurfExp}$ (more precisely, we allow the mixture use of syntactic sugars and core expressions) to define recursive syntactic sugars, as seen in the following example.
\[
\begin{array}{l}
\drule{(\m{Odd}~e)}{(\m{if}~(>~e~0)~(\m{Even}~(-~e~1))~\false)}\\
\drule{(\m{Even}~e)}{(\m{if}~(>~e~0)~(\m{Odd}~(-~e~1))~\true)}
\end{array}
\]

We assume that all desugaring rules are not overlapped in the sense that for a syntactic sugar expression, only one desugaring rule is applicable for a single sugar in the expression.


\subsubsection{Mixed Language}
\begin{figure}[t]
\begin{centering}
{\footnotesize
\[
			\begin{array}{lcl}
			\m{Exp} &::=& \m{DisplayableExp}\\
			&|& \m{UndisplayableExp}\\
\\
			\m{DisplayableExp} &::=& \m{SurfExp}\\
			&|& \m{CommonExp}
\\
			\m{UndisplayableExp} &::=& \m{CoreExp'}\\
			&|& \m{OtherSurfExp}\\
			&|& \m{OtherCommonExp}\\
\\
			\m{CoreExp} &::=& \m{CoreExp'}\\
						 &|& \m{CommonExp}\\
						 &|& \m{OtherCommonExp}\\
\\
			\m{CoreExp'} &::=& (\m{CoreHead'}~\m{Exp*})\\
\\
			\m{SurfExp} &::=& (\m{SurfHead}~\m{DisplayableExp*})\\
\\
			\m{CommonExp} &::=& (\m{CommonHead}~\m{DisplayableExp*})\\
			&|& c \qquad \note{// constant value}\\
			&|& x \qquad \note{// variable} \\
\\
			\m{OtherSurfExp} &::=& (\m{SurfHead}~\m{Exp*}~\m{UndisplayableExp}~\m{Exp*})\\
\\
			\m{OtherCommonExp} &::=& (\m{CommonHead}~\m{Exp*}~\m{UndisplayableExp}~\m{Exp*})
			\end{array}
			\]
}

\end{centering}
\caption{Our Mixed Language}
\label{fig:mix}
\end{figure}

Our mixed language for resugaring combines the surface language and the core language, described in Figure \ref{fig:mix}.
%
The differences between expressions in our core language and those in our surface language are identified by their \m{Head}. But there may be some expressions in the core language which are also used in the surface language for convenience, or we need some core language's expressions to help us getting better resugaring sequences. So we take \m{CommonHead} out from the \m{CoreHead}, which can be displayed in resugaring sequences (the rest of \m{CoreHead} becomes \m{CoreHead'}). The \m{CoreExp'} denotes an expression with undisplayable \m{CoreHead} (named \m{CoreHead'}). The \m{SurfExp} denotes an expression that have \m{SurfHead} and their subexpressions being displayable. The \m{CommonExp} denotes an expression with displayable \m{Head} (named \m{CommonHead}) in the core language, together with displayable subexpressions. There exist some other expressions during our resugaring process, which have displayable \m{Head}, but one or more of their subexpressions should not be displayed. They are of \m{UndisplayableExp}. We distinguish these two kinds of expressions for the \emph{abstraction} property (discussed in Section \ref{mark:abs}).

As an example, for the core language in Figure \ref{fig:core},
we may assume \m{if}, \m{let}, \m{lambdaN} (call-by-name lambda calculus), \m{listop} as \m{CoreHead'}, \m{arithop}, \m{lambda} (call-by-value lambda calculus), \m{cons} as \m{CommonHead}. This will allow \m{arithop}, \m{lambda} and \m{cons} to appear in the resugaring sequences, and thus display more useful intermediate steps during resugaring.

Note that some expressions with \m{CoreHead} contain subexpressions with \m{SurfHead}, they are of \m{CoreExp} but not in the core language. In the mixed language, we process these expressions by the context rules of the core language, so that the reduction rules of core language and the desugaring rules of surface language can be mixed as a whole (the $\redc{}{}$ in next section). For example, suppose we have the context rule of \m{if} expression
\[
\infer{(\m{if}~e_1~e_2~e_3) \rightarrow (\m{if}~e_1'~e_2~e_3)}{e_1 \rightarrow e_1'}
\]
then if $e_1$ is a reducible expression in the core language, it will be reduced by the reduction rule in the core language; if $e_1$ is a \m{SurfExp}, it will be reduced by the desugaring rule of $e_1$'s \m{Head} (actually, how the subexpression reduced does not matter, because it is just to mark the location where it should be reduced); if $e_1$ is also a \m{CoreExp} which has one or more non-core subexpressions, a recursive reduction by $\redc{}{}$ is needed.


\subsection{Resugaring Algorithm}

Our resugaring algorithm works on the mixed language, based on the evaluation rules of the core language and the desugaring rules for defining the surface language. Let $\redc{}{}$ denote the one-step reduction of the \m{CoreExp} (described in previous section), and $\drule{}{}$ the one-step desugaring of outermost sugar. We define $\redm{}{}$, a one-step reduction of our mixed language, as follows.
\label{mark:mixedreduction}
\infrule[CoreRed1]
{
\exists i.\, \redc{(\m{CoreHead}~e_1~\ldots~e_i~\ldots~e_n)}
                  {(\m{CoreHead}~e_1~\ldots~e_i'~\ldots~e_n)}~and~e_i~\in~\m{SurfExp}\\
                  \redm{e_i}{e_i''} }
{\redm{(\m{CoreHead}~e_1~\ldots~e_i~\ldots~e_n)}{(\m{CoreHead}~e_1~\ldots~e_i''~\ldots~e_n)}}
\infrule[CoreRed2]
{
\not \exists i.\, \redc{(\m{CoreHead}~e_1~\ldots~e_i~\ldots~e_n)}
                  {(\m{CoreHead}~e_1~\ldots~e_i'~\ldots~e_n)}~and~e_i~\in~\m{SurfExp}\\
\redc{(\m{CoreHead}~e_1~\ldots~e_n)}{e'}}
%\neg \exists i.\, \redc{(\m{CoreHead}~e_1~\ldots~e_i~\ldots~e_n)}{(\m{CoreHead}~e_1~\ldots~e_i'~\ldots~e_n)}~and~e_i~\in~\m{SurfExp}}
{\redm{(\m{CoreHead}~e_1~\ldots~e_i~\ldots~e_n)}{e'}}
\infrule[SurfRed1]
{\drule{(\m{SurfHead}~x_1~\ldots~x_i~\ldots~x_n)}{e}~\\
\exists i.\, \redm{e[e_1/x,\ldots,e_i/x_i,\ldots,e_n/x_n]}{e[e_1/x,\ldots,e_i'/x_i,\ldots,e_n/x_n]}\\
\redm{e_i}{e_i''}
}
{\redm{(\m{SurfHead}~e_1~\ldots~e_i~\ldots~e_n)}{(\m{SurfHead}~e_1~\ldots~e_i''~\ldots~e_n)}}
\infrule[SurfRed2]
{\drule{(\m{SurfHead}~x_1~\ldots~x_i~\ldots~x_n)}{e}\\
\not \exists i.\, \redm{e[e_1/x_1,\ldots,e_i/x_i,\ldots,e_n/x_n]}{e[e_1/x_1,\ldots,e_i'/x_i,\ldots,e_n/x_n]}
}
{\redm{(\m{SurfHead}~e_1~\ldots~e_i~\ldots~e_n)}{e[e_1/x_1,\ldots,e_i/x_i,\ldots,e_n/x_n]}
}

Putting them in simple words, for an expression with \m{CoreHead}, we just use the evaluation rules of core language combined with desugaring rules as reduction rules. As described in the previous section, if a \m{SurfExp} subexpression is to be reduced (\m{CoreRed1}), we should mark the location and apply $\redm{}{}$ on it recursively; otherwise (\m{CoreRed2}), the expression is just reduced by the $\redc{}{}$'s rule.
For the expression with \m{SurfHead}, we will first expand the outermost sugar (identified by the \m{SurfHead}), then recursively apply $\redm{}{}$ on the partly desugared expression. In the recursive call, if one of the original subexpression $e_i$ is reduced then the original sugar is not necessarily desugared, we should only reduce the subexpression $e_i$ (\m{SurfRed1}); otherwise, the outermost sugar should be desugared (\m{SurfRed2}).


%\todo{Add explanantion of the above rule.}

Now, our desugaring algorithm can be easily defined based on $\redm{}{}$.

\[
\begin{array}{llll}
\m{resugar} (e) &=& \key{if}~\m{isNormal}(e)~\key{then}~return\\
              & & \key{else}~\\
							& & \qquad \key{let}~\redm{e}{e'}~\key{in}\\
							& & \qquad \key{if}~e' \in~\m{DisplayableExp} \\
							& & \qquad \qquad \m{output}(e'),~\m{resugar}(e')\\
							& & \qquad \key{else}~\m{resugar}(e')
\end{array}
\]
During the resugaring, we just apply the reduction ($\redm{}{}$) on the input expression step by step until it becomes a normal form, while outputting those intermediate expressions that belong to \m{DisplayableExp}. This algorithm is more explicit and simpler than the traditional one. Note that $\redm{}{}$ is executed recursively on the subexpressions, which provides rooms for further optimization (see in Section \ref{mark:optimize}).
%, because the current description is easier to understand.)

\subsection{Correctness}
\label{mark:correct}

Following the traditional resugaring approach \cite{resugaring,hygienic}, we show that our resugaring approach satisfied the following three properties:
(1)
\emph{Emulation}: For each reduction of an expression in our mixed language, it should reflect on one-step reduction of the expression totally desugared in the core language, or one step desugaring on a syntactic sugar;
(2)
\emph{Abstraction}: Only displayable expressions (\m{DisplayableExp}) defined in our mixed language appear in our resugaring sequences; and
(3)
\emph{Coverage}: No syntactic sugar is desugared before its sugar structure should be destroyed in the  core language.

\subsubsection{Emulation} This is the basic property for a correct resugaring. Since our desugaring does not change an expression after it is totally desugared, what we need to show is that a non-desugaring reduction in the mixed language is exactly the same as reduction which should appear after the expression is totally desugared in the core language. Note that for \m{CoreExp}, the reduction is exactly same as reduction in core language, unless a subexpression headed with \m{SurfHead} is to be reduced recursively by $\redm{}{}$, what we need to prove is the following lemma. We use \m{fulldesugar}(\m{exp}) to represent the expression after \m{exp} totally desugared in this section.

\begin{lemma}[Emulation]
\label{lemma1}
For~\m{exp} = $(\m{SurfHead}~e_1~\ldots~e_i~\ldots~e_n)$, if~$\redm{\m{exp}}{\m{exp'}}$ and \m{fulldesugar}(\m{exp}) $\not=$ $\m{fulldesugar}(\m{exp'})$,~then $\redc{\m{fulldesugar}(\m{exp})}{\m{fulldesugar}(\m{exp'})}$.

\end{lemma}

\begin{lemma}
\label{lemma2}

For~\m{exp} = $(\m{SurfHead}~e_1~\ldots~e_i~\ldots~e_n)$, if $\redc{}{}$ reduces $\m{fulldesugar}(\m{exp})$ at the expression original from $e_i$ in one step, then $\redm{}{}$ will reduce \m{exp} at $e_i$.

%then $\redc{\m{fulldesugar}(\m{Exp})}{\m{fulldesugar}(\m{Exp'})}$
\end{lemma}

Correctness of lemma \ref{lemma2} is obvious, because the $\redm{}{}$ always expands the outermost sugar to check which exact subexpression is to be reduced by $\redc{}{}$. And if the expansion of outermost sugar is not enough to get the location of reduced subexpression, the $\redm{}{}$ will be called recursively, so finally get the correct location.
\begin{proof}[Proof of Lemma \ref{lemma1}]
We need to consider two cases: reduction rules \m{SurfRed1} and \m{SurfRed2}, because they are rules for expression headed with \m{SurfHead}.

For \m{SurfRed1} rule, $\redm{(\m{SurfHead}~e_1~\ldots~e_i~\ldots~e_n)}{(\m{SurfHead}~e_1~\ldots~e_i''$$\ldots~e_n)}$, where $\redm{e_i}{e_i''}$. 
If $\m{fulldesugar}(e_i)$ = $\m{fulldesugar}(e_i'')$, that means the i-th subexpression only involves desugaring, then \m{fulldesugar}(\m{exp}) = \m{fulldesugar}(\m{exp'}). If $\m{fulldesugar}(e_i)$ $\not=$ $\m{fulldesugar}(e_i'')$, non-desugaring reduction occurs at the subexpression, then what we need to prove is that, $\redc{\m{fulldesugar}(\m{exp})}{\m{fulldesugar}(\m{exp'})}$, that means the one-step reduction in $\redm{}{}$ shows what exactly reduced in the totally desugared expression. Note that the only difference between \m{exp} and \m{exp'} is at the i-th subexpression. Because of lemma \ref{lemma2}, the $\redm{}{}$ reduces \m{exp} at the correct subexpression, what we need to prove is $\redc{\m{fulldesugar}(e_i)}{\m{fulldesugar}(e_i'')}$, that is the emulation of the subexpression $e_i$. So the equation can be proved recursively.

For \m{SurfRed2} rule, \m{exp'} is \m{exp} after the outermost sugar desugared. So the expressions after totally desugared are equal, that is \m{fulldesugar}(\m{exp}) =
\m{fulldesugar}(\m{exp'}).
\end{proof}

%In summary, our resugaring approach satisfies emulation property.

\subsubsection{Abstraction}
\label{mark:abs}
The correctness of abstraction is obvious, because the resugaring algorithm checks whether the intermediate step is of \m{DisplayableExp} before outputting. The abstraction property is related to why we need resugaring: sugar expressions become unrecognizable after desugaring. So different from the abstraction of the traditional approach which only shows the changes of original expression, our approach defines the abstraction by catalog of the expressions in the mixed language. Thus, users can decide which kinds of expressions are recognizable, so that resugaring sequences will be abstract enough. For example, a recursive sugar expression's resugaring sequences will show useful information during the execution. Lazy resugaring, as the key idea of our approach, makes any intermediate steps retain as many sugar structures as possible, so the abstraction property of our approach does achieve a good abstraction.

One may ask what if we just want to display core language's expression originated from the input expression (just as what the traditional approach does). The solution is that we can write surface mirror term for core expressions. For example, if we want to show resugaring sequences of \Code{(and (if (and \#t \#f) ...) ...)} without displaying \m{if} expression produced by \m{and}, we can write a \m{IF} term as \m{CommonExp}, with the same evaluation rules as \m{if}. Then just input \Code{(and (IF (and \#t \#f) ...) ...)} to get the resugaring sequences. It is really flexible.

\subsubsection{Coverage}
The coverage property is important, because resugaring sequences are useless if losing intermediate steps. By lazy desugaring, it becomes obvious, because there is no chance to lose. In Lemma \ref{lemma3}, we want to show that our reduction rules in the mixed language is \emph{lazy} enough. Because it is obvious, we only give a proof sketch here.
\begin{lemma}[Coverage]
\label{lemma3}
A syntactic sugar only desugars when \emph{necessary} during the reduction of the mixed language, that means after a core reduction on the fully-desugared expression, the sugar's structure is destroyed.
\end{lemma}


\begin{proof}[Proof sketch of Lemma \ref{lemma3}]
From Lemma \ref{lemma2}, we know the $\redm{}{}$ recursively reduces an expression at correct subexpression, or the $\redm{}{}$ will expand the outermost sugar (of the current expression processed by a one-step try) in rule \m{SurfRed2}. Note that \m{SurfRed2} is the only reduction rule to desugar sugars directly (other rules only desugar sugars when recursively call \m{SurfRed2}), we can prove the lemma recursively if \m{SurfRed2} is \emph{lazy} enough.

In \m{SurfRed2} rule, we firstly expand the outermost sugar and get a temp expression with the structure of the outermost sugar. Then when we recursively call $\redm{}{}$, the reduction result shows the structure has been destroyed, so the outermost sugar has to be desugared. Since the recursive reduction of a terminable (some bad sugars may never stop which are pointless) sugar expression will finally terminate, the lemma can be proved recursively.
\end{proof}
