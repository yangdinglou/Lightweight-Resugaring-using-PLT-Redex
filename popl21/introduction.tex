%!TEX root = ./main.tex
\section{Introduction}

%What is the research background and and what motivate you to do this research?

%What is the research issue and how the issue has been addressed so far?

%What is the remained research problem and how challenge it is?

%What is your key idea (insight) of your solution to be discussed in this paper?

%What are the three main technical contributions of this paper?

%The rest of the paper is organized as follows. ...



Syntactic sugar, first coined by Peter J. Landin in 1964 \cite{syntacticsugar}, was introduced to describe the surface syntax of a simple ALGOL-like programming language which was defined semantically in terms of the applicative expressions of the core lambda calculus. It has proved to be very useful for defining domain-specific languages (DSLs) and extending languages \cite{FellFFKBMT18,CulpFFK19}.
Unfortunately, when syntactic sugar is eliminated by transformation, it obscures the relationship between the user’s source program and the transformed program.



Resugaring is a powerful technique to resolve this problem \cite{resugaring,hygienic}. It  can automatically convert the evaluation sequences of desugared expression in the core language into representative sugar's syntax in the surface language.
As demonstrated in Section \ref{sec2},
the key idea in this resugaring is "tagging" and "reverse desugaring": it tags each desugared core term with the corresponding desugared rule, and follows the evaluation steps in the core language but keep applying the desugaring rules reversibly as much as possible to find surface-level representations of the tagged core terms.

While it is natural to do resugaring by reverse desugaring of tagged core terms, it introduces complexity and inefficiency.
\begin{itemize}
\item {\em Impractical to handle recursive sugar}. While tagging is used to remember the position of desugaring so that reverse desugaring can be done at the correct position when a desugared core expression is evaluated, it  becomes very tricky and complex when recursive sugars are considered. Moreover, it can only handle the recursive sugar which can be written by pattern-based desugaring rules \cite{resugaring}.%\todo{pattern-based?}

\item {\em Complicated to handle hygienic sugar}. For reverse desugaring, we need to match part of the core expression on the RHS (which could be much larger than LHS) of the desugaring rule and to get the surface term by substitution. But when a syntactic sugar introduces variable bindings, this match-and-substitute turns out to be very complex if we consider local bindings (hygienic sugars) \cite{hygienic}.

\item {\em Inefficient in reverse desugaring.} It needs to keep checking whether reverse desugaring is applicable during the evaluation of desugared core expressions, which is costive. Moreover, the match-and-substitute for reverse desugaring is expensive particularly when the core term is large.

\end{itemize}

In this paper, we propose a novel approach to resugaring, which does not use tagging and reverse desugaring at all.
The key idea is "lazy desugaring", in the sense that desugaring is delayed so that the reverse application of desugaring rules becomes unnecessary.
To this end, we consider the surface language and the core language as one language, and reduce expressions in such a smart way that all resugared terms can be fully produced based on the reduction rules in the core language and  the desugaring rules for defining syntactic sugars. To gain more efficiency, we propose an automatic method to compress a sequence of core expression reductions into a one-step reduction of the surface language, by automatically deriving evaluation rules of the syntactic sugars based on the syntactic sugar definition and evaluation rules (consist of the context rules and reductions rules) of the core language.


Our main technical contributions can be summarized as follow.
\begin{itemize}
\item We propose a novel approach to resugaring by lazy desugaring, where reverse application of desugaring rules becomes unnecessary. We recognize a sufficient and necessary condition for a syntactic sugar to be desugared, and propose a reduction strategy, based on evaluator of the core languages and the desugaring rules, which is sufficient to produce all necessary resugared terms on the surface language. We prove the correctness of our approach.



\item We show that evaluation rules for syntactic sugars can be fully derived for a wide  class of desugaring rules and the evaluation rules of the core language. We design an inference automaton to capture symbolic execution of a syntactic sugar based on the evaluation and desugaring rules, and derive evaluation rules for new syntactic sugars from the inference automaton. These evaluation rules, if derivable, can be seamlessly used with lazy desugaring to make resugaring more efficient.



\item We have implemented a system based on the new resugaring approach. It is much more efficient than the traditional approach, because it completely avoids unnecessary complexity of the reverse desugaring. It is more powerful in that it cannot only deal with all cases (such as hygienic and simple recursive sugars) published so far \cite{resugaring,hygienic}, but can do more allowing more flexible recursive sugars. All the examples in this paper have passed the test of the system.


\end{itemize}

The rest of our paper is organized as follows. We start with an overview of our approach in Section \ref{sec2}. We then discuss the core of resugaring by lazy desugaring in Section \ref{sec3}, and automatic derivation of evaluation rules for syntactic sugars in Section \ref{sec:ruleDerivation}. We describe the implementation and show some applications as case studies in Section \ref{sec5}. We discuss related  work in Section \ref{sec6}, and conclude the paper in Section \ref{sec7}.
