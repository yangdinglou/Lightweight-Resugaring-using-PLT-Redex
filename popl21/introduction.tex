%!TEX root = ./main.tex
\section{Introduction}

%What is the research background and and what motivate you to do this research?

%What is the research issue and how the issue has been addressed so far?

%What is the remained research problem and how challenge it is?

%What is your key idea (insight) of your solution to be discussed in this paper?

%What are the three main technical contributions of this paper?

%The rest of the paper is organized as follows. ...

%Domain-specific languages \cite{dsl} are becoming more and more important in our daily lives; the IFTTT apps and IOS's shortcuts use many DSLs to describing various  tasks to make our lives more convenient.
%The users of DSL are no longer limited to programmers, but people from all walks of life.(to be completed)

Syntactic sugar, first coined by Peter J. Landin in 1964 \cite{syntacticsugar}, was introduced to describe the surface syntax of a simple ALGOL-like programming language which was defined semantically in terms of the applicative expressions of the core lambda calculus. It has proved to be very useful for defining domain specific languages (DSLs) and extending languages \cite{FellFFKBMT18,CulpFFK19}.
Unfortunately, when syntactic sugar is eliminated by transformation, it obscures the relationship between the user’s source program and the transformed program.

%
% has an obvious problem. DSL based on syntactic sugars contains many components of its host language. Then its interpretation will be outside the DSL itself. The evaluation sequences of syntactic sugar expressions will contain many terms of the host language, which may confuse the users of DSL.

Resugaring is a powerful technique to resolve this problem \cite{resugaring,hygienic}. It  can automatically convert the evaluation sequences of desugared expression in the core language into representative sugar's syntax in the surface language.
As demonstrated in Section \ref{sec2},
%to guarantee the properties of emulation, %(preserving correctness in semantics),
%abstraction,
%(keeping abstraction of syntax sugar),
%and coverage,
% (doing conversion as much as possible),
the key idea in this resugaring is "tagging" and "reverse desugaring": it tags each desugared core term with the corresponding desugared rule, and follows the evaluation steps in the core language but keep applying the desuagring rules reversibly as much as possible to find surface-level representations of the tagged core terms.

While it is natural to do resugaring by reverse desugaring of tagged core terms, it introduces complexity and inefficiency.
\begin{itemize}
\item {\em Tricky to handle reursive sugar}. While tagging is used to remember the position of desugaring so that reverse desugaring can be done at correct position when desugared core expression is evaluated, it  becomes very tricky and complex when recursive sugars are considered \cite{resugaring}.

\item {\em Complicated to handle ygienic sugar}. For reverse desugaring, we need to match part of the core expression on the RHS of the desugar rule and to get the surface term by substitution. This match-and-substitute turns out to be very complex if we consider local bindings (hygienic sugars) \cite{hygienic}.

\item {\em Inefficient in reverse desugaring.} It need to keep checking whether reverse desugaring is applicable during evaluation of desugared expression, which is very costive. Moreover, the match-and-substitute for reverse desugaring is costive particularly when the core term is big.

\end{itemize}

In this paper, we propose a novel approach to resugaring, which does not use tagging and reverse desugaring at all.
% are unnecessay if we delay desugaring.
%But we found the existing resugaring approach using match and substitution is kind of redundant. The biggest deficiency of existing resugaring method is that the syntactic sugars in an expression have to fully desugar before evaluation. This limits the processing ability of the method. Moreover, it limits the complexity of getting the resugaring sequences. If we need to resugar a very huge expression, the match and substitution processes will cost so much. Also, processing of hygienic macros is a little bit complex due to the extra data structure. Finally, we found the existing approach only assumes a stepper for core language, when the semantics of core languages can be got in some cases. We want to figure out how the semantics of core language will help.
%
%In this paper, We propose a lightweight approach to get resugaring sequences based on syntactic sugars. The key idea of our approach is---syntactic sugar expression only desugars at the point that it have to desugar. We guess that we don't have to desugar the whole expression at the initial time of evaluation under the premise of keeping the properties of expression.
%
%In this paper, we propose an resugaring approach by lazy dusugaring mixed with a dynamic approach and a static approach.
The key idea is "lazy desugaring", in the sense that desuagring is delayed so that the reverse application of desugaring rules become unnecessary.
%of the whole approach is---syntactic sugar expressions only desugar at the point they have to desugar, which is what the word "lazy" means. It would be correct for resugaring if we can prove the whole sugar expressions will keep the properties by such lazy processes.
We consider the surface language and the core language as one language, and reduce expressions dynamically either by the reduction rules in the core language or by the desugaring rules for defining syntactic sugars. To gain more efficiency, we can make a shortcut of a sequence of core expression reduction to a one-step reduction of the surface language, by automatically deriving reduction rules on the surface language from those on the core language.
%
%The dynamic approach uses the reduction semanticsof core language to decide whether desugaring the sugar. The static approach uses the reduction semantics \cite{reduction}  of core language to get reduction semantics of surface language based on sugars' syntax, then execute the syntactic sugar programs on the surface's semantics.
% The context rules of surface language decide which subexpression can be reduced, or desugaring is necessary because of the reduction rules.

Our main technical contributions can be summarized as follow. \todo{The following contributions will be revised later.}
\begin{itemize}
\item {\em A mixture approach of resugaring.} We introduce an mixture of two different resugaring approachs to combine the advances of following approaches. The lazy dusugaring is common feature of two approaches, which give each approach some good properties.
\item {\em A lightweight but powerful dynamic approach.} The dynamic approach we proposed is based on core language's reduction semantics. It takes surface language and core language as a whole, then decided whether expanding the sugars or reducing the subexpressions according to properties that make the resugaring correct. Thus, it is lightweight because many match and substitution processes can be omitted. We test the dynamic approach on many applications. The result shows that in addition to handle what existing work can handle, our dynamic approach can process recursive sugar easily, which makes it powerful. And the rewriting system based on reduction semantics makes it possible to write syntactic sugar easily.
\item {\em An independent and efficient static approach.} The static approach we proposed also used core language's reduction semantics. But instead of executing at the level of core language, we turn the core language's semantics into automata. Then for each syntactic sugar, we would generate the surface language's semantics without depending on some rules in core language. (some meta-functions may be necessary.) Thus, it is efficient because many steps in core language can be omitted. todo: complete
\item {\em Correctness}.
\end{itemize}

We have implemented lazy desugaring and automatic derivation of reduction rules for syntactic sugars. All the example in this paper have passed the test of the system.

The rest of our paper is organized as follow. We start with an overview of our approach in Section \ref{sec2}. We then discuss the core of resuarging by lazy desugaring in Section \ref{sec3}, and automatic derivation of reduction rules for syntactic sugars in Section \ref{sec4}. We discuss relative work in Section \ref{sec5}, and conclude the paper in Section \ref{sec6}.
