%!TEX root = ./main.tex
\section{Introduction}

%What is the research background and and what motivate you to do this research?

%What is the research issue and how the issue has been addressed so far?

What is the remained research problem and how challenge it is?

%What is your key idea (insight) of your solution to be discussed in this paper?

What are the three main technical contributions of this paper?

The rest of the paper is organized as follows. ...

Domain-specific language\cite{dsl} is becoming useful for people's daily tasks. For example, the IFTTT app and IOS's shortcuts designed DSLs describing some tasks to make our lives more convenient. So the users of DSL are no longer limited to programmers, but people from all walks of life.(to be completed)

Syntactic sugar\cite{syntacticsugar}, as a simple ways design DSL, has a obvious problem. DSL based on syntactic sugars contains many components of its host language. Then its interpretation will be outside the DSL itself. It will be better if we can get interpreter of DSL without host language's components, when given host language's interpreter and DSL's rewriting rules. It's important for DSL becoming a real programming language---its interpreter should not depend on something outside itself. Then the real language will be more concise during execution, or some program analysis, optimization tasks.

There is an existing work---resugaring\cite{resugaring}\cite{hygienic}, which partially solved the problem upon. It lifts the evaluation sequences of desugared expression to sugar's syntax. The evaluation sequences shown by resugaring will not contain components of host language. But we found the resugaring method using match and substitution is kind of redundant. Resugaring also cannot perfectly solve some syntactic sugar's feature, such as recursive sugar, higher-order sugar.  (todo: static approachs' advantages)

(challenge)


We propose two diffenent approachs to get better interpreters for DSLs based on syntactic sugars. Initially, our work focused on improving current resugaring method(dynamic approach). After finishing that, we found the dynamic approach could be abstracted to evalation rules of DSL(static approach). 

The key idea of the dynamic approach is---syntactic sugar expression only desugars at the point that it have to desugar. We guess that we don't have to desugar the whole expression at the initial time of evaluation under the premise of keeping the properties of expression. (todo: static approach insight)

2