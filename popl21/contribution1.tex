%!TEX root = ./main.tex
\section{Lightweight Algorithm}
\label{sec3}

\subsection{Language setting}

\subsubsection{Grammatical restrictions}
\hfill\\
Firstly, the whole language should restrict to tree-structured disjoint expression.

\begin{Def}[disjoint]
For every sub-expression in a expression, its reduction rule is decided by itself.
\end{Def}

This restriction limits the scope of language. Every sub-expression must have no side effect. We will discuss more on side effect in ...

\begin{Def}[tree-structured]
The grammar of the whole language is defined as follow.
\[
\begin{array}{rcl}
\mbox{Exp} &::=& (\mbox{Headid}~\mbox{Exp}*)\\
&|& \mbox{Value}\\
&|& \mbox{Variable}
\end{array}
\]
\end{Def}

The grammatical restrictions give our language a similiar property as church-rosser theorem for lambda calculus. 

todo:church-rosser?

\subsubsection{Context restrictions}
\hfill\\
For expressions in CoreLang, the context rule should restrict it to have only one reduction path. The context rules can limit the order of evaluation. This restriction is normal, because a program in general-purposed language should have only one execution path.\label{mark:ctx}

For expressions in SurfLang, context rules should allow every sub-expressions reduced. It's the same as full-$\beta$ reduction.

\subsubsection{Restriction of syntactic sugar}
\hfill\\
The form of syntactic sugar is as follow.
\fbox{
$(\mbox{Surfid}\;e_{1}\;e_{2}\;\ldots)$ ~→~ $(\mbox{Headid}\; \ldots)$
}

An counter example of this restriction is $(\mbox{Surfid}\;\ldots\;(e1\;e2)\ldots))$ in LHS. It's for simpler algorithm form, and the expression ability of syntactic sugar will not be changed.

\begin{Def}[Unambiguous]
For every syntactic sugar expression, it can only desugar to one expression in CoreLang.
\end{Def}

\subsubsection{Grammar Description}
\hfill\\
In our language setting, we regard SurfLang and CoreLang as a whole language. The whole language is under restrictions above, and its grammar is defined as follow.

\begin{centering}
	\framebox[38em][c]{
		\parbox[t]{38em}{
			\[
			\begin{array}{rcl}
			\mbox{Exp} &::=& \mbox{DisplayableExp}\\
			&|& \mbox{UndisplayableExp}\\
			\end{array}
			\]
			\[
			\begin{array}{rcl}
			\mbox{DisplayableExp} &::=& \mbox{Surfexp}\\
			&|& \mbox{Commonexp}
			\end{array}
			\]

			\[
			\begin{array}{rcl}
			\mbox{UndisplayableExp} &::=& \mbox{Coreexp}\\
			&|& \mbox{OtherSurfexp}\\
			&|& \mbox{OtherCommonexp}
			\end{array}
			\]
			
			\[
			\begin{array}{rcl}
			\mbox{Coreexp} &::=& (\mbox{CoreHead}~\mbox{Exp}*)
			\end{array}
			\]
			
			\[
			\begin{array}{rcl}
			\mbox{Surfexp} &::=& (\mbox{SurfHead}~\mbox{DisplayableExp}*)
			\end{array}
			\]
			
			\[
			\begin{array}{rcl}
			\mbox{Commonexp} &::=& (\mbox{CommonHead}~\mbox{DisplayableExp}*)\\
			&|& \mbox{Value}\\
			&|& \mbox{Variable}
			\end{array}
			\]
			
			\[
			\begin{array}{rcl}
			\mbox{OtherSurfexp} &::=& (\mbox{SurfHead}~\mbox{Exp}*~\mbox{UndisplayableExp}~\mbox{Exp}*)
			\end{array}
			\]
			
			\[
			\begin{array}{rcl}
			\mbox{OtherCommonexp} &::=& (\mbox{CommonHead}~\mbox{Exp}*~\mbox{UndisplayableExp}~\mbox{Exp}*)
			\end{array}
			\]
		}
	}
\end{centering}

The difference between CoreLang and SurfLang is identified by $Headid$. But there are some terms in CoreLang should be displayed during evaluation. Or we need some terms to help us getting better resugaring sequences. So we defined {\bfseries Commonexp}, which origin from CoreLang, but can be displayed in resugaring sequences. The {\bfseries Coreexp} terms are terms with undisplayable CoreLang's Headid. The {\bfseries Surfexp} terms are terms with SurfLang's Headid and all sub-expressions are displayable. The {\bfseries Commonexp} terms are terms with displayable CoreLang's Headid, together with displayable sub-expressions. There exists some other expression during our resugaring process. They have Headid which can be displayed, but one or more subexpressions can't. They are UndisplayableExp.

\subsection{Algorithm defination}

Our lightweight resugaring algorithm is based on a core algorithm f. For every expression during resugaring process, it may have one or more reduction rules. The core algorithm f chooses the one that satisfies three properties of resugaring, then applies it on the given expression. The core algorithm f is defined as \ref{alg:f}.
\begin{algorithm}
	\caption{Core-algorithm f}
	\label{alg:f}     % 给算法一个标签,以便其它地方引用该算法
	\begin{algorithmic}[1]       % 数字 "1" 表示为算法显示行号的时候,每几行显示一个行号,如:"1" 表示每行都显示行号,"2" 表示每两行显示一个行号,也是为了方便其它地方的引用
		\REQUIRE ~~\\      % 算法的输入参数说明部分
		Any expression $Exp$=$(Headid~Subexp_{1}~\ldots~Subexp_{\ldots})$ which satisfies Language setting
		\ENSURE ~~\\     % 算法的输出说明
		$Exp'$ reduced from $Exp$, s.t. the reduction satisfies three properties of resugaring
		\STATE     Let $ListofExp'$ = $\{Exp'_{1}\;,Exp'_{2}~\ldots\}$
		\IF {$Exp$ is Coreexp or  Commonexp or OtherCommonexp}
		\IF {Lengthof($ListofExp'$)==0}
		\RETURN null; \hfill Case1
		\ELSIF {Lengthof($ListofExp'$)==1}
		\RETURN first($ListofExp'$); \hfill Case2
		\ELSE 
		\RETURN $Exp'_{i}$ = $(Headid~Subexp_{1}~\ldots~Subexp'_{i}~\ldots)$; //where i is the index of subexp which have to be reduced. \hfill Case3
		\ENDIF
		\ELSE 
		\IF {$Exp$ have to be desugared}
		\RETURN desugarsurf($Exp$); \hfill Case4
		\ELSE
		\STATE Let $DesugarExp$ = desugarsurf(Exp)
		\IF {$Subexp_{i}$ is reduced to $Subexp'_{i}$ during $f(DesugarExp)$}
		\RETURN $Exp'_{i}$ = $(Headid~Subexp_{1}~\ldots~Subexp'_{i}~\ldots)$; \hfill Case5
		\ELSE
		\RETURN $DesugarExp$; \hfill Case6
		\ENDIF
		\ENDIF
		\ENDIF
		
	\end{algorithmic}
\end{algorithm}

We briefly describe the core algorithm f in words.

For Exp in language defined as last section, try all reduction rules in the language, get a list of possible expressions ListofExp'=\{$Exp'_{1}$,$Exp'_{2}$,$\ldots$\}. 

Line 2-9 deal with the case when Exp has a CoreLang's Headid. When Exp is value or variable (line 3-4), ListofExp' won't have any element (not reducible). When Exp is of Coreexp or Commonexp (line 5-6, due to the context restriction of CoreLang, only one reduction rule can be applied. When Exp is OtherCommonexp (line 7-8), due to the context restriction of CoreLang, only one sub-expression can be reduced, then just apply core algorithm recursively on the sub-expression.

Line 10-21 deal with the case then Exp has a SurfLang's Headid. When Exp only has one reduction rule (line 11-12), the syntactic sugar has to desugar. If not, we should expand outermost sugar and find the sub-expression which should be reduced (line 14-16), or the sugar has to desugar (line 17-18).


Then, our lightweight-resugaring algorithm is defined as \ref{alg:lwresugar}.

\begin{algorithm}
	\caption{Lightweight-resugaring}
	\label{alg:lwresugar}     % 给算法一个标签,以便其它地方引用该算法
	\begin{algorithmic}[1]       % 数字 "1" 表示为算法显示行号的时候,每几行显示一个行号,如:"1" 表示每行都显示行号,"2" 表示每两行显示一个行号,也是为了方便其它地方的引用
		\REQUIRE ~~\\      % 算法的输入参数说明部分
		Surfexp $Exp$
		\ENSURE ~~\\     % 算法的输出说明
		$Exp$'s evaluation sequences within DSL
		\WHILE {$tmpExp$ = f($Exp$)}
		\IF {$tmpExp$ is empty}
		\RETURN
		\ELSIF {$tmpExp$ is Surfexp or Commonexp}
		\PRINT $tmpExp$;
		\STATE Lightweight-resugaring($tmpExp$);
		\ELSE 
		\STATE Lightweight-resugaring($tmpExp$);
		\ENDIF
		\ENDWHILE
		
	\end{algorithmic}
\end{algorithm}

The whole process executes core algorithm f, and output sequences which is of Surfexp or Commonexp.

\subsection{Proof of correctness}

First of all, because the difference between our lightweight resugaring algorithm and the existing one is that we only desugar the syntactic sugar when needed, and in the existing approach, all syntactic sugar desugars firstly and then executes on CoreLang.

Second, to prove convenience, define some terms.

$Exp~=~(Headid\;Subexp_{1}\;Subexp_{\ldots} \ldots)$ is any reducible expression in our language.

If we use the reduction rule that desugar Exp's outermost syntactic sugar, then the reduction process is called {\bfseries Outer Reduction}.

If the reduction rule we use reduce $Subexp_{i}$, where $Subexp_{i}$ is $(Headid_{i}~Subexp_{i1}~Subexp_{i\ldots} \ldots)$
\begin{itemize}
	\item If the reduction process is Outer Reduction of $Subexp_{i}$ = $(Headid_{i}~Subexp_{i1}~Subexp_{i\ldots} \ldots)$, then it is called {\bfseries Surface Reduction}.
	\item If the reduction process reduces $Subexp_{ij}$, then it is called {\bfseries Inner Reduction}.
\end{itemize}

{\bfseries Example:}

$(\mbox{if}\; \#t\; Exp_{1}\; Exp_{2})$ → $Exp1$ \hfill Outer Reduction

$(\mbox{if}\; (\mbox{And}\; \#t\; \#f)\; Exp_{1}\; Exp_{2})$ → $(\mbox{if}\; (\mbox{if}\; \#t\; \#f\; \#f)\; Exp_{1}\; Exp_{2})$ \hfill Surface Reduction

$(\mbox{if}\; (\mbox{And}\; (\mbox{And}\; \#t\; \#t)\; \#t)\; \#f)\; Exp_{1}\; Exp_{2})$ → $(\mbox{if}\; (\mbox{And}\; \#t\; \#t)\; Exp_{1}\; Exp_{2})$ \hfill Inner Reduction

\begin{Def}[Upper and lower expression]
For $Exp$=$(Headid\;Subexp_{1}\;Subexp_{\ldots} \ldots)$, $Exp$ is called {\bfseries upper expression}$,Subexp_{i}$is called {\bfseries lower expression}.
\end{Def}

We only need to prove that all the 6 cases of core algorithm f won't effect its properties. Case 1 and case 3 won't effect any properties, because it does what CoreLang should do.

\begin{proof}[Proof of Emulation]
\hfill\\
For case 4 and 6, desugaring won't change Emulation property, because desugaring and resugaring are interconvertible.

For case 2 and 5, our core algorithm reduces the sub-expression which should be reduced. So if applying core algorithm f on the subexpression satisfies emulation property, then this two cases satisfy. A recursive proof it is.
todo:case5
\end{proof}

\begin{proof}[Proof of Abstraction]
\hfill\\
It's true, because we only display the sequence which satisfies abstraction property.
\end{proof}

\begin{lemma}
If no syntactic sugar desugared before it has to, then coverage property is satisfied.
\end{lemma}

\begin{proof}[Proof of Lemma]
Assume that no syntactic sugar not necessarily expanded desugars too early, existing an expression in CoreLang

$Exp$ = $(Headid\;Subexp_{1}\;Subexp_{\ldots} \ldots)$ which can be resugared to

$ResugarExp'$ = $(Surfid\;Subexp'_{1}\;Subexp'_{\ldots}\ldots)$, and $ResugarExp'$ is not displayed during lightweight-resugaring. Then

\begin{itemize}
	\item Or existing
	
	$ResugarExp$=$(Surfid\;Subexp'_{1}\;\ldots\;Subexp_{i}\;Subexp'_{\ldots}\ldots)$ in resugaring sequences, such that the expression after $ResugarExp$ desugaring reduces to $Exp$, and the reduction reduces $ResugarExp$'s sub-expression $Subexp_{i}$. If so, outermost syntactic sugar of $ResugarExp$ is not expanded. So if $ResugarExp'$ is not displayed, then the sugar not necessarily expanded desugars too early, which is contrary to assumption.
	
	
	\item Or existing
	
	$ResugarExp$=$(Surfid'\;\ldots\;ResugarExp'\;\ldots)$ in resugaring sequences, such that the expression after $ResugarExp$ desugaring reduces to $Exp$, and $Exp$ is desugared from $ResugarExp'$'s sub-expression. If $ResugarExp'$ is not displayed, then the outermost syntactic sugar is expanded early, which is contrary to assumption.
	todo
	%使得$ResugarExp$解糖后得到的表达式单步规约得到$Exp$,且该$Exp$是从$ResugarExp$中的子表达式$ResugarExp'$解糖得到,说明此步单步规约不涉及$ResugarExp'$的规约。而如果不能展示$ResugarExp'$,则说明该语法糖在执行前面的序列被提前破坏了,也与假设矛盾。

\end{itemize}
\end{proof}

\begin{proof}[Proof of Coverage]
\hfill\\
For case 4 and 6, the syntactic sugar has to desugar.

For case 2 and 5, the reduction occurs in sub-expression of $Exp$. So if applying core algorithm f on the subexpression doesn't desugar syntactic sugars not necessarily expanded, then this two cases don't. A recursive proof it is.
\end{proof}