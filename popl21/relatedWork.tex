%!TEX root = ./main.tex
\section{Related Work}
\label{sec5}
%Explain the work that are related to your problem, and to your three contributions. 

\emph{Resugaring sequences }\cite{resugaring,hygienic} As we have discussed many times, the concept of resugaring is original from their work, by the main idea of "tagging" and "reverse desugaring". Our approach is more lightweight, powerful and effiecnt, as discussed before. In summary, we also find some common issues about resugaring.
\begin{itemize}
\item Side effects in resugaring. In the first paper of resugaring, they try a \m{letrec} sugar based on \m{set!} term in core language and get no intermediate steps. After trying some syntactic sugars which contain side effect, we would say a syntactic sugar including side-effect is bad for resugaring, because after a side effect takes effect, the desugared expression should never resugar to the sugar expression. Thus, we don't think resugaring is useful for syntactic sugars  including side effects, though it can be done by marking any expressions which have a side effect.
\item Hygienic resugaring. As we showed in both approaches, hygiene is easily and naturally resolved by lazy desugaring, because it may behave as what the sugar ought to express. The second paper of resugaring presents a DAG to solve the problem, which is a smart but not concise way.
\item Assumption on core language. The traditional resugaring and the dynamic approach both use a blackbox evaluator of core language, while the dynamic approach use the semantics of core language. We found that if given the semantics of core language, the resugaring will be more convienent. The blackbox evaluator in our dynamic approach will not need the extension, while the rules getting by our static approach is more express. 
\end{itemize}

The type resugaring work\cite{resugaringtype} indicates that it is possible to automatically construct surface language's semantics by unification. But after trying to do this as type resugaring does, we found it impossible because \todo{the reason}


{\bfseries Galois slicing for Imperative Functional Programs\cite{slicing}} is a work for dynamic analyzing functional programs during execution. The forward component of the Galois connection maps a partial input x to the greatest partial output y that can be computed from x; the backward component of the Galois connection maps a partial output y to the least partial input x from which we can compute y. Our approach used a similiar idea on slicing expressions and processing on subexpressions. The dynamic approach is like the forward component, so the method to handle side effects in functional programs may be useful for a better resugaring with side effects.

{\bfseries Macros as Multi-Stage Computations\cite{multistage}} is a work similar to lazy expansion for macros. Some other researches\cite{modularstaging} about multi-stage programming\cite{MSP} indicate that it is an useful idea for implementing domain-specific languages. Our resugaring approach combines the idea of multi-stage programming with syntactic sugars, which achieves a better approach. Macro systems in some language (such as Racket\cite{racket}) have support lazy expansion. Our dynamic approach is a combination of existing resugaring and lazy expansion, which achieves a more powerful approach.

\emph{Origin tracking}\cite{origintracking} is about tracking the origins of terms in rewriting system, which is similar to existing resugaring approaches. Our approach, as an unidirectional resugaring, is quite suitable for domain-specific languages. The reason is that syntactic sugars used to be seen as an extension of host language, while our approach regards them as components of a new language.

\emph{Ziggurat}\cite{Ziggurat} is a semantic-extension framework. It allows defining new macros with semantics based on existing terms in a language. It is quite useful for static analysis on macros. Instead of semantics based on core language, the reduction rules of sugar got by our static approach is independent of core language, which may be more concise for static analysis.

Addition to PLT Redex\cite{SEwPR} we used to engineer the semantics, there are some other semantics engineering tools\cite{dynsem,Ksemantic} which aim to test or verify the semantics of languages. The methods of these researches can be easily combined with our static approach.

% \subsection{Comments on resugaring}

% \subsubsection{Side effects in resugaring}
% \label{mark:side}
% The previous resugaring approach used to tried a $Letrec$ sugar and found no useful sequences shown. We explain the reason from the angle of side effects. We also used to try some syntactic sugars which contain side effect. We would say a syntactic sugar including side-effect is bad for resugaring, because after a side effect takes effect, the desugared expression should never resugar to the sugar expression. Thus, we don't think resugaring is useful for syntactic sugars  including side effects, though it can be done by marking any expressions which have a side effect.

% \subsubsection{Hygienic resugaring}As mentioned in Sec\ref{mark:hygienic}, our approaches can deal with hygienic resugaring without much afford as the existing approach\cite{hygienic}. (Of course with the help of core language's semantics, see in next discussion) The dynamic approach uses a trivial, not beautiful tricky to handle the hygienic macros, so that we decide to make the rewriting system hygienic instead. ($\#:binding-forms$ keyword in PLT Redex) But the static approach handle the hygienic macro very easily, by adding a substitution's hash table. The dynamic approach can also use this method, but a hygienic rewriting system is enough.

% \subsubsection{Assumption on CoreLang's evaluator}
% \label{mark:assumption} As mentioned in Sec , the work "resugaring" originated from has weaker assumption on the core language---it just required a stepper of core languages' expression, when our approach needed the whole reduction semantics. Thus, the intent of our resugaring is not a tool for supporting resugaring for languages, but a tool for implementing DSL better. We will discuss this in feature work for details.  