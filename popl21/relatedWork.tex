%!TEX root = ./main.tex
\section{Related Work}

%Explain the work that are related to your problem, and to your three contributions. 

The most related work is the series of resugaring\cite{resugaring,hygienic,resugaringtype,resugaringscope}. The first two work is about resugaring evaluation sequences, the third one is about resugaring scope rules, and the last one is about resugaring type rules. The whole series is for better syntactic sugar. Our approach considered to implement a method for better DSLs, then regarded core language and surf language as a whole language.

Macrofiction\cite{Macrofication} is for generating macros for programs automatically to refactor the codes. Our work may try resugaring some macros generated automatically to test the practicality of resugaring method. Since the setup of our tools is simple, it will be easy to intergrate the two tools.

Galois slicing for Imperative Functional Programs\cite{slicing} is a work for analyzing functional programs during execution. Our approach used a similiar idea on slicing programs and processing on sub-programs.

\subsection{Comments on resugaring}

\subsubsection{Side effects in resugaring}
\label{mark:side}
The previous resugaring approach used to tried a $Letrec$ sugar and found no useful sequences shown. We explain the reason from the angle of side effects. We also used to try some syntactic sugars which contain side effect. We would say a syntactic sugar including side-effect is bad for resugaring, because after a side effect takes effect, the desugared expression should never resugar to the sugar expression. Thus, we don't think resugaring is useful for syntactic sugars  including side effects, though it can be done by marking any expressions which have a side effect.

\subsubsection{hygienic resugaring}As mentioned in Sec\ref{mark:hygienic}, our approach can deal with hygienic resugaring without much afford (just another case in core algorithm). Compare to existing hygienic resugaring\cite{hygienic}, our approach is a more general approach. As we learned from the existing approach, it will also work if the rewriting system itself is hygienic, so is our approach. During implementing our tools, we found using $\#refers-to$ keyword of PLT Redex would get more concise intermediate process, so we just use it.

\subsubsection{assumption on CoreLang's evaluator}
\label{mark:assumption} As mentioned in Sec , the work "resugaring" originated from has weaker assumption on the core language---it just required a stepper of core languages' expression, when our approach needed the whole reduction semantics. Thus, the intent of our resugaring is not a tool for supporting resugaring for languages, but a tool for implementing DSL better. We will discuss this in feature work for details. 