%!TEX root = ./main.tex
\section{Related Work}
\label{sec6}
%Explain the work that are related to your problem, and to your three contributions.
\todo{TSL}

As discussed many times before, our work is much related to the pioneer work of resugaring in \cite{resugaring,hygienic}. The idea of "tagging" and "reverse desugaring" is a clear explanantion of "resugaring", but it becomes very complex when the rhs of the desugaring rule becomes complex. Our approach does not need reverse desuharing, and is more lightweight, powerful and efficient.
For hygienic resugaring, compared with approach of using DAG to solve the variable binding  problem in \cite{hygienic}, our approach of "lazy desugaring" is more natural, because it  behaves as what the sugar ought to express.

\ignore{
\begin{itemize}
\item Side effects in resugaring.\label{mark:side} In the first paper of resugaring, they try a \m{letrec} sugar based on \m{set!} term in core language and get no intermediate steps. After trying some syntactic sugars which contain side effect, we would say a syntactic sugar including side-effect is bad for resugaring, because after a side effect takes effect, the desugared expression should never resugar to the sugar expression. Thus, we don't think resugaring is useful for syntactic sugars  including side effects, though it can be done by marking any expressions which have a side effect.
\item Hygienic resugaring. As we showed in both approaches, hygiene is easily and naturally resolved by lazy desugaring, because it may behave as what the sugar ought to express. The second paper of resugaring presents a DAG to solve the problem, which is a smart but not concise way.
\item Assumption on core language. The traditional resugaring and the dynamic approach both use a blackbox evaluator of core language, while the dynamic approach use the semantics of core language. We found that if given the semantics of core language, the resugaring will be more convenient. The blackbox evaluator in our dynamic approach will not need the extension, while the rules getting by our static approach is more express.
\end{itemize}
}

Our work on reduction rule derivaiton for the surface language was inpired by the work of \emph{Type resugaring} \cite{resugaringtype}, which shows that it is possible to automatically construct the typing rules for the surface language by statically analyzing type inference trees. But it cannot be applied for our reduction rule derivation, because it has an assumption that there is only one rule for multi-branchs of one syntactic sugar \todo{I don't understand the meaning here.}, which is not the case for evaluation rules.
\todo{We may explain the key idea about how we deal with this assumption.}


\emph{Macros as multi-stage computations} \cite{multistage} is a work related to our lazy expansion for sugars. Some other researches \cite{modularstaging} about multi-stage programming \cite{MSP} indicate that it is useful for implementing domain-specific languages. However, multi-stage programming is a metaprogramming method, which mainly works for run-time code generation and optimization. In contrast, our lazy resugaring approach treat sugars as part of a mixed language, rather than seperate them by staging. Moreover, the lazy desugaring gives us chance to derive evaluation rules of sugars, which is a new good point compared to multi-stage programming.

Our work is related to the \emph{Galois slicing for imperative functional programs} \cite{slicing}, a work for dynamic analyzing functional programs during execution. The forward component of the Galois connection maps a partial input $x$ to the greatest partial output $y$ that can be computed from $x$; the backward component of the Galois connection maps a partial output $y$ to the least partial input $x$ from which we can compute $y$.
%Our approach used a similar idea on slicing expressions and processing on subexpressions.
This can also be considered as a bidirectional transformation \cite{bx,lens07} and the round-tripping between desugaring and resugaring in traditional approach. In contrast to these work, our resugaring approach is basically unidirectional, with a local bidirectional step
for a one-step try in our lazy desugaring. It should be noted that Galois slicing may be useful to handle side effects in resugaring in the future (for example, slicing the part where side effects appear).

Our method for derving reduction rule is influenced by work on \emph{origin tracking} \cite{origintracking}, which is to  track the origins of terms in rewriting system.
For desugaring rules in a good form as given in Section \ref{sec:ruleDerivation}, we can easily track the terms and make use of them for derivation of reduction rules. \todo{Add more related work about derivation of reductions rules by Zhichao.}
%
%When the tracking in their notation can be easily done for sugar whose rules can be derived automatically.

Our implementation is built upon the PLT Redex \cite{SEwPR}, a semantics engineering tool, but it is possible to implement our approach on other semantics engineeing tools such as those in \cite{dynsem,Ksemantic} which aim to test or verify the semantics of languages. The methods of these researches can be easily combined with our approach to implement more general rule derivation. \emph{Ziggurat} \cite{Ziggurat} is a semantic-extension framework, also allowing defining new macros with semantics based on existing terms in a language. It is should be useful for static analysis of macros.
%Instead of semantics based on core language, the reduction rules of sugar derived by our approach is independent of core language, which may be more concise for static analysis.

\ignore{
related to semantic-extension frameworks.

\emph{Ziggurat} \cite{Ziggurat} is a semantic-extension framework. It allows defining new macros with semantics based on existing terms in a language. It is quite useful for static analysis on macros. Instead of semantics based on core language, the reduction rules of sugar derived by our approach is independent of core language, which may be more concise for static analysis. In addition to PLT Redex\cite{SEwPR} which we used to engineer the semantics, there are some other semantics engineering tools \cite{dynsem,Ksemantic} which aim to test or verify the semantics of languages. The methods of these researches can be easily combined with our approach to implement more general rule derivation.
}

% \subsection{Comments on resugaring}

% \subsubsection{Side effects in resugaring}
% \label{mark:side}
% The previous resugaring approach used to tried a $Letrec$ sugar and found no useful sequences shown. We explain the reason from the angle of side effects. We also used to try some syntactic sugars which contain side effect. We would say a syntactic sugar including side-effect is bad for resugaring, because after a side effect takes effect, the desugared expression should never resugar to the sugar expression. Thus, we don't think resugaring is useful for syntactic sugars  including side effects, though it can be done by marking any expressions which have a side effect.

% \subsubsection{Hygienic resugaring}As mentioned in Sec\ref{mark:hygienic}, our approaches can deal with hygienic resugaring without much afford as the existing approach\cite{hygienic}. (Of course with the help of core language's semantics, see in next discussion) The dynamic approach uses a trivial, not beautiful tricky to handle the hygienic macros, so that we decide to make the rewriting system hygienic instead. ($\#:binding-forms$ keyword in PLT Redex) But the static approach handle the hygienic macro very easily, by adding a substitution's hash table. The dynamic approach can also use this method, but a hygienic rewriting system is enough.

% \subsubsection{Assumption on CoreLang's evaluator}
% \label{mark:assumption} As mentioned in Sec , the work "resugaring" originated from has weaker assumption on the core language---it just required a stepper of core languages' expression, when our approach needed the whole reduction semantics. Thus, the intent of our resugaring is not a tool for supporting resugaring for languages, but a tool for implementing DSL better. We will discuss this in feature work for details.
