% !TEX program = pdflatex
%!TEX spellcheck
%% For double-blind review submission, w/o CCS and ACM Reference (max submission space)
\documentclass[acmsmall,review,anonymous]{acmart}\settopmatter{printfolios=true,printccs=false,printacmref=false}
%% For double-blind review submission, w/ CCS and ACM Reference
%\documentclass[acmsmall,review,anonymous]{acmart}\settopmatter{printfolios=true}
%% For single-blind review submission, w/o CCS and ACM Reference (max submission space)
%\documentclass[acmsmall,review]{acmart}\settopmatter{printfolios=true,printccs=false,printacmref=false}
%% For single-blind review submission, w/ CCS and ACM Reference
%\documentclass[acmsmall,review]{acmart}\settopmatter{printfolios=true}
%% For final camera-ready submission, w/ required CCS and ACM Reference
%\documentclass[acmsmall]{acmart}\settopmatter{}


%% Journal information
%% Supplied to authors by publisher for camera-ready submission;
%% use defaults for review submission.
\acmJournal{PACMPL}
\acmVolume{1}
\acmNumber{CONF} % CONF = POPL or ICFP or OOPSLA
\acmArticle{1}
\acmYear{2018}
\acmMonth{1}
\acmDOI{} % \acmDOI{10.1145/nnnnnnn.nnnnnnn}
\startPage{1}

%% Copyright information
%% Supplied to authors (based on authors' rights management selection;
%% see authors.acm.org) by publisher for camera-ready submission;
%% use 'none' for review submission.
\setcopyright{none}
%\setcopyright{acmcopyright}
%\setcopyright{acmlicensed}
%\setcopyright{rightsretained}
%\copyrightyear{2018}           %% If different from \acmYear

%% Bibliography style
\bibliographystyle{ACM-Reference-Format}
%% Citation style
%% Note: author/year citations are required for papers published as an
%% issue of PACMPL.
\citestyle{acmauthoryear}   %% For author/year citations


%%%%%%%%%%%%%%%%%%%%%%%%%%%%%%%%%%%%%%%%%%%%%%%%%%%%%%%%%%%%%%%%%%%%%%
%% Note: Authors migrating a paper from PACMPL format to traditional
%% SIGPLAN proceedings format must update the '\documentclass' and
%% topmatter commands above; see 'acmart-sigplanproc-template.tex'.
%%%%%%%%%%%%%%%%%%%%%%%%%%%%%%%%%%%%%%%%%%%%%%%%%%%%%%%%%%%%%%%%%%%%%%


%% Some recommended packages.
\usepackage{booktabs}   %% For formal tables:
                        %% http://ctan.org/pkg/booktabs
\usepackage{subcaption} %% For complex figures with subfigures/subcaptions
                        %% http://ctan.org/pkg/subcaption

\usepackage{algorithm}
\usepackage{algorithmic}
\usepackage{proof}

\renewcommand{\algorithmicrequire}{\textbf{Input:}}

\renewcommand{\algorithmicensure}{\textbf{Output:}}

\newtheorem{Def}{Defination}[section]
\newtheorem{mythm}{Theorem}[section]
\begin{document}

%% Title information
\title%[Short Title]
{A lightweight resugaring approach based on reduction semantics}         %% [Short Title] is optional;
                                        %% when present, will be used in
                                        %% header instead of Full Title.
\titlenote{with title note}             %% \titlenote is optional;
                                        %% can be repeated if necessary;
                                        %% contents suppressed with 'anonymous'
\subtitle{Subtitle}                     %% \subtitle is optional
\subtitlenote{with subtitle note}       %% \subtitlenote is optional;
                                        %% can be repeated if necessary;
                                        %% contents suppressed with 'anonymous'


%% Author information
%% Contents and number of authors suppressed with 'anonymous'.
%% Each author should be introduced by \author, followed by
%% \authornote (optional), \orcid (optional), \affiliation, and
%% \email.
%% An author may have multiple affiliations and/or emails; repeat the
%% appropriate command.
%% Many elements are not rendered, but should be provided for metadata
%% extraction tools.

%% Author with single affiliation.
\author{First1 Last1}
\authornote{with author1 note}          %% \authornote is optional;
                                        %% can be repeated if necessary
\orcid{nnnn-nnnn-nnnn-nnnn}             %% \orcid is optional
\affiliation{
  \position{Position1}
  \department{Department1}              %% \department is recommended
  \institution{Institution1}            %% \institution is required
  \streetaddress{Street1 Address1}
  \city{City1}
  \state{State1}
  \postcode{Post-Code1}
  \country{Country1}                    %% \country is recommended
}
\email{first1.last1@inst1.edu}          %% \email is recommended

%% Author with two affiliations and emails.
\author{First2 Last2}
\authornote{with author2 note}          %% \authornote is optional;
                                        %% can be repeated if necessary
\orcid{nnnn-nnnn-nnnn-nnnn}             %% \orcid is optional
\affiliation{
  \position{Position2a}
  \department{Department2a}             %% \department is recommended
  \institution{Institution2a}           %% \institution is required
  \streetaddress{Street2a Address2a}
  \city{City2a}
  \state{State2a}
  \postcode{Post-Code2a}
  \country{Country2a}                   %% \country is recommended
}
\email{first2.last2@inst2a.com}         %% \email is recommended
\affiliation{
  \position{Position2b}
  \department{Department2b}             %% \department is recommended
  \institution{Institution2b}           %% \institution is required
  \streetaddress{Street3b Address2b}
  \city{City2b}
  \state{State2b}
  \postcode{Post-Code2b}
  \country{Country2b}                   %% \country is recommended
}
\email{first2.last2@inst2b.org}         %% \email is recommended


%% Abstract
%% Note: \begin{abstract}...\end{abstract} environment must come
%% before \maketitle command
\begin{abstract}
With the rapid development of computer science, domain-specific language (DSL) is quite useful in our daily life, not only for programmers or computer scientists, but for people from all walks of life. Syntactic sugar is a good way to implement embedded DSLs, because it can make good use of existing general-purposed language's feature. However, the evaluation sequences became unrecognizable after the sugar expression desugared. 


Resugaring is an method to solve the problem above. In this paper, we purposed a lightweight approach of resugaring based on reduction semantics---getting evaluation sequences without fully desugaring the whole syntactic sugar expression. We implement a tool based on our method using PLT Redex and test our approach on some applications. The results show that our lightweight approach can even deal with more syntactic sugar's feature.
\end{abstract}


%% 2012 ACM Computing Classification System (CSS) concepts
%% Generate at 'http://dl.acm.org/ccs/ccs.cfm'.
\begin{CCSXML}
<ccs2012>
<concept>
<concept_id>10011007.10011006.10011008</concept_id>
<concept_desc>Software and its engineering~General programming languages</concept_desc>
<concept_significance>500</concept_significance>
</concept>
<concept>
<concept_id>10003456.10003457.10003521.10003525</concept_id>
<concept_desc>Social and professional topics~History of programming languages</concept_desc>
<concept_significance>300</concept_significance>
</concept>
</ccs2012>
\end{CCSXML}

\ccsdesc[500]{Software and its engineering~General programming languages}
\ccsdesc[300]{Social and professional topics~History of programming languages}
%% End of generated code


%% Keywords
%% comma separated list
\keywords{Domain-specific Language, Syntactic Sugar, Interpreter}  %% \keywords are mandatory in final camera-ready submission


%% \maketitle
%% Note: \maketitle command must come after title commands, author
%% commands, abstract environment, Computing Classification System
%% environment and commands, and keywords command.
\maketitle

%!TEX root = ./main.tex
\section{Introduction}

%What is the research background and and what motivate you to do this research?

%What is the research issue and how the issue has been addressed so far?

%What is the remained research problem and how challenge it is?

%What is your key idea (insight) of your solution to be discussed in this paper?

%What are the three main technical contributions of this paper?

%The rest of the paper is organized as follows. ...

Domain-specific language\cite{dsl} is becoming useful for people's daily tasks. For example, the IFTTT app and IOS's shortcuts designed DSLs describing some tasks to make our lives more convenient. So the users of DSL are no longer limited to programmers, but people from all walks of life.(to be completed)

Syntactic sugar\cite{syntacticsugar}, as a simple ways design DSL, has a obvious problem. DSL based on syntactic sugars contains many components of its host language. Then its interpretation will be outside the DSL itself. The evaluation sequences of syntactic sugar expression will contain many terms of the host language, which may confuse the users of DSL.

There is an existing work---resugaring\cite{resugaring}\cite{hygienic}, which aimed to solve the problem upon. It lifts the evaluation sequences of desugared expression to sugar's syntax. The evaluation sequences shown by resugaring will not contain components of host language. But we found the resugaring method using match and substitution is kind of redundant. The biggest deficiency of existing resugaring method is that the syntactic sugars in an expression have to fully desugar before evaluation. This limits the processing ability of the method. Moreover, it limits the complexity of getting the resugaring sequences. If we need to resugar a very huge expression, the match and substitution processes will cost so much. Also, processing of hygienic macros is complex due to the extra data structure.

In this paper, We propose a lightweight approach to get resugaring sequences based on syntactic sugars. The key idea of our approach is---syntactic sugar expression only desugars at the point that it have to desugar. We guess that we don't have to desugar the whole expression at the initial time of evaluation under the premise of keeping the properties of expression. 

Initially, our work focused on improving current resugaring method. After finishing that, we found our lightweight resugaring approach could process some syntactic sugars' feature that current approach cannot do. Finally, we implement our algorithm using PLT Redex\cite{SEwPR} and test our approach on some applications. The result shows that our approach does handle more features of syntactic sugar.

In the rest of this paper, we present the technical details of our approach together with the proof of correctness. In details, the rest of our paper is organized as follow:

\begin{itemize}
\item An overview of our approach with some background knowledge.[sec \ref{sec2}]
\item The algorithm defination and proof of correctness.[sec \ref{sec3}]
\item The implementation of our lightweight resugaring algorithm using PLT Redex.[sec \ref{sec4}]
\item sth else?[sec \ref{sec5}]
\item Evaluation of our lightweight resugaring approach.[sec \ref{sec6}]
\end{itemize}

%!TEX root = ./main.tex
\section{Overview}
\label{sec2}


In this section, we give a brief overview of our approach. To be concrete, we will consider the following simple core language, defining boolean expressions using the \m{if} construct:
\[
\begin{array}{lllll}
e &::=& \m{CoreExp}\\
\m{CoreExp} &::=& \Code{(if~$e$~$e$~$e$)} &\note{// if construct}\\
& |& \true  & \note{// true value}\\
& |& \false & \note{// false value}
\end{array}
\]
The semantics of the language is simple, consisting of the following context rule to specify the computational order:
\[
\begin{array}{lcl}
C &:=& \Code{(if C $e$ $e$)}\\
&|&[\bigcdot] \qquad \note{//evaluation context's hole}\\
\end{array}
\]
and two reduction rules (the letter c means core):
\[
\centering
 \redc{(\m{if}~\true~e_1~e_2)}{e_1}  \qquad \redc{(\m{if}~\false~e_1~e_2)}{e_2}
\]
Assume that our surface language is defined by two syntactic sugars defined by:
%---\emph{and} sugar and \emph{or} sugar on the core language.
\[
\begin{array}{c}
\drule{(\m{And}~e_1~e_2)}{(\m{if}~e_1~e_2~\false)}\\
\drule{(\m{Or}~e_1~e_2)}{(\m{if}~e_1~\true~e_2)}
\end{array}
\]
Now let us demonstrate how to execute \Code{(And (Or \true~\false) (And \false ~\true))}, and get the following resugaring sequence by our approach.
{\small
\begin{Codes}
    (And (Or \true \false) (And \false \true))
\OneStep{ (And \true (And \false \true))}
\OneStep{ (And \false \true)}
\OneStep{ #f}
\end{Codes}
}

Our new resugaring approach eliminates costive "reverse desugaring" by "lazy desugaring", where a syntactic sugar will be expanded only when it is necessary. It consists of following three steps.

{\em Step 1: Calculating Context Rules for Sugars.}
While giving the evaluation rules of the core language, we can derive the following context rules of the surface language.
\[
\begin{array}{lcl}
C &:=& (\m{And}~C~e)\\
&|& (\m{Or}~C~e)\\
&|&[\bigcdot]\\
\end{array}
\]
From the context rules of \m{if}, we can find that the condition ($e_1$) is always evaluated first. Therefore, for expression \Code{(And $e_1$ $e_2$)} defined by syntactic sugar, $e_1$ is also evaluated first, which is the context rule of \m{And}. Similarly, we can calculate the context rule of \m{Or}.

{\em Step 2: Deriving Reduction Rules for the Mixed Language.}
We mix the surface language with the core language as in Fig. \ref{fig:mixexample}, where \m{CommonExp} means the expressions used in both the surface language and the core language both, and derive $\to_m$, a one-step reduction in our mixed language, from the the reduction rules of fhe core language and the given desugaring rules. By using $\to_m$, we can get the following evaluation sequence in the mixed language for
the program \Code{(And (Or \true~\false) (And \false~\true))}

{\footnotesize
\begin{Codes}
    (And (Or \true \false) (And \false \true))
\OneStep{ (And (if \true \true \false) (And \false \true))}
\OneStep{ (And \true (And \false \true))}
\OneStep{ (if \true (And \false \true) \false)}
\OneStep{ (And \false \true)}
\OneStep{ (if \false \true \false)}
\OneStep{ \false}
\end{Codes}
}


\begin{figure}[t]
\centering
\begin{subfigure}{\linewidth}{\footnotesize
    \begin{flushleft}
        \[
        \begin{array}{lll}
        e &::=& \m{CoreExp} \\
        &|&\m{SurfExp}\\
        &|&\m{CommonExp}\\
        \m{CoreExp} &::=& \Code{(if~$e$~$e$~$e$)}\\
        \m{SurfExp} &::=& \Code{(And~$e$~$e$)}\\
        &|&\Code{(Or~$e$~$e$)}\\
        \m{CommonExp} &::=& \true\\
        &|& \false\\
        \end{array}
        \]
    \end{flushleft}
    \caption{Syntax}
    \label{fig:mixsyntax}
}
\end{subfigure}
\begin{subfigure}{\linewidth}{\footnotesize
    \begin{flushleft}
        \[\footnotesize
        \begin{array}{lcl}
        C &:=& (\m{if}~C~$e$~$e$)\\
        &|& (\m{And}~C~$e$)\\
        &|& (\m{Or}~C~$e$)\\
        &|&[\bigcdot]\\
        \end{array}
        \]
        \end{flushleft}
    \caption{Context Rules}
    \label{fig:mixcontext}
}
\end{subfigure}

\begin{subfigure}{\linewidth}{\footnotesize
    \begin{flushleft}
        \[
        \begin{array}{c}
        \redm{(\m{And}~e_1~e_2)}{(\m{if}~e_1~e_2~\false)}\\
        \redm{(\m{Or}~e_1~e_2)}{(\m{if}~e_1~\true~e_2)}\\
        \redm{(\m{if}~\true~e_1~e_2)}{e_1}\\
        \redm{(\m{if}~\false~e_1~e_2)}{e_2}
        \end{array}
        \]
    \end{flushleft}
    \caption{Reduction Rules}
    \label{fig:mixreduction}
}
\end{subfigure}

\caption{Mixed Language Example}
\label{fig:mixexample}
\end{figure}


{\em Step 3: Removing Unnecessary Expressions.}
We can just keep the intermediate sequences without \m{Coreexp} in any sub-expressions to get the resugaring sequence above.

%Note that the context rules should restrict the computational order of a sugar expression's sub-expressions, thus we should let the context rules be correct---reflecting what should be executed in the desugared expression.
Note that as the goal of resugaring is to present the evaluation of sugar programs, we are given a way to clearly specify which expression should be outputted. For the example in this section, of course, the sugar \m{And} and \m{Or} should be outputted, and also the boolean values should be. So we set boolean values as \m{CommonExp} in Fig. \ref{fig:mixsyntax}, so that they can be displayed though they are in the core language. By clearly separating what should be displayed, we can always get the resugaring evaluation sequences we need (this is slightly different from the existing approach's setting, as we will discuss in Section \ref{mark:correctness}.)

%!TEX root = ./main.tex
\section{Dynamic Approach}

The goals of our dynamic approach is similar with Resugaring\cite{resugaring}\cite{hygienic}, that is, get evaluation sequences of surface language's expression using surface expression's syntax. However, their approach make it by converting evaluation sequences of desugared expressions into surface language's expression, using match and substitution on syntactic sugars' rule. We do this by{\bfseries not expanding syntactic sugar until necessary}. Our approach shows some better properties for implementing DSL.

\subsection{Language setting}

\[
\begin{array}{rcl}
\mbox{Exp} &::=& (\mbox{Headid}~\mbox{Exp}*)\\
&|& \mbox{Value}\\
&|& \mbox{Variable}
\end{array}
\]

\begin{center}
	\framebox[35em][l]{
		\parbox[t]{35em}{
			\[
			\begin{array}{rcl}
			\mbox{Exp} &::=& \mbox{Coreexp}\\
			&|& \mbox{Surfexp}\\
			&|& \mbox{Commonexp}\\
			&|& \mbox{OtherSurfexp}\\
			&|& \mbox{OtherCommonexp}
			\end{array}
			\]
			
			\[
			\begin{array}{rcl}
			\mbox{Coreexp} &::=& (\mbox{CoreHead}~\mbox{Exp}*)
			\end{array}
			\]
			
			\[
			\begin{array}{rcl}
			\mbox{Surfexp} &::=& (\mbox{SurfHead}~(\mbox{Surfexp}~|~\mbox{Commonexp})*)
			\end{array}
			\]
			
			\[
			\begin{array}{rcl}
			\mbox{Commonexp} &::=& (\mbox{CommonHead}~(\mbox{Surfexp}~|~\mbox{Commonexp})*)\\
			&|& \mbox{Value}\\
			&|& \mbox{Variable}
			\end{array}
			\]
			
			\[
			\begin{array}{rcl}
			\mbox{OtherSurfexp} &::=& (\mbox{SurfHead}~\mbox{Exp}*~\mbox{Coreexp}~\mbox{Exp}*)
			\end{array}
			\]
			
			\[
			\begin{array}{rcl}
			\mbox{OtherCommonexp} &::=& (\mbox{CommonHead}~\mbox{Exp}*~\mbox{Coreexp}~\mbox{Exp}*)
			\end{array}
			\]
		}
	}
\end{center}

\subsection{Algorithm}

\begin{algorithm}
	\caption{Core-algorithm f}
	\label{alg:f}     % 给算法一个标签,以便其它地方引用该算法
	\begin{algorithmic}[1]       % 数字 "1" 表示为算法显示行号的时候,每几行显示一个行号,如:"1" 表示每行都显示行号,"2" 表示每两行显示一个行号,也是为了方便其它地方的引用
		\REQUIRE ~~\\      % 算法的输入参数说明部分
		Any expression $Exp$=$(Headid~Subexp_{1}~\ldots~Subexp_{\ldots})$ which satisfies Language setting
		\ENSURE ~~\\     % 算法的输出说明
		$Exp'$ reduced from $Exp$, s.t. the reduction satisfies three properties of resugaring
		\STATE     Let $ListofExp'$ = $\{Exp'_{1}\;,Exp'_{2}~\ldots\}$
		\IF {$Exp$ is Coreexp or  Commonexp or OtherCommonexp}
		\IF {Lengthof($ListofExp'$)==0}
		\RETURN null; //\hfill Rule1.1
		\ELSIF {Lengthof($ListofExp'$)==1}
		\RETURN first($ListofExp'$); //\hfill Rule1.2
		\ELSE 
		\RETURN $Exp'_{i}$ = $(Headid~Subexp_{1}~\ldots~Subexp'_{i}~\ldots)$; //where i is the index of subexp which have to be reduced. \hfill Rule1.3
		\ENDIF
		\ELSE 
		\IF {$Exp$ have to be desugared}
		\RETURN desugarsurf($Exp$); //\hfill Rule2.1
		\ELSE
		\STATE Let $DesugarExp'$ = desugarsurf(Exp)
		\IF {$Subexp_{i}$ is reduced to $Subexp'_{i}$ during $f(DesugarExp')$}
		\RETURN $Exp'_{i}$ = $(Headid~Subexp_{1}~\ldots~Subexp'_{i}~\ldots)$; //\hfill Rule2.2.1
		\ELSE
		\RETURN desugarsurf($Exp$); //\hfill Rule2.2.2
		\ENDIF
		\ENDIF
		\ENDIF
		
	\end{algorithmic}
\end{algorithm}

\begin{algorithm}
	\caption{Lightweight-resugaring}
	\label{alg:lwresugar}     % 给算法一个标签,以便其它地方引用该算法
	\begin{algorithmic}[1]       % 数字 "1" 表示为算法显示行号的时候,每几行显示一个行号,如:"1" 表示每行都显示行号,"2" 表示每两行显示一个行号,也是为了方便其它地方的引用
		\REQUIRE ~~\\      % 算法的输入参数说明部分
		Surfexp $Exp$
		\ENSURE ~~\\     % 算法的输出说明
		$Exp$'s evaluation sequences within DSL
		\WHILE {$tmpExp$ = f($Exp$)}
		\IF {$tmpExp$ is empty}
		\RETURN
		\ELSIF {$tmpExp$ is Surfexp or Commonexp}
		\PRINT $tmpExp$;
		\STATE Lightweight-resugaring($tmpExp$);
		\ELSE 
		\STATE Lightweight-resugaring($tmpExp$);
		\ENDIF
		\ENDWHILE
		
	\end{algorithmic}
\end{algorithm}

%!TEX root = ./main.tex
\section{Contribution2 ...}
\label{sec4}

Explain your second technical contribution.
%!TEX root = ./main.tex
\section{Contribution3 ...}
\label{sec5}

Explain your third technical contribution.
\section{Evaluation}

Explain how your system is implemented and how the experiment is performed to evaluate your approach.
%!TEX root = ./main.tex
\section{Related Work}
\label{sec6}
%Explain the work that are related to your problem, and to your three contributions.
\todo{checking the sentense}

As discussed many times before, our work is much related to the pioneering work of \emph{resugaring} in \cite{resugaring}. The idea of "tagging" and "reverse desugaring" is a clear explanation of "resugaring", but it becomes very complex when the RHS of the desugaring rule becomes complex. Our approach does not need to reverse desugaring and is more powerful, and efficient.
For hygienic resugaring, compared with the approach of using DAG to solve the variable binding problem in \cite{hygienic}, our approach of "lazy desugaring" can achieve a kind of natural hygiene within our core language.



\emph{Macros as multi-stage computations} \cite{multistage} is a work related to our lazy expansion for sugars. Some other researches \cite{modularstaging} about multi-stage programming \cite{MSP} indicate that it is useful for implementing domain-specific languages. However, multi-stage programming is a metaprogramming method, which mainly works for run-time code generation and optimization. In contrast, our lazy resugaring approach treats sugars as part of a mixed language, rather than separate them by staging. Moreover, the lazy desugaring gives us a chance to derive evaluation rules of sugars, which is a new good point compared to multi-stage programming.

Our work is related to the \emph{Galois slicing for imperative functional programs} \cite{slicing}, a work for dynamic analyzing functional programs during execution. The forward component of the Galois connection maps a partial input $x$ to the greatest partial output $y$ that can be computed from $x$; the backward component of the Galois connection maps a partial output $y$ to the least partial input $x$ from which we can compute $y$.
%Our approach used a similar idea on slicing expressions and processing on subexpressions.
This can also be considered as a bidirectional transformation \cite{bx,lens07} and the round-tripping between desugaring and resugaring in the existing approach. In contrast to these works, our resugaring approach is basically unidirectional. 


There is a long history of hygienic macro expansion\cite{hygienicmacro}, and a formal specific hygiene definition was given \cite{10.5555/1792878.1792884} by specific the binding scopes of macros. another formal definition of the hygienic macro\cite{EssenceofHygiene} is based on nominal logic\cite{10.1007/s001650200016}. Instead of using the desugaring rule or something else to achieve hygiene, we use the lazy desugaring with the small core language to avoid hygienic problems in our approach.
%
%When the tracking in their notation can be easily done for sugar whose rules can be derived automatically.

Our implementation is built upon the PLT Redex \cite{SEwPR}, a semantics engineering tool, but it is possible to implement our approach on other semantics engineering tools such as those in \cite{dynsem,Ksemantic} which aim to test or verify the semantics of languages. The methods of these researches can be easily combined with our approach to implementing more general rule derivation. \emph{Ziggurat} \cite{Ziggurat} is a semantic extension framework, also allowing defining new macros with semantics based on existing terms in a language. It is should be useful for static analysis of macros.
%Instead of semantics based on core language, the reduction rules of sugar derived by our approach are independent of the core language, which may be more concise for static analysis.


%!TEX root = ./main.tex
\section{Conclusion}
\label{sec7}

%Summarize the paper, explaining what you have shown, what results you have achieved, and what future work is.

In this paper, we propose a novel resugaring approach using lazy desugaring. We design the approach based on a  core language, with a simple desugaring system. Our algorithm then can output the evaluation sequence in the surface syntax, given some syntactic sugars together with an input program. In our approach, the most important insight is delaying the expansion of syntactic sugars by calculating context rules (Section \ref{sec:language} and \ref{sec:algo}), which decide whether the mixed language should reduce the sub-expression by core's reduction rules or expand the sugar. We show that the system can handle a variety of syntactic sugars and can achieve better efficiency (Section \ref{mark:resugaringexample} and \ref{sec:implementation}). Moreover, the approach is flexible to make some extensions (Section \ref{sec:ext}).

We find the extensions may work if the core language and the syntactic sugar have some properties such like compositional, clear semantics, and unique computational order. So one possible future work of this is to extend the core language and the desugaring system with other components of language design like type system, analyzer, and optimizer. Also, we find it is possible to derive stand-alone evaluation rules for the surface language by means similar to how we calculate context rules, making it more convenient to develop domain-specific languages. This functionality can be added to future systems.


%% Acknowledgments
\begin{acks}                            %% acks environment is optional
                                        %% contents suppressed with 'anonymous'
  %% Commands \grantsponsor{<sponsorID>}{<name>}{<url>} and
  %% \grantnum[<url>]{<sponsorID>}{<number>} should be used to
  %% acknowledge financial support and will be used by metadata
  %% extraction tools.
  This material is based upon work supported by the
  \grantsponsor{GS100000001}{National Science
    Foundation}{http://dx.doi.org/10.13039/100000001} under Grant
  No.~\grantnum{GS100000001}{nnnnnnn} and Grant
  No.~\grantnum{GS100000001}{mmmmmmm}.  Any opinions, findings, and
  conclusions or recommendations expressed in this material are those
  of the author and do not necessarily reflect the views of the
  National Science Foundation.
\end{acks}


%% Bibliography
\bibliography{reference}


%% Appendix
\appendix
\section{Appendix}

Text of appendix \ldots

\end{document}
