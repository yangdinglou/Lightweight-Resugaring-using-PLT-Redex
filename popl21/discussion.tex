%!TEX root = ./main.tex
\section{Discussion}
\label{sec:discuss}

\subsection{Compare to previous work}

As mentioned many times before, the biggest difference between previous resugaring approach and our approach, is that our approach doesn't need to desugar the sugar expresssion totally. Thus, our approach has the following advantages compared to previous work.

\begin{itemize}
	\item {\bfseries Lightweight} As the example at sec\ref{mark:simple}, the match and substitution process searchs all intermediate sequences many times. It will cause huge cost for a large program. So out approach---only expanding a syntactic sugar when necessarily, is a lightweight approach.
	\item {\bfseries Friendly to hygienic macro} Previous hygienic resugaring approach use a new data structure---abstract syntax DAG, to process resugaring of hygienic macros. Our approach simply finds hygienic error after expansion, and gets the correct reduction instead. 
	\item {\bfseries More syntactic sugar features} The ability of processing recursive sugar is a superiority compared to previous work. The key point is that recursive syntactic sugar must handle boundary conditions. Our approach handle them easily by not necessarily desugaring all syntactic sugars. Higher-order functions, as an important feature of functional programming, was supported by many daily programming languages. So the ability on higher-order sugar is important. 
	\item {\bfseries Rewriting rules based on reduction semantics} Any syntactic sugar that can expressed by reduction semantics can be used in our approach. It will give more possible forms for constructing syntactic sugars. todo:example?
\end{itemize}

\subsection{Comments on resugaring}

\subsubsection{Side effects in resugaring}The previous resugaring approach used to tried a $Letrec$ sugar and found no useful sequences shown. We explain the reason from the angle of side effects. We also used to try some syntactic sugars which contain side effect. We would say a syntactic sugar including side-effect is bad for resugaring, because after a side effect takes effect, the desugared expression should never resugar to the sugar expression. Thus, we don't think resugaring is useful for syntactic sugars  including side effects, though it can be done by marking any expressions which have a side effect.

\subsubsection{hygienic resugaring}As mention in Sec\ref{mark:hygienic}, our approach can deal with hygienic resugaring without much afford (just another case in core algorithm). Compare to existing hygienic resugaring\cite{hygienic}, our approach is a more general approach. As we learned from the existing approach, it will also work if the rewriting system itself is hygienic, so is our approach. During implementing our tools, we found using $\#refers-to$ keyword of PLT Redex would get more concise intermediate process, so we just use it.