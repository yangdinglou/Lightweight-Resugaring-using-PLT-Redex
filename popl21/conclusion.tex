%!TEX root = ./main.tex
\section{Conclusion}
\label{sec6}

%Summarize the paper, explaining what you have shown, what results you have achieved, and what future work is.

In this paper, we purpose a new approach (see Fig \ref{fig:mixture}) or resugaring mixed with a dynamic apporach and static approach, which has some advances compared to existing approaches. The two approaches are seemingly similar in lazy desugaring. Essentially, we would see the static approach is the abstract(todo:another express?) of dynamic approach. In the dynamic approach, the most important part is \emph{reduction in mixed language} (see in sec \ref{mark:mixedreduction}), which decides whether reducing the subexpression or desugaring the outermost sugar. Reducing subexpressions are just the same as context rules in static approach; desugaring the outermost sugar is similar to reduction rules in static approach. However, the reduction rules is more convinent and efficent than dynamic resugaring, because the static approach evolves a process like abstract interpretation\cite{AbstractInterpretation}, then reduces many steps executed in core language. Moreover, the semantics got by static approach make it possible to do some optimization at the surface language level, which is important for implementing a DSL. In contrast, the dynamic approach is more powerful by supporting recursive sugars' resugaring. Besides, the rewriting based on reduction semantics makes the sugar represented in many ways.

As we mentioned before, the original intent of our research is finding a better method (or building a tool) for implementing DSL. We could see static approach is better for achieving the goal, because getting the semantics of DSL (based on syntactic sugar) will be very useful for applying any other techniques on the DSL. But it will be better if the defects of expressiveness in the static approach can be solved. So the first future work may be achieving a more powerful static approach as our dynamic approach. Then we will carefully design a core language for as the host language of our dream system and find a better type resugaring approach for the system. Finally, a general optimazation method for DSL in our system is needed.