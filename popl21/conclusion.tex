%!TEX root = ./main.tex
\section{Conclusion}
\label{sec7}

%Summarize the paper, explaining what you have shown, what results you have achieved, and what future work is.

In this paper, we purpose a lightweight, efficient and powerful resugaring approach by lazy desugaring. Essentially, we would see the derivation of rules is the abstract of the basic resugaring approach. In the basic approach, the most important part is \emph{reduction in mixed language} (see in sec \ref{mark:mixedreduction}), which decides whether reducing the subexpression or desugaring the outermost sugar. Reducing subexpressions are just the same as derivate context rules; desugaring the outermost sugar is similar to derivate reduction rules. However, the derivate evaluation rules is more convenient and efficient than the basic resugaring, because the derivation evolves a process like abstract interpretation\cite{AbstractInterpretation}, then reduces many steps executed in core language. Moreover, the semantics got by derivation make it possible to do some optimization at the surface language level, which is important for implementing a DSL. In contrast, the dynamic approach is more powerful by supporting recursive sugars' resugaring. Besides, the rewriting based on reduction semantics makes the sugar represented in many ways.

The original intent of our research is finding a better method (or building a tool) for implementing DSL. We could see derivate evaluation rules is better for achieving the goal, because getting the semantics of DSL (based on syntactic sugar) will be very useful for applying any other techniques on the DSL. But it will be better if the defects of expressiveness of sugar which the derivation can handle are improved. So we may be achieving a more powerful derivation which can recursive recursive sugar as a future work. \todo{describe defect of derivation}
% Then we will carefully design a core language for as the host language of our dream system and find a better type resugaring approach for the system. Finally, a general optimazation method for DSL in our system is needed.