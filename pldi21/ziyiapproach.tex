%!TEX root = ./main.tex

\section{Resugaring by Lazy Desugaring}
\label{sec3}

In this section, we present our new approach to resugaring. Different from the existing approach that clearly separates the surface from the core languages, we intentionally combine them as one mixed language, allowing free use of the language constructs in both languages. We will show that any expression in the mixed language can be evaluated in such a smart way that a sequence of all expressions that are necessary to be resugared can be correctly produced.

\subsection{Mixed Language for Resugaring}

\begin{figure}[t]
\begin{flushleft}
{\footnotesize
\[
\begin{array}{lllll}
\m{CoreExp} &::=& x  & \note{variable}\\
&~|~& c  & \note{constant}\\
&~|~& (\m{CoreHead}~\m{CoreExp}_1~\ldots~\m{CoreExp}_n) & \note{constructor}\\
\\
\m{SurfExp} &::=& x  & \note{variable}\\
&~|~& c  & \note{constant}\\
&~|~& (\m{SurfHead}~\m{SurfExp}_1~\ldots~\m{SurfExp}_n) & \note{sugar constructor}\\
\end{array}
\]
}
\end{flushleft}


	\caption{Core and Surface Expressions}
	\label{fig:expression}
\end{figure}

As a preparation for our resugaring algorithm, we define a mixed language that combines a core language with a surface language (defined by syntactic sugars over the core language). An expression in this language is reduced step by step by the evaluation rules for the core language and the desugaring rules for the syntactic sugars in the surface language. We assume that the evaluation of the core language is  compositional (as the definition in \cite{hygienic}), that is, for evaluation contexts $E_1$ and $E_2$, $E_1[E_2]$ is also an evaluation context.
\subsubsection{Core Language}


The evaluator of our core language is driven by evaluation rules (context rules and reduction rules), with three natural assumptions. First, the evaluation is deterministic, in the sense that any expression in the core language will be reduced by a unique reduction sequence (restricted by context rules). Second, evaluation of a sub-expression has no side-effect on other parts of the expression. Third, the context rules have no conditions, which means the following rules are not permitted.

{\footnotesize
\[
\begin{array}{lll}
\m{C}& ::= & (\m{notif}~[\bigcdot]~e_2~e_3)\\
&|& (\m{notif}~v_1~[\bigcdot]~e_3), \qquad \m{if}~(\m{equal}?~v_1~\m{\true})\\
&|& (\m{notif}~v_1~e_2~[\bigcdot]), \qquad \m{if}~(\m{equal}?~v_1~\m{\false})
\end{array}
\]}

The form of the expressions in our core language is defined in Fig. \ref{fig:expression}. It is a variable, a constant, or a (language) constructor expression. Here, $\m{CoreHead}$ stands for a language constructor such as $\m{if}$ and $\m{let}$. To be concrete, we will use the core language defined in Fig.  \ref{fig:core} to demonstrate our approach. It is a usual functional language and its semantics is defined by the context rules and the reduction rules. Here the [e/x] denotes capture-avoiding substitution.

\begin{figure*}[thb]
% \subcaptionbox{Syntax \label{fig:coresyntax}}[0.32\linewidth]{
% \begin{flushleft}
% \[
% {\footnotesize
% 		\begin{array}{lcl}
% 		\m{CoreExp} &::=& \Code{(CoreExp~CoreExp~...)} ~~\note{// apply}\\

% 		&|& \m{(if~CoreExp~CoreExp~CoreExp)} ~~\note{// condition}\\
% 		&|& \m{(let~((x~CoreExp)~...)~CoreExp)} ~~\note{// binding}\\
% 		&|& \m{(listop~CoreExp)} ~~\note{// first, rest, empty?}\\
% 		&|& \m{(cons~CoreExp~CoreExp)} ~~\note{// data structure of list}\\
% 		&|& \m{(arithop~CoreExp~CoreExp)} ~~\note{// +, -, *, /, >, <, =}\\
% 		&|& \m{x} ~~\note{// variable}\\
% 		&|& \m{value}\\
% 		\m{value} &::=& \m{($\lambda$~(x~...)~CoreExp)} ~~\note{// call-by-value}\\
% 		&|& \m{c} ~~\note{// boolean, number and list}
% 		\end{array}
% }
% \]
% \end{flushleft}

% }
\begin{subfigure}{0.45\linewidth}
\[
{\footnotesize
		\begin{array}{lcl}
		\m{CoreExp} &::=& \Code{(apply~CoreExp~CoreExp~...)}\\

		&|& \m{(if~CoreExp~CoreExp~CoreExp)} ~~\note{// condition}\\
		&|& \m{(let~(x~CoreExp))~CoreExp)} ~~\note{// binding}\\
		&|& \m{(listop~CoreExp)} ~~\note{// first, rest, empty?}\\
		&|& \m{(cons~CoreExp~CoreExp)} ~~\note{// data structure of list}\\
		&|& \m{(arithop~CoreExp~CoreExp)} ~~\note{// +, -, *, /, >, <, =}\\
		&|& \m{x} ~~\note{// variable}\\
		&|& \m{value}\\
		\m{value} &::=& \m{($\lambda$~(x~...)~CoreExp)} ~~\note{// call-by-value}\\
		&|& \m{c} ~~\note{// boolean, number and list}
		\end{array}
}
\]
\caption{Syntax}
\end{subfigure}
\begin{subfigure}{0.5\linewidth}
\[
{\footnotesize
		\begin{array}{lcl}
		\Code{C} &::=& \Code{(apply~value~...~C~CoreExp~...)}\\
		&|& \Code{(if~C~CoreExp~CoreExp)}\\
		&|& \Code{(let~(x~value)~CoreExp)}\\
		&|& \Code{(listop~C)}\\
		&|& \Code{(cons~C~CoreExp)}\\
		&|& \Code{(cons~value~C)}\\
		&|& \Code{(arithop~C~CoreExp)}\\
		&|& \Code{(arithop~value~C)}\\
		&|& [\bigcdot]\\[1.5em]
		\end{array}
}
\]
\caption{Context Rules}
\end{subfigure}

\begin{subfigure}{0.8\linewidth}
	\[
{\footnotesize
		\begin{array}{lcl}
		\Code{(($\lambda$~($x_1$~$x_2$~...)~CoreExp)~$\m{value}_1$~$\m{value}_2$~...)} &\redc{}{}& \Code{(($\lambda$~($x_2$~...)~CoreExp[$\m{value}_1$/$x_1$])~$\m{value}_2$~...)}\\

		\Code{(if~$\m{\true}$~$\m{CoreExp}_1$~$\m{CoreExp}_2$)}&\redc{}{}& \Code{$\m{CoreExp}_1$}\\
		\Code{(if~$\m{\false}$~$\m{CoreExp}_1$~$\m{CoreExp}_2$)}&\redc{}{}& \Code{$\m{CoreExp}_2$}\\
		\Code{(let~($x$~$\m{value}$)~CoreExp)}&\redc{}{}&\Code{CoreExp[\m{value}/$x$]}\\
		\ldots&&
		\end{array}
}
\]
	\caption{Part of Reduction Rules}
\end{subfigure}

\caption{A Core Language's Example}
\label{fig:core}
\end{figure*}



%For simplicity, we use the prefix notation. For instance, we write $\m{if-then-else}~e_1~e_2~e_3$, which would be more readable if we write $\m{if}~e_1~\m{then}~e_2~\m{else}~e_3$. In this paper, we may write both if it is clear from the context.

\subsubsection{Surface Language}

Our surface language is defined by a set of syntactic sugars, together with some elements in the core language, such as constants and variables. The expression of a surface language has a similar form as shown in Fig.  \ref{fig:expression}. To separate surface language's \m{Head} from that of the core language, we capitalize the first letter of surface language's \m{Head}.

%Here we just assume a simple desugaring system for a syntactic sugar expression. We will show how this approach can be combined with other complex desugaring.
A syntactic sugar \m{SurfHead} is defined by a desugaring rule in the following form

\[
\drule{(\m{SurfHead}~e_1~e_2~\ldots~e_n)}{\m{exp}} %\todo{\text{font size in drule}}
\]
where its left-hand-side (LHS) is a pattern and its left-hand-side (RHS) is an expression of the surface language or the core language. The LHS may be nested , so we can write sugars like \Code{(SurfHead ($e_1$ ($e_2$ $e_3$)) $\ldots$ $e_n$)}. And any pattern variable (e.g., $e_1$) in LHS appears only once in RHS. For instance, we may define syntactic sugar \m{And} by
\[
\drule{(\m{And}~e_1~e_2)}{(\m{if}~e_1~e_2~\m{\false})}.
\]
If we need to use a pattern variable multiple times in RHS, a \m{let} binding may be used (a normal way in syntactic sugar). To see the problem of multiple uses of variable, suppose we define a sugar as follow:
\[
\drule{(\m{Twice}~e_1)}{(+~e_1~e_1)}.
\]
If we execute \Code{(Twice (+ 1 1))}, it will first be desugared to \Code{(+ (+ 1 1) (+ 1 1))}, then reduced to \Code{(+ 2 (+ 1 1))} by one step. The sub-expression \Code{(+ 1 1)} has been reduced but should not be resugared to the surface, because the other \Code{(+ 1 1)} has not been reduced yet.
So we just use a \m{let} binding to resolve this problem. The RHS should be \Code{(let (x $e_1$) (+ x x))} in this case.


Note that in the desugaring rule, we relax the ordinary restriction that RHS must be a $\m{CoreExp}$. This makes it possible to define recursive sugars:
%, which We can use $\m{SurfExp}$ (more precisely, we allow the mixture use of syntactic sugars and core expressions) to define recursive syntactic sugars, as seen in the following example.
\[
\begin{array}{l}
\drule{(\m{Odd}~e)}{(\m{let}~(x~e)~(\m{if}~(>~x~0)~(\m{Even}~(-~x~1))~\m{\false}))}\\
\drule{(\m{Even}~e)}{(\m{let}~(x~e)~(\m{if}~(>~x~0)~(\m{Odd}~(-~x~1))~\m{\true}))}
\end{array}
\]

%As described above, we only assume the desugaring is a transformer without other helper function, thus there will be many kinds of ill-formed syntactic sugar which cannot desugared well (just as the \m{Odd}, \m{Even} sugars above, although can be processed by our lazy desugaring), or the semantics of the sugar cannot be defined clearly.

Without loss of generosity, we assume all desugaring rules are not overlapped in the sense that for any syntactic sugar in an expression, only one desugaring rule is applicable.


\subsubsection{Mixed Language}
\begin{figure}[t]
\begin{centering}
{\footnotesize
\[
			\begin{array}{lcl}
			\m{Exp} &::=& \m{DisplayableExp}\\
			&|& \m{MixedExp}\\
			\\
			\m{DisplayableExp} &::=& \m{(SurfHead~DisplayableExp~$\ldots$)}\\
			&|& \m{(CommonHead~DisplayableExp~$\ldots$)}\\
			&|& c\\
			&|& x\\
			\\
			\m{MixedExp} &::=& \m{(SurfHead~MixedExp~$\ldots$)}\\
			&|& \m{(CoreHead~MixedExp~$\ldots$)}\\
			&|& c\\
			&|& x\\
			\end{array}
			\]
}

\end{centering}
\caption{Our Mixed Language}
\label{fig:mix}
\end{figure}

Our mixed language for resugaring combines the surface language and the core language, described in Fig.  \ref{fig:mix}.
%
The differences between expressions in our core language and those in our surface language are identified by their \m{Head}. But there may be some expressions in the core language which are also used in the surface language for convenience, or to say, we need some core language's expressions to help us get better resugaring sequences. So we take \m{CommonHead} as a subset of the \m{CoreHead}, which can be displayed in resugaring sequences (just as the \m{CommonExp} in our Section \ref{sec2}). Then if any subexpression in an expression contains no \m{CoreHead} except for \m{CommonHead}, we should let them displayed during the evaluation process (named \m{DisplayableExp}). Otherwise, the expression should not be displayed. We just use a \m{MixedExp} expression to represent the expressions that are unnecessarily displayed for concision, and discuss more on \m{DisplayableExp}  in Section \ref{sec5}.

 % The \m{SurfExp} denotes an expression that have \m{SurfHead} and their subexpressions being displayable. The \m{CommonExp} denotes an expression with displayable \m{Head} (named \m{CommonHead}) in the core language, together with displayable subexpressions. There exist some other expressions during our resugaring process, which have displayable \m{Head}, but one or more of their subexpressions should not be displayed. They are of \m{UndisplayableExp}.



As an example, for the core language in Fig.  \ref{fig:core},
we may assume \m{arithop}, \m{$\lambda$} (call-by-value lambda calculus), \m{cons} as \m{CommonHead}, \m{if}, \m{let}, \m{listop} as \m{CoreHead} but out of \m{CommonHead}. This will allow \m{arithop}, \m{$\lambda$} and \m{cons} to appear in the resugaring sequences, and thus display more useful intermediate steps during resugaring.

In the mixed language some expressions with \m{CoreHead} may contain subexpressions with \m{SurfHead}. They will be treated as \m{CoreExp} in the mixed language but not \m{CoreExp} in the core language. In the mixed language, we process these expressions by the context rules of the core language, so that the reduction rules of the core language and the desugaring rules of surface language can be mixed as a whole
 (the $\redm{}{}$ in Fig. \ref{fig:mixexample}). For example, suppose we have the context rule of \m{if} expression\footnote{Another presentation of \Code{(if $[\bigcdot]$ e e)}. Use this here for convenience.}
\[
\infer{(\m{if}~e_1~e_2~e_3) \rightarrow (\m{if}~e_1'~e_2~e_3)}{e_1 \rightarrow e_1'}
\]
then if $e_1$ is a reducible expression in the core language, it will be reduced by the reduction rule in the core language: if $e_1$ is a \m{SurfExp}, it will be reduced by the desugaring rule of $e_1$'s \m{Head}; if $e_1$ is also a \m{CoreExp} which has non-core sub-expressions, a recursive reduction by $\redc{}{}$ is needed.


\subsection{Resugaring Algorithm}

Our resugaring algorithm works on the mixed language, based on the evaluation rules of the core language and the desugaring rules for defining the surface language. The process of  getting the resugaring sequence contains two separate parts.

\begin{enumerate}
\item Calculating the context rules of syntactic sugars.
\item Filtering \m{DisplayableExp} during the execution of the mixed language.
\end{enumerate}

\begin{algorithm}
	\caption{\m{calcontext}}
	\label{alg:f}     % 给算法一个标签,以便其它地方引用该算法
	\begin{algorithmic}[1]       % 数字 "1" 表示为算法显示行号的时候,每几行显示一个行号,如:"1" 表示每行都显示行号,"2" 表示每两行显示一个行号,也是为了方便其它地方的引用
		\REQUIRE ~~\\      % 算法的输入参数说明部分
		\Code{currentLHS = (SurfHead~$t_1$~$t_2$~$\ldots$~$t_n$)}\hfill \\
		\note{//where $t_i$ is $e$ or $v$(value).}\\
		\Code{currentContext = (Head~$\ldots~e_1'~e_2'~\ldots~e_m'$)} \\
		\note{//where $e_i'$ can be at any depth of sub-expressions.}\\
		\Code{currentIncal = $\{\ldots\}$} \note{//list of contexts in calculation.}
		\ENSURE ~~\\     % 算法的输出说明
		\Code{ListofRule, tmpLHS}
		\STATE     \Code{Let ListofRule = \{\}, tmpLHS = currentLHS, InCal = append(currentIncal, SurfHead)}
		\IF {$\not \exists~\text{contexts~rules~of}~\m{Head}$}
			\IF {$\exists$ \m{Head} in \m{InCal}}
				\RETURN error
			\ELSE
				\STATE \Code{ListofRule = append(ListofRule,}
				\STATE \qquad\Code{calcontext(\m{Head}.LHS,\m{Head}.RHS,InCal))}
			\ENDIF
		\ENDIF
		\STATE \Code{Let OrderList = $\{e_i',~e_j',~\ldots\}$}\hfill\\ \note{//RHS's computational order got by context rules}
		\FOR {\Code{subExp} in \m{OrderList}}
			\IF {$\exists i, s.t. e_i=\Code{subExp}$}
				\STATE \Code{ListofRule= append(ListofRule,}
				\STATE \qquad\Code{tmpLHS$[[\bigcdot]/e_i]$)}
			\ELSE
				\STATE \Code{Let recRule, recLHS = calcontext(}
				\STATE \qquad\Code{tmpLHS,~subExp,~Incal)}
				\STATE \Code{tmpLHS = recLHS}
				\STATE \Code{ListofRule = append(ListofRule, recRule)}
				\STATE {\bfseries break}\note{//means the RHS has to be destroyed.}
			\ENDIF
		\ENDFOR
		\RETURN \Code{ListofRule, tmpLHS}

	\end{algorithmic}
\end{algorithm}

\subsubsection{Context Rule Derivation}
Given a sugar \m{SurfHead} defined by
\[
\drule{(\m{SurfHead}~e_1~e_2~\ldots~e_n)}{(\m{Head}~\ldots~e_1'~e_2'~\ldots~e_m')}
\]
the context rule derivation is to infer which and in which order $e_i$ should be computed before this desugaring rule is applied. This information can be derived by analyzing computational order of each sub-expression in RHS of the desugaring rule. The reason we can derive the context rules of syntactic sugar is that, for any sugar's RHS, a reduction will either reduce on the $e_i$ of its $LHS$, or reduction on other component of RHS to destroy the RHS's form (which means the sugar has to be expanded). So we can trace the order before the destroying. The algorithm for this derivation, \texttt{calcontext}, is described as Algorithm \ref{alg:f}. Running
\[
 \Code{calcontext(\m{SurfHead}.LHS,~\m{SurfHead}.RHS, \{\})}
\]
will yield the context rules for \m{SurfHead}.

Before explaining the algorithm in detail, we use a simple example to illustrate the idea.
Consider a sugar \m{Sg0} defined by the following desugaring rule:
\[
\drule{\m{(Sg0}~e_1~e_2~e_3~e_4\m{)}}{\m{(+}~e_1~\m{(if}~e_2~e_3~e_4\m{))}}
\]
where we assume that $+$ computes its first argument and then its second argument before performing reduction, i.e., the context rules for $+$ are  $(+~[\bigcdot]~e_2)$ and $(+~v_1~[\bigcdot]~e_2)$. Now from $\m{(+}~e_1~\m{(if}~e_2~e_3~e_4\m{))}$, we can infer that $e_1$ should be computed first from the context rule of $+$, and then $e_2$ is computed from the context rule of the inner $\m{if}$, and finally it stops because no context rule is applicable for any remaining subexpression. Therefore we get the following context rules for \m{Sg0} as
$
(\m{Sg0}~[\bigcdot]~e_2~e_3~e_4),~(\m{Sg0}~v_1~[\bigcdot]~e_3~e_4).
$

Return to the algorithm. If \m{Head} is a \m{CoreHead}, for each context rule of the \m{Head} in order, we should just recursively make context rules for each hole, until a whole sub-expression is iterated. For instance of \m{Sg1}:
\[
\drule{\m{(Sg1}~e_1~e_2~e_3~e_4\m{)}}{\m{(+}~e_1~\m{(+}~e_2~\m{(+}~e_3~e_4\m{)))}}
\]
the sugar does not have to be expanded until all four sub-expressions are reduced to value, as demonstrated in the following execution trace.

\begin{footnotesize}
\[
\begin{array}{l}
\mbox{$OrderList$ = $\{e_1,~(+~e_2~(+~e_3~e_4))\}$} \\
\quad \mbox{$\Rightarrow$ \Code{(Sg1 $[\bigcdot]~e_2~e_3~e_4$)}}\\
\mbox{$OrderList$ = $\{e_2,~(+~e_3~e_4)\}$}\\
\quad \mbox{$\Rightarrow$ \Code{(Sg1 $v_1~[\bigcdot]~e_3~e_4$)}}\\
\mbox{$OrderList$ = $\{e_3,~e_4\}$} \\
\quad \mbox{$\Rightarrow$ \Code{(Sg1 $v_1~v_2~[\bigcdot]~e_4$),(Sg1 $v_1~v_2~v_3~[\bigcdot]$)}}
\end{array}
\]
\end{footnotesize}
But for instance of \m{Sg2}:
\[
\drule{(\m{Sg2}~e_1~e_2~e_3~e_4)}{\m{(+~(+~(+}~e_1~e_2\m{)}~e_3\m{)}~e_4\m{)}}
\]
once the sub-expressions $e_1$ and $e_2$ are reduced to value, the sugar has to be expanded, because if being desugared to the core language, the sub-expression \Code{(+ $v_1~v_2$)} will be reduced, then the sugar form of RHS is destroyed.
\begin{footnotesize}
	

	\[
\begin{array}{l}
\mbox{$OrderList$ = $\{(+~(+~e_1~e_2)~e_3),~e_4\}$}\\
\quad \Rightarrow \mbox{nothing}\\
\mbox{$OrderList$ = $\{(+~e_1~e_2),~e_3\}$}\\
\quad \Rightarrow \mbox{nothing}\\
\mbox{$OrderList$ = $\{e_1,~e_2\}$}\\
\quad \Rightarrow \mbox{\Code{(Sg2 $[\bigcdot]~e_2~e_3~e_4$), (Sg2 $v_1~[\bigcdot]~e_3~e_4$)})}
\end{array}
\]
\end{footnotesize}


If the \m{Head} is a \m{SurfHead} with its context rules calculated, then we regard it as \m{CoreHead}. If it has no context rule, we will try calculating its context rules first. However, if an infinite recursive process happens, it means that the original recursive sugars are of ill-form, such as the following:
\[
\begin{array}{l}
\drule{(\m{Odd}~e)}{(\m{Even}~(-~e~1))}\\
\drule{(\m{Even}~e)}{(\m{Odd}~(-~e~1))}.
\end{array}
\]

After calculating all context rules, we can add them to the mixed language's context rule; we can also add the desugaring rules to the mixed language's reduction rule as what we showed in Section \ref{sec2}.

%\todo{Add explanantion of the above rule.}

\subsubsection{Filtering and Main Algorithm}

As the second part of the whole process, our resugaring algorithm can be defined based on evaluation rules of the mixed language. Let $\redm{}{}$ be one-step reduction in the mixed language.

\[
\begin{array}{llll}
\m{resugar} (e) &=& \key{if}~\m{isNormal}(e)~\key{then}~return\\
              & & \key{else}~\\
							& & \qquad \key{let}~\redm{e}{e'}~\key{in}\\
							& & \qquad \key{if}~e' \in~\m{DisplayableExp} \\
							& & \qquad \qquad \m{output}(e'),~\m{resugar}(e')\\
							& & \qquad \key{else}~\m{resugar}(e')
\end{array}
\]
During the resugaring, we just apply the reduction ($\redm{}{}$) on the input program step by step until no reduction can be applied (\m{isNormal}, \m{value} in our setting), while outputting the intermediate expressions that belong to \m{DisplayableExp}.


\subsection{Correctness}
\label{mark:correct}

We define following properties of our desugaring algorithm for correctness.

First of all, we define a function $\mathtt{D}$ to fully remove sugars in a program using desugaring rules.
\[
\begin{array}{lll}
	\mathtt{D}(\Code{value}) = \Code{value}\\
	\mathtt{D}(\Code{CoreHead}~e_1~e_2~ ...) = (\Code{CoreHead}~(\mathtt{D}(e_1))~(\mathtt{D}(e_2))~...)\\
	\mathtt{D}(\Code{SurfHead}~e_1~e_2~ ...) = \mathtt{D}(e[e_i/x_i])\\
\quad \mbox{\bf where}~\drule{(\Code{SurfHead}~x_1~x_2~ ...)}{e}
%\mbox{\Code{$\mathtt{D}(\Code{value}) =$ value}}\\
%\mbox{\Code{$\mathtt{D}(\Code{x}) =$ x}}\\
%\mbox{\Code{$\mathtt{D}(\Code{(CoreHead $e_1$ $e_2$ ...)}) =
%%\Code{(CoreHead $\mathtt{D}(e_1')$ $\mathtt{D}(e_2')$ ...)}}}
%\mbox{\Code{$\mathtt{D}(\Code{(SurfHead $e_1$ $e_2$ ...)})$} = \Code{(CoreHead $\mathtt{D}(e_1')$ $\mathtt{D}(e_2')$ ...)}}\\
%\mbox{\scriptsize{where $\drule{\Code{(SurfHead $e_1$ $e_2$ ...)}}{\Code{(CoreHead $e_1'$ $e_2'$ ...)}}$}}
\end{array}
\]

An expression \m{P} can be fully desugared if \Code{$\mathtt{D}(\m{P})$} terminates. We use $\m{E}[\m{P}]$ to denote filling in the hole of the evaluation context \m{E} with \m{P}.  The fully desugaring of the evaluation context is also the same form, by following desugar rules of evaluation context.

\begin{Def}[Desugaring rule of evaluation context]
	For syntactic sugar S
	\[
	\drule{(\m{SurfHead}~e_1~e_2~\ldots~e_n)}{(\m{Head}~\ldots~e_1'~e_2'~\ldots~e_m')}
	\]
	and evaluation context \m{C} = $\m{S}.LHS[[\bigcdot]/e_i]$, where $[\bigcdot]$ is at $e_i$'s location, then
	\[
	\drule{\m{C}}{\m{S}.RHS[[\bigcdot]/e_i]}
	\]

\end{Def}
As the evaluation rules of the mixed language defined, there are only two kinds of reductions---(1) desugaring on an expression headed with \m{SurfHead}; (2) reduction on an expression headed with \m{CoreHead}. And because we need the mixed language's reduction corresponds to the execution of the fully desugared program, for any program \m{P} in the mixed language, if it reduces by expanding a sugar, then the reduction will occurs in the the expression after the sugar expanded in $\mathtt{D}(\m{P})$; otherwise (reduced by \m{CoreHead}), the reduction will be also by the same \m{CoreHead} in $\mathtt{D}(\m{P})$.

\begin{property} \label{thm1}
	For a program \m{P}=$\m{E}[\m{S}]$ of the mixed language which can be fully desugared, where \m{P'}=$\mathtt{D}(\m{P})$=$\m{E'}[\m{C}]$, and \m{S}=\Code{(SurfHead $e_1$ ... $e_n$)} in the program \m{P} together with \m{E'}=$\mathtt{D}(\m{E})$ (then of course \m{C}=$\mathtt{D}(\m{S})$); if $\redm{\m{E}[\m{S}]}{\m{E}[\m{S'}]}$ and $\drule{\m{S}}{\m{S'}}$, then $\redc{\m{E'}[\m{C}]}{\m{E'}[\m{C'}]}$ together with destroying the sugar's RHS form of \m{S} by $\redc{\m{C}}{\m{C'}}$. An example in Fig. \ref{example:ppt1}.
\end{property}
\example{\tiny
\begin{tabular}{|l | l | l |}\hline
    \m{P}/\m{P'} & \m{E}/\m{E'} & \m{S}/\m{C}\\ \hline
    \Code{(And (And \#t \#f) \#f)} & \Code{(And $[\bigcdot]$ \#f)} & \Code{(And \#t \#f)}  \\ \hline
    \Code{(if (if \#t \#f \#f) \#f \#f)} & \Code{(if $[\bigcdot]$ \#f \#f)} & \Code{(if \#t \#f \#f)}   \\ \hline
  \end{tabular}
\begin{flushleft}
	$\redm{\Code{(And (And \#t \#f) \#f)}}{\Code{(And (if \#t \#f \#f) \#f)}}$, reduced by \m{And}.\\
	$\redc{\Code{(if (if \#t \#f \#f) \#f \#f)}}{\Code{(if \#f \#f \#f)}}$, reduced on the expression expanded from \m{And} sugar\.\
	So \m{S'}=\Code{(if \#t \#f \#f)}, \m{C'}=\Code{\#f}; thus the $\redc{}{}$ destroyed the sugar form of \Code{(And \#t \#f)}.

\end{flushleft}

}{Example of property \ref{thm1}}{example:ppt1}

\begin{property} \label{thm2}
	For a program \m{P}=$\m{E}[\m{CC}]$ of the mixed language which can be fully desugared, where \m{P'}=$\mathtt{D}(\m{P})$=$\m{E'}[\m{C}]$, and \m{CC}=\Code{(CoreHead $e_1$ ... $e_n$)} in the program \m{P} together with \m{E'}=$\mathtt{D}(\m{E})$ (then of course \m{C}=$\mathtt{D}(\m{CC})$); if $\redm{\m{E}[\m{CC}]}{\m{E}[\m{CC'}]}$ reduced by the \m{CoreHead}'s reduction rule on \m{CC}, then for $\redm{\m{E'}[\m{C}]}{\m{E'}[\m{C'}]}$, it also reduced by the \m{CoreHead}'s reduction rule on \m{C}. An example in Fig. \ref{example:ppt2}.
\end{property}

\example{\tiny
\begin{tabular}{|l | l | l |}\hline
    \m{P}/\m{P'} & \m{E}/\m{E'} & \m{C}/\m{CC}\\ \hline
    \Code{(if (if \#t (And \#t \#f) \#t) \#f \#f)} & \Code{(if $[\bigcdot]$ \#f \#f)} & \Code{(if \#t (And \#t \#f) \#t)}  \\ \hline
    \Code{(if (if \#t (if \#t \#f \#f) \#t) \#f \#f)} & \Code{(if $[\bigcdot]$ \#f \#f)} & \Code{(if \#t (if \#t \#f \#t) \#f)}   \\ \hline
  \end{tabular}
\begin{flushleft}
	$\redm{\Code{(if (if \#t (And \#t \#f) \#t) \#f \#f)}}{\Code{(if (And \#t \#f) \#f \#f)}}$, reduced on \m{if}.\\
	$\redc{\Code{(if (if \#t (if \#t \#f \#f) \#t) \#f \#f)}}{\Code{((if (if \#t \#f \#f) \#f \#f)}}$, on the same \m{if}.\\
	So \m{C'}=\Code{(And \#t \#f)},  \m{CC'}=\Code{(if \#t \#f \#f)}; thus \m{C} and \m{CC} are both reduced by \m{if}'s reduction rule.

\end{flushleft}

}{Example of property \ref{thm2}}{example:ppt2}

The properties restrict how the lazy desugaring of our mixed language should be---the resugaring sequences should behave as the sequences after desugared to the core language.

Proof sketch is in Appendix. We discuss the hygiene of our approach in Section \ref{mark:hygiene}.

\ignore{
\begin{lemma}
The context rules of syntactic sugar are correct if computable.

Or to say, for sugar \m{S}, the context rules $C_1$, $C_2$ $\ldots$, if the context rules limit reduction order of $S.LHS$ in $\{e_i,~\ldots,~e_j\}$, then the reduction order of $S.RHS$ is also of this sequence.

\end{lemma}

\begin{proof}
In algo \ref{alg:f}, we recursively search the sugar's RHS to find $e_i$ $\ldots$ in RHS's computational order, so the sugar's context rules are correct.
\end{proof}



\begin{proof}[Proof of Theorem \ref{thm1}]
Consider a context \m{C}, where the sugar \m{S} in the hole. So the P is the program \m{C[S/hole]}.

If no sugar expression in \m{C}, then it is to prove the program \m{C[fulldesugar(S)/hole]} will reduce on \m{fulldesugar(S)}. It is obvious because that is where the hole at.

If there is any sugar in \m{C}, then it is to prove the program \m{fulldesugar(C[S/hole])} will reduce on \m{fulldesugar(S)}. According to the lemma, it is true because the hole will be at the same place after any sugar recursively desugared.
\end{proof}

\begin{proof}[Proof of Theorem \ref{thm2}]
According to the lemma, any sugar has the correct context rules. So the hole is also correct for the mixed language.
\end{proof}
}

\subsection{Possible Extensions}

So far, our desugaring rules are of a simple form. What if we need a more complex form to describe more complex semantics for syntactic sugar? For example, we may need a type system for checking during desugaring; we may specify the binding of syntactic sugar for more general hygiene; we may use some other functions to help the desugaring. All of these extensions are possible as long as the following conditions are satisfied.
\begin{enumerate}
	\item \emph{Compositional}: Generally speaking, the desugaring order and should not affect the semantics of a sugar expression. Otherwise, the lazy desugaring will not be correct.
	\item \emph{Unique Computational Order}: For any rules of syntactic sugar, the context rules should limit an expression to have only one computational order. Otherwise, the algorithm \m{calcontext} will not be deterministic.
	\item \emph{Clear Semantic}: If a syntactic sugar's desugaring rule is ambiguous or wrong, the algorithm \m{calcontext} may go wrong.
\end{enumerate}

For instance, we may need a desugaring rule like
\[
\drule{\Code{(Sugar $e_1$ $e_2$ $\ldots$ $e_n$)}}{\Code{(if (Helper $e_1$ $e_2$) $\ldots$)}}
\]
where \m{Helper} is an external function, that means we don't have the evaluation rules of \m{Helper}. In this case, we have to force the expansion of sugar expressions headed by \m{Sugar}. We describe how to force the desugaring in Section \ref{mark:simple}.
