%!TEX root = ./main.tex
\section{Introduction}

%What is the research background and and what motivate you to do this research?

%What is the research issue and how the issue has been addressed so far?

%What is the remained research problem and how challenge it is?

%What is your key idea (insight) of your solution to be discussed in this paper?

%What are the three main technical contributions of this paper?

%The rest of the paper is organized as follows. ...



Syntactic sugar, first coined by Peter J. Landin in 1964 \cite{syntacticsugar}, was introduced to describe the surface syntax of a simple ALGOL-like programming language which was defined semantically in terms of the applicative expressions of the core lambda calculus. It has been proved to be very useful for defining domain-specific languages (DSLs) and extending languages \cite{FellFFKBMT18,CulpFFK19}.
Unfortunately, when syntactic sugar is eliminated by transformation, it obscures the relationship between the user’s source program and the transformed program.


Resugaring is a powerful technique that has been proposed to resolve this problem \cite{resugaring,hygienic}. It  can automatically convert the evaluation sequences of a desugared program in the core language into representative sugar's syntax in the surface language.
%Just like the existing approach, it is natural to try matching the expressions after the desugared with syntactic sugars' dusugaring rules to reversely expand the sugars---that why it was named "resugaring". Here we use the sugar \Code{Newor} to see how the existing resugaring approach works, with following desugaring rule. (Considering the \m{not} as a core language's constructor)
%
As a simple example, consider \Code{Or1}, a syntactic sugar defined by the following desugaring rule:
\[
\drule{\Code{(Or1 $e_1$ $e_2$)}}{\Code{(let (t $e_1$) (if t t $e_2$))}}.
\]
Figure \ref{fig:resugar1} demonstrates a resugaring process in the existing approach for
\[
\Code{(Or1 (not \#t) \#t)}
\]
by the existing approach. The sequence of expressions on the left is obtained from the evaluation sequence of the desugared program (in the core language) on the right by repeated reverse expansion of the sugar (reverse application of the desugaring rule through match and substitution).
\begin{figure}
\begin{center}
	\[
	{\footnotesize
		\begin{array}{lcl}
		\text{Surface}&&\text{Core}\\
		\Code{(Or1 (not \#t) }&\xrightarrow{desugar}&\Code{(let ((t (not \#t))) }\\
		\Code{\qquad\hspace{0.5em}\#t)}&\xleftarrow{resugar}&\Code{\qquad\hspace{0.5em}(if t t \#t))}\\
		\qquad\quad\dashdownarrow& &\qquad\qquad\downarrow\\
		\Code{(Or1 \#f }& &\Code{(let ((t \#f)) }\\
		\Code{\qquad\hspace{0.5em}\#t)}&\xleftarrow[resugar]&\Code{\qquad\hspace{0.5em}(if t t \#t))}\\
		\qquad\quad\dashdownarrow& &\qquad\qquad\downarrow\\
		\text{no resugaring}& &\Code{(if \#f \#f \#t)}\\
		\qquad\quad\dashdownarrow& &\qquad\qquad\downarrow\\
		\Code{\#t}& & \Code{\#t}\\
	\end{array}
	}
	\]
\end{center}
\caption{A Resugaring Example}
\label{fig:resugar1}
\end{figure}


\ignore{
But it is not as easy as the example above. Sometimes the program in the core language contains a desugared form of a syntactic sugar, but the form may belong to the original (not desugared) program. In the following example, the reverse expansion of sugar should be noticed.
\begin{center}
\[
{\footnotesize
	\begin{array}{lcl}
	Surface&&Core\\
	\Code{(let ((t \#t)))}&\xrightarrow{desugar}&\Code{(let ((t \#t)))}\\
	\Code{\quad(if t t (Newor \#t \#f)))}&\xleftarrow[resugar]&\Code{\quad(if t t (let\ldots))}\\
	\qquad\qquad\dashdownarrow& &\qquad\qquad\downarrow\\
	\qquad\qquad\vdots& &\qquad\qquad\vdots\\
	\qquad\qquad\dashdownarrow& &\qquad\qquad\downarrow\\
	\Code{(Newor \#t \#t)}& &\Code{(let ((t \#t))}\\
	&\xleftarrow[resugar?]&\Code{\quad(if t t \#t))}\\
	\qquad\qquad\vdots& &\qquad\qquad\vdots\\
\end{array}
}
\]
\end{center}
}

Resugaring becomes complex, if we use core language constructs in  surface programs. For instance, for the following surface program
\[
\Code{(let (t \#t) (if t t (Or1 \#t \#f)))}
\]
after one-step reduction on the surface program, we get
\[
\Code{(let (t \#t) (if t t \#t))}
\]
but we do not want to further get \Code{(Or1 \#t \#t)} by reverse application of the desugaring rule of \Code{Or1}, because this \Code{let} was originally used in the surface program.

Resugaring becomes hard, if we allow sugars to be recursive of higher-order, and with binding.
For instance, when meeting hygienic sugar, the reverse expansion by a simple match and substitution will not work. Consider the following hygienic sugar:
\[
\drule{\Code{(Double $e_1$)}}{\Code{(let (t $e_1$) (* 2 t))}}
\]
which uses a local binding for \Code{t}. For the surface program
\[
\Code{(let (t 1) (Double t))}
\]
which introduces a local binding for \Code{t} again, a more involved reverse expansion is needed.
\ignore{
\[
{\footnotesize
	\begin{array}{lcl}
	Surface&&Core\\
	\Code{(let (t 1)}&\xrightarrow{desugar}&\Code{(let (t 1)}\\
	\Code{\quad(Double t))}&&\Code{\quad(let (t' t) (* 2 t')))}\\
	\qquad\qquad\dashdownarrow& &\qquad\qquad\downarrow\\
	\Code{(Double 1)}&\xdashleftarrow[resugar?]&\Code{(let (t' 1) (* 2 t')))}\\
\end{array}
}
\]
\end{center}
If we use binder renaming for solving the case as above, some other information is needed (such as the permutation $\m{t} \to \m{t'}$).
}
% \todo{Examples of 2 resugaring method(what is, why hard)}

Some attempts have been made to tackle these problems by "tagging" \cite{resugaring} and "abstract syntax DAG" \cite{hygienic}.
They remember where the reverse expansion should be done, and design new representation of programs and improve the match-substitution algorithms.
However, there is a key step that makes the existing approaches less practical---the reverse expansion of sugars needs to match the desugared expressions to see if they can be resugared. It would be very expensive if a surface program uses many syntactic sugars or some syntactic sugars are desugared to large core expressions.
%Also, as the debugging for surface languages can be a good application of resugaring, the efficiency matters.


% \todo{why existing resugaring has a problem, examples. good but complicate, novel without desugaring}

\label{mark:mention}
In this paper, we propose a new approach to resugaring, which does not use reverse expansion (reverse desugaring) at all.
The key idea is "lazy desugaring", in the sense that desugaring is done lazily  so that the reverse application of desugaring rules becomes unnecessary. Rather than assuming a black-box stepper for the core language as the existing approaches, we treat the core language as a white-box, explicitly using
%our resugaring approach is based on the
its evaluation rules (consisting of the context rules and reduction rules) of the core language.
%So our approach seems to be a meta-level language feature rather than a tool for existing languages. (Although it can also work with a black-box stepper, as we will discuss in Section \ref{sec5.1})
Our idea is to consider the surface language and the core language as one mixed language. We regard desugaring rules as reduction rules of the surface language and calculate the context rules of surface language to show when desugaring
should be done. Our method can guarantee that the intermediate evaluation steps of the mixed language will contain the resugaring evaluation sequence for a surface program.

Our main technical contributions can be summarized as follows.
\begin{itemize}
\item We propose a novel approach to resugaring by lazy desugaring, where the reverse application of desugaring rules is unnecessary. The new approach has two new features: (1) it is powerful, providing a new way to deal with hygenic sugars and solving the open problem of recursive sugars; and (2) it is efficient because it completely avoids unnecessary complexity of the reverse desugaring.

\item We present an automatic algorithm to calculate the context rules for syntactic sugars, and propose a new reduction strategy, based on the evaluator of the core languages and the desugaring rules together with the context rules of syntactic sugar, guaranteeing to produce all necessary resugared expressions on the surface language.

\item We have implemented a system based on the new approach, and the experimental results show that it is much more efficient than the existing approach. We demonstrate with many examples that our system is powerful enough to deal with hygienic, recursive and higher-order sugars. All the examples in this paper have passed the test of the system.


\end{itemize}

The rest of our paper is organized as follows. We start with an overview of our approach in Section \ref{sec2}. We then discuss the core of resugaring by lazy desugaring in Section \ref{sec3}, give examples in Section \ref{sec:cases}, and describe our experiment and evaluation in Section \ref{sec4}. We discuss some other issues in Section \ref{sec5} and related work in Section \ref{sec6}, and conclude the paper in Section \ref{sec7}.

Note that there are two pairs of similar words in the whole paper.
\begin{enumerate}
	\item \emph{program} v.s. \emph{expression};
	\item \emph{desugar/desugaring} v.s. \emph{expand/expansion}.
\end{enumerate}
We use the word \emph{program} if an expression is the input program in the  resugaring process, while using \emph{expression} otherwise. We use the word \emph{expand/expansion} when discussing desugaring on a syntactic sugar, while using \emph{desugar/desugaring} when discussing desugaring on an expression or else.
