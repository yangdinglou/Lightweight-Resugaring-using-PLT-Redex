% !TEX program = pdflatex
%!TEX spellcheck
%% For double-blind review submission, w/o CCS and ACM Reference (max submission space)
\documentclass[sigplan,10pt,review,anonymous]{acmart}\settopmatter{printfolios=true,printccs=false,printacmref=false}
%% For double-blind review submission, w/ CCS and ACM Reference
%\documentclass[acmsmall,review,anonymous]{acmart}\settopmatter{printfolios=true}
%% For single-blind review submission, w/o CCS and ACM Reference (max submission space)
%\documentclass[acmsmall,review]{acmart}\settopmatter{printfolios=true,printccs=false,printacmref=false}
%% For single-blind review submission, w/ CCS and ACM Reference
%\documentclass[acmsmall,review]{acmart}\settopmatter{printfolios=true}
%% For final camera-ready submission, w/ required CCS and ACM Reference
%\documentclass[acmsmall]{acmart}\settopmatter{}


%% Journal information
%% Supplied to authors by publisher for camera-ready submission;
%% use defaults for review submission.
\acmJournal{PACMPL}
\acmVolume{1}
\acmNumber{CONF} % CONF = POPL or ICFP or OOPSLA
\acmArticle{1}
\acmYear{2018}
\acmMonth{1}
\acmDOI{} % \acmDOI{10.1145/nnnnnnn.nnnnnnn}
\startPage{1}

%% Copyright information
%% Supplied to authors (based on authors' rights management selection;
%% see authors.acm.org) by publisher for camera-ready submission;
%% use 'none' for review submission.
\setcopyright{none}
%\setcopyright{acmcopyright}
%\setcopyright{acmlicensed}
%\setcopyright{rightsretained}
%\copyrightyear{2018}           %% If different from \acmYear

%% Bibliography style
\bibliographystyle{ACM-Reference-Format}
%% Citation style
%% Note: author/year citations are required for papers published as an
%% issue of PACMPL.
%\citestyle{acmauthoryear}   %% For author/year citations


%%%%%%%%%%%%%%%%%%%%%%%%%%%%%%%%%%%%%%%%%%%%%%%%%%%%%%%%%%%%%%%%%%%%%%
%% Note: Authors migrating a paper from PACMPL format to traditional
%% SIGPLAN proceedings format must update the '\documentclass' and
%% topmatter commands above; see 'acmart-sigplanproc-template.tex'.
%%%%%%%%%%%%%%%%%%%%%%%%%%%%%%%%%%%%%%%%%%%%%%%%%%%%%%%%%%%%%%%%%%%%%%


%% Some recommended packages.
\usepackage{booktabs}   %% For formal tables:
                        %% http://ctan.org/pkg/booktabs
\usepackage{subcaption} %% For complex figures with subfigures/subcaptions
                        %% http://ctan.org/pkg/subcaption

\usepackage{algorithm}
\usepackage{tabularx}
\usepackage{algorithmic}
\usepackage{proof}
\usepackage{alltt}
\usepackage{pgf}
\usepackage{bcproof}

\renewcommand{\algorithmicrequire}{\textbf{Input:}}

\renewcommand{\algorithmicensure}{\textbf{Output:}}

\newtheorem{Def}{Defination}[section]
\newtheorem{mythm}{Theorem}[section]
\newtheorem{property}{Property}[section]
\newtheorem{lemma}{Lemma}[section]
\newtheorem{Asm}{Assumption}

\newcommand{\Code}[1]{\texttt{#1}}
\newenvironment{Codes}
  {\begin{alltt}\leftskip=1.5em} % \tiny
  {\end{alltt}}

\newenvironment{smallCodes}
  {\begin{alltt}\leftskip=1.5em\small} %
  {\end{alltt}}

\newcommand{\OneStep}{{\rule{0pt}{1.2\baselineskip}{\ensuremath\longrightarrow}}}
\newcommand{\DeStep}{{\rule{0pt}{1.2\baselineskip}{\ensuremath\dashrightarrow}}}

\newcommand\m[1]{\mbox{\tt #1}}
\newcommand\key[1]{\mbox{\rm \bf #1}}
\newcommand\drule[2]{#1 ~\rightarrow_d~ #2}
\newcommand\redc[2]{#1 ~\rightarrow_c~#2}
\newcommand\rede[2]{#1 ~\rightarrow_e~#2}
\newcommand\redm[2]{#1 ~\rightarrow_m~#2}
\newcommand\note[1]{\mbox{{\scriptsize #1}}}
\newcommand\ignore[1]{}

\def\coreId{\m{cId}}
\def\surfId{\m{sId}}
\def\headId{\m{hId}}

\def\true{\#t}
\def\false{\#f}

\def\myend{\flushright{\qed}}

% some macros for editing/commenting the paper

\def\modify#1#2#3{{\small\underline{\sf{#1}}:} {\color{red}{\small #2}}
{{\color{red}\mbox{$\Rightarrow$}}} {\color{blue}{#3}}}

\newcommand{\hmodify}[2]{\modify{Hu}{#1}{#2}}
\newcommand\mymargin[1]{\marginpar{{\flushleft\textsc\footnotesize {#1}}}}
\newcommand\hmargin[1]{\mymargin{Hu:\;#1}}

\newcommand{\hmodifyok}[2]{#2}

\newcommand{\mycomment}[2]{{\small\color{magenta}\underline{\sf{#1}}:} {\color{magenta}{\small #2}}}
\newcommand{\hcomment}[1]{\mycomment{Hu}{#1}}
\newcommand{\gcomment}[1]{\mycomment{G}{#1}}
\newcommand{\todo}[1]{\mycomment{Todo}{#1}}

\newcommand{\xcomment}[1]{\mycomment{X}{#1}}

\newcommand{\reduce}[1]{{\color{blue}{#1}}}
\newcommand{\reducedversion}[1]{{\color{blue}{#1}}}

\newcommand{\example}[3]{
\begin{figure}[thb]
\begin{center}
#1
\end{center}
\caption{#2}
\label{#3}
\end{figure}
}

\makeatletter
\newcommand{\xleftrightarrow}[2][]{\ext@arrow 3359\leftrightarrowfill@{#1}{#2}}
\newcommand{\xdashrightarrow}[2][]{\ext@arrow 0359\rightarrowfill@@{#1}{#2}}
\newcommand{\xdashleftarrow}[2][]{\ext@arrow 3095\leftarrowfill@@{#1}{#2}}
\newcommand{\xdashleftrightarrow}[2][]{\ext@arrow 3359\leftrightarrowfill@@{#1}{#2}}
\def\rightarrowfill@@{\arrowfill@@\relax\relbar\rightarrow}
\def\leftarrowfill@@{\arrowfill@@\leftarrow\relbar\relax}
\def\leftrightarrowfill@@{\arrowfill@@\leftarrow\relbar\rightarrow}
\def\arrowfill@@#1#2#3#4{%
  $\m@th\thickmuskip0mu\medmuskip\thickmuskip\thinmuskip\thickmuskip
   \relax#4#1
   \xleaders\hbox{$#4#2$}\hfill
   #3$%
}
\makeatother

\makeatletter
\newcommand*{\dashdownarrow}{%
  \mathrel{%
    \mathpalette\dasharrow@vert{-90}%
  }%
}
\newcommand*{\dashuparrow}{%
  \mathrel{%
    \mathpalette\dasharrow@vert{90}%
  }%
}
\newcommand*{\dasharrow@vert}[2]{%
  \sbox0{$#1\vcenter{}$}%
  \sbox2{$#1\dashrightarrow\m@th$}%
  \dimen@=1.2\dimexpr\ht2-\ht0\relax
  % 1/2 width of the new symbol with side bearing
  \sbox2{\raisebox{-\ht0}{\unhcopy2}}%
  \ht2=\z@
  \dp2=\z@
  \vcenter{\hbox to 2\dimen@{\hfill\rotatebox{#2}{\box2}\hfill}}%
}
\makeatother

\makeatletter
\newcommand*\bigcdot{\mathpalette\bigcdot@{.5}}
\newcommand*\bigcdot@[2]{\mathbin{\vcenter{\hbox{\scalebox{#2}{$\m@th#1\bullet$}}}}}
\makeatother

\begin{document}

%% Title information
\title%[Short Title]
{Compositional Resugaring by Lazy Desugaring}
%{A lightweight resugaring approach based on reduction semantics}
%% [Short Title] is optional;
                                        %% when present, will be used in
                                        %% header instead of Full Title.
%\titlenote{with title note}             %% \titlenote is optional;
                                        %% can be repeated if necessary;
                                        %% contents suppressed with 'anonymous'
%\subtitle{Subtitle}                     %% \subtitle is optional
%\subtitlenote{with subtitle note}       %% \subtitlenote is optional;
                                        %% can be repeated if necessary;
                                        %% contents suppressed with 'anonymous'


%% Author information
%% Contents and number of authors suppressed with 'anonymous'.
%% Each author should be introduced by \author, followed by
%% \authornote (optional), \orcid (optional), \affiliation, and
%% \email.
%% An author may have multiple affiliations and/or emails; repeat the
%% appropriate command.
%% Many elements are not rendered, but should be provided for metadata
%% extraction tools.

%% Author with single affiliation.
\author{First1 Last1}
\authornote{with author1 note}          %% \authornote is optional;
                                        %% can be repeated if necessary
\orcid{nnnn-nnnn-nnnn-nnnn}             %% \orcid is optional
\affiliation{
  \position{Position1}
  \department{Department1}              %% \department is recommended
  \institution{Institution1}            %% \institution is required
  \streetaddress{Street1 Address1}
  \city{City1}
  \state{State1}
  \postcode{Post-Code1}
  \country{Country1}                    %% \country is recommended
}
\email{first1.last1@inst1.edu}          %% \email is recommended

%% Author with two affiliations and emails.
\author{First2 Last2}
\authornote{with author2 note}          %% \authornote is optional;
                                        %% can be repeated if necessary
\orcid{nnnn-nnnn-nnnn-nnnn}             %% \orcid is optional
\affiliation{
  \position{Position2a}
  \department{Department2a}             %% \department is recommended
  \institution{Institution2a}           %% \institution is required
  \streetaddress{Street2a Address2a}
  \city{City2a}
  \state{State2a}
  \postcode{Post-Code2a}
  \country{Country2a}                   %% \country is recommended
}
\email{first2.last2@inst2a.com}         %% \email is recommended
\affiliation{
  \position{Position2b}
  \department{Department2b}             %% \department is recommended
  \institution{Institution2b}           %% \institution is required
  \streetaddress{Street3b Address2b}
  \city{City2b}
  \state{State2b}
  \postcode{Post-Code2b}
  \country{Country2b}                   %% \country is recommended
}
\email{first2.last2@inst2b.org}         %% \email is recommended


%% Abstract
%% Note: \begin{abstract}...\end{abstract} environment must come
%% before \maketitle command
\begin{abstract}


Syntactic sugar, first coined by Peter J. Landin in 1964, has proved to be very useful for defining domain-specific languages and extending languages. Unfortunately, when syntactic sugar is eliminated by transformation, it obscures the relationship between the user’s source program and the transformed program. Resugaring is a powerful technique to resolve this problem, which automatically converts the evaluation sequences of desugared program in the core language into representative sugar's syntax in the surface language. However, the existing approach relies on the reverse application of desugaring rules to desugared core expression whenever possible. When a desugaring rule is complex and a desugared program is large, such reverse expansion becomes very complex and costive.

In this paper, we propose a novel approach to compositional resugaring by lazy desugaring, where the reverse application of desugaring rules is unnecessary. We recognize a sufficient and necessary condition for a syntactic sugar to be expanded, and propose a reduction strategy, based on the evaluator of the core languages and the desugaring rules, which can produce all necessary resugared expressions on the surface language. We have implemented a system based on this new approach. The result shows that our new approach is more efficient compared to the existing approach, and the lazy desugaring also provides powerful expressiveness even for a simple desugaring so that it can also deal with the common hygienic sugars and flexible recursive sugars.

\end{abstract}


%% 2012 ACM Computing Classification System (CSS) concepts
%% Generate at 'http://dl.acm.org/ccs/ccs.cfm'.
\begin{CCSXML}
<ccs2012>
<concept>
<concept_id>10011007.10011006.10011008</concept_id>
<concept_desc>Software and its engineering~General programming languages</concept_desc>
<concept_significance>500</concept_significance>
</concept>
<concept>
<concept_id>10003456.10003457.10003521.10003525</concept_id>
<concept_desc>Social and professional topics~History of programming languages</concept_desc>
<concept_significance>300</concept_significance>
</concept>
</ccs2012>
\end{CCSXML}

\ccsdesc[500]{Software and its engineering~General programming languages}
\ccsdesc[300]{Social and professional topics~History of programming languages}
%% End of generated code


%% Keywords
%% comma separated list
\keywords{Resugaring, Syntactic Sugar, Interpreter, Domain-Specific Language, Reduction Semantics}  %% \keywords are mandatory in final camera-ready submission


%% \maketitle
%% Note: \maketitle command must come after title commands, author
%% commands, abstract environment, Computing Classification System
%% environment and commands, and keywords command.
\maketitle

\section{Introduction}

What is the research background and and what motivate you to do this research?

What is the research issue and how the issue has been addressed so far?

What is the remained research problem and how challenge it is?

What is your key idea (insight) of your solution to be discussed in this paper?

What are the three main technical contributions oof this paper?

The rest of the paper is organized as follows. ...
%!TEX root = ./main.tex
\section{Overview}
\label{sec2}

Use a simple but sharp example to give an overview of your approach.

\subsection{Defination of resugaring}
This subsection is partially similiar to original defination in\cite{resugaring}.
\begin{Def}[Resugaring]
Given core language (named {\bfseries CoreLang}) and its evalation rules, together with surface language based on syntactic sugars of CoreLang (named {\bfseries Surflang}). For any expression of Surflang, getting the evaluation sequences of the expression in terms of SurfLang. 
\end{Def}
For correctness of the resugaring, the evaluation sequences should maintain the following three properties:
\begin{enumerate}
\item {\bfseries Emulation} Each term in the generated surface evaluation sequence desugars into the core term which it is meant to represent.
\item {\bfseries Abstraction} The resugaring sequences should only contains terms in SurfLang, and each term of SurfLang should originate from initial expression.
\item {\bfseries Coverage} No sequence is skipped during the process.
\end{enumerate}

Given an example below.

For syntactic sugar {\bfseries and} and {\bfseries or}, the sugar rules are:
\begin{center}
	\parbox[t]{\textwidth}{%
		\begin{center}  
			and(e1, e2) → if(e1, e2, \#f)\\
			or(e1, e2) → if(e1, \#t, e2)
		\end{center}  
	}%  
\end{center}
which forms a simple SurfLang.

The evaluation rules of {\bfseries if} is:
\begin{center}
	\parbox[t]{\textwidth}{%
		\begin{center}  
			if(\#t, e1, e2) → e1\\
			if(\#f, e1, e2) → e2
		\end{center}  
	}% 
\end{center}

Then for SurfLang's expression
%!TEX root = ./main.tex

\section{dynamic approach}
\label{sec3}

\subsection{Language setting}

\subsubsection{Grammatical restrictions}
Firstly, the whole language should be restricted to tree-structured disjoint expression.

\begin{Def}[disjoint]
For every sub-expression in a expression, its reduction rule is decided by itself.
\end{Def}

This restriction limits the scope of language. Every sub-expression must have no side effect. We will discuss more on side effect in section \ref{mark:side}.

\begin{Def}[tree-structured]
The grammar of the whole language is defined as follow.
\[
\begin{array}{rcl}
\mbox{Exp} &::=& (\mbox{Headid}~\mbox{Exp}*)\\
&|& \mbox{Value}\\
&|& \mbox{Variable}
\end{array}
\]
\end{Def}

The grammatical restrictions give our language a similiar property as church-rosser theorem\cite{churchrosser} for lambda calculus. 

\subsubsection{Context restrictions}
For expressions in CoreLang, the context rules should restrict it to have only one reduction path. The context rules can limit the order of evaluation. This restriction is normal, because a program in general-purposed language should have only one execution path.\label{mark:ctx}

For expressions in SurfLang, context rules should allow every sub-expressions reduced. It's the same as full-$\beta$ reduction.

\subsubsection{Restriction of syntactic sugar}
The form of syntactic sugar is as follow.

\fbox{
$(\mbox{Surfid}\;e_{1}\;e_{2}\;\ldots)$ ~→~ $(\mbox{Headid}\; \ldots)$
}

An counter example of this restriction is $(\mbox{Surfid}\;\ldots\;(e1\;e2)\ldots))$ in LHS. It's for simpler algorithm form, and the expression ability of syntactic sugar will not be changed.

\begin{Def}[Unambiguous]
For every syntactic sugar expressions, it can only desugar to one expression in CoreLang.
\end{Def}

\subsubsection{Grammar Description}
In our language setting, we regard SurfLang and CoreLang as a whole language. The whole language is under restrictions above, and its grammar is defined as follow.

\begin{centering}
	\framebox[38em][c]{
		\parbox[t]{38em}{
			\[
			\begin{array}{rcl}
			\mbox{Exp} &::=& \mbox{DisplayableExp}\\
			&|& \mbox{UndisplayableExp}\\
			\end{array}
			\]
			\[
			\begin{array}{rcl}
			\mbox{DisplayableExp} &::=& \mbox{Surfexp}\\
			&|& \mbox{Commonexp}
			\end{array}
			\]

			\[
			\begin{array}{rcl}
			\mbox{UndisplayableExp} &::=& \mbox{Coreexp}\\
			&|& \mbox{OtherSurfexp}\\
			&|& \mbox{OtherCommonexp}
			\end{array}
			\]
			
			\[
			\begin{array}{rcl}
			\mbox{Coreexp} &::=& (\mbox{CoreHead}~\mbox{Exp}*)
			\end{array}
			\]
			
			\[
			\begin{array}{rcl}
			\mbox{Surfexp} &::=& (\mbox{SurfHead}~\mbox{DisplayableExp}*)
			\end{array}
			\]
			
			\[
			\begin{array}{rcl}
			\mbox{Commonexp} &::=& (\mbox{CommonHead}~\mbox{DisplayableExp}*)\\
			&|& \mbox{Value}\\
			&|& \mbox{Variable}
			\end{array}
			\]
			
			\[
			\begin{array}{rcl}
			\mbox{OtherSurfexp} &::=& (\mbox{SurfHead}~\mbox{Exp}*~\mbox{UndisplayableExp}~\mbox{Exp}*)
			\end{array}
			\]
			
			\[
			\begin{array}{rcl}
			\mbox{OtherCommonexp} &::=& (\mbox{CommonHead}~\mbox{Exp}*~\mbox{UndisplayableExp}~\mbox{Exp}*)
			\end{array}
			\]
		}
	}
\end{centering}

The difference between CoreLang and SurfLang is identified by $Headid$. But there are some terms in CoreLang should be displayed during evaluation, or we need some terms to help us getting better resugaring sequences. So we defined {\bfseries Commonexp}, which origin from CoreLang, but can be displayed in resugaring sequences. The {\bfseries Coreexp} terms are terms with undisplayable CoreLang's Headid. The {\bfseries Surfexp} terms are terms with SurfLang's Headid and all sub-expressions are displayable. The {\bfseries Commonexp} terms are terms with displayable CoreLang's Headid, together with displayable sub-expressions. There exists some other expression during our resugaring process, which have Headid which can be displayed, but one or more subexpressions can't. They are UndisplayableExp.

Take some terms in CoreLang as examples. 

\begin{flushleft}
\begin{tabularx}{\textwidth}%
{|>{\setlength{\hsize}{.4\hsize}\centering\arraybackslash}X  |>{\setlength{\hsize}{1.6\hsize}\centering\arraybackslash}X|}
%{|*{2}{>{\centering\arraybackslash}X|}}
\hline
Syntax & Reduction rules \\ \hline
(if e e e) &\qquad\qquad\qquad(if \#t e2 e3) $\rightarrow$ e2 \newline ~(if \#f e2 e3) $\rightarrow$ e3\\ \hline
(($\lambda$ (x ...) e) e ...) & (($\lambda$ (x0 x1 ...) e) v0 v1 ...) $\rightarrow$ (let ((x0 v0) (($\lambda$ (x1 ...) e) v1 ...))\\ \hline
(($\lambda_N$ (x ...) e) e ...) & (($\lambda_N$ (x0 x1 ...) e) e0 e1 ...) $\rightarrow$ (let ((x0 e0) (($\lambda_N$ (x1 ...) e) e1 ...))\\ \hline
(let ((x e) ...) e) & (let ((x0 e0) (x1 e1) ...) e) $\rightarrow$ (let ((x1 e1) ...) (subst x0 e0 e))\newline (let () e)  $\rightarrow$ e (where subst is a meta function)\\ \hline
(first e) & (first (list v1 v2 ...)) $\rightarrow$ v1\\ \hline
(rest e) & (rest (list v1 v2 ...)) $\rightarrow$ (list v2 ...)\\ \hline
(empty e) & \qquad\qquad\qquad(empty (list)) $\rightarrow$ \#t \newline (empty (list v1 ...)) $\rightarrow$ \#f\\ \hline
(cons e e) & (cons v1 (list v2 ...)) $\rightarrow$ (list v1 v2 ...)\\ \hline
(op e e) \newline op=+-*/><== & (op v1 v2) $\rightarrow$ arithmetic result\\ \hline
\end{tabularx}
\end{flushleft}

We let {\bfseries if}, {\bfseries let}, {\bfseries $\lambda _{N}$}, {\bfseries empty}, {\bfseries first}, {\bfseries rest} as Coreexp's Headid, {\bfseries Op}, {\bfseries $\lambda$}, {\bfseries cons} as Commonexp's Headid. Then we could show some useful intermediate steps.


\subsection{Algorithm defination}

Our lightweight resugaring algorithm is based on a core algorithm core-algo. For every expression during resugaring process, it may have one or more reduction rules. The core algorithm core-algo chooses the one that satisfies three properties of resugaring, then applies it on the given expression. The core algorithm core-algo is defined as \ref{alg:f}.
\begin{algorithm}
	\caption{Core-algorithm core-algo}
	\label{alg:f}     % 给算法一个标签,以便其它地方引用该算法
	\begin{algorithmic}[1]       % 数字 "1" 表示为算法显示行号的时候,每几行显示一个行号,如:"1" 表示每行都显示行号,"2" 表示每两行显示一个行号,也是为了方便其它地方的引用
		\REQUIRE ~~\\      % 算法的输入参数说明部分
		Any expression $Exp$=$(Headid~Subexp_{1}~\ldots~Subexp_{\ldots})$ which satisfies Language setting
		\ENSURE ~~\\     % 算法的输出说明
		$Exp'$ reduced from $Exp$, s.t. the reduction satisfies three properties of resugaring
		\STATE     Let $ListofExp'$ = $\{Exp'_{1}\;,Exp'_{2}~\ldots\}$
		\IF {$Exp$ is Coreexp or  Commonexp or OtherCommonexp}
		\IF {Lengthof($ListofExp'$)==0}
		\RETURN null; \hfill Case1
		\ELSIF {Lengthof($ListofExp'$)==1}
		\RETURN first($ListofExp'$); \hfill Case2
		\ELSE 
		\RETURN $Exp'_{i}$ = $(Headid~Subexp_{1}~\ldots~Subexp'_{i}~\ldots)$; //where i is the index of subexp which have to be reduced. \hfill Case3
		\ENDIF
		\ELSE 
		\IF {Lengthof($ListofExp'$)==1}
		\RETURN desugarsurf($Exp$); \hfill Case4
		\ELSE
		\STATE Let $DesugarExp$ = desugarsurf(Exp)
		\IF {$Subexp_{i}$ is reduced to $Subexp'_{i}$ during $f(DesugarExp)$}
		\RETURN $Exp'_{i}$ = $(Headid~Subexp_{1}~\ldots~Subexp'_{i}~\ldots)$; \hfill Case5
		\ELSE
		\RETURN $DesugarExp$; \hfill Case6
		\ENDIF
		\ENDIF
		\ENDIF
		
	\end{algorithmic}
\end{algorithm}

We briefly describe the core algorithm core-algo in words.

For Exp in language defined as last section, try all reduction rules in the language, get a list of possible expressions ListofExp'=\{$Exp'_{1}$,$Exp'_{2}$,$\ldots$\}. 

Line 2-9 deal with the case when Exp has a CoreLang's Headid. When Exp is value or variable (line 3-4), ListofExp' won't have any element (not reducible). When Exp is of Coreexp or Commonexp (line 5-6), due to the context restriction of CoreLang, only one reduction rule can be applied. When Exp is OtherCommonexp (line 7-8), due to the context restriction of CoreLang, only one sub-expression can be reduced, then just apply core algorithm recursively on the sub-expression.

Line 10-21 deal with the case then Exp has a SurfLang's Headid. When Exp only has one reduction rule (line 11-12), the syntactic sugar has to desugar. If not, we should expand outermost sugar and find the sub-expression which should be reduced (line 14-16), or the sugar has to desugar (line 17-18), because it will never be resugared. The steps in line 14 to 16 are the critical part of our algorithm (call {\bfseries one-step try}\label{mark:onesteptry}).


Then, our lightweight-resugaring algorithm is defined as \ref{alg:lwresugar}.

\begin{algorithm}
	\caption{Lightweight-resugaring}
	\label{alg:lwresugar}     % 给算法一个标签,以便其它地方引用该算法
	\begin{algorithmic}[1]       % 数字 "1" 表示为算法显示行号的时候,每几行显示一个行号,如:"1" 表示每行都显示行号,"2" 表示每两行显示一个行号,也是为了方便其它地方的引用
		\REQUIRE ~~\\      % 算法的输入参数说明部分
		Surfexp $Exp$
		\ENSURE ~~\\     % 算法的输出说明
		$Exp$'s evaluation sequences within DSL
		\WHILE {$tmpExp$ = f($Exp$)}
		\IF {$tmpExp$ is empty}
		\RETURN
		\ELSIF {$tmpExp$ is Surfexp or Commonexp}
		\PRINT $tmpExp$;
		\STATE Lightweight-resugaring($tmpExp$);
		\ELSE 
		\STATE Lightweight-resugaring($tmpExp$);
		\ENDIF
		\ENDWHILE
		
	\end{algorithmic}
\end{algorithm}

The whole process of the lightweight resugaring executes core algorithm core-algo, and output sequences which is of Surfexp or Commonexp.

\subsection{Proof of correctness}

First of all, because the difference between our lightweight resugaring algorithm and the existing one is that we only desugar the syntactic sugar when needed, and in the existing approach, all syntactic sugar desugars firstly and then executes on CoreLang.

Then, to prove convenience, define some terms.

$Exp~=~(Headid\;Subexp_{1}\;Subexp_{\ldots} \ldots)$ is any reducible expression in our language.

If we use the reduction rule that desugar Exp's outermost syntactic sugar, then the reduction process is called {\bfseries Outer Reduction}.

If the reduction rule we use reduce $Subexp_{i}$, where $Subexp_{i}$ is $(Headid_{i}~Subexp_{i1}~Subexp_{i\ldots} \ldots)$
\begin{itemize}
	\item If the reduction process is Outer Reduction of $Subexp_{i}$ = $(Headid_{i}~Subexp_{i1}~Subexp_{i\ldots} \ldots)$, then it is called {\bfseries Surface Reduction}.
	\item If the reduction process reduces $Subexp_{ij}$, then it is called {\bfseries Inner Reduction}.
\end{itemize}

{\bfseries Example:}

$(\mbox{if}\; \#t\; Exp_{1}\; Exp_{2})$ → $Exp1$ \hfill Outer Reduction

$(\mbox{if}\; (\mbox{And}\; \#t\; \#f)\; Exp_{1}\; Exp_{2})$ → $(\mbox{if}\; (\mbox{if}\; \#t\; \#f\; \#f)\; Exp_{1}\; Exp_{2})$ \hfill Surface Reduction

$(\mbox{if}\; (\mbox{And}\; (\mbox{And}\; \#t\; \#t)\; \#t) \; Exp_{1}\; Exp_{2})$ → $(\mbox{if}\; (\mbox{And}\; \#t\; \#t)\; Exp_{1}\; Exp_{2})$ \hfill Inner Reduction

\begin{Def}[Upper and lower expression]
For $Exp$=$(Headid\;Subexp_{1}\;Subexp_{\ldots} \ldots)$, $Exp$ is called {\bfseries upper expression}$,Subexp_{i}$is called {\bfseries lower expression}.
\end{Def}

Case 2, 4, 6 in the core algorithm are of outer reduction. And case 3 or 5 are of surface reduction if the reduced subexpression is processed by outer reduction, or they are of inner reduction.
What we need to prove is that all the 6 cases of core algorithm core-algo satisfy the properties. Case 1 and case 2 won't effect any properties, because it does what CoreLang should do.

\begin{proof}[Proof of Emulation]
\hfill\\
For case 4 or 6, desugaring won't change Emulation property, because desugaring and resugaring are interconvertible.

For case 3 or 5, our core algorithm reduces the sub-expression which should be reduced. So if applying core algorithm core-algo on the subexpression satisfies emulation property, then this two cases satisfy. As we mentioned above, if the reduction is surface reduction, the subexpression is processed by case 2, 4 or 6, which have been proved to satisfy the emulation property; if the reduction is inner reduction, the subexpression is processed by case 3 or 5, which can be proved recursively, because the depth of expressions is finite, the subexpression will finally be reduced by an outer reduction. Thus, the reduction of the subexpression satisfies the emulation property, so it is for case 3 or 5.

\end{proof}

\begin{proof}[Proof of Abstraction]
\hfill\\
It's true, because we only display the sequence which satisfies abstraction property.
\end{proof}

\begin{lemma}
If no syntactic sugar desugared before it has to, then coverage property is satisfied.
\end{lemma}

\begin{proof}[Proof of Lemma]
Assume that no syntactic sugar not necessarily expanded desugars too early, existing an expression in CoreLang

$Exp$ = $(Headid\;Subexp_{1}\;Subexp_{\ldots} \ldots)$ which can be resugared to

$ResugarExp'$ = $(Surfid\;Subexp'_{1}\;Subexp'_{\ldots}\ldots)$, and $ResugarExp'$ is not displayed during lightweight-resugaring process. Then

\begin{itemize}
	\item Or existing
	
	$ResugarExp$=$(Surfid\;Subexp'_{1}\;\ldots\;Subexp_{i}\;Subexp'_{\ldots}\ldots)$ in resugaring sequences, such that the expression after $ResugarExp$ desugaring reduces to $Exp$, and the reduction reduces $ResugarExp$'s sub-expression $Subexp_{i}$. If so, outermost syntactic sugar of $ResugarExp$ is not expanded. So if $ResugarExp'$ is not displayed, then the sugar not necessarily expanded desugars too early, which is contrary to assumption.
	
	
	\item Or existing
	
	$ResugarExp$=$(Surfid'\;\ldots\;ResugarExp'\;\ldots)$ in resugaring sequences, such that the expression after $ResugarExp$ desugaring reduces to $Exp$, and $Exp$ is desugared from $ResugarExp'$'s sub-expression. If $ResugarExp'$ is not displayed, then the outermost syntactic sugar is expanded early, which is contrary to assumption.
	
	\item Or though the $Exp$ exists, it doesn't from $ResugarExp$.

\end{itemize}
\end{proof}

\begin{proof}[Proof of Coverage]
\hfill\\
For case 4 and 6, the syntactic sugar has to desugar.

For case 3 and 5, the reduction occurs in sub-expression of $Exp$. So if applying core algorithm core-algo on the subexpression doesn't desugar syntactic sugars not necessarily expanded, then this two cases don't. If the reduction is surface reduction, then the reduction of the subexpression is processed by case 2, 4 or 6, which don't desugar sugars not necessarily expanded; if the reduction is inner reduction, then it's another recursive proof as emulation. So in these two cases, the core-algo only desugar the sugar which has to be desugared.


\end{proof}

\subsection{Implementation}

Our lightweight resugaring approach  is implemented using PLT Redex\cite{SEwPR}, which is an semantic engineering tool based one reduction semantics\cite{reduction}. The whole framework is as Fig\ref{fig:frame}.

\begin{figure}[h]
	\centering
	\includegraphics[width=12cm]{images/frame.png}
	\caption{framework of implementation}
	\label{fig:frame}
\end{figure}

The grammar of the whole language contains Coreexp, Surfexp and Commonexp as the language setting in sec\ref{sec3}. OtherSurfexp is of Surfexp and OtherCommonexp is of Commonexp. The identifier of any kind of expression is Headid of expression. If we need to add a syntactic sugar to the whole language, only three steps is needed.

\begin{enumerate}
\item Add grammar of the syntactic sugar.
\item Add context rules of the sugar, such that any sub-expressions can be reduced.
\item Add desugar rules of the sugar to reduction rules of the whole language.
\end{enumerate}

Then inputting an expression of the syntactic sugar to lightweight-resugaring will get the resugaring sequences.

\subsection{Evaluation}

We test some applications on the tool implemented using PLT Redex. Note that we set CBV's lambda calculus as terms in commonexp, because we need to output some intermediate sequences including lambda expressions in some examples. It's easy if we want to skip them.

\subsubsection{simple sugar}
\label{mark:simple}

We construct some simple syntactic sugar and try it on our tool. Some sugar is inspired by the first work of resugaring\cite{resugaring}. The result shows that our approach can process all sugar features of their first work.

We take a SKI combinator syntactic sugar as an example. We will show why our approach is lightweight.

\begin{flushleft}
	$S$ $\rightarrow$ $(\lambda _{N}~(x_{1}~x_{2}~x_{3})~(x_{1}~x_{2}~(x_{1}~x_{3})))$
	
	$K$ $\rightarrow$ $(\lambda _{N}~(x_{1}~x_{2})~x_{1})$
	
	$I$ $\rightarrow$ $(\lambda _{N}~(x)~x)$
\end{flushleft}

Althought SKI combinator calculus is a reduced version of lambda calculus, we can construct combinators' sugar based on call-by-need lambda calculus in our CoreLang. For expression

 $(S~(K~(S~I))~K~xx~yy)$, we get the following resugaring sequences as following.
\begin{Codes}
    (S (K (S I)) K xx yy)
\CoreStep (((K (S I)) xx (K xx)) yy)
\CoreStep (((S I) (K xx)) yy)
\CoreStep (I yy ((K xx) yy))
\CoreStep (yy ((K xx) yy))
\CoreStep (yy xx)
\end{Codes}
% \begin{figure}[ht]
% 	\centering
% 	\parbox[t]{\textwidth}{
% 				\begin{center}
% 				{
% 					\small\selectfont
% 					(S (K (S I)) K xx yy)\\
% 					↓\\
% 					(((K (S I)) xx (K xx)) yy)\\
% 					↓\\
% 					(((S I) (K xx)) yy)\\
% 					↓\\
% 					(I yy ((K xx) yy))\\
% 					↓\\
% 					(yy ((K xx) yy))\\
% 					↓\\
% 					(yy xx)
% 				}
% 				\end{center}
% 			}
% 	\caption{SKI's resugaring sequences}
% 	\label{fig:SKI}
% \end{figure}

For existing approach, the sugar expression should firstly desugar to
\begin{flushleft}
$((\lambda _{N}
   (x_{1} x_{2} x_{3})
   (x_{1} x_{3} (x_{2} x_{3})))
  ((\lambda _{N} (x_{1} x_{2}) x_{1})
   ((\lambda _{N}
     (x_{1} x_{2} x_{3})
     (x_{1} x_{3} (x_{2} x_{3})))
    (\lambda _{N} (x) x)))
  (\lambda _{N} (x_{1} x_{2}) x_{1})
  xx
  yy)$
\end{flushleft}

Then in our CoreLang, the execution of expanded expression will contain 33 steps. For each step, there will be many attempts to match and substitute the syntactic sugars. It will omit more steps for a larger expression. 

So the unidirectional resugaring algorithm makes our approach lightweight, because no attempts for resugaring the expression take place.

\subsubsection{hygienic macro}
\label{mark:hygienic}

The second work\cite{hygienic} mainly processes hygienic macro compared to first work. We try a $Let$ sugar , which is a common hygienic sugar example, on our tool. Our algorithm can easily process hygienic macro without special data structure. The $Let$ sugar is define as follow

$(Let\;x\;v\;exp)$ $\rightarrow$ $(Apply\;(\lambda\;(x)\;exp)\;v)$

Take $(Let~x~1~(+~x~(Let~x~2~(+~x~1))))$ for an example. First, a temp expression

$(Apply\;(\lambda\;(x)\;(+~x~(Let~x~2~(+~x~1))))\;1)$

is needed. (case 5 or 6)Then one-step try on the temp expression, we will get

$(+~1~(Let~1~2~(+~1~1)))$ which is out of the whole language's grammar. In this case, it is not a good choice to desugar the outermost $Let$ sugar. Then we just apply the core-algo f on the sub-expression where the error occurs ($(+~x~(Let~x~2~(+~x~1)))$ in this example). So the right intermediate sequence $(Let~x~1~(+~x~3))$ will be get.

In practical application, we think resugaring for a unhygienic rewriting system is not interesting at all, because hygienic macro can be easily processed by rewriting system. So in the finally implementation of our tool, we just use PLT Redex's binding forms to deal with hygienic macros. But we did try it on the version without hygienic rewriting system.

\subsubsection{recursive sugar}
Recursive sugar is a kind of syntactic sugars where call itself or each other during the expanding. For example,

$(Odd\;e)$ $\rightarrow$ $(if\;(>\;e\;0)\;(Even\;(-\;e\;1)\;\#f))$

$(Even\;e)$ $\rightarrow$ $(if\;(>\;e\;0)\;(Odd\;(-\;e\;1)\;\#t))$

are typical recursive sugars. The previous works can process this kind of syntactic sugar easily, because boundary conditions are in the sugar itself.

Take $(Odd~2)$ as an example. The previous work will firstly desugar the expression using the rewriting system. Then the rewriting system will never start resugaring as Fig\ref{fig:odd} shows.

\begin{figure}[ht]
	\centering
	\parbox[t]{\textwidth}{
				\begin{center}
				{
					\small\selectfont
					(Odd 2)\\
					↓\\
					(if (> 2 0) (Even (- 2 1) \#f))\\
					↓\\
					(if (> (- 2 1) 0) (Odd (- (- 2 1) 1) \#t))\\
					↓\\
					(if (> (- (- 2 1) 1) 0) (Even (- (- (- 2 1) 1) 1) \#f))\\
					↓\\
					{\ldots}
				}
				\end{center}
				
			}
	\caption{Odd2's desugaring process}
\label{fig:odd}
\end{figure}

Then the advantage of our approach is embodied. Our lightweight approach doesn't require a whole expanding of sugar expression, which gives the framework chances to judge boundary conditions in sugars themselves, and showing more intermediate sequences. We get the resugaring sequences as Fig \ref{fig:rec} of the former example using our tool.

\begin{figure}[ht]
	\centering
	\parbox[t]{\textwidth}{
				\begin{center}
				{
					\small\selectfont
					(Odd 2)\\
					↓\\
					(Even (- 2 1))\\
					↓\\
					(Even 1)\\
					↓\\
					(Odd (- 1 1))\\
					↓\\
					(Odd 0)\\
					↓\\
					\#f
				}
				\end{center}
				
			}
	\caption{Odd2's resugaring sequences}
\label{fig:rec}
\end{figure}

We also construct some higher-order syntactic sugars and test them. The higher-order feature is important for constructing practical syntactic sugar. And for syntactic sugar's feature, it is of recursive sugar. Giving the following two higher-order syntactic sugar as examples.

\begin{flushleft}
	$(map\;e\;(list\;v_1\ldots))$→
	
	$(if\;(empty\;(list\;v_1\ldots))\;(list)\;(cons\;(e\;(first\;(list\;v_1\ldots)))\;(map\;e\;(rest\;(list\;v_1\ldots)))))$
\end{flushleft}

\begin{flushleft}
	$(filter\;e\;(list\;v_1\;v_2\ldots))$→
	
	$(if\;(e\;v_1)\;(cons\;v_1\;(filter\;e\;(list\;v_2\ldots)))\;(filter\;e\;(list\;v_2\ldots)))$
	
	$(filter\;e\;(list))$ → $(list)$
\end{flushleft}
These two syntactic sugars use different sugar forms to implement. For $Map$ sugar, we use if expression in CoreLang to constrain the boundary conditions. For $Filter$ sugar, we use two different parameters' form, which is another easy way for constructing syntactic sugar. The testing results show as Fig\ref{fig:map} \ref{fig:filter}.

\begin{figure}[ht]
	\centering
	\parbox[t]{\textwidth}{
				\begin{center}
				{
					\small\selectfont
					(map (λ (x) (+ 1 x)) (list 1 2 3))\\
					↓\\
					(cons 2 (map (λ (x) (+ 1 x)) (list 2 3)))\\
					↓\\
					(cons 2 (cons 3 (map (λ (x) (+ 1 x)) (list 3))))\\
					↓\\
					(cons 2 (cons 3 (cons 4 (map (λ (x) (+ 1 x)) (list)))))\\
					↓\\
					(cons 2 (cons 3 (cons 4 (list))))\\
					↓\\
					(cons 2 (cons 3 (list 4)))\\
					↓\\
					(cons 2 (list 3 4))\\
					↓\\
					(list 2 3 4)
				}
				\end{center}
				
			}
	\caption{Map's resugaring sequences}
\label{fig:map}
\end{figure}

\begin{figure}[ht]
	\centering
	\parbox[t]{\textwidth}{
	
				\begin{center}
				{
					\small\selectfont
					(filter (λ (x) (and (> x 3) (< x 6))) (list 1 2 3 4 5 6 7))\\
					↓\\
					(filter (λ (x) (and (> x 3) (< x 6))) (list 2 3 4 5 6 7))\\
					↓\\
					(filter (λ (x) (and (> x 3) (< x 6))) (list 3 4 5 6 7))\\
					↓\\
					(filter (λ (x) (and (> x 3) (< x 6))) (list 4 5 6 7))\\
					↓\\
					(cons 4 (filter (λ (x) (and (> x 3) (< x 6))) (list 5 6 7)))\\
					↓\\
					(cons 4 (cons 5 (filter (λ (x) (and (> x 3) (< x 6))) (list 6 7))))\\
					↓\\
					(cons 4 (cons 5 (filter (λ (x) (and (> x 3) (< x 6))) (list 7))))\\
					↓\\
					(cons 4 (cons 5 (filter (λ (x) (and (> x 3) (< x 6))) (list))))\\
					↓\\
					(cons 4 (cons 5 (list)))\\
					↓\\
					(cons 4 (list 5))\\
					↓\\
					(list 4 5)
				}
					
				\end{center}
				
			}
	\caption{Filter's resugaring sequences}
\label{fig:filter}
\end{figure}
\subsection{Compare to previous work}

As mentioned many times before, the biggest difference between previous resugaring approach and our approach, is that our approach doesn't need to desugar the sugar expresssion totally. Thus, our approach has the following advantages compared to previous work.

\begin{itemize}
	\item {\bfseries Lightweight} As the example at sec\ref{mark:simple}, the match and substitution process searchs all intermediate sequences many times. It will cause huge cost for a large program. So out approach---only expanding a syntactic sugar when necessarily, is a lightweight approach.
	\item {\bfseries Friendly to hygienic macro} Previous hygienic resugaring approach use a new data structure---abstract syntax DAG, to process resugaring of hygienic macros. Our approach simply finds hygienic error after expansion, and gets the correct reduction instead. 
	\item {\bfseries lazy expansion for recursive sugar} The ability of processing recursive sugar is a superiority compared to previous work. The key point is that recursive syntactic sugar must handle boundary conditions. Our approach handle them easily by lazy expanding the syntactic sugars. Higher-order functions, as an important feature of functional programming, was supported by many daily programming languages, and many higher-order functions should run recursively. So it's an important feature of our dynamic approach.
	\item {\bfseries Rewriting rules based on reduction semantics} Any syntactic sugar that can expressed by reduction semantics can be used in our approach. It will give more possible forms for constructing syntactic sugars. todo:example?
\end{itemize}

The most obvious shortage compared to existing approach is that our approach needs a whole semantic of core languages. The reason is because in case 5 and 6, we need to expand the outermost syntactic sugar and try one step, which may contain unexpanded sugars. Theoretically, our dynamic approach would also work with only a core language's stepper, by totally expand all sugar expressions and marked where each term is originated from. Simple modifications are needed in core-algo. But we did not try it, because of the intent we would discussed in Sec\ref{mark:assumption}.
% %!TEX root = ./main.tex
\section{Derivation of Evaluation Rules}
\label{sec:ruleDerivation}

So far, we have shown that our resugaring algorithm can use lazy desugaring to avoid costive reverse desugaring in the traditional approach. However, as shown in the reduction rules {\sc SurfRed1} and {\sc SurfRed2}, we still need a one-step try to check whether a syntactic sugar is required or not. It would be more efficient if such one-step try could be avoided. In this section, we show that this is possible, by giving an automatic method to derive evaluation rules for the syntactic sugar through symbolic computation.
%As demonstrated in Section \ref{sec2}, this will significantly improve the efficiency of our resugaring.
%In this section, we purpose such a method to make the resugaring of syntactic sugar more efficient.

%As discussed in the overview, frequent attempts on reverse desugaring during traditional resugaring processes is costive and inefficient. If we can statically derive syntactic sugar's evaluation rules through its structure, it will greatly improve the efficiency of resugaring. In this section, we purpose such a method to make the resugaring of syntactic sugar more efficient.

To see our idea, consider a simple syntactic sugar defined by $\drule{(\m{not}~e_1)}{(\m{if}~e_1~\m{\#f}~\m{\#t})}$. To derive reductions rules for \m{not} from those of \m{if}, we design \textit{inference automaton} (IFA) that can be used to express and manipulate a set of evaluation rules for a language construct.
%
Assuming that we already have the IFAs of all language constructs of the core language, our method to construct the evaluation rules of a syntactic sugar is as follows: First, we construct an IFA for the syntactic sugar according to the desugaring rules, then we transform and simplify the IFA, and finally we generate evaluation rules for the syntactic sugar from the IFA.

In this following, we start with some examples of IFA, its formal definition and its normal form, before we proceed to give our algorithm for conversion between evaluation rules and IFA.
%We discuss the role of IFA in dealing with syntactic sugar.

\subsection{Inference Automaton}

As mentioned above, IFA intuitively describes the evaluation rules of a certain language construct. To help readers better understand it, we start with some examples, then we give the formal definition of IFA.

\begin{example}[IFA of~~ \m{if}]

Recall the three evaluation rules of \m{if} in the overview. Given a term \m{if}~$e_1$~$e_2$~$e_3$, from these rules, we can see that $e_1$ should be evaluated first, then $e_2$ or $e_3$ will be chosen to evalute depending on the value of $e_1$. The evaluation result of $e_2$ or $e_3$ will be the value of the term. This evaluation procedure (or the three evaluation rules) for \m{if}~$e_1$~$e_2$~$e_3$ can be represented by the IFA in Figure \ref{fig:ifa-if}.

\begin{figure}[t]
    \centering
    \includegraphics[scale=0.25]{images/ifa/ifa-if.png}
    \caption{IFA of \m{if}}
    \label{fig:ifa-if}
\end{figure}

A node of IFA indicates that the term needs to be evaluated, and the nodes before this have been evaluated. The arrow from $q_0$ to $q_1$ indicates that this branch is selected when the evaluation result of $e_1$ is \m{\#t}. The arrow between $e_1$ and $e_3$ is similar. The double circles of $e_2$ and $e_3$ denote that their evaluation result is the result of the term with this syntactic structure. In most cases, the transition condition is the evaluation result (an abstract value) of the previous node or a specific value. To simplify the representation of IFA in the figures of this paper, in the former case, we omit the condition on the transition edge; and in the latter case, we only mark the value on the transition edge.

When a term headed with \m{if} needs to be evaluated (for example \m{(if (if \#t \#t \#f) \#f \#t)}), first evaluating the $e_1$ (\m{(if \#t \#t \#f)}). Note that in this process, evaluating a subexpression requires running another automaton based on its syntax, while the outer automaton hold the state at $q_0$. According to the result of $e_1$ (\m{\#f}), the IFA selects the branch ($e_3$). Then the result of $e_3$ (\m{\#t}) is the evaluation result of the term.
\myend
\end{example}

\begin{example}[IFA of \m{nand}]

Sometimes the rules may be more complex, such as being reduced into another syntactic structure, or a term containing other syntactic structures. For example, we can express \m{nand}'s evaluation rules as follows. Based on the method discussed above, we can draw \m{nand}'s IFA as Figure \ref{fig:ifa-nand-a}.
\[
    \infer{(\m{nand}~e_1~e_2) \to (\m{nand}~e_1'~e_2)}{e_1 \to e_1'}
\]\[
    (\m{nand}~v_1~e_2) \to (\m{if}~(\m{if}~v_1~e_2~\m{\#f})~\m{\#f}~\m{\#t})
\]

% \infrule[E-Nand]{e_1 \to e_1'}{(\m{nand}~e_1~e_2) \to (\m{nand}~e_1'~e_2)}
% \infax[E-NandV]{(\m{nand}~v_1~e_2) \to (\m{if}~(\m{if}~v_1~e_2~\m{\#f})~\m{\#f}~\m{\#t})}

\begin{figure}[t]
    \centering
    \subcaptionbox{IFA according to the rules \label{fig:ifa-nand-a}}[.31\linewidth]{
        \includegraphics[scale=0.25]{images/ifa/ifa-nand-1-small.png}
    }
    \subcaptionbox{After expanding the outer \m{if} \label{fig:ifa-nand-b}}[.31\linewidth]{
        \includegraphics[scale=0.25]{images/ifa/ifa-nand-2-small.png}
    }
    \subcaptionbox{After expanding the inner \m{if} \label{fig:ifa-nand-c}}[.33\linewidth]{
        \includegraphics[scale=0.25]{images/ifa/ifa-nand-3-small.png}
    }
    \caption{IFA of \m{nand}}
    \label{fig:ifa-nand}
\end{figure}

When the automaton runs into the terminal node of Figure \ref{fig:ifa-nand-a}, it derives the \m{if} term. In fact, we have known how \m{if} works through the IFA of \m{if}. Thus we can replace the last node with an $\text{IFA}_{\m{\scriptsize if}}$ and substitute $e_1$ to $e_3$ of $\text{IFA}_{\m{\scriptsize if}}$ with the parameters of the node. Use the termination nodes of $\text{IFA}_{\m{\scriptsize if}}$ as the termination nodes of new $\text{IFA}_{\m{\scriptsize \scriptsize nand}}$. The results are shown in Figure \ref{fig:ifa-nand-b}. Further decomposing the inner \m{if} node, connecting the terminating nodes of $\text{IFA}_{\m{\scriptsize nand}}$ to the nodes pointed to by the original output edge, we get Figure \ref{fig:ifa-nand-c}.

As can be seen, the nodes of IFA in Figure \ref{fig:ifa-nand-c} have no other composite syntactic structures. Such an IFA completely expresses the semantics of a syntactic structure, and no longer cares about the evaluation rules of other syntactic structures. We do similar steps for syntactic sugar, which will be discussed later.
\myend
\end{example}

\begin{example}[IFA of \m{and}]

We represent the evaluation rule of \m{and} in a more complex way, as follows.
\[
    (\m{and}~e_1~e_2) \to (\m{let}~x~e_1~(\m{if}~x~e_2~x))
\]
In this case, we use the let binding. So we need to record the term represented by each variable at each node called symbol table. The representation of $\text{IFA}_{\m{\scriptsize and}}$ is shown in Figure \ref{fig:ifa-and-a}.

Further discuss the syntactic structure. We first evaluate $e_1$ and save the result in $x$. When evaluating the term of \m{if}, what is actually evaluated is $(\m{if}~x~e_2~x)[e_1/x]$. We represent it in the form of Figure \ref{fig:ifa-and-b}, where node $q_1$ contains a symbol table for recording variables and the nodes referred to. It is similar to context, but since we mapped the variable to the node, we distinguished it. Mapping to nodes is to ensure that the variable data will not be lost when the IFA is transformed.
\myend
\end{example}


\begin{figure}[t]
    \centering
    \subcaptionbox{IFA according to the rules \label{fig:ifa-and-a}}[.4\linewidth]{
        \includegraphics[scale=0.25]{images/ifa/ifa-and-1.png}
    }
    \subcaptionbox{After expanding \m{let} \label{fig:ifa-and-b}}[.4\linewidth]{
        \includegraphics[scale=0.25]{images/ifa/ifa-and-2.png}
    }
    \caption{IFA of \m{and}}
    \label{fig:ifa-and}
\end{figure}

%--------------------------------
%\subsubsection{Definition of IFA}

\begin{Def}[Inference Automaton]

    An inference automaton (IFA) of syntactic structure \m{CoreHead} is a 5-tuple, $(Q, \Sigma, q_0, F, \delta)$, consisting of

    \begin{itemize}
        \item A finite set of nodes $Q$, each node contains a expression and a symbol table. The symbol table maps a variable to a node.
        \item A finite set of conditions $\Sigma$
        \item A start node $q_0 \in Q$
        \item A set of terminal nodes $F \subseteq Q$
        \item A transition function $\delta: (Q-F) \times \Sigma' \to Q$ where $\Sigma' \subseteq \Sigma$
    \end{itemize}

    and for each node $q$, there is no sequence of conditions $C = (c_1,c_2,\ldots,c_n)\subseteq \Sigma^*$, which makes that after $q$ transfers sequentially according to $P$, it returns $q$.

\end{Def}

The last constraint requires that there be no circles in our IFA.

In IFA, state transition does not depend on input. The only input of IFA is the term to be evaluated with this syntactic structure. The state transition is through whether the term meets the condition. Note that IFA is associated with syntactic structure. At Each IFA only represents the current evaluation of a syntactic structure. The state indicates that some sub-expressions of the syntactic structure have been evaluated, and the rest have not.

%======================
\subsection{Normal IFA}

Although IFA can intuitively show the behavior of a syntactic structure for evaluation, IFA itself has a complicated form. For example, the IFA of a syntactic structure may contain other syntactic structures as shown in Figure \ref{fig:ifa-nand-a}. We try to deform and simplify IFA to make it easier to analyze.

\begin{Def}[Normal IFA]
    \label{def:nmlifa}
    If an IFA meets following constraints, we call it a normal IFA.
    \begin{itemize}
        \item The expression of node $q \in Q$ can only be a pattern variable $e_i$ or a local variable $x$. If it is a local variable, it cannot be in the symbol table of $q$.
        \item For any $q_1,q_2 \in Q$ and $c_1, c_2 \in \Sigma$, $\delta(q_1, c_1) \neq \delta(q_2, c_2)$.
        \item On each branch, each pattern variable $e_i$ can only be evaluated once.
    \end{itemize}
\end{Def}

If an IFA is normal, it means there are no more composite syntactic structures in it. And it is a tree structure. We  prove that it is always feasible to convert IFA to normal IFA.

\begin{mythm}[Normalizability of IFA]
    \label{mythm:nmlifa}
    An IFA can be transformed into a normal IFA, if the normal IFAs of all sub-syntactic structures in the IFA are known.
\end{mythm}

In order to prove this theorem, we give the following steps and prove their correctness.

%------------------------------------
\subsubsection{Transform into a Tree}

When evaluating a term, different branches do not influence each other. Therefore, a node with at least two sources can be cloned and applied to different branches as shown in Figure \ref{fig:nmlifa-tree}. Replace the node referenced by the branch with the cloned node in conditions and symbol tables. The correctness is obvious. In this way, we can transform an IFA into a tree so that it satisfies the second constraint of normal IFA.

\begin{figure}[t]
    \centering
    \includegraphics[scale=0.25]{images/nmlifa/nmlifa-tree.png}
    \caption{Transform into a Tree}
    \label{fig:nmlifa-tree}
\end{figure}

%-------------------------------------------------
\subsubsection{Substitute Sub-Syntactic Structure}
\label{mark:hygieneinderive}

If the node $q$ in the tree IFA contains a term of a syntactic structure \m{H}, we replace this node with the normal IFA of \m{H}, replace the parameters and pass the symbol table of $q$ into the sub-IFA. Connect all the termination nodes of \m{H} to the original output node $q'$, and then convert it into a tree IFA according to the method described in the previous step as shown in Figure \ref{fig:nmlifa-subst}. Replace the node referenced by the branch with the \textit{new} one in conditions and symbol tables. This step can guarantee correctness, for $(\m{H}~e_1~\ldots~e_n)[e/x] = (\m{H}~e_1[e/x]~\ldots~e_n[e/x])$.

In particular, if we guarantee that a term $e$ does not contain a certain variable $x$, we can remove it from the symbol table. For example, in our problem, each pattern variable $e_i$ of syntactic sugar cannot contain unbound variables. So we can remove all symbol tables of nodes with $e_i$.

For hygienic sugar, this step of substitution can also guarantee correctness. Because what we require to substitute is a normal IFA, if the expression of a node $q$ in sub-IFA is a local variable $x$, it must be unknowable in the substructure ($x$ not in the symbol table of $q$). So the value of $x$ must be passed from the outer syntactic structure.

\begin{figure}[t]
    \centering
    \includegraphics[scale=0.25]{images/nmlifa/nmlifa-subst.png}
    \caption{Substitute Sub-Syntactic Structure}
    \label{fig:nmlifa-subst}
\end{figure}

%----------------------------------------------------
\subsubsection{Replace Variables in the Symbol Table}

If the expression of a node $q$ is a local variable $x$ and $x$ is in the symbol table of $q$, we can replace $x$ with the expression of the node it points to, then remove $x$ from the symbol table as Figure \ref{fig:nmlifa-replace}. Because $x[e_1/x, e_2/y]=e_1[e_2/y]$, it is correct.

\begin{figure}[t]
    \centering
    \includegraphics[scale=0.25]{images/nmlifa/nmlifa-replace.png}
    \caption{Replace Variables in the Symbol Table}
    \label{fig:nmlifa-replace}
\end{figure}

%---------------------------------------------------------------------
\subsubsection{Remove Evaluated Nodes and Merge Transition Conditions}

If an IFA is a tree, for each branch, remove the non-terminal nodes that have been evaluated with the same symbol table, and merge the conditions on the transition edge as Figure \ref{fig:nmlifa-merge}. Replace the node referenced by the branch with the first-evaluated node in conditions and symbol tables. This step can make an IFA satisfy the third constraint of normal IFA.

Since on any branch, if $e_i$ is removed, it must have been evaluated. Therefore, when IFA runs to this node, there is no need to do any evaluation on $e_i$. At the same time, the transfer edge ensures that the conditions are not lacking. Correctness is guaranteed.

\begin{figure}[t]
    \centering
    \includegraphics[scale=0.25]{images/nmlifa/nmlifa-merge.png}
    \caption{Remove Evaluated Nodes and Merge Transition Conditions}
    \label{fig:nmlifa-merge}
\end{figure}

%------------------------------------
\subsubsection{Remove Constant Value}

If the expression of a node is a constant value, remove the node and merge the conditions on the transition edge just like evaluated value. Because IFA does not do anything in this node, this does not affect the correctness of IFA.

%-------------------------------------
\subsubsection{Normalizability of IFA}

\begin{proof}[Proof of Theorem \ref{mythm:nmlifa}]
    By repeating the above steps continuously, we can always get normal IFA. And it is equivalent to the original IFA for the correctness of each step.
\end{proof}

%===========================================
\subsection{Convert evaluation rules to IFA}

In the examples at the beginning of this section, we construct IFAs based on their semantics. Now, we give an algorithm that can automatically convert the evaluation rules to IFA and ensure its correctness. But at the same time, it has stricter requirements on the evaluation rules.

%--------------------------
\subsubsection{Assumptions}

\begin{Asm}
    \label{Asm:rules}
    A syntactic structure $CoreHead$ only contains the following evaluation rules.

    \[
        \infer
        {(\m{CoreHead}~e_1 \ldots e_i \ldots e_n) \to (\m{CoreHead}~e_1 \ldots e_i' \ldots e_n)}{e_i \to e_i'\quad T}
    \]
    \[
        \infer{(\m{CoreHead}~v_1 \ldots v_p~e_1 \ldots e_q) \to e}{T}
    \]
    where $e$ can be any value, one of the parameters or a term of another syntactic structure. $T$ is a set of constraints of the parameters containing $e_j$ is a value or $e_j$ is a specific value where $j \in 1,2,\ldots,n$.
\end{Asm}

This assumption specifies the form of the evaluation rules to ensure that IFAs can be generated. The first one is a context rule, and the other one is a reduction rule. Rule $(\m{if}~\m{\#t}~e_1~e_2) \to e_1$ can be seen as \[\infer{\m{if}~e~e_1~e_2 \to e_1}{e~\key{is}~\m{\#t}}.\]

\begin{Asm}[Orderliness of Syntactic Structure]
    \label{Asm:orderliness}
    The syntactic structure in CoreLang is finite. Think of all syntactic structures as points in a directed graph. If one of $CoreHead$'s evaluation rules can generate a term containing $CoreHead'$, then construct an edge that points from $CoreHead$ to $CoreHead'$. The directed graph generated in this method has no circles.
\end{Asm}

IFAs are not able to construct syntactic structures that contain recursive rules. This assumption qualifies that we can find an order for all syntactic structures, and when we try to construct IFA for $CoreHead$, IFA of $CoreHead'$ is known.

\begin{Asm}[Determinacy of One-Step Evaluation]
    \label{Asm:determinacy}
    The rules satisfy the determinacy of one-step evaluation.
\end{Asm}

By assumption \ref{Asm:determinacy}, we can get the following lemma, which points out the feasibility of using a node in IFA to represent the evaluation of subexpressions.

\begin{lemma}
    \label{lemma:one-step}
    If a term $(\m{CoreHead}~e_1~\ldots~e_n)$ does a one-step evaluation by rule (E-Head) of $\m{CoreHead}$, which is a one-step evaluation of pattern variable $e_i$, then it continues to use this rule until $e_i$ becomes a value.
\end{lemma}

\begin{proof}[Proof of Lemma \ref{lemma:one-step}]
    According to Assumption \ref{Asm:determinacy}, this lemma is trivial.
\end{proof}

%------------------------
\subsubsection{Algorithm}

\begin{mythm}[IFA Can Be Constructed by Evaluation Rules]
    \label{mythm:Rule2IFA}
    If all the syntactic structures in CoreLang satisfy these assumptions, we can construct IFAs for all syntactic structures in CoreLang.
\end{mythm}

\begin{proof}[Proof of Theorem \ref{mythm:Rule2IFA}]

    We prove this theorem by giving an algorithm that converts evaluation rules to IFA. By Assumption \ref{Asm:orderliness}, we get an order of syntactic structures. We construct the IFA for each structure in turn.

    We generate a node for each rule of the syntactic structure \m{CoreHead} and insert them into $Q$. If the rule is a context rule for a pattern varibale $e_i$, set $e_i$ as the expression of the node. If the rule is a reduction rule, add them into $F$ as terminal nodes and set the reduced term $e$ as the expression of the node. Symbol tables of these nodes are set to be empty. Next we connect these nodes.

    For a term like $(\m{CoreHead}~e_1~\ldots~e_n)$, considering that $e_1\cdots e_n$ are not value, According to Lemma \ref{lemma:one-step}, we have the unique rule $r$ of \m{CoreHead} for one-step evaluation. Let node $q$ corresponding to $r$ be $q_0$.

    If $r$ is a context rule for $e_i$, let the term of $q$ be $e_i$. Assume that the evaluation of $e_i$ results in $v_i$, we get term $(\m{CoreHead}~e_1 \ldots e_{i-1}~v_i~e_{i+1} \ldots e_n)$. For each possible value of $v_i$, choose the rules $r'$ that should be used. The node is $q'$. Set a condition as $c=q~\key{is}~v_i$ Let $\delta(q, c)$ be $q'$. For each branch, seem $r'$ as $r$ and keep doing this until $r$ is a reduction rule.
\end{proof}

In this way, we got an IFA of \m{CoreHead}. According to the lemma \ref{mythm:nmlifa}, we can also get a normal IFA of \m{CoreHead}.

%---------------------------------------------------------
\subsubsection{Example: Construct IFA of \m{xor} by Rules}

We give an example to show how to convert evaluation rules to IFA using the algorithm mentioned above. Since the symbol tables of all nodes are empty, we omit not to write.

\infrule[E-Xor]{e_1 \to e_1'}{(\m{xor}~e_1~e_2) \to (\m{xor}~e_1'~e_2)}
\infax[E-XorTrue]{(\m{xor}~\m{\#t}~e_2) \to (\m{if}~e_2~\m{\#f}~\m{\#t})}
\infrule[E-XorFalse]{e_2 \to e_2'}{(\m{xor}~\m{\#f}~e_2) \to (\m{xor}~\m{\#f}~e_2')}
\infax[E-XorFalseTrue]{(\m{xor}~\m{\#f}~\m{\#t}) \to \m{\#t}}
\infax[E-XorFalseFalse]{(\m{xor}~\m{\#f}~\m{\#f}) \to \m{\#f}}

Suppose that \m{xor} is a syntactic structure in CoreLang. There are five rules for it. Therefore, we construct five nodes for the rules and set the expression as Figure \ref{fig:ifa-xor-a}.

Considering a term $(\m{xor}~e_1~e_2)$, where $e_1$ and $e_2$ are not values. It will be derived by rule (E-Xor). Therefore, set the node of the rule as the start node $q_0$. According to the rules of \m{xor}, the evaluation result of $e_1$ can be \m{\#t} or \m{\#f}. If the value is \m{\#t}, the term will be $(\m{xor}~\m{\#t}~e_2)$ and use rule (E-XorTrue) to derive. Then connect $q_0$ and $q_1$ with condition $q_0~\key{is}~\m{\#t}$. Connect $q_0$ and $q_2$ with condition $q_0~\key{is}~\m{\#f}$ similarly as Figure \ref{fig:ifa-xor-b}. Then connect $q_2$ to the last two nodes with conditions according to the value of $e_2$. The IFA of \m{xor} can be expressed as Figure \ref{fig:ifa-xor-c}.

\begin{figure}[t]
    \centering
    \subcaptionbox{\label{fig:ifa-xor-a}}[.3\linewidth]{
        \includegraphics[scale=0.22]{images/ifa/ifa-xor-1.png}
    }
    \subcaptionbox{\label{fig:ifa-xor-b}}[.33\linewidth]{
        \includegraphics[scale=0.22]{images/ifa/ifa-xor-2.png}
    }
    \subcaptionbox{\label{fig:ifa-xor-c}}[.33\linewidth]{
        \includegraphics[scale=0.22]{images/ifa/ifa-xor-3.png}
    }
    \caption{IFA of \m{xor}: Constructed by evaluation rules}
    \label{fig:ifa-xor}
\end{figure}

%--------------------------
\subsubsection{Correctness}

Before proving the correctness of the algorithm, we first prove the following lemma.

\begin{lemma}
    \label{lemma:accessibility}
    If a term $e$ of syntactic structure \m{CoreHead} uses rule $r$ to derive, we can find a path from $q_0$ to $q$ generated by $r$.
\end{lemma}

\begin{proof}[Proof of Lemma \ref{lemma:accessibility}]
    Suppose $e=(\m{CoreHead}~v_1~\ldots~v_p~e_1~\ldots~e_q)$. We build a term $e_0'=(\m{CoreHead}~\m{id}(v_1)~\ldots~\m{id}(v_p)~e_1~\ldots~e_q)$, where $\m{id}(x)=x$. $e_0'$ should have the same value as $e$. If $e_0'$ is derived by rule $r_1$, $r_1$ must be a context rule, or $e$ can also be reduced by $r_1$. Also, the node $q_1$ generated by $r_1$ will be the start node. Without loss of generality, assume that $r_1$ is a context rule for $e_1$. We get $e_1'=(\m{CoreHead}~v_1~\m{id}(v_2)~\ldots~\m{id}(v_p)~e_1~\ldots~e_q)$ after derived by $r_1$. Similarly, we find a rule $r_2$ which is a context rule for $e_2$. The node $q_2$ generated by $r_2$ satisfies $\delta(q_1, (e_1~\key{is}~v_1))=q_2$. By analogy, we can get $e_p'=(\m{CoreHead}~v_1~\ldots~v_p~e_1~\ldots~e_q)=e$ and a path $q_1(=q_0), q_2, \ldots, q_p$. At last, we use rule $r$ to derive $e$ and add $q$ to the path.
\end{proof}

\begin{lemma}
    \label{lemma:rule2ifa-correct}
    For any syntactic structure \m{CoreHead}, if its evaluation rules meets the above assumptions, the normal IFA got by the algorithm in Theorem \ref{mythm:Rule2IFA} has the same semantics as the rules.
\end{lemma}

In other words, for any term of syntactic structure \m{CoreHead}, evaluating the term by IFA and evaluating by rule get the same derivation sequence.

\begin{proof}[Proof of Lemma \ref{lemma:rule2ifa-correct}]
    We only need to discuss that in a one-step derivation, both get the same result.

    Considering a term $e=(\m{CoreHead}~e_1 \ldots e_n)$ use rule $r$ to derive. Suppose $r$ generate node $q$. By Lemma \ref{lemma:accessibility}, we find a path from $q_0$ to $q$. The term $e$ must meet the conditions from $q_0$ to $q$. Therefore, the one-step derivation of this term $e$ in IFA must be located at $q$. If $r$ is a context rule for a pattern variable $e_i$, the expression of $q$ is $e_i$ as well, and $e_i$ of $e$ is not a value. Thus both of the one-step derivation of $e$ are one-step derivation of $e_i$. If $r$ is a reduction rule to $e'$, both of them are $e'$.

    Similarly, if an term can be derived in one step in IFA, then it must be able to use the corresponding rule for one-step derivation.
\end{proof}

%-----------------------------
\subsubsection{IFA of \m{let}}

If a certain syntactic structure does not meet the above assumptions, it does not mean that this syntactic structure does not have an IFA. We can define its IFA according to its semantics. However, this method cannot be automated and requires users to ensure its correctness. For example, given the evaluation rules of \m{let}, we can specify the IFA of \m{let} as Figure \ref{fig:ifa-let}.
\[
    \infer{\m{let}~x~e_1~e_2 \to \m{let}~x~e_1'~e_2}{e_1 \to e_1'}
\]\[
    (\m{let}~x~e_1~e_2) \to e_2[e_1/x]
\]

% \infrule[E-Let]{e_1 \to e_1'}{\m{let}~x~e_1~e_2 \to \m{let}~x~e_1'~e_2}
% \infax[E-LetSubst]{(\m{let}~x~e_1~e_2) \to e_2[e_1/x]}

\begin{figure}[t]
    \centering
    \includegraphics[scale=0.25]{images/ifa/ifa-let.png}
    \caption{IFA of \m{let}}
    \label{fig:ifa-let}
\end{figure}

In the evaluation rules of \m{let}, there is a substitution. Therefore, in IFA of \m{let}, we need symbol table to express this. When $e_2$ is evaluated or expanded, it is necessary to replace $x$ with the value of node $q_0$ in $e_2$.




%===========================================
\subsection{Convert IFA to Evaluation Rules}

Next we try to convert IFA to evaluation rules. Because IFA can be converted into normal IFA by Theorem \ref{mythm:nmlifa}, we only need to convert normal IFA into rules. Unfortunately, IFA can express some derivation methods whose evaluation rules are difficult to describe. Therefore, we also need to have stricter constraints on IFA to ensure that evaluation rules can be generated.

\begin{Asm}
    \label{Asm:st}
    In a normal IFA, if $q \notin F$, then the symbol table of $q$ is empty, and the expression of $q$ cannot be a local variable.
\end{Asm}

In fact, this is a very strong assumption, which requires that only termination nodes could have substitution. Because it is difficult for us to generate a context rule for the term after substitution.

%------------------------
\subsubsection{Algorithm}

Similarly, we first give its algorithm, and then prove its correctness.

\begin{mythm}[Rules Can Be Constructed by Normal IFA]
    \label{mythm:ifa2rule}
    For each normal IFA satisfy Assumption \ref{Asm:st}, it can be converted to evaluation rules.
\end{mythm}

\begin{proof}[Proof of Theorem \ref{mythm:ifa2rule}]

    Suppose that the IFA stands for the syntactic structure $H$, then we build evaluation rules for $H$. First traverse all nodes to find the set of all terms in nodes, which is the parameters of the syntactic structure $H$ like $(\m{H}~e_1 \ldots e_n)$. Then generate evaluation rule for each node.

    Begin with $q_0$, traverse the IFA. Let $q$ be $q_0$. Record the conditions by a set $T$.

    Suppose that $q$ is a terminal node, the expression of $q$ is $e$ and the symbol table of $q$ is like $\{x:q_x; y:q_y; \ldots\}$. Let $e_x,e_y,\ldots$ be the expressions of $q_x, q_y, \ldots$. Add a reduction rule like
    \[
        \infer{(\m{H}~e_1 \ldots e_n) \to e[e_x/x][e_y/y]\ldots}{T}
    \]
    % \infrule[E-Hr]{T}{(\m{H}~e_1 \ldots e_n) \to e[e_x/x][e_y/y]\ldots}

    If $q$ is not a termination node, and the expression of $q$ is $e_i$, add a context rule like
    \[
        \infer{(\m{H}~e_1~\ldots~e_i~\ldots~e_n) \to (\m{H}~e_1~\ldots~e_i'~\ldots~e_n)}{e_i \to e_i' \quad T}
    \]

    %\infrule[E-Hi]{e_i \to e_i' \quad T}{(\m{H}~e_1~\ldots~e_i~\ldots~e_n) \to (\m{H}~e_1~\ldots~e_i'~\ldots~e_n)}

    For each condition $c$ and node $q'$ satisfying $\delta(q, c)=q'$, do the following steps separately. Replace the node in $c$ with its expression and add it to $T$. Let $q'$ be $q$. Keep doing this until $q$ is a terminal node.
\end{proof}

%----------------------------------------------------------
\subsubsection{Example: Construct Rules of \m{nand} by IFA}

Figure \ref{fig:ifa-nand-bs} is the IFA of \m{nand} which have been constructed. We can simplify it and get a normal IFA as Figure \ref{fig:ifa-nand-as}.

\begin{figure}[t]
    \centering
    \subcaptionbox{Before Simplification\label{fig:ifa-nand-bs}}[.45\linewidth]{
        \includegraphics[scale=0.28]{images/ifa/ifa-nand-4.png}
    }
    \subcaptionbox{After Simplification\label{fig:ifa-nand-as}}[.45\linewidth]{
        \includegraphics[scale=0.28]{images/ifa/ifa-nand.png}
    }
    \caption{IFA of \m{nand}}
    \label{fig:ifa-nand-std}
\end{figure}

Start with $q_0$, we get a context rule as
\[
    \infer{(\m{nand}~e_1~e_2) \to (\m{nand}~e_1'~e_2)}{e_1 \to e_1'}
\]
% \infrule[E-Nand]{e_1 \to e_1'}{(\m{nand}~e_1~e_2) \to (\m{nand}~e_1'~e_2)}

We first discuss the branch of $q_2$. Add $e_1~\key{is}~\m{\#t}$ to $T$. Because $q_2$ is not a terminal node, add a new context rule for $q_2$.
\[
    \infer{(\m{nand}~e_1~e_2) \to (\m{nand}~e_1~e_2')}{e_2 \to e_2'\quad e_1~\key{is}~\m{\#t}}
\]
% \infrule[E-NandTrue]{e_2 \to e_2'\quad e_1~\key{is}~\m{\#t}}{(\m{nand}~e_1~e_2) \to (\m{nand}~e_1~e_2')}

Since $q_4$ is a reduction rule, append $e_2~\key{is}~\m{\#t}$ to $T$ and add a new reduction rule for $q_4$. Reduction rule for $q_5$ is similar.
\[
    \begin{array}{c}
        \infer{(\m{nand}~e_1~e_2) \to \m{\#f}}{e_1~\key{is}~\m{\#t}\quad e_2~\key{is}~\m{\#t}}
        \quad
        \infer{(\m{nand}~e_1~e_2) \to \m{\#t}}{e_1~\key{is}~\m{\#t}\quad e_2~\key{is}~\m{\#f}}
    \end{array}
\]
% \infrule[E-NandTrueTrue]{e_1~\key{is}~\m{\#t}\quad e_2~\key{is}~\m{\#t}}{(\m{nand}~e_1~e_2) \to \m{\#f}}
% \infrule[E-NandTrueFalse]{e_1~\key{is}~\m{\#t}\quad e_2~\key{is}~\m{\#f}}{(\m{nand}~e_1~e_2) \to \m{\#t}}

Back to $q_0$, for $q_4'$ and $q_5'$ are also terminal nodes, we can build reduction rules for $q_4'$ and $q_5'$ in the same way.
\[
    \begin{array}{c}
        \infer{(\m{nand}~e_1~e_2) \to \m{\#f}}{e_1~\key{is}~\m{\#f}\quad \m{\#f}~\key{is}~\m{\#t}}
        \quad
        \infer{(\m{nand}~e_1~e_2) \to \m{\#t}}{e_1~\key{is}~\m{\#f}\quad \m{\#f}~\key{is}~\m{\#f}}
    \end{array}
\]
% \infrule[E-NandFalse1]{e_1~\key{is}~\m{\#f}\quad \m{\#f}~\key{is}~\m{\#t}}{(\m{nand}~e_1~e_2) \to \m{\#f}}
% \infrule[E-NandFalse2]{e_1~\key{is}~\m{\#f}\quad \m{\#f}~\key{is}~\m{\#f}}{(\m{nand}~e_1~e_2) \to \m{\#t}}

We can judge that \m{\#f} is not \m{\#t}, so we can remove the rule (NandFalse1) from the rules, for it contains a condition that is never met. At the same time, we rewrite the remaining rules into a more customary form.
\[
    \begin{array}{cc}
        \infer{(\m{nand}~e_1~e_2) \to (\m{nand}~e_1'~e_2)}{e_1 \to e_1'}
         &
        \infer{(\m{nand}~\m{\#t}~e_2) \to (\m{nand}~\m{\#t}~e_2')}{e_2 \to e_2'}
    \end{array}
\]\[
    \begin{array}{ccc}
        (\m{nand}~\m{\#t}~\m{\#t}) \to \m{\#f}
         &
        (\m{nand}~\m{\#t}~\m{\#f}) \to \m{\#t}
         &
        (\m{nand}~\m{\#f}~e_2) \to \m{\#t}
    \end{array}
\]

% \infrule[E-Nand]{e_1 \to e_1'}{(\m{nand}~e_1~e_2) \to (\m{nand}~e_1'~e_2)}
% \infrule[E-NandTrue]{e_2 \to e_2'}{(\m{nand}~\m{\#t}~e_2) \to (\m{nand}~\m{\#t}~e_2')}
% \infax[E-NandTrueTrue]{(\m{nand}~\m{\#t}~\m{\#t}) \to \m{\#f}}
% \infax[E-NandTrueFalse]{(\m{nand}~\m{\#t}~\m{\#f}) \to \m{\#t}}
% \infax[E-NandFalse]{(\m{nand}~\m{\#f}~e_2) \to \m{\#t}}


%--------------------------
\subsubsection{Correctness}

\begin{lemma}
    \label{lemma:ifa2rule-correct}
    For any syntactic structure \m{H}, if its normal IFA meets the above assumptions, the evaluation rules obtained according to the algorithm in Theorem \ref{mythm:ifa2rule} have the same semantics as IFA.
\end{lemma}

In other words, for any term of syntactic structure \m{H}, evaluating the term by IFA and evaluating by rule get the same derivation sequence.

\begin{proof}[Proof of Lemma \ref{lemma:ifa2rule-correct}]
    We only need to discuss that in a one-step derivation, both get the same result.

    Considering a term $e=(\m{H}~e_1 \ldots e_n)$ use rule $r$ to derive. The term $e$ must meet the condition $T$ of $r$ such as some parameters must be value or a specific value. Suppose $r$ is generated by node $q$. Then $T$ is the set of all transition conditions from $q_0$ to $q$. Therefore, the one-step derivation of this term $e$ in IFA must be located at $q$. If $r$ is a context rule for a pattern variable $e_i$, the expression of $q$ is $e_i$ as well, and $e_i$ of $e$ is not a value. Thus both of the one-step derivation of $e$ are one-step derivation of $e_i$.

    Similarly, if an term can be derived in one step in IFA, then it must be able to use the corresponding rule for one-step derivation.
\end{proof}

%===========================
\subsection{Syntactic Sugar}

We can find that although rules of \m{nand} we used when constructing the IFA included the syntactic structure of \m{if}, the final result does not. For syntactic sugar, we also deal with it with a similar idea, that is, construct the IFA of syntactic sugar, simplify it and convert it into rules. With the IFA, we can easily get the evaluation rules for syntactic sugars.

Similarly, we also need to add some constraints on syntactic sugar.

\begin{Asm}[Orderliness of Syntactic Sugar]
    \label{Asm:orderliness-sugar}
    The definition of each syntactic sugar can only use the syntactic structure in coreLang and the syntactic sugar that has been defined.
\end{Asm}

\begin{Def}
    \label{def:ifa-sugar}
    Considering the following syntactic sugar
    \[
        \drule{(\m{SurfHead}~x_1~\ldots~x_n)}{e},
    \]
    the IFA of \m{SurfHead} is defined as the IFA of syntactic structure \m{SurfHead'} whose evaluation rule is
    \[
        (\m{SurfHead'}~x_1~\ldots~x_n) \to e
    \]

\end{Def}

%--------------------------------------------
\subsubsection{An Example of Syntactic Sugar}

Suppose that there are only \m{if} and \m{let} in our CoreLang, whose IFAs are known. Now we build rules for syntactic sugar \m{or} and \m{sg}.
\[
    \drule{(\m{or}~e_1~e_2)}{(\m{let}~x~e_1~(\m{if}~x~x~e_2))}
\]

\m{or} syntactic sugar only uses the syntax structure of CoreLang, which meets Assumption \ref{Asm:orderliness-sugar}. Then we generate evaluation rules for \m{or}.

\begin{figure}[t]
    \centering
    \subcaptionbox{Rule of Definition \ref{def:ifa-sugar} \label{fig:ifa-ex-or-1}}[.31\linewidth]{
        \includegraphics[scale=0.3]{images/ifa/ifa-ex-or-1.png}
    }
    \subcaptionbox{IFA generated with the algorithm of Theorem \ref{mythm:Rule2IFA} \label{fig:ifa-ex-or-2}}[.31\linewidth]{
        \includegraphics[scale=0.3]{images/ifa/ifa-ex-or-2.png}
    }
    \subcaptionbox{Expand syntactic structure of \m{let} \label{fig:ifa-ex-or-3}}[.31\linewidth]{
        \includegraphics[scale=0.3]{images/ifa/ifa-ex-or-3.png}
    }
    \subcaptionbox{Expand syntactic structure of \m{if} \label{fig:ifa-ex-or-4}}[.31\linewidth]{
        \includegraphics[scale=0.3]{images/ifa/ifa-ex-or-4.png}
    }
    \subcaptionbox{Replace $x$ with expression according to the symbol table \label{fig:ifa-ex-or-5}}[.31\linewidth]{
        \includegraphics[scale=0.3]{images/ifa/ifa-ex-or-5.png}
    }
    \subcaptionbox{Rules generated with the algorithm of Theorem \ref{mythm:ifa2rule} \label{fig:ifa-ex-or-6}}[.31\linewidth]{
        \includegraphics[scale=0.3]{images/ifa/ifa-ex-or-6.png}
    }
    \caption{Example: Syntactic Sugar of \m{or}}
    \label{fig:ifa-nand-std}
\end{figure}

The IFA of \m{or} is the same as the IFA of \m{or'} whose rule is shown in Figure \ref{fig:ifa-ex-or-1}. Therefore, we can generate an IFA according to the rule as shown in Figure \ref{fig:ifa-ex-or-2}. Next we transform IFA to make it a normal IFA, as shown in Figure \ref{fig:ifa-ex-or-3} to Figure \ref{fig:ifa-ex-or-5}. Finally, according to the structure of IFA, generate or evaluation rules, as shown in Figure \ref{fig:ifa-ex-or-6}.

% \infrule[E-Or]{e_1 \to e_1'}{(\m{or}~e_1~e_2) \to (\m{or}~e_1'~e_2)}
% \infrule[E-OrTrue]{e_1~\key{is}~\m{\#t}}{(\m{or}~e_1~e_2) \to e_1}
% \infrule[E-OrFalse]{e_1~\key{is}~\m{\#f}}{(\m{or}~e_1~e_2) \to e_2}

%--------------------------
\subsubsection{Correctness}

As discussed in Section \ref{sec3}, our approach should satisfy the three properties: emulation, abstraction and coverage. The property of abstraction is obvious, for the rules we generate only contain the syntactic structure itself, without any other structures in core language. However, our approach does not perfectly satisfy emulation and coverage. In the derivation, we lost the information that a syntactic structure is reduced to another syntactic structure. This makes the derivation sequence different. But through Theorem \ref{mythm:nmlifa}, Lemma \ref{lemma:rule2ifa-correct} and Lemma \ref{lemma:ifa2rule-correct}, we can guarantee the correctness of the evaluation results.

%!TEX root = ./main.tex
\section{Experiment}
\label{sec4}

In this section, we present several examples to show different aspects of our approach.

\subsection{Case Study on Expressiveness}

We have implemented our resugaring approach using PLT Redex \cite{SEwPR}, which is a semantic engineering tool based on reduction semantics \cite{reduction}. We show several case studies to demonstrate the power of our approach. Some examples we will discuss in this section are in Fig.  \ref{fig:resugaring}. Note that we set call-by-value lambda calculus as terms in \m{CommonExp}, because we want to output some intermediate expressions including lambda calculus in some examples. It's easy if we want to skip them.

\begin{figure*}[t]
\centering
\begin{subfigure}{.3\linewidth}
	\begin{spacing}{0.5}
		\begin{flushleft}
			{\scriptsize
			\begin{Codes}
			\qquad (S (K (S I)) K xx yy)\\
			\OneStep{ (((K (S I)) xx (K xx)) yy)}\\
			\OneStep{ (((S I) (K xx)) yy)}\\
			\OneStep{ (I yy ((K xx) yy))}\\
			\OneStep{ (yy ((K xx) yy))}\\
			\OneStep{ (yy xx)}\\
			\end{Codes}
			}
		\end{flushleft}
	\end{spacing}
	
	\caption{Example of SKI}
	\label{fig:SKI}
\end{subfigure}
\begin{subfigure}{.3\linewidth}
	\begin{spacing}{0.5}
		\begin{flushleft}
			{\scriptsize
			\begin{Codes}
			    (let x 2 (Hygienicadd 1 x))\\
			\OneStep{ (Hygienicadd 1 2)}\\
			\OneStep{ (+ 1 2)}\\
			\OneStep{ 3}\\[2.25em]
			
			\end{Codes}
			
			}
		\end{flushleft}
		
	\end{spacing}
	
	\caption{Example of \m{Hygienicadd}}
	\label{fig:hygienicadd}
\end{subfigure}
\begin{subfigure}{.25\linewidth}
	\begin{spacing}{0.5}
		\begin{flushleft}
			{\scriptsize
			\begin{Codes}
			    (Odd 2)\\
			\OneStep{ (Even (- 2 1))}\\
			\OneStep{ (Even 1)}\\
			\OneStep{ (Odd (- 1 1))}\\
			\OneStep{ (Odd 0)}\\
			\OneStep{ \#f}\\
			\end{Codes}
			}
		\end{flushleft}
		
	\end{spacing}
	
	\caption{Example of \m{Odd} and \m{Even}}
	\label{fig:oddeven}
\end{subfigure}
%newline
\begin{subfigure}{.4\linewidth}
		\begin{flushleft}
			{\scriptsize
			% \begin{Codes}
				\quad\;\;\;\Code{ (Map ($\lambda$ (x) (+ x 1)) (cons 1 (list 2)))}\\
			\OneStep{ \Code{ (Map ($\lambda$ (x) (+ x 1)) (list 1 2))}}\\
			\OneStep{ \Code{ (cons 2 (Map ($\lambda$ (x) (+ 1 x)) (list 2)))}}\\
			\OneStep{ \Code{ (cons 2 (cons 3 (Map ($\lambda$ (x) (+ 1 x)) (list))))}}\\
			\OneStep{ \Code{ (cons 2 (cons 3 (list)))}}\\
			\OneStep{ \Code{ (cons 2 (list 3))}}\\
			\OneStep{ \Code{ (list 2 3)}}\\[2em]
			% \end{Codes}
			}
		\end{flushleft}
		
	
	\caption{Example of \m{Map}}
	\label{fig:Map}
\end{subfigure}
\begin{subfigure}{.5\linewidth}
		\begin{flushleft}
			{\scriptsize
			\quad\;\;\;\Code{ (Filter ($\lambda$ (x) (and (> x 1) (< x 4))) (list 1 2 3 4))}\\
			\OneStep{ \Code{ (Filter ($\lambda$ (x) (and (> x 1) (< x 4))) (list 2 3 4))}}\\
			\OneStep{ \Code{ (cons 2 (Filter ($\lambda$ (x) (and (> x 1) (< x 4))) (list 3 4)))}}\\
			\OneStep{ \Code{ (cons 2 (cons 3 (Filter ($\lambda$ (x) (and (> x 1) (< x 4))) (list 4))))}}\\
			\OneStep{ \Code{ (cons 2 (cons 3 (Filter ($\lambda$ (x) (and (> x 1) (< x 4))) (list))))}}\\
			\OneStep{ \Code{ (cons 2 (cons 3 (list)))}}\\
			\OneStep{ \Code{ (cons 2 (list 3))}}\\
			\OneStep{ \Code{ (list 2 3)}}
			
			}
		\end{flushleft}
	
	\caption{Example of \m{Filter}}
	\label{fig:Filter}
\end{subfigure}

\caption{Resugaring Examples}
\label{fig:resugaring}
\end{figure*}


\subsubsection{Simple Sugars}
\label{mark:simple}

We construct some simple syntactic sugars and try it on our tool. Some sugars are inspired by the first work of resugaring \cite{resugaring}. 
Take an SKI combinator syntactic sugar as an example. We can regard \m{S} as an expression headed with S, without sub-expression. And for showing a concise result, we add the call-by-need lambda calculus in the core language for this example.
\[
\begin{array}{l}
\drule{\m{S}}{\Code{($\lambda_N$ ($x_1$ $x_2$ $x_3$) ($x_1$ $x_2$ ($x_1$ $x_3$)))}}\\
\drule{\m{K}}{\Code{($\lambda_N$ ($x_1$ $x_2$) $x_1$)}}\\
\drule{\m{I}}{\Code{($\lambda_N$ ($x$) $x$)}}
\end{array}
\]




Although SKI combinator calculus is a reduced version of lambda calculus, we can construct combinators' sugars based on call-by-need lambda calculus in our core language. For the sugar program \Code{(S (K (S I)) K xx yy)}, we get the resugaring sequences as Fig.  \ref{fig:SKI}. Here the sugars contain no sub-expression, then the sugar should just desugar to the core expression. It is interesting that the sugar without sub-expressions (written by lambda calculus) and the sugar with sub-expressions will behave differently. For example, in this case, we can write the sugar as \Code{(S e e e)} and so on, then the sugar may not have to be expanded. Moreover, we can use this difference to force a syntactic sugar desugared using call-by-need lambda calculus. See if we want a sugar \m{ForceAnd} which does not
want to use the context rules of \m{if} to getting the resugaring sequence, we can just write the following sugar.

\[
\drule{\Code{ForceAnd}}{\Code{($\lambda_N$ ($x_1$ $x_2$) (if $x_1$ $x_2$ \false))}}
\]



\ignore{
  For the existing approach, the sugar expression should firstly desugar to
\begin{Codes}
((lambdaN (x1 x2 x3) (x1 x3 (x2 x3)))
  ((lambdaN (x1 x2) x1)
   ((lambdaN  (x1 x2 x3) (x1 x3 (x2 x3)))
    (lambdaN (x) x)))
  (lambdaN (x1 x2) x1)
  xx yy)
\end{Codes}
\reduce{can be removed}

Then in our core language, the execution of expanded expression will contain 33 reduction steps in our implementation. For each step, there will be many attempts to match and substitute the syntactic sugars to resugar the expression. It will omit more steps for a larger expression.
}


\subsubsection{Hygienic Sugars}
\label{mark:hygienic}


The second work \cite{hygienic} of existing resugaring approach mainly processes hygienic sugar compared to the first work. It uses a DAG to represent the expression. However, hygiene is not hard to be handled by our lazy desugaring strategy. Our algorithm can easily process hygienic sugar without a special data structure.
A typical hygienic problem is as the following example.
\[
\drule{\Code{(Hygienicadd $e_1$ $e_2$)}}{\Code{(let ((x $e_1$)) (+ x $e_2$))}}
\]
% \begin{Codes}
% 	(Hygienicadd e1 e2) \DeStep{ (let ((x e1)) (+ x e2))}
% \end{Codes}
For the existing resugaring approach, if we want to get sequences of \Code{(let (x 2) (Hygienicadd 1 x))}, it will firstly desugar to \Code{(let (x 2) (let x 1 (+ x x)))}, which is awful because the two $x$ in \Code{(+ x x)} should be bound to different values. So the existing hygienic resugaring approach uses abstract syntax DAG to distinct different \m{x} in the desugared expression. But for our approach based on lazy desugaring, the \m{Hygienicadd} sugar does not have to expand until necessary, thus, getting resugaring sequences as Fig.  \ref{fig:hygienicadd} based on a non-hygienic transformer system. We will discuss hygienic problem in Section \ref{mark:hygiene}.


% The lazy desugaring is also convenient for achieving hygiene for non-hygienic core language. For example, \Code{(let x 1 (+ x (let x 2 (+ x 1))))} may be reduced to \Code{(+ 1 (let 1 2 (+ 1 1)))} by a simple core language whose \Code{let} expression does not handle cases like that. But by writing a sugar (not syntactic sugar in the usual sense, because we do not want it to behave as \m{let}) \m{Let}
% \[\drule{\Code{(Let~$e_1$~$e_2$~$e_3$)}}{\Code{(let~($e_1$~$e_2$)~$e_3$)}}\]
% and making the following modifies in the reduction of mixed language: rejecting the one-step try (i.e., delaying the expansion of the sugar, another form of lazy desugaring) if an error occurs, then recursively applying $\redm{}{}$ on the subexpression where the error takes place. In this example, it is to delay the expansion of outermost \m{Let} and apply $\redm{}{}$ on \Code{(+ x (Let x 2 (+ x 1)))}. We will get the resugaring sequences as Figure \ref{fig:Let} in our tool. It is not resugaring in the usual sense for violating the emulation property, but can be useful for implementing lightweight hygiene.

In practical application, we think hygiene can be easily processed by more complex transformer systems (such as \cite{10.5555/1792878.1792884}). Overall, our results show lazy desugaring is a good way to handle hygienic sugars in any system.

\subsubsection{Recursive Sugars}
\label{sec:recursiveSugar}

Recursive sugar is a kind of syntactic sugars where calls itself or each other during the expansion. For example,
\[
\begin{array}{l}
\drule{(\m{Odd}~e)}{(\m{let}~((x~e))~(\m{if}~(>~x~0)~(\m{Even}~(-~x~1))~\false))}\\
\drule{(\m{Even}~e)}{(\m{let}~((x~e))~(\m{if}~(>~x~0)~(\m{Odd}~(-~x~1))~\true))}
\end{array}
\]
are recursive sugars. The existing resugaring approach can't process syntactic sugar written like this (non-pattern-based) easily, because boundary conditions are in the sugar itself.

Take \Code{(Odd 2)} as an example. The previous work will firstly desugar the program using the rules. Then the desugaring will never terminate as the following shows.
\begin{footnotesize}
\begin{Codes}
   (Odd 2)
\DeStep{ (let ((x 2)) (if (> x 0) (Even (- x 1)) \#f))}
\DeStep{ (let ((x 2)) (if (> x 0)}
\qquad\quad(let ((x1 (- x 1))) (if (> x1 0) (Odd (- x1 1)) \#t))
\qquad\quad\#f))
\DeStep{ ...}
\end{Codes}
\end{footnotesize}



Then the advantage of our approach is embodied. Our approach does not require a whole expansion of sugar expression, which gives the framework chances to judge boundary conditions in sugars themselves and showing more intermediate sequences. We get the resugaring sequences as Fig.  \ref{fig:oddeven} of the former example using our tool.



We also construct some higher-order syntactic sugars and test them. The higher-order feature is important for constructing practical syntactic sugars. And many higher-order sugars should be constructed by recursive definition. The first sugar is \m{Filter}, implemented by pattern matching.
\[\begin{array}{l}
\drule{\Code{\small(Filter $e$ (list $v_1$ $v_2$ ...))}}{}\\
\quad
\Code{\small (let ((f $e$)) (if (f $v_1$)}\\
\qquad\qquad\qquad\qquad\;\;\Code{\small (cons $v_1$ (Filter f (list $v_2$ ...)))}\\
\qquad\qquad\qquad\qquad\;\;\Code{\small (Filter f (list $v_2$ ...))))}\\

\drule{\Code{\small(Filter $e$ (list))}}{\Code{\small(list)}}
\end{array}\]
and getting resugaring sequences as Fig.  \ref{fig:Filter}.
Here, although the sugar can be processed by the existing resugaring approach, it will be redundant. The reason is that a \m{Filter} program for a list of length $n$ will match to find possible resugaring $n*(n-1)/2$ times. Thus, lazy desugaring is really important to reduce the resugaring complexity of recursive sugar.

Moreover, just like the \m{Odd} and \m{Even} sugar above, there are some simple desugaring systems which do not allow pattern-based desugaring. Or there are some sugars that need to be expressed by the expressions in core language as branching conditions. Take the example of another higher-order sugar \m{Map} as an example, and get resugaring sequences as Fig.  \ref{fig:Map}.
\[
\begin{array}{l}
\drule{\Code{\small(Map $e_1$ $e_2$)}}{}\\
\quad\Code{\small (let ((f $e_1$))}\\
\qquad\Code{\small(let ((x $e_2$))}\\
\qquad\quad

\Code{\small(if (empty? x)}\\
\qquad\qquad\;\;\;\Code{\small(list)}\\
\qquad\qquad\;\;\;\Code{\small(cons (f (first x)) (Map f (rest x))))))}


\end{array}
\]



Note that the \m{let} expression is to limit the sub-expression only appears once in RHS. In this example, we can find that the list \Code{(cons 1 (list 2))}, though equal to \Code{(list 1 2)}, is represented by core language's expression. So it will be difficult to naturally handle such inline boundary conditions for existing desugaring systems. (The case can be specific by some setting, such as local-expansion\cite{10.1017/S0956796812000093} in Racket language's macro.) But our approach is easy to handle cases like this without specifying the expansion.

\subsubsection{Limitation on Presentation}
One may note that the context rules of our sugar setting limit the presentation of syntactic sugar. For example, it is difficult to present a sugar with ellipses, because the form of its context rule may vary. It is still possible if we add some 
restriction (so that the algorithm \ref{alg:f} will work), or we can just make it using the list operation just as what the sugar \m{Map, Filter} work. Overall, the presentation of our sugar system is not so flexible, but it won't affect the expressiveness.

\subsection{Efficiency}
% \begin{figure}[thb]
%   \begin{center}\small
%   \begin{tabular}{l | l | l |l}
%     \emph{Sugar} & \emph{Ours Steps} & \emph{Existing's Steps} & \emph{Description} \\ \hline
%     And, Or & 4  & 2+0+2=4 & \\
%     And, Newor & 10 & 7+6+3=16 & Or with binding \\
%     Filter & 29 & 22+78+10=110 &  \\
%     Sg, and, or & 11 & 5+0+10=15 & Sg is nested \\ 
%     HygienicAdd & 6 & 5+1+1=7 &  \\
%     S, K, I & 16 & 11+13+10=34 &  \\
%   \end{tabular}
%   \end{center}
%   \caption{Comparison on Reduction Steps}
%   \label{fig:step}
%   \end{figure}

\begin{figure}[t]
	\centering
	\includegraphics[width=0.48\textwidth]{images/efficiency.png}
	\caption{Comparison on Reduction Steps}
	\label{fig:step}
\end{figure}
To show the efficiency of our approach, we use the reduction steps comparing to the existing approach as the metrics. The following Fig. \ref{fig:step} shows the difference between the two approaches. Notice that both approaches have pre-processions---for the existing one, it is to desugar the programs to the core language together with some tags; for ours, it is to calculate the context rules of syntactic sugars. We do not consider the steps during these pre-processes. Besides, we derive the reduction steps of the existing approach into three different kinds---the reductions in the core language, the reverse expansion with failed resugaring, the reverse expansion with successful resugaring.  Use the following example to see the difference. Consider a sugar named \m{Hard} with two arguments, which has many reduction steps after desugared. Assuming for specific $e_1$ and $e_2$, the \Code{(Hard $e_1$ $e_2$)} after fully desugared has 100 reduction steps (finally to \#f, for example), and only 1 intermediate step can be resugared (to \Code{(Hard $v_1$ $e_2$)}, for example). Then for \Code{(And (Hard $e_1$ $e_2$) \#t)}, although all the 100 steps in the core language try the reverse desugaring, only \Code{(if stepn \#t \#f)} (2 steps on \m{And}, \m{Hard}) and \Code{(if \#f \#t \#f)} (1 step on \m{And}) are successful. Other 98 attempts will be failed together with 98 steps on \m{And}.

\[
{\footnotesize
	\begin{array}{lcl}
	Surface&&Core\\
	\Code{(And (Hard $e_1$ $e_2$) \#t)}&\xrightarrow{desugar}&\Code{(if step0 \#t \#f)}\\
	\qquad\quad\dashdownarrow& &\qquad\qquad\downarrow\\
	\Code{(And step1 \#t)}&\xleftarrow{resugar}&\Code{(if step1 \#t \#f)}\\
	\qquad\quad\vdots& &\qquad\qquad\vdots\\
	\Code{(And (Hard $v_1$ $e_2$) \#t)}&\xleftarrow{resugar}&\Code{(if stepn \#t \#f)}\\
	\qquad\quad\vdots& &\qquad\qquad\vdots\\
	\Code{(And \#f \#t)}&\xleftarrow{resugar}& \Code{(if \#f \#t \#f)}\\
	\qquad\quad\dashdownarrow& &\qquad\qquad\downarrow\\
	\Code{\#f}&& \Code{\#f}\\
\end{array}
}
\]


The general regularity is---the more complex the sugar is, the more steps will our approach save. Note that if the RHS of a syntactic sugar is huge, one-step reduction of the reverse desugaring will also be more complex, because the huge sugars will contain many failed attempts to resugar. So avoiding reverse expansion of syntactic sugar can improve the efficiency for practical use because there are not always small programs like the demos.
%!TEX root = ./main.tex
\section{Discussion}
\label{sec5}

\subsection{Model Assumption and A Black-Box Extension}
\label{sec5.1}


As we mentioned in the introduction (Section \ref{mark:mention}), our approach assumes a more specific model (evaluation rules) compared to the existing approach (black-box stepper). Here is a small gap between the motivation of the existing approach and ours---the existing approach focused mainly on a tool for existing language, while our approach considered more on a meta-level feature for language implementation. The examples in Section \ref{sec:recursiveSugar} have shown how the lazy desugaring solves some problems in practice.

In addition, as what we need for the lazy desugaring is just the computational order of the syntactic sugar, we can make an extension for the resugaring algorithm to work with only a black-box core language stepper. The most important difference between the black-box stepper and the evaluation rules is the computational order---while the same language behaves uniquely, the evaluation rules can show the computational order statically (without running the program). So when meeting the black-box stepper for the core language, we can just use some simple program to "get" the computational order of the core language as the following example  shows: we simply let the sub-expressions of a \m{Head} be some reducible expressions and test the computational order.

\begin{center}\footnotesize
	\Code{(if $\m{tmpe}_1$ $\m{tmpe}_2$ $\m{tmpe}_3$)}\\ $\Downarrow_{stepper}$\\ \Code{\qquad\qquad\qquad\qquad\;\;(if $\m{tmpe}_1$' $\m{tmpe}_2$ $\m{tmpe}_3$)}\note{//getting a context rule}\\ $\Downarrow_{getnext}$\\ \Code{(if $\m{tmpv}_1$ $\m{tmpe}_2$ $\m{tmpe}_3$)}\\ $\Downarrow_{stepper}$\\ \qquad\Code{$\m{tmpe}_i$}\note{//no more rules}\\
\end{center}


But that's not enough---the core language and the surface language cannot be mixed easily because of the lack of evaluation rules for the core language. We should do the same try during the evaluation to make the core language's stepper useful when meeting some surface language's expression. Here we give a dynamic on-step reduction of the mixed language. Note that here we only define the reduction for unnested syntactic sugar for convenience. It is easy to extend to nested sugar (but so huge to express). 

\begin{Def}[Dynamic mixed language's one-step reduction $\redm{}{}$] Defined in Fig.  \ref{fig:dynamic}.
\end{Def}
\begin{figure*}[t]\footnotesize
\infrule[CoreRed]
{ \forall~i.~e_i\in \m{CoreExp}\\
\redc{(\m{CoreHead}~e_1~\ldots~e_n)}{e'}}
{\redm{(\m{CoreHead}~e_1~\ldots~e_n)}{e'}}
\infrule[CoreExt1]
{ \forall~i.~tmp_i= (e_i \in \m{SurfExp}~?~\m{tmpe}~:~e_i)\\
\redc{(\m{CoreHead}~tmp_1~\ldots~tmp_i~\ldots~tmp_n)}{(\m{CoreHead}~tmp_1~\ldots~tmp_i'~\ldots~tmp_n)}}
{\redm{(\m{CoreHead}~e_1~\ldots~e_i~\ldots~e_n)}{(\m{CoreHead}~e_1~\ldots~e_i'~\ldots~e_n)}\\where~\redm{e_i}{e_i'}}
\infrule[CoreExt2]
{ \forall~i.~tmp_i= (e_i \in \m{SurfExp}~?~\m{tmpe}~:~e_i)\\
\redc{(\m{CoreHead}~tmp_1~\ldots~tmp_n)}{e'}~\note{// not reduced in sub-expressions}}
{\redm{(\m{CoreHead}~e_1~\ldots~e_n)}{e'[e_1/tmp_1~\ldots~e_n/tmp_n]}}
\infrule[SurfRed1]
{\drule{(\m{SurfHead}~x_1~\ldots~x_i~\ldots~x_n)}{e}~\\
\exists i.\, \redm{e[e_1/x,\ldots,e_i/x_i,\ldots,e_n/x_n]}{e[e_1/x,\ldots,e_i'/x_i,\ldots,e_n/x_n]}~\&~
\redm{e_i}{e_i''}
}
{\redm{(\m{SurfHead}~e_1~\ldots~e_i~\ldots~e_n)}{(\m{SurfHead}~e_1~\ldots~e_i''~\ldots~e_n)}}
\infrule[SurfRed2]
{\drule{(\m{SurfHead}~x_1~\ldots~x_i~\ldots~x_n)}{e}\\
\not \exists i.\, \redm{e[e_1/x_1,\ldots,e_i/x_i,\ldots,e_n/x_n]}{e[e_1/x_1,\ldots,e_i'/x_i,\ldots,e_n/x_n]~\&~
\redm{e_i}{e_i''}}
}
{\redm{(\m{SurfHead}~e_1~\ldots~e_i~\ldots~e_n)}{e[e_1/x_1,\ldots,e_i/x_i,\ldots,e_n/x_n]}}
\footnotesize{where~\m{tmpe}~is~any~reduciable~\m{CoreExp}~expression}
\caption{Dynamic Reduction}
\label{fig:dynamic}
\end{figure*}

Putting them in words. For expression \Code{(SurfHead $e_1$ $\ldots$ $e_n$)}, as we discussed in Section \ref{mark:correct}, only reduction on the $e_i$ of the sugar's $LHS$ will not destroy the $RHS$'s form. So we can just take a try after the expansion of \m{SurfHead}. 

For an expression \Code{(CoreHead $e_1$ $\ldots$ $e_n$)}  whose sub-expressions contain \m{SurfExp}, replacing all \m{SurfExp} sub-expressions with any reducible core language's expression $\m{tmpe}_i$ . Then getting a result after inputting the new expression $e'$ to the original black-box stepper. Then two possible cases come.

If reduction appears at a sub-expression at $\m{tmpe}_i$'s location, then the stepper with the extension should return \Code{(CoreHead $e_1$ $\ldots$ $e_i'$ $\ldots$ $e_n$)}, where $e_i'$ is $e_i$ after the mixed language's one-step reduction ($\redm{}{}$) as the following example (rule $\mathtt{CoreExt1}$)
\begin{center}\scriptsize
	\Code{(if (and e1 e2) (or e1 e2) \#f)}\\ $\Downarrow_{replace}$\\ \Code{(if (not \#t) (not \#t) \#f)}\\ $\;\;\Downarrow_{blackbox}$\\ \Code{(if \#f (not \#t) \#f)}\\ $\quad\Downarrow_{reduction}$\\ \Code{(if (if e1 e2 \#f) (or e1 e2) \#f)}
\end{center}

Otherwise (no reduction on $\m{tmp}_i$), the stepper should return \Code{$e'$}, with all the replaced sub-expressions replacing back (rule $\mathtt{CoreExt2}$).
\begin{center}\scriptsize
	\Code{(if \#t (and ...) (or ...))}\\ $\Downarrow_{replace}$ \\\Code{ (if \#t $\m{tmpe}_2$ $\m{tmpe}_3$)}\\ $\;\;\Downarrow_{blackbox}$\\  \Code{$\m{tmpe}_2$}\\ $\quad\;\;\;\Downarrow_{replaceback}$\\ \Code{(and ...)}
\end{center}
We call the extension "one-step-try", because it tries one step on the expression in the black-box stepper. The extension will not violate the properties of the original core language's evaluator. It is obvious that the evaluator with the extension will reduce at the sub-expression as it needs in the core language, if the reduction appears in a sub-expression. The stepper with extension behaves the same as mixing the evaluation rules of the core language and desugaring rules of surface language.

But something goes wrong when substitution takes place during \m{CoreExt2}. For a program like \Code{(let (x 2) (Sugar x y))} as an example, it should reduce to \Code{(Sugar 2 y)} by the \m{CoreRed2} rule, but got \Code{(Sugar x y)} by the \m{CoreExt2} rule. So when using the extension of black-box stepper's rule (\m{ExtRed2}), we need some other information about in which sub-expression a substitution will occur. Then for these sub-expressions, we need to do the same substitution before replacing back. The substitution can be got by a similar idea as the dynamic reduction in our simple core language's setting. For example, we know the third sub-expression of an expression headed with \m{let} is to be substituted. we should first try \Code{(let (x 2) x)}, \Code{(let (x 2) y)} in one-step reduction to get the substitution [2/x], then, getting \Code{(Sugar 2 y)}.

Then for any sugar expression, we can process them dynamically by "one-step-try" like the example in Fig.  \ref{example:try}. (The bold \m{Head} means trying on this expression.)
\example{
\[
{\footnotesize
	\begin{array}{lcl}
	\text{resugaring}&&\text{one-step-try}\\
	\Code{({\bfseries And} (Or \#t \#f)}&\xrightarrow{try}&\Code{(if ({\bfseries Or} \#t \#f)}\\
	\Code{\qquad\hspace{0.5em}(And \#f \#t))}&&\Code{\qquad(And \#f \#t)}\\
	& &\Code{\qquad\#f)}\\
	\qquad\quad\dashdownarrow& &\qquad\qquad\downarrow\\
	\Code{(And ({\bfseries Or} \#t \#f)}& &\Code{(And ({\bfseries if} \#t \#t \#f)}\\
	\Code{\qquad\hspace{0.5em}(And \#f \#t))}&&\Code{\qquad\hspace{0.5em}(And \#f \#t))}\\
	\qquad\quad\dashdownarrow& &\qquad\qquad\downarrow\\
	\Code{({\bfseries And} \#t}&\xrightarrow{try}&\Code{({\bfseries if} \#t}\\
	\Code{\qquad\hspace{0.5em}(And \#f \#t))}&&\Code{\qquad\hspace{0.5em}(And \#f \#t)}\\
	& &\Code{\qquad\hspace{0.5em}\#f)}\\
	\qquad\quad\dashdownarrow& &\qquad\qquad\downarrow\\
	\Code{({\bfseries And} \#f \#t)}&\xrightarrow{try}&\Code{({\bfseries if} \#f \#t \#f)}\\
	\qquad\quad\dashdownarrow& &\\
	\Code{\#f}& &\\
\end{array}
}
\]
}{Example of One-Step-Try}{example:try}



\subsection{Properties and Trade-off}
\label{mark:correctness}

The existing resugaring approaches \cite{resugaring,hygienic} proposed the following three properties to define the correctness.

\begin{quote}
Emulation:
Each term in the generated surface evaluation sequence desugars into the core term which it is meant to represent.\\
Abstraction:
Code introduced by desugaring is never revealed in the surface evaluation sequence, and code originating from the original input program is never hidden by resugaring.\\
Coverage: Resugaring is attempted on every core step, and as few core steps are skipped as possible.\\
\end{quote}
Here we will show what are the similarities and differences between theirs and our properties.

\emph{Emulation}: The properties in Section \ref{sec3} is just the same as the emulation property. It is the most basic one.

\emph{Abstraction and Coverage}: Our reduction in the mixed language has some similarities to theirs. But since our framework has no execution for the fully desugared program and no reverse desugaring, there are some differences in details.

Overall, our approach restricts the output by the \m{Head} of an expression and its sub-expressions. It is quite natural since the motivation of the resugaring is to show useful intermediate sequences, we think it will be better than restricting the output by judging whether the intermediate expressions contain some components desugared from the original program's components. Take the following sugar definitions as an example.
\[
\drule{\Code{(Nor~x~y)}}{\Code{(And~(not~x)~(not~y))}}
\]
\[
\drule{\Code{(And~x~y)}}{\Code{(if~x~y~\false)}}
\]
Then for a logic domain, what should be a resugaring sequence of the program \Code{(not (And (Nor \false~\true) \true))} ?

In our opinion, if the outer \m{not}, \m{And} can be displayed, so they should be after desugared.
The existing approach will produce the sequences as follows.
\begin{footnotesize}
\begin{Codes}
    (not (And (Nor \false \true) \true))
\OneStep{ (not (And \false \true))}
\OneStep{ (not \false)}
\OneStep{ \true}
\end{Codes}
\end{footnotesize}
while ours will produce the following sequences.
\begin{footnotesize}
\begin{Codes}
    (not (And (Nor \false \true) \true))
\OneStep{ (not (And (And (not \false) (not \true)) \true))}
\OneStep{ (not (And (And \true (not \true)) \true))}
\OneStep{ (not (And (not \true) \true))}
\OneStep{ (not (And \false \true))}
\OneStep{ (not \false)}
\OneStep{ \true}
\end{Codes}
\end{footnotesize}

Also, if we want to display the core language's expression only when it is originated from the input program, we can just make a mirror for it as a \m{CommonHead}. For example, when we want to show resugaring sequences of \Code{(And (if (And \#t \#f) ...) ...)}
without showing the \m{if} expression expanded from \m{And}, we only need to set \m{If} as \m{CommonHead} together with its evaluation rules same as \m{if}.

In summary, our approach chooses a slightly different way for the \emph{abstraction} for better \emph{coverage} in the real application.
\subsection{Hygiene}
\label{mark:hygiene}

As an important property for sugar or macro system, we used to think it necessary to achieve hygiene by combining the approach with an existing hygienic desugaring system. But during the experiment, we find it naturally solve the hygienic problem with the original desugaring system in our language setting.

In our approach, the sugar can contain some bindings, written by the core language's \m{let}. The hygienic problem only happens when binders of an expanded sugar conflict with other binders. We classify them into following two cases. Any hygienic problems are composite by the two cases.

The first one is that, a sugar expression exists in binding's evaluation context. For the sugar \m{Or1} with following rule,
\[\drule{\Code{(Or1~$e_1$~$e_2$)}}{\Code{(let (t $e_1$) (if t t $e_2$))}}\]
The program \Code{(let (t \#t) (Or1 \#f t))} is of the case. But because of the context rule of \m{let}, the sugar \m{Or1} will not be expanded before the \m{t} is substituted. So the program reduces to \Code{(Or1 \#f \#t)} first, so avoiding the hygienic problem. Because the bound variables in sugar expressions are only introduced by let-binding, all of them can "delay" the expansion of the syntactic sugar.

The second one is that, a sugar expression which introduced binding by the sugar expansion contains bindings in its sub-expression. For the sugar \m{Subst} with following rule,
\[
\drule{\Code{(Subst $e_1$ $e_2$ $e_3$)}}{\Code{(let ($e_2$ $e_3$) $e_1$)}}
\]
The program \Code{(Subst (+ f (let (f 1) f)) f 5)} is of the case. The sugar introduces a local-binding on the variable \m{f}, with its sub-expression contains multiple \m{f}. By calculating the context rules of \m{Subst}, the sugar has to be expanded after the $e_3$ being a value. After desugaring to \Code{(let (f 5) (+ f (let (f 1) (+ f 1))))},  no hygienic problem will take place because of the capture-avoiding substitution in the core language.

Because of the definition of desugaring in our approach, we cannot achieve hygiene by proving the $\alpha-equivalence$.
Here what we want to show is that, even without complex things like macro systems, scope specification and so on, the lazy desugaring itself will solve the common hygienic problem with carefully-designed core language. And of course the lazy desugaring will also work together with a hygienic desugaring system (e.g., by specific the binding scope \cite{10.5555/1792878.1792884}). Also, the scope inference of syntactic sugar\cite{resugaringscope} provides a good perspective for solving the hygienic problem.
\subsection{Limitation on Presentation}
The context rules of our sugar setting limit the presentation of syntactic sugar. For example, it is difficult to present a sugar with ellipses as pattern variables in our current tool, because the form of its context rule may vary. It is still possible if we add some
restriction (so that the algorithm \ref{alg:f} will work). On the other hand, we can just make it using the list operation just as what the sugar \m{Map, Filter} work. Overall, the presentation of our sugar system is not so flexible, but it won't affect the expressiveness.
%!TEX root = ./main.tex
\section{Related Work}
\label{sec6}
%Explain the work that are related to your problem, and to your three contributions.

As discussed many times before, our work is much related to the pioneering work of \emph{resugaring} in \cite{resugaring,hygienic}. The idea of "tagging" and "reverse desugaring" is a clear explanation of "resugaring", but it becomes very complex when the RHS of the desugaring rule becomes complex. Our approach does not need to reverse desugaring, and is more lightweight, powerful, and efficient.
For hygienic resugaring, compared with the approach of using DAG to solve the variable binding  problem in \cite{hygienic}, our approach of "lazy desugaring" can achieve kind of natural hygiene within our core language.



\emph{Macros as multi-stage computations} \cite{multistage} is a work related to our lazy expansion for sugars. Some other researches \cite{modularstaging} about multi-stage programming \cite{MSP} indicate that it is useful for implementing domain-specific languages. However, multi-stage programming is a metaprogramming method, which mainly works for run-time code generation and optimization. In contrast, our lazy resugaring approach treats sugars as part of a mixed language, rather than separate them by staging. Moreover, the lazy desugaring gives us a chance to derive evaluation rules of sugars, which is a new good point compared to multi-stage programming.

Our work is related to the \emph{Galois slicing for imperative functional programs} \cite{slicing}, a work for dynamic analyzing functional programs during execution. The forward component of the Galois connection maps a partial input $x$ to the greatest partial output $y$ that can be computed from $x$; the backward component of the Galois connection maps a partial output $y$ to the least partial input $x$ from which we can compute $y$.
%Our approach used a similar idea on slicing expressions and processing on subexpressions.
This can also be considered as a bidirectional transformation \cite{bx,lens07} and the round-tripping between desugaring and resugaring in the existing approach. In contrast to these works, our resugaring approach is basically unidirectional. It should be noted that Galois slicing may be useful to handle side effects in resugaring in the future (for example, slicing the part where side effects appear).


There is a long history of hygienic macro expansion\cite{hygienicmacro}, and a formal specific hygiene definition was given \cite{10.5555/1792878.1792884} by specific the binding scopes of macros. another formal definition of the hygienic macro\cite{EssenceofHygiene} is based on nominal logic\cite{10.1007/s001650200016}. Instead of using the desugaring rule or something else to achieve hygiene, we use the lazy desugaring with the small core language to avoid hygienic problem in our approach.
%
%When the tracking in their notation can be easily done for sugar whose rules can be derived automatically.

Our implementation is built upon the PLT Redex \cite{SEwPR}, a semantics engineering tool, but it is possible to implement our approach on other semantics engineering tools such as those in \cite{dynsem,Ksemantic} which aim to test or verify the semantics of languages. The methods of these researches can be easily combined with our approach to implementing more general rule derivation. \emph{Ziggurat} \cite{Ziggurat} is a semantic extension framework, also allowing defining new macros with semantics based on existing terms in a language. It is should be useful for static analysis of macros.
%Instead of semantics based on core language, the reduction rules of sugar derived by our approach is independent of core language, which may be more concise for static analysis.


%!TEX root = ./main.tex
\label{sec7}\section{Conclusion}

Summarize the paper, explaining what you have shown, what results you have achieved, and what future work is.

%% Acknowledgments
\begin{acks}                            %% acks environment is optional
                                        %% contents suppressed with 'anonymous'
  %% Commands \grantsponsor{<sponsorID>}{<name>}{<url>} and
  %% \grantnum[<url>]{<sponsorID>}{<number>} should be used to
  %% acknowledge financial support and will be used by metadata
  %% extraction tools.
  This material is based upon work supported by the
  \grantsponsor{GS100000001}{National Science
    Foundation}{http://dx.doi.org/10.13039/100000001} under Grant
  No.~\grantnum{GS100000001}{nnnnnnn} and Grant
  No.~\grantnum{GS100000001}{mmmmmmm}.  Any opinions, findings, and
  conclusions or recommendations expressed in this material are those
  of the author and do not necessarily reflect the views of the
  National Science Foundation.
\end{acks}


%% Bibliography
\bibliography{reference}


%% Appendix
% \appendix
% \section{Appendix}

% Text of appendix \ldots

\end{document}
