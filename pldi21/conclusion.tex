%!TEX root = ./main.tex
\section{Conclusion}
\label{sec7}

%Summarize the paper, explaining what you have shown, what results you have achieved, and what future work is.

In this paper, we purpose a novel resugaring approach by lazy desugaring.
Overall, rather than a tool for existing programming languages, our approach is more likely to be a feature of meta-language. We design the approach based on a small core language, with a simple desugaring system.
In our resugaring approach, the most important insight is delaying the expansion of syntactic sugars by calculating context rules(see in Section \ref{sec3}), which decide whether the mixed language should reduce the sub-expression or expand the sugar. The lazy desugaring gives our approach chances to achieve better efficiency and expressiveness.



As for the future work, we found side effects are troublesome to handle in resugaring, because once a side effect is taken in RHS of a desugaring rule, the sugar cannot be easily resugared according to \emph{emulation} property. We need to find a gentler way to handle sugars with side effects. We also want to extend the core language and the desugaring system with things like type systems, analyzers. In addition, we found it is possible to derivate the stand-alone evaluation rules for the surface language by means the same as calculating the context rules. Maybe there is a more gentle way for developing domain-specific languages.
