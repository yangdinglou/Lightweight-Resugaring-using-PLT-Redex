%!TEX root = ./main.tex
\section{Conclusion}
\label{sec7}

%Summarize the paper, explaining what you have shown, what results you have achieved, and what future work is.

In this paper, we purpose a novel resugaring approach by lazy desugaring.
%Essentially, we would see the derivation of evaluation rules is the abstract of the basic resugaring approach.
In our resugaring approach, the most important part is calculating context rules for the syntactic sugars (see in Section \ref{sec3}), which decides whether it should reduce the subexpression or desugar the outermost sugar. The lazy desugaring gives our approach chances to achieve better efficiency and expressiveness.



As for the future work, we found side effects are troublesome to handle in resugaring, because once a side effect is taken in RHS of a desugaring rule, the sugar cannot be easily resugared according to \emph{emulation} property. We need to find a gentler way to handle sugars with side effects. In addition, we found it is possible to derivate the stand-alone evaluation rules for the surface language by means same as calculating the context rules. Maybe there is a more gentle way for developing domain-specific languages.
