% !TEX program = pdflatex
%!TEX spellcheck
%% For double-blind review submission, w/o CCS and ACM Reference (max submission space)
\documentclass[sigplan,10pt,review,anonymous]{acmart}\settopmatter{printfolios=true,printccs=false,printacmref=false}
%% For double-blind review submission, w/ CCS and ACM Reference
%\documentclass[acmsmall,review,anonymous]{acmart}\settopmatter{printfolios=true}
%% For single-blind review submission, w/o CCS and ACM Reference (max submission space)
%\documentclass[acmsmall,review]{acmart}\settopmatter{printfolios=true,printccs=false,printacmref=false}
%% For single-blind review submission, w/ CCS and ACM Reference
%\documentclass[acmsmall,review]{acmart}\settopmatter{printfolios=true}
%% For final camera-ready submission, w/ required CCS and ACM Reference
%\documentclass[acmsmall]{acmart}\settopmatter{}


%% Journal information
%% Supplied to authors by publisher for camera-ready submission;
%% use defaults for review submission.
\acmJournal{PACMPL}
\acmVolume{1}
\acmNumber{CONF} % CONF = POPL or ICFP or OOPSLA
\acmArticle{1}
\acmYear{2018}
\acmMonth{1}
\acmDOI{} % \acmDOI{10.1145/nnnnnnn.nnnnnnn}
\startPage{1}

%% Copyright information
%% Supplied to authors (based on authors' rights management selection;
%% see authors.acm.org) by publisher for camera-ready submission;
%% use 'none' for review submission.
\setcopyright{none}
%\setcopyright{acmcopyright}
%\setcopyright{acmlicensed}
%\setcopyright{rightsretained}
%\copyrightyear{2018}           %% If different from \acmYear

%% Bibliography style
\bibliographystyle{ACM-Reference-Format}
%% Citation style
%% Note: author/year citations are required for papers published as an
%% issue of PACMPL.
%\citestyle{acmauthoryear}   %% For author/year citations


%%%%%%%%%%%%%%%%%%%%%%%%%%%%%%%%%%%%%%%%%%%%%%%%%%%%%%%%%%%%%%%%%%%%%%
%% Note: Authors migrating a paper from PACMPL format to traditional
%% SIGPLAN proceedings format must update the '\documentclass' and
%% topmatter commands above; see 'acmart-sigplanproc-template.tex'.
%%%%%%%%%%%%%%%%%%%%%%%%%%%%%%%%%%%%%%%%%%%%%%%%%%%%%%%%%%%%%%%%%%%%%%


%% Some recommended packages.
\usepackage{booktabs}   %% For formal tables:
                        %% http://ctan.org/pkg/booktabs
\usepackage{subcaption} %% For complex figures with subfigures/subcaptions
                        %% http://ctan.org/pkg/subcaption

\usepackage{algorithm}
\usepackage{tabularx}
\usepackage{algorithmic}
\usepackage{proof}
\usepackage{alltt}
\usepackage{pgf}
\usepackage{bcproof}

\renewcommand{\algorithmicrequire}{\textbf{Input:}}

\renewcommand{\algorithmicensure}{\textbf{Output:}}

\newtheorem{Def}{Defination}[section]
\newtheorem{mythm}{Theorem}[section]
\newtheorem{property}{Property}[section]
\newtheorem{lemma}{Lemma}
\newtheorem{Asm}{Assumption}

\newcommand{\Code}[1]{\texttt{#1}}
\newenvironment{Codes}
  {\begin{alltt}\leftskip=1.5em} % \tiny
  {\end{alltt}}

\newenvironment{smallCodes}
  {\begin{alltt}\leftskip=1.5em\small} %
  {\end{alltt}}

\newcommand{\OneStep}{{\rule{0pt}{1.2\baselineskip}{\ensuremath\longrightarrow}}}
\newcommand{\DeStep}{{\rule{0pt}{1.2\baselineskip}{\ensuremath\dashrightarrow}}}

\newcommand\m[1]{\mbox{\tt #1}}
\newcommand\key[1]{\mbox{\rm \bf #1}}
\newcommand\drule[2]{#1 ~\rightarrow_d~ #2}
\newcommand\redc[2]{#1 ~\rightarrow_c~#2}
\newcommand\rede[2]{#1 ~\rightarrow_e~#2}
\newcommand\redm[2]{#1 ~\rightarrow_m~#2}
\newcommand\note[1]{\mbox{{\scriptsize #1}}}
\newcommand\ignore[1]{}

\def\coreId{\m{cId}}
\def\surfId{\m{sId}}
\def\headId{\m{hId}}

\def\true{\#t}
\def\false{\#f}

\def\myend{\flushright{\qed}}

% some macros for editing/commenting the paper

\def\modify#1#2#3{{\small\underline{\sf{#1}}:} {\color{red}{\small #2}}
{{\color{red}\mbox{$\Rightarrow$}}} {\color{blue}{#3}}}

\newcommand{\hmodify}[2]{\modify{Hu}{#1}{#2}}
\newcommand\mymargin[1]{\marginpar{{\flushleft\textsc\footnotesize {#1}}}}
\newcommand\hmargin[1]{\mymargin{Hu:\;#1}}

\newcommand{\hmodifyok}[2]{#2}

\newcommand{\mycomment}[2]{{\small\color{magenta}\underline{\sf{#1}}:} {\color{magenta}{\small #2}}}
\newcommand{\hcomment}[1]{\mycomment{Hu}{#1}}
\newcommand{\gcomment}[1]{\mycomment{G}{#1}}
\newcommand{\todo}[1]{\mycomment{Todo}{#1}}

\newcommand{\xcomment}[1]{\mycomment{X}{#1}}

\newcommand{\reduce}[1]{{\color{blue}{#1}}}
\newcommand{\reducedversion}[1]{{\color{blue}{#1}}}

\newcommand{\example}[3]{
\begin{figure}[thb]
\begin{center}
#1
\end{center}
\caption{#2}
\label{#3}
\end{figure}
}

\makeatletter
\newcommand{\xleftrightarrow}[2][]{\ext@arrow 3359\leftrightarrowfill@{#1}{#2}}
\newcommand{\xdashrightarrow}[2][]{\ext@arrow 0359\rightarrowfill@@{#1}{#2}}
\newcommand{\xdashleftarrow}[2][]{\ext@arrow 3095\leftarrowfill@@{#1}{#2}}
\newcommand{\xdashleftrightarrow}[2][]{\ext@arrow 3359\leftrightarrowfill@@{#1}{#2}}
\def\rightarrowfill@@{\arrowfill@@\relax\relbar\rightarrow}
\def\leftarrowfill@@{\arrowfill@@\leftarrow\relbar\relax}
\def\leftrightarrowfill@@{\arrowfill@@\leftarrow\relbar\rightarrow}
\def\arrowfill@@#1#2#3#4{%
  $\m@th\thickmuskip0mu\medmuskip\thickmuskip\thinmuskip\thickmuskip
   \relax#4#1
   \xleaders\hbox{$#4#2$}\hfill
   #3$%
}
\makeatother

\makeatletter
\newcommand*{\dashdownarrow}{%
  \mathrel{%
    \mathpalette\dasharrow@vert{-90}%
  }%
}
\newcommand*{\dashuparrow}{%
  \mathrel{%
    \mathpalette\dasharrow@vert{90}%
  }%
}
\newcommand*{\dasharrow@vert}[2]{%
  \sbox0{$#1\vcenter{}$}%
  \sbox2{$#1\dashrightarrow\m@th$}%
  \dimen@=1.2\dimexpr\ht2-\ht0\relax
  % 1/2 width of the new symbol with side bearing
  \sbox2{\raisebox{-\ht0}{\unhcopy2}}%
  \ht2=\z@
  \dp2=\z@
  \vcenter{\hbox to 2\dimen@{\hfill\rotatebox{#2}{\box2}\hfill}}%
}
\makeatother

\makeatletter
\newcommand*\bigcdot{\mathpalette\bigcdot@{.5}}
\newcommand*\bigcdot@[2]{\mathbin{\vcenter{\hbox{\scalebox{#2}{$\m@th#1\bullet$}}}}}
\makeatother

\begin{document}

%% Title information
\title%[Short Title]
{Compositional Resugaring by Lazy Desugaring}
%{A lightweight resugaring approach based on reduction semantics}
%% [Short Title] is optional;
                                        %% when present, will be used in
                                        %% header instead of Full Title.
%\titlenote{with title note}             %% \titlenote is optional;
                                        %% can be repeated if necessary;
                                        %% contents suppressed with 'anonymous'
%\subtitle{Subtitle}                     %% \subtitle is optional
%\subtitlenote{with subtitle note}       %% \subtitlenote is optional;
                                        %% can be repeated if necessary;
                                        %% contents suppressed with 'anonymous'


%% Author information
%% Contents and number of authors suppressed with 'anonymous'.
%% Each author should be introduced by \author, followed by
%% \authornote (optional), \orcid (optional), \affiliation, and
%% \email.
%% An author may have multiple affiliations and/or emails; repeat the
%% appropriate command.
%% Many elements are not rendered, but should be provided for metadata
%% extraction tools.

%% Author with single affiliation.
\author{First1 Last1}
\authornote{with author1 note}          %% \authornote is optional;
                                        %% can be repeated if necessary
\orcid{nnnn-nnnn-nnnn-nnnn}             %% \orcid is optional
\affiliation{
  \position{Position1}
  \department{Department1}              %% \department is recommended
  \institution{Institution1}            %% \institution is required
  \streetaddress{Street1 Address1}
  \city{City1}
  \state{State1}
  \postcode{Post-Code1}
  \country{Country1}                    %% \country is recommended
}
\email{first1.last1@inst1.edu}          %% \email is recommended

%% Author with two affiliations and emails.
\author{First2 Last2}
\authornote{with author2 note}          %% \authornote is optional;
                                        %% can be repeated if necessary
\orcid{nnnn-nnnn-nnnn-nnnn}             %% \orcid is optional
\affiliation{
  \position{Position2a}
  \department{Department2a}             %% \department is recommended
  \institution{Institution2a}           %% \institution is required
  \streetaddress{Street2a Address2a}
  \city{City2a}
  \state{State2a}
  \postcode{Post-Code2a}
  \country{Country2a}                   %% \country is recommended
}
\email{first2.last2@inst2a.com}         %% \email is recommended
\affiliation{
  \position{Position2b}
  \department{Department2b}             %% \department is recommended
  \institution{Institution2b}           %% \institution is required
  \streetaddress{Street3b Address2b}
  \city{City2b}
  \state{State2b}
  \postcode{Post-Code2b}
  \country{Country2b}                   %% \country is recommended
}
\email{first2.last2@inst2b.org}         %% \email is recommended


%% Abstract
%% Note: \begin{abstract}...\end{abstract} environment must come
%% before \maketitle command



%% 2012 ACM Computing Classification System (CSS) concepts
%% Generate at 'http://dl.acm.org/ccs/ccs.cfm'.
\begin{CCSXML}
<ccs2012>
<concept>
<concept_id>10011007.10011006.10011008</concept_id>
<concept_desc>Software and its engineering~General programming languages</concept_desc>
<concept_significance>500</concept_significance>
</concept>
<concept>
<concept_id>10003456.10003457.10003521.10003525</concept_id>
<concept_desc>Social and professional topics~History of programming languages</concept_desc>
<concept_significance>300</concept_significance>
</concept>
</ccs2012>
\end{CCSXML}

\ccsdesc[500]{Software and its engineering~General programming languages}
\ccsdesc[300]{Social and professional topics~History of programming languages}
%% End of generated code


%% Keywords
%% comma separated list
\keywords{Resugaring, Syntactic Sugar, Interpreter, Domain-Specific Language, Reduction Semantics}  %% \keywords are mandatory in final camera-ready submission


%% \maketitle
%% Note: \maketitle command must come after title commands, author
%% commands, abstract environment, Computing Classification System
%% environment and commands, and keywords command.
\maketitle



%% Acknowledgments
\begin{acks}                            %% acks environment is optional
                                        %% contents suppressed with 'anonymous'
  %% Commands \grantsponsor{<sponsorID>}{<name>}{<url>} and
  %% \grantnum[<url>]{<sponsorID>}{<number>} should be used to
  %% acknowledge financial support and will be used by metadata
  %% extraction tools.
  This material is based upon work supported by the
  \grantsponsor{GS100000001}{National Science
    Foundation}{http://dx.doi.org/10.13039/100000001} under Grant
  No.~\grantnum{GS100000001}{nnnnnnn} and Grant
  No.~\grantnum{GS100000001}{mmmmmmm}.  Any opinions, findings, and
  conclusions or recommendations expressed in this material are those
  of the author and do not necessarily reflect the views of the
  National Science Foundation.
\end{acks}


%% Bibliography



%% Appendix
\appendix
\section{Appendix}
\subsection{Proof of Properties}
\begin{lemma}
  For a syntactic sugar $S$, with the desugaring rule
  \[
  \drule{(\m{SurfHead}~e_1~e_2~\ldots~e_n)}{(\m{Head}~\ldots~e_1'~e_2'~\ldots~e_m')}
  \]
  If the algorithm \m{calcontext} gets the context rules as follows.\\
$S.LHS[[\bigcdot]/e_i]$\\
$S.LHS[v_i/e_i, [\bigcdot]/e_j]$\\
$\ldots$\\
$S.LHS[v_i/e_i, v_j/e_j, \ldots, [\bigcdot]/e_p]$\\
$S.LHS[v_i/e_i, v_j/e_j, \ldots, v_p/e_p, [\bigcdot]/e_q]$\\
$\ldots$\\
$S.LHS[v_i/e_i, v_j/e_j, \ldots, v_p/e_p, v_q/e_q, \ldots, [\bigcdot]/e_x]$\\
Then for any program \m{P} headed with \m{SurfHead}, if the sub-expressions $\{e_i, e_j, \ldots, e_p\}$ are all values, then $\redc{\mathtt{D}(\m{P})}{\m{P'}}$ will reduce on $\mathtt{D}(e_q)$.
\end{lemma}
\begin{proof}[Proof Sketch of lemma 1]
  In the algorithm \m{calcontext}, the computational order of $S.RHS$ is iterated. So the context rules of the sugar correspond to the computational order.
  So if the program \m{P} will reduce on $e_q$, the desugared $\mathtt{D}(\m{P})$ will also reduce on the same part.
\end{proof}

\begin{lemma}
  For a syntactic sugar $S$, with the desugaring rule
\[
\drule{(\m{SurfHead}~e_1~e_2~\ldots~e_n)}{(\m{Head}~\ldots~e_1'~e_2'~\ldots~e_m')}
\]
If the algorithm \m{calcontext} gets the context rules as follows.\\
$S.LHS[[\bigcdot]/e_i]$\\
$S.LHS[v_i/e_i, [\bigcdot]/e_j]$\\
$\ldots$\\
$S.LHS[v_i/e_i, v_j/e_j, \ldots, [\bigcdot]/e_x]$\\
Then for any program \m{P} headed with \m{SurfHead}, if the sub-expressions $\{e_i, e_j, \ldots, e_x\}$ are all values, $\redc{\mathtt{D}(\m{P})}{\m{P'}}$ will destroy the $S.RHS$'s form.
\end{lemma}
\begin{proof}[Proof Sketch of lemma 2]
  In the algorithm \m{calcontext}, the iteration recursively runs on an inner sub-expression which is not $e_i$ or values, then it should be an expression with a constructor. Because of the computational order on it, the inner sub-expression can be reduced (which means the $RHS$'s form is destroyed) if its context rules are iterated. So whenever the recursive call on \m{calcontext} is returned, the whole \m{calcontext} should break.

  If $\redc{\mathtt{D}(\m{P})}{\m{P'}}$ does not destroy the $S.RHS$'s from, the reduction will be on $e_i$, which is conflicted with our algorithm \m{calcontext}.
\end{proof}


\begin{proof} [Proof Sketch of Property 3.1]
  If $\redm{\m{E}[\m{S}]}{\m{E}[\m{S'}]}$ and $\drule{\m{S}}{\m{S'}}$, according to lemma 1 and context rules of core language's expression, the program \m{P} reduces recursively on the correct sub-expression, so $\redc{\m{E'}[\m{C}]}{\m{E'}[\m{C'}]}$.
  According to lemma 2, the reduction will destroy the $RHS$'s form of the sugar $S$.

\end{proof}

\begin{proof} [Proof Sketch of Property 3.2]
  If $\redm{\m{E}[\m{CC}]}{\m{E}[\m{CC'}]}$ reduced by the \m{CoreHead}'s reduction rule on \m{CC}, according to lemma 1 and context rules of core language's expression, the program \m{P} reduces recursively on the correct sub-expression, so for $\redm{\m{E'}[\m{C}]}{\m{E'}[\m{C'}]}$, it will reduce on the same sub-expression.

  Because the sub-expression \m{CC} can be reduced by its \m{CoreHead}, so matter how its inner expressions desugared, according to the context rules, the fully desugared \m{C} will also reduced by its \m{CoreHead}.
  
\end{proof}


\end{document}
